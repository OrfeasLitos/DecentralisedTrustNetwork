
    \begin{lemma}[No Evil Edges in the $MinCut$] \ \\
       \label{mincutmany}
       Let $S \subset \mathcal{V}, A \notin S$. When calculating $MaxFlow\left(A, S\right)$, it is impossible to have an edge
       $\left(v, w\right) \in MinCut\left(A, S\right) : v \in S$.
    \end{lemma}
    \begin{proof}[No Evil Edges in the $MinCut$ Lemma (\ref{mincutmany}]
    \label{mincutmanyproof}
       Let $T$ be the auxiliary node. It is
       \begin{equation}
          \forall v \in S, c_{vT} = \infty \enspace.
       \end{equation}
       We can see that $out_A < \infty$ and thus
       \begin{equation}
       \label{maxflowbounded}
          maxFlow\left(A, S\right) < \infty \enspace.
       \end{equation}
       Since all edges in the $MinCut$ are saturated and due to (\ref{maxflowbounded}), we have
       \begin{equation}
          \nexists v \in S : \left(v, T\right) \in MinCut \enspace.
       \end{equation}
       Suppose that
       \begin{equation}
          \exists v \in S, w \in \mathcal{V} : \left(v, w\right) \in MinCut \enspace.
       \end{equation}
       Then this edge must be saturated, that is $x_{vw} = c_{vw} > 0$. However, there exists an alternative flow
       configuration $X'$ where
       \begin{equation}
       \begin{gathered}
          \forall \left(u, u'\right) \in \mathcal{E} \setminus \{\left(v, w\right), \left(v, T\right)\}, x_{u,u'}' =
          x_{u,u'} \enspace, \\
          x_{vw}' = 0 \mbox{ and} \\
          x_{vT}' = x_{vT} + x_{vw} \enspace,
       \end{gathered}
       \end{equation}
       which is valid because
       \begin{equation}
          \begin{rcases}
             \sum\limits_{w \in N^{+}\left(v\right)}x_{vw} = \sum\limits_{w \in N^{+}\left(v\right)}x_{vw}' \\
             c_{vT} = \infty
          \end{rcases}
          \Rightarrow x_{vT}' \leq c_{vT}
       \end{equation}
       and $X'$ is maximum as well because it carries exactly the same flow as $X$. Thus
       \begin{equation}
          \left(v, w\right) \notin MinCut \enspace.
       \end{equation}
       \qed
    \end{proof}

    \begin{sepproof} (Risk Invariance Theorem (\ref{riskinv}))
       Let
       \begin{align*}
          \forall v,w \in \mathcal{V}', c'_{vw} &= DTr'_{v \rightarrow w} \mbox{ and} \\
          \forall v,w \in \mathcal{V}'', c''_{vw} &= DTr''_{v \rightarrow w} \enspace.
       \end{align*}
       Then
       It is
       \begin{equation}
       \label{ccompare}
          \forall v, w \in \mathcal{V}, c'_{vw} \leq c''_{vw}
       \end{equation}
       and any valid flow on $\mathcal{G}'$ is a valid flow on $\mathcal{G}''$ as well. Furthermore, any
       $MaxFlow\left(A, B\right)$ chooses $x_{AB} = c_{AB}$, thus
       \begin{equation}
       \label{xcompare}
          x''_{AB} = x'_{AB} + l \enspace.
       \end{equation}
       From (\ref{ccompare}) and (\ref{xcompare}) we see that
       \begin{equation}
       \label{doublebigger}
          maxFlow_{\mathcal{G}''}\left(A, B\right) \geq maxFlow_{\mathcal{G}'}\left(A, B\right) + l \enspace.
       \end{equation}
       Now suppose that
       \begin{equation}
       \label{mfsupposition}
          maxFlow_{\mathcal{G}''}\left(A, B\right) > maxFlow_{\mathcal{G}'}\left(A, B\right) + l \enspace.
       \end{equation}
       Then 
       \begin{equation*}
          \sum\limits_{v \in N^{-}\left(B\right)'' \setminus \{A\}}x''_{vB} > \sum\limits_{v \in N^{-}\left(B\right)'
          \setminus \{A\}}x'_{vB} \enspace.
       \end{equation*}
       However, it holds that
       \begin{equation}
       \label{cequal}
          \forall e \in \mathcal{V} \setminus \{\left(A, B\right)\}, c'_e = c''_e \enspace,
       \end{equation}
       and $x_{AB}$ flows directly from $A$ to $B$ without adding to the incoming or outgoing flow of any intermediate node,
       thus $MaxFlow_{\mathcal{G}''}$ can choose
       \begin{equation*}
          \forall e \in \mathcal{V} \setminus \{\left(A, B\right)\}, x''_e = x'_e
       \end{equation*}
       and thus, by contradiction with (\ref{mfsupposition}), it holds that
       \begin{equation}
       \label{singlebigger}
          maxFlow_{\mathcal{G}''}\left(A, B\right) \leq maxFlow_{\mathcal{G}'}\left(A, B\right) + l \enspace.
       \end{equation}
       From (\ref{doublebigger}) and (\ref{singlebigger}) we get
       \begin{equation}
       \label{mfequal}
          maxFlow_{\mathcal{G}''}\left(A, B\right) = maxFlow_{\mathcal{G}'}\left(A, B\right) + l \enspace.
       \end{equation}
       Finally, it holds that
       \begin{equation*}
       \begin{gathered}
          Tr''_{A \rightarrow B} = maxFlow_{\mathcal{G}''}\left(A, B\right) \overset{\left(\ref{mfequal}\right)}{=} \\
          = maxFlow_{\mathcal{G}'}\left(A, B\right) + l = Tr'_{A \rightarrow B} + l
          \overset{\left(\ref{primetrust}\right)}{=} Tr_{A \rightarrow B} \enspace.
       \end{gathered}
       \end{equation*}
       The proposition is proved. \qed
    \end{sepproof}

     \href{http://www.ebay.com}{ebay} is
     centralized and as such it is vulnerable to ddos attacks \cite{ddosattacks} and can be considered as a single point of
     failure, it charges fees for the use of its services \cite{ebayfees} and maintains a private database of personal data,
     but it can give money-back guarantees \cite{ebayguarantee} since it is run by a single company that has a financial
     advantage in keeping its clients satisfied. On the other hand, \href{https://openbazaar.org/}{OpenBazaar} is a
     decentralized platform built on bitcoin \cite{bitcoin}, where individual stores or its \href{https://duosear.ch}{search
     engine} are vulnerable to ddos attacks \cite{ddosattacks}, but not the platform as a whole. Additionally, it does not
     charge fees for its usage \cite{openbazaar} and there is no central agent recording all the transactions alongside with
     private data \cite{openbazaar} but it is possible for a buyer or a seller colluding with a  moderator to scam the third
     party and there exists no central authority able to verify the truth of her claim and reimburse her
     \cite{multisigfraud}. Even though trust (or distrust) should be directed to each store individually, it is very likely
     that the whole platform will be discarded as untrusted by the scammed party.

       Without loss of generality, we can suppose that the turn in which we are interested is 0 ($\mathcal{G} =
       \mathcal{G}_0$). First we will show that the $MaxFlow$ can be a result of a valid execution of algorithm
       \ref{transitivegame} and afterwards we will show that each valid execution of algorithm \ref{transitivegame}
       corresponds to a valid flow from $A$ to $B$. Thus we will have proven that $Tr_{A \rightarrow B} = maxFlow(A, B)$.
       \begin{itemize}
          \item We will first show that there exists an execution of algorithm \ref{transitivegame} such that $Loss_A =
          maxFlow(A, B)$. Let $X$ be the flows as returned by an execution of the $MaxFlow(A, B)$ algorithm on $\mathcal{G}$.
          It is known that all flows are DAGs [citation needed] and that all DAGs are a partial order of their nodes based on
          the partial ordering $x_{vw} \leq 0 \Rightarrow v < w$ [citation needed]. From this partial order, we can create a
          total order with an algorithm such as topoSort \cite{toposort}. The maximum element of the total order is a node
          that does not have any outgoing flow. Removing any node from a DAG cannot create a cycle, thus the graph that
          remains after removing a node from a DAG is also a DAG, thus it has a total order as well, which can be chosen to
          be the same total order as before removing the node, except for the removed node. If the removed node was maximum
          or minimum, the new total order is obvious. We will prove our claim using induction. \\
          \begin{itemize}
             \item Player $B$ is the maximum node in turn 0 because she is the sink of the MaxFlow algorithm, thus she is the
             first to be chosen to play and steals all her incoming and outgoing trust. $\forall v \in N^{-}(B)_0, x_{vB}
             \leq DTr_{v \rightarrow B, 0}$ and $\sum\limits_{v \in N^{-}(B)_0}x_{vB} = maxFlow(A, B)$. The graph
             $\mathcal{G}_1 = \mathcal{G}_0 \setminus \{B\}$ is also a DAG and corresponds to the previous total order if we
             remove the maximum element, $B$.
             \item Suppose that $\forall j \in [k], k > 0$, the player $v$ corresponding to the maximum element is chosen to
             play for the first time, that $\forall w \in N^{-}(v)_{j-1} (= N^{-}(v)_0), x_{wv} \leq y$ where $Steal(y,w) \in
             Turn_j \wedge \sum\limits_{w \in N^{-}(v)_0}x_{wv} = \sum\limits_{w \in N^{+}(v)_0}x_{vw}$.
             \item For $j = k+1$, $Player(k+1) = v'$ corresponds to the maximum element of the previous total order with the
             element $v$ removed and it is the first time player $v'$ plays, since $v > v'$ in all previous steps thus $v'$
             was not maximum. It also holds that $\forall w \in N^{-}(v')_0, x_{wv'} \leq DTr_{w \rightarrow v', 0}$ since
             the $x_{wv'}$ are chosen by the maxFlow algorithm with corresponding capacities the direct trusts and, since
             $\sum\limits_{w \in N^{-}(v')_0}x_{wv'} = \sum\limits_{w \in N^{+}(v')_0}x_{v'w}$ and player $v'$ has already
             been stolen value equal to $\sum\limits_{w \in N^{+}(v')_0}x_{v'w}$ (since she has no outgoing flow in turn
             $j$), player $v'$ can choose to steal from each player $w \in N^{-}(v')_0$ value at least equal to $x_{wv'}$
             without violating the conservative strategy.
          \end{itemize}
          We have proven using induction that if the algorithm chooses only maximum nodes, after exactly $|V(\mathcal{G}_0)|-
          1$ turns (we do not count idle player $A$) every player except for $A$ will have stolen at least value equal to the
          flow passing through them and player $A$ will have been stolen value exactly equal to $maxFlow(A, B) \Rightarrow
          Loss_A = maxFlow(A, B)$.
          \item We will now show that for any valid execution of algorithm \ref{transitivegame} there exists at least one
          valid flow from $A$ to $B$, such that $Loss_A = \sum\limits_{v \in N^{+}(A)_0}x_{Av}$. Let $j$ be a turn where
          \ref{transitivegame} has converged ($j$ exists, according to theorem \ref{convergence}). Then $Loss_{A, j} =
          out_{A, 0} - out_{A, j}$. We create a new graph $\mathcal{G}'$ such that $V(\mathcal{G}') = V(\mathcal{G}) \cup
          \{T\}, E(\mathcal{G}') = E(\mathcal{G}) \cup \{(T, v) : v \in Sad_j\} \cup (T, A), \forall (v, w) \in E(\mathcal{G}),
          c'_{vw} = DTr_{v \rightarrow w, 0} - DTr_{v \rightarrow w, j}, \forall v \in Sad_j, c'_{Tv} = c'_{TA} = \infty$
          ($T$ is an auxiliary source that trusts infinitely $A$ and all the $Sad$ nodes). We execute the $MaxFlow(T, B)$
          algorithm on $\mathcal{G}'$ and we get a flow $X' : \forall v,w \in V(\mathcal{G}), x'_{vw} = c'_{vw}$ (all the
          edges, except for the auxiliary ones, saturated). Thus $\sum\limits_{v \in N^{+}(A)_0}x'_{Av} = Loss_{A, j}$.
          If we create a new graph $\mathcal{G}''$ with $V(\mathcal{G}'') = V(\mathcal{G}'), E(\mathcal{G}'') =
          E(\mathcal{G}'), \forall v \in Sad_j, c''_{Tv} = 0, c''_{TA} = \infty, \forall v,w \in V(\mathcal{G}), c''_{vw} =
          c_{vw}$ (the auxiliary node trusts only $A$) and execute the $MaxFlow(T, B) = X''$ algorithm on $\mathcal{G}''$,
          $\sum\limits_{v \in N^{+}(A)_0}x''_{Av} = \sum\limits_{v \in N^{+}(A)_0}x'_{Av}$ (the outgoing flow from
          $A$ will remain the same as in $\mathcal{G}'$) since no capacity accesible from $A$ has been modified (the only
          changed capacities are those that begin from $T$ and there is no incoming flow to $T$) thus $\sum\limits_{v \in
          N^{+}(A)_0}x''_{Av} \geq \sum\limits_{v \in N^{+}(A)_0}x'_{Av}$ and $\sum\limits_{v \in N^{+}(A)_0}x'_{Av} =
          \sum\limits_{v \in N^{+}(A)_0}c'_{Av} = \sum\limits_{v \in N^{+}(A)_0}c''_{Av}$ (the outgoing edges from $A$ were
          already saturated) thus $\sum\limits_{v \in N^{+}(A)_0}x''_{Av} \leq \sum\limits_{v \in N^{+}(A)_0}x'_{Av}$. Thus
          the resulting flow is equal to $Loss_{A, j}$, or $\sum\limits_{v \in N^{+}(A)_0}x''_{Av} = \sum\limits_{v \in
          N^{+}(A)_0}c'_{Av} = Loss_{A, j}$.
       \end{itemize}
       We finally conclude that $Tr_{A \rightarrow B} = MaxFlow(A, B)$.
          \begin{itemize}
             \item The flow $X$ is obviously valid for the initial graph because $\forall (v,w) \in E(FG), c(v,w) = DTr_{v
             \rightarrow w, 0} \geq DTr_{v \rightarrow w, 0} - DTr_{v \rightarrow w, j} = c'(v,w) \geq x_{vw}$ and it
             already holds that $\forall v \in V(FG'), \sum\limits_{w \in N^{-}(v)}x'_{wv} = \sum\limits_{w \in N^{+}(v)}
             x'_{vw}$, thus it also holds for the flows of $X$.
             \item We can easily see that $Loss_{A,j} \geq \sum\limits_{v \in N{+}(A)}x_{Av}$ because $Loss_{A,j} =
             out_{A,0} - out_{A,j} = \sum\limits_{v \in N^{+}(A)}c'(A,v)$. To show that $Loss_{A,j} \leq
             \sum\limits_{v \in N{+}(A)}x_{Av}$, we first suppose that $Loss_{A,j} > \sum\limits_{v \in N{+}(A)}x_{Av}$. We
             will now prove that there exists a residual path from $A$ to $B$. $Loss_{A,j} = \sum\limits_{v \in N^{+}(A)}
             (DTr_{A \rightarrow v, 0} - DTr_{A \rightarrow v, j}) = \sum\limits_{v \in N^{+}(A)}c_{Av}$. From the
             supposition we can see that $\sum\limits_{v \in N^{+}(A)}c_{Av} > \sum\limits_{v \in N^{+}(A)}x_{Av}
             \Rightarrow \exists v \in N^{+}(A) : c_{Av} > x_{Av}$. \\
             Since $\forall v \in V(FG) \setminus \{A,B\}, \sum\limits_{w \in N^{-}(v)}c_{wv} \overset{conservative}{=}
             \sum\limits_{w \in N^{+}(v)}c_{vw} \wedge \sum\limits_{w \in N^{-}(v)}x_{wv} \overset{flow}{=}
             \sum\limits_{w \in N^{+}(v)}x_{vw}$, it holds that $\forall v \in V(FG) \setminus \{A,B\}, \sum\limits_{w \in
             N^{-}(v)}(c_{wv} - x_{wv}) = \sum\limits_{w \in N^{+}(v)}(c_{vw} - x_{vw})$. \\
             We will now show that $\forall v \in V(FG) \setminus
             \{A,B\}, (\exists w \in V(FG) : c_{wv} > x_{wv} \Rightarrow \exists u \in V(FG) : c_{vu} > x_{vu})$. Suppose
             that the previous statement is false. Then it would hold that $\exists v \in V(FG) \setminus \{A,B\} :
             (\exists w \in V(FG) : c_{wv} > x_{wv} \wedge \forall u \in V(FG), c_{vu} = x_{vu})$ (1). But then we have
             $\sum\limits_{w \in V(FG)}c_{wv} \overset{(1)}{>} \sum\limits_{w \in V(FG)}x_{wv} \overset{flow}{=}
             \sum\limits_{w \in V(FG)}x_{vw} \overset{(1)}{=} \sum\limits_{w \in V(FG)}c_{vw} \overset{conservative}{=}
             \sum\limits_{w \in V(FG)}c_{wv} \Rightarrow \sum\limits_{w \in V(FG)}c_{wv} > \sum\limits_{w \in V(FG)}c_{wv}$
             which is a contradiction. Thus we showed that $\forall v \in V(FG) \setminus \{A,B\}, (\exists w \in V(FG) :
             c_{wv} > x_{wv} \Rightarrow \exists u \in V(FG) : c_{vu} > x_{vu})$. \\
             The flow graph that resulted from
             $MaxFlow(A, B)$ is a DAG, thus there exists a corresponding total ordering, as we saw before.
             Obviously $A = v_0$ and $B = v_{|V(FG)|}$. When an element $v_k$ is in the $k$-th position in this total
             ordering, it has incoming flow only from smaller elements and outgoing flow only to bigger elements, that is
             $\forall l < k, x_{kl} = 0 \wedge \forall m > k, x_{mk} = 0$.
             Thus the previous result can be rewritten this way: $\forall k \in [|V(FG)|],
             (\exists l < k : c_{v_lv_k} > x_{v_lv_k} \Rightarrow \exists m > k : c_{v_kv_m} > x_{v_kv_m})$.
             Thus, the supposition $Loss_{A, j} > \sum\limits_{v \in N^{+}(A)}x_{Av}$ combined with the previous result shows
             that there exists a residual path from $A$ to $B$ since we can start from $A$ and find a series of sequential
             edges that all have flows smaller than the corresponding capacities and eventually reach $B$ in at most
             $|E(FG)|$ steps, thus $X'$ is not a maximum flow, which is a contradiction. Thus $Loss_{A,j} \leq
             \sum\limits_{v \in N{+}(A)}x_{Av}$ and, since also $Loss_{A,j} \geq \sum\limits_{v \in N{+}(A)}x_{Av}$, we
             deduce that $Loss_{A,j} = \sum\limits_{v \in N{+}(A)}x_{Av}$.
          \end{itemize}
2nd bullet
          \item We will now show that for any valid execution of algorithm \ref{transitivegame} there exists at least one
          valid flow from $A$ to $B$, $X$, such that $Loss_A = \sum\limits_{v \in N^{+}(A)}x_{Av}$. Let $j$ be a turn where
          \ref{transitivegame} has converged ($j$ exists, according to theorem \ref{convergence}). Then $Loss_{A, j} =
          out_{A, 0} - out_{A, j}$. Let $\forall v \in N^{+}(A)_0, x_{Av} = DTr_{A \rightarrow v, 0} - DTr_{A \rightarrow
          v, j}$. For any conservative player $v \in N^{+}(A)_0$, let $\forall w \in N^{+}(v)_0, x_{vw} \leq DTr_{v
          \rightarrow w, 0} - DTr_{v \rightarrow w, j}, \sum\limits_{w \in N^{+}(A)_0}x_{vw} = x_{Av}$. This is possible
          because $v$ is conservative, thus the value she stole from $A$ must have been stolen previously from her. More
          generally, $\forall v \in \mathcal{V}_0 \setminus \{A,B\}, \forall w \in N^{+}(v)_0, x_{vw} \leq
          DTr_{v \rightarrow w, 0} - DTr_{v \rightarrow w, j}, \sum\limits_{w \in N^{+}(v)_0}x_{vw} = \sum\limits_{w \in
          N^{-}(v)_0}x_{wv}$. Since the graph we build is a DAG in every step, which corresponds to a partial order, there
          always exists a total order that we can get using an algorithm such as topoSort [citation needed]. Thus, by
          choosing to calculate the outgoing flows only of the minimum element of this total order, it is possible to create
          a valid flow network from $A$ to $B$ in exactly $|V(FG)| - 1$ iterations of the above steps.
OLD
          \item The flow to $A$ is the flow that results from the following process: After the execution of
          \ref{transitivegame}, for each sad player iteratively replenish the $DTr$ stolen from the sad player by the one
          that stole from her (if multiple players stole from the sad player, then replenish all the stolen $DTr$). Repeat
          the process until the evil player replenishes the initially stolen $DTr$. This is always possible because if there
          is no player who stole from each one who is replenished, then the $Steal()$ she did in the first place would not be
          according to the conservative strategy. Also this process will end with the evil player replenishing $DTr$ equal
          to the sum of $DTr$ that was stolen from sad players because the conservative players cannot avoid replenishing,
          or else they do not follow the conservative strategy. The $DTr$ stolen from $A$ will not be replenished, since
          the player(s) that have stolen from $A$ will not replenish the stolen value and, inductively, this value will not
          be replenished. Thus $A$ will have been stolen the exact same value that the modified evil player has stolen,
          $\forall w,v \in V(FG), DTr_{v \rightarrow w} \geq x_{vw}$ (1st requirement for flows) and there would be no node
          that gets more flow than it pushes, except for $A$ and $B$ (2nd requirement for flows), thus it is a valid flow.
          \item Let $X$ be the flows as returned by an execution of the $maxFlow$ algorithm. The evil player can steal
          the values denoted by $X$ and every other player can steal exactly as much as the $X$ flows denote, since they
          have the 1st property and thus are stealable in any strategy and also hold the 2nd property, thus they comply with
          the conservative strategy. More concretely, $\forall v,w \in V(FG), DTr_{v \rightarrow w}' = x_{vw}$. Then the two
          properties of flows hold:
          \begin{itemize}
             \item $\forall v,w \in V(FG),x_{vw} \leq DTr_{v \rightarrow w}$ and thus any set of strategies that include only
             $Steal()$ actions such that $\sum\limits_{y : Steal(y,w) \in Turn_j, Player(j) = v}y = DTr_{v \rightarrow w} -
             x_{vw}$ is feasible.
             \item $\forall v \in V(FG) \setminus \{A,B\}, \sum\limits_{w \in N^{+}(v)}x_{wv} =
             \sum\limits_{w \in N^{-}(v)}x_{vw}$ thus $\forall v \in V(FG) \setminus \{A,B\}, Strategy(v) = Conservative$.
          \end{itemize}
             
       \end{itemize}
       Thus the maximum value $A$ can lose if $B$ is evil is $Tr_{A \rightarrow B} = maxFlow(A, B)$.
\ \\ OLDER
       \begin{enumerate}
	   \item We will show that $Tr_{A \rightarrow B} \leq MaxFlow(A, B)$.
          We know that $MaxFlow(A, B) = MinCut_{A \rightarrow B}$. We will show that, if everybody except
          A and B follows the conservative strategy,  $Tr_{A \rightarrow B} \leq MinCut_{A \rightarrow B}$. Suppose that in
          round $i$ all the members of the MinCut, $P$, have stolen the maximum value they can from members that belong
          in the MaxFlow graph and nobody in the partition in which $A$ belongs has stolen yet any value. Let the total
          stolen value from the MinCut members be $St$. It is obvious that $St_i \leq MinCut_{A \rightarrow B}$, because
          otherwise there would exist $u \in P$ that doesn't follow the conservative strategy, since they stole more than they
          were stolen from. The same argument holds for any round $i' > i$ because in each round an conservative player can
          steal only up to the value she has been stolen. It is also impossible that the $St$ increase further due to
          stolen value from members of the partition of $B$ since members of $P$ disconnect the two partitions and have
          already played their turns, thus $\forall i' > i, St_{i'} \leq St_i$. There exists a round, $k$, when all the
          conservative players stop stealing, so in the worst case $A$ will have been stolen
          $Tr_{A \rightarrow B} = St_k \leq MinCut_{A \rightarrow B} = MaxFlow(A, B)$.
          \item We can see that $Tr_{A \rightarrow B} \geq MaxFlow(A, B)$ because the strategy where each
          one of the non-idle players steals value equal to the incoming flows from their respective friends is a valid
          strategy that does not contradict with the conservative strategy, since for every conservative player $w$ it holds that
          $\sum\limits_{v \in N^{-}(w)}x_{vw} = \sum\limits_{v \in N^{+}(w)}x_{wv}$ and according to the strategy each
          conservative player will have been stolen value equal to $\sum\limits_{v \in N^{+}(w)}x_{wv}$. More concretely,
          let $Player(j) = B$ and $Player(j+d) = C :$
       \end{enumerate}
       Combining the two results, we see that $Tr_{A \rightarrow B} = MaxFlow(A, B)$.
        OLD PROOF START
        \begin{enumerate}
           \item $Tr_{A \rightarrow B} \geq MaxFlow(A, B)$ because by the definition of $Tr_{A \rightarrow B}$,
           B leaves taking with him all the incoming trust, so there is no trust flowing towards him after leaving.
           $Tr_{A \rightarrow B} < MaxFlow(A, B)$ would imply that after B left, there would still remain trust
           flowing from A to B.
           \item $Tr_{A \rightarrow B} \leq MaxFlow(A, B)$ \\
           Suppose that $Tr_{A \rightarrow B} > MaxFlow(A, B)$ (1). Then, using the min cut - max flow theorem we
           see that there is a set of capacities $U= \{u_1,\dots,u_n\}$ with flows $X = \{x_1,\dots,x_n\}$ such that
           $\sum\limits_{i=1}^{n}{x_i} = MaxFlow(A, B)$ and, if severed $(\forall i \in [n] \: u_i' = 0)$
           the flow from A to B would be $0$, or, put differently, there would be no directed trust path from A to B. No
           strategy followed by B could reduce the value of A, so our supposition (1) cannot be true.
        \end{enumerate}
        OLD PROOF END

     Further Research, short version
     First of all, concrete trust manipulation algorithms are needed to make use of Risk Invariance theorem. Secondly, an
     extension of this work to a dynamic setting where players can enter, leave and execute turns simultaneously and where
     there is no need for an algorithm like TrustIsRisk Game. Furthermore, the fact that $MaxFlow$ needs full network
     knowledge may be undesirable for some parties, consequently there should be further research on zero knowledge methods.
     Moreover, extended game theoretic analysis for cases more complex than the Transitive Game is needed to expand our
     comprehension on the proposed system. Obviously an implementation of the wallet is necessary to make the system
     available and related experimental results can give more insight on its dynamics. Finally, alternative multisig types,
     such as 1-of-3 can be explored.

Zero knowledge
     Our network evaluates indirect trust by computing the max flow in the graph of lines-of-credit. In order to do that,
     complete information about the network is required. However, disclosing the network topology may be undesirable, as it
     subverts the identity of the participants even when participants are treated pseudonymously, as deanonymisation
     techniques can be used \cite{deanonymisation}. To avoid such issues, exploring the ability to calculate flows in a zero
     knowledge fashion may be desirable. However, performing network queries in zero knowledge may allow an adversary to
     extract topological information. More research is required to establish how flows can be calculated effectively in zero
     knowledge and what bounds exist in regards to information revealed in such fashion.

     The current description of TrustIsRisk refers to a static setting where the game evolves in turns. In each turn only one
     user changes the state of the network. In the dynamic setting, the users should be able to play simultaneously, freely
     join, leave and disconnect temporarily from the network.


Definition of individual players in neighbourhood
           \item Let $S \subset \mathcal{V}_j$. Let $N\left(A\right)_{j,i}$ (respectively $N^{+}\left(A\right)_{j,i},
              N^{-}\left(A\right)_{j,i}, N\left(S\right)_{j,i},$ $N^{+}\left(S\right)_{j,i}, N^{-}\left(S\right)_{j,i}$) be
              the $i$-th element of set $N\left(A\right)_j$ (respectively of $N^{+}\left(A\right)_j,  N^{-}\left(A\right)_j,
              N\left(S\right)_j, N^{+}\left(S\right)_j, N^{-}\left(S\right)_j$), according to an arbitrary but fixed
              enumeration of the set players.

    \begin{proofsketch}
       If everybody is conservative, nobody can initiate the chain of steals.
    \end{proofsketch}
    \begin{proof} \ \\
       Suppose that we are interested in graphs $\mathcal{G}_j$. Let $(j_k)$ an increasing sequence of positive integers,
       \begin{equation*}
       \begin{gathered}
          \mbox{let } S_{j_k} \subseteq N^{-}\left(Player\left(j_k\right)\right)_{j_k-1} \mbox{ and} \\
          \mbox{let } \forall v \in S_{j_k}, y_{v, j_k} > 0\enspace.
       \end{gathered}
       \end{equation*}
       Suppose that there exists a subsequence of History, $(Turn_{j_k})$, where
       \begin{equation*}
          Turn_{j_k} = \bigcup\limits_{v \in S_{j_k}}\{Steal(y_{v, j_k},v)\} \enspace,
       \end{equation*}
       This subsequence must have an initial element, $Turn_{j_1}$. However, $Player(j_1)$ follows the conservative strategy,
       thus somebody must have stolen from her as well, so $Player(j_1)$ cannot be the initial element. We have a
       contradiction, thus the theorem holds.
    \end{proof}

proof that the chosen flow for the transitive game is the maxflow (false since the particular transitive game is not guaranteed to be the max)
       We will now prove that
       \begin{equation}
          X = MaxFlow_{\mathcal{G}}\left(A, B\right) \enspace.
       \end{equation}
       \begin{itemize}
          \item Let $X^A = MaxFlow_{\mathcal{G}}$. If we suppose that
          \begin{equation}
             maxFlow_{\mathcal{G}}\left(A, B\right) > \sum\limits_{v \in \mathcal{V}''}x''_{Av} \enspace,
          \end{equation}
          then we can set
          \begin{equation}
             X^T = X \cup \{(T, A)\} \mbox{ with}
          \end{equation}
          \begin{equation}
             \forall v, w \in \mathcal{V}'', x^T_{vw} = x^A_{vw} \mbox{ and}
          \end{equation}
          \begin{equation}
             x^T_{TA} = \sum\limits_{v \in \mathcal{V}''}x^A_{Av} \enspace.
          \end{equation}
          Since
          \begin{equation}
             \sum\limits_{v \in \mathcal{V}''}x^A_{Av} = maxFlow_{\mathcal{G}}\left(A, B\right) \enspace,
          \end{equation}
          we see that
          \begin{equation}
             \sum\limits_{v \in \mathcal{V}''}x^T_{Tv} = x^T_{TA} > x''_{TA} \enspace,
          \end{equation}
          thus $X''$ is not $MaxFlow_{\mathcal{G}}\left(T, B\right)$, which is a contradiction. Thus
          \begin{equation}
             maxFlow_{\mathcal{G}}\left(A, B\right) \leq \sum\limits_{v \in \mathcal{V}''}x''_{Av} \enspace,
          \end{equation}
          therefore
          \begin{equation}
             maxFlow_{\mathcal{G}}\left(A, B\right) \leq \sum\limits_{v \in \mathcal{V}}x_{Av} \enspace.
          \end{equation}
          \item If we suppose that
          \begin{equation}
             maxFlow_{\mathcal{G}}\left(A, B\right) < \sum\limits_{v \in \mathcal{V}''}x''_{Av} \enspace,
          \end{equation}
          we can likewise choose $X^A$ such that
          \begin{equation}
             \forall v, w \in \mathcal{V}'' \setminus \{T\}, x^A_{vw} = x''_{vw} \enspace,
          \end{equation}
          thus
          \begin{equation}
             \sum\limits_{v \in \mathcal{V}'' \setminus \{T\}}x^A_{Av} = \sum\limits_{v \in \mathcal{V}''}x''_{Av} \enspace.
          \end{equation}
          We deduce that
          \begin{equation}
             \sum\limits_{v \in \mathcal{V}'' \setminus \{T\}}x^A_{Av} > maxFlow_{\mathcal{G}}\left(A, B\right) \enspace,
          \end{equation}
          which is a contradiction. Thus
          \begin{equation}
             maxFlow_{\mathcal{G}}\left(A, B\right) \geq \sum\limits_{v \in \mathcal{V}''}x''_{Av} \enspace,
          \end{equation}
          therefore
          \begin{equation}
             maxFlow_{\mathcal{G}}\left(A, B\right) \geq \sum\limits_{v \in \mathcal{V}}x_{Av} \enspace.
          \end{equation}
       \end{itemize}

Formalism for colluding players playing consecutively
       The colluding players follow the evil strategy. Suppose that until some turn $j$ only good players are chosen to play.
       Starting from turn $j$, suppose that there exist $|\mathcal{B} \cup \mathcal{C}|$ consecutive turns for the first game
       and $|\mathcal{B}|$ consecutive turns for the second game during which all the colluding players are chosen to play.
       More formally, suppose that
       \begin{equation}
       \begin{gathered}
          \exists j \in \mathbb{N} : \forall d_1 \in [|\mathcal{B} \cup \mathcal{C}|], Player(j+d) \in \mathcal{B} \cup
          \mathcal{C} \wedge \\
          \wedge \forall d_1, d_2 \in [|\mathcal{B} \cup \mathcal{C}|], d_1 \neq d_2, Player(j + d_1) \neq Player(j + d_2)
          \wedge \\
          \wedge \forall d \in [|\mathcal{B} \cup \mathcal{C}|], Strategy(Player(j+d)) = Evil \wedge \\
          \wedge \forall j' \in [j] Player(j') \notin \mathcal{B} \cup \mathcal{C}
       \end{gathered}
       \end{equation}
       for the first and likewise for the second game. \\

unneeded mincut arguments under sybil resilience proof
       From lemma (\ref{mincutmany}), we know that
       \begin{equation}
          \forall \left(v, w\right) \in MinCut_1\left(A, T_1\right), v \notin \mathcal{B} \cup \mathcal{C} \wedge \forall
          \left(v, w\right) \in MinCut_2\left(A, T_2\right), v \notin \mathcal{B}
       \end{equation}
       and thus
       \begin{equation}
       \begin{gathered}
          e \in MinCut_1 \Rightarrow e \in \mathcal{E}_2 \wedge \\
          \wedge e \in MinCut_2 \Rightarrow e \in \mathcal{E}_1 \wedge \\
          \wedge \forall e \in MinCut_1 \cup MinCut_2, c_1(e) = c_2(e) \enspace.
       \end{gathered}
       \end{equation}

rephrased part of sybil resilience proof
       used to construct a valid flow of equal value for the first case if we set
       \begin{align}
          \forall v \in \mathcal{V}_1 \setminus \left(\mathcal{B} \cup \mathcal{C}\right), \forall w \in \mathcal{V}_1&,
          x_{vw,1} = x_{vw,2} \enspace, \\
          \forall v \in \mathcal{B}&, x_{vT_1,1} = \sum\limits_{w \in N^{+}(v)}x_{vw,2} \enspace, \\
          \forall v \in \mathcal{C}, \forall w \in \mathcal{V}_1&, x_{vw,1} = 0 \enspace.
       \end{align}
       Observe that
       \begin{equation}
          \forall v \in \mathcal{V}_1 \setminus \left(\mathcal{B} \cup \mathcal{C}\right), \forall w \in \mathcal{C},
          x_{vw, 1} = 0 \enspace.
       \end{equation}
       From these two observations, we deduce that there exists a function, say $F_2(X_1)$, that transforms
       the $MaxFlow_1$ of the first graph into a valid flow for the second graph that has the same amount of flow as
       $MaxFlow_1$ and there also exists a similar function $F_1(X_2)$ that transforms the $MaxFlow_2$ of the second graph
       into a valid flow for the first graph that has the same amount of flow as $MaxFlow_2$. Suppose that
       \begin{equation}
         maxFlow_1 < maxFlow_2 \enspace.
       \end{equation}
       Then
       \begin{equation}
          F_1(MaxFlow_2) > maxFlow_1 \enspace,
       \end{equation}
       which is a contradiction. Likewise, suppose that
       \begin{equation}
          maxFlow_1 > maxFlow_2 \enspace.
       \end{equation}
       Then
       \begin{equation}
          F_2(MaxFlow_1) > maxFlow_2 \enspace,
       \end{equation}

requirement for new capacities corollary and proof
   \begin{corollary}[Requirement for $\sum\limits_{i=1}^{n}{u_{s, i}'} = F - V$, $u_{s, i}' \leq x_{s, i}$] \ \\
      In the setting of \ref{trusttransfer}, it is impossible to have $maxFlow' = F - V$ if
      $\sum\limits_{i=1}^{n}{u_{s, i}'} > F - V \wedge \forall i \in [n],u_{s, i}' \leq x_{s, i}$.
   \end{corollary}
   \begin{proof}
      Due to \ref{trusttransfer}, $maxFlow' = F - V$ if $\sum\limits_{i=1}^{n}{u_{s, i}'} = F - V
      \wedge \forall i \in [n], u_{s, i}' \leq x_{s, i}$. If we create new capacities such that
      $\forall i \in [n], u_{s,i}'' \leq x_{s,i}$, then obviously $maxFlow'' = \sum\limits_{i=1}^{n}{u_{s,i}''}$. If
      additionally $\sum\limits_{i=1}^{n}{u_{s,i}''} > F - V$, then $maxFlow'' > F - V$.
   \end{proof}

Old saturation proof
   Suppose that $\exists k \in [n] : x_k' 
   < u_k'$ as calculated by $MaxFlow_{\mathcal{G}_j}$. Then $maxFlow_{\mathcal{G}_j} =
   \sum\limits_{i=1}^{n}x_i' < \sum\limits_{i=1}^{n}y_i$ since $x_k' < y_k$ and (\ref{saturation:flowleqcap}).
   This however is impossible because the configuration $\forall i \in [n], x_i' = y_i$ is valid in $\mathcal{G}_j$ since
   $\forall i \in [n], y_i = c_i'$ and also has a higher flow, thus the maxFlow algorithm will 
   prefer the configuration with the higher flow. Thus we deduce that $\forall i \in [n], x_i' = c_i'$.

old trust transfer theorem
   Let $s$ source, $t$ sink, $n = N^{+}(s)$ \\
   $X = \{x_1, \dots, x_n\}$ outgoing flows from $s$, \\
   $U = \{u_1, \dots, u_n\}$ outgoing capacities from $s$, \\
   $V$ the value to be transferred. \\
   Nodes apart from $s$, $t$ follow the conservative strategy. \\
   Obviously $maxFlow = F = \sum\limits_{i=1}^{n}{x_i}$.
   {\em \begin{lstlisting}
        /                      ....                     \
       / x_s1/u_s1                         x_1t/u_1t     \
      /                                                   \
     /                                                     \
    / x_s2/u_s2                               x_2t/u_2t     \
   s-------------              ....          ------------t
    \      .                                           .    /
     \     .                                           .   /
      \    .                                           .  /
       \ x_sn/u_sn             ....        x_mt/u_mt     /
        \                                               /
   \end{lstlisting}}
   We create a new graph where
   \begin{enumerate}
     \item  $\sum\limits_{i=1}^{n}{u_i'} = F - V$
     \item $\forall i \in [n] \: u_i' \leq x_i$
   \end{enumerate}

   It holds that $maxFlow' = F' = F - V$.

old abs naive algorithm
\begin{algorithm}[H]
   \label{abs}
   \SetKwInOut{Input}{Input}
   \SetKwInOut{Output}{Output}
   \Input{$x_i$ flows, $n = |N^{+}(s)|$, $V$ value}
   \Output{$u_i'$ capacities}
   \caption{Absolute equality trust transfer ($||\Delta_i||_\infty$ minimizer)}
   $F \gets \sum\limits_{i=1}^{n}x_i$ \\
   \If{$F < V$}{\Return $\bot$}
   \For{$i \gets 1$ to $n$}
      {$u_i' \gets x_i$ \label{abscapinit}}
   $reduce \gets {V \over n}$ \\
   $reduction \gets 0$ \\
   $empty \gets 0$ \\
   $i \gets 0$ \label{absiinit} \\
   \While{$reduction < V$ \label{absloop}}
      {\If{$u_i' > 0$ \label{absifcappositive}}{\If{$x_i < reduce$ \label{absifflowlessreduce}}
            {$empty \gets empty + 1$ \label{absemptyincrement} \\
             \If{$empty < n$ \label{absifemptylessn}}
                {$reduce \gets reduce + \frac{reduce - x_i}{n - empty}$ \label{absreducemodify}}
             $reduction \gets reduction + u_i'$ \label{absreductionincrease} \\
             $u_i' \gets 0$ \label{abscapzero} \\}
       \ElseIf{$x_i \geq reduce$}{$reduction \gets reduction + u_i' - (x_i - reduce)$ \label{absreductionmodify} \\
             $u_i' \gets x_i - reduce$ \label{abscapreduce}}}
       $i \gets (i + 1) mod \:n$ \label{absiincrement}}
   \Return $U' = \bigcup\limits_{k=1}^{n}\{u_k'\}$ \label{absreturn}
\end{algorithm}
  A variation of this algorithm using a Fibonacci heap with complexity $O(n)$ can be created, but that is part of
  further research.

old abs naive preproofs
  We will start by showing some results useful for the following proofs. Let $j$ be the number of iterations of the
  \textbf{while} loop for the rest of the proofs for algorithm \ref{abs} (think of $i$ from line~\ref{absiincrement}
  without the $mod\:n$).\\
  First we will show that $empty \leq n$. $empty$ is only modified on line~\ref{absemptyincrement} where it is
  incremented by 1. This happens only when $u_i' > 0$ (line~\ref{absifcappositive}), which is assigned the value 0 on
  line~\ref{abscapzero}. We can see that the incrementation of $empty$ can happen at most $n$ times because
  $|U'| = n$. Since $empty_0 = 0$, $empty \leq n$ at all times of the execution. \\
  Next we will derive the recursive formulas for the various variables. \\
  $empty_0 = 0$ \\
  $empty_{j+1} = 
     \begin{cases}
        empty_j, & u_{(j+1)\:mod\:n}' = 0 \\
        empty_j+1, & u_{(j+1)\:mod\:n}' > 0 \: \wedge \: x_{(j+1) \:mod\:n} < reduce_j \\
        empty_j, & u_{(j+1)\:mod\:n}' > 0 \: \wedge \: x_{(j+1) \:mod\:n} \geq reduce_j
     \end{cases}$ \\ \ \\
  $reduce_0 = \frac{V}{n}$ \\
  $reduce_{j+1} =
     \begin{cases}
        reduce_j, & u_{(j+1)\:mod\:n}' = 0 \\
        reduce_j + \frac{reduce_j-x_{(j+1)\:mod\:n}}{n-empty_{j+1}}, & u_{(j+1)\:mod\:n}' > 0 \: \wedge \:
           x_{(j+1) \:mod\:n} < reduce_j \\
        reduce_j, & u_{(j+1)\:mod\:n}' > 0 \: \wedge \: x_{(j+1) \:mod\:n} \geq reduce_j
     \end{cases}$ \\ \ \\
  $reduction_0 = 0$ \\
  $reduction_{j+1} =
     \begin{cases}
        reduction_j, & u_{(j+1)\:mod\:n}' = 0 \\
        reduction_j + u_{(j+1)\:mod\:n}', & u_{(j+1)\:mod\:n}' > 0 \: \wedge \: x_{(j+1) \:mod\:n} < reduce_j \\
        reduction_j + u_{(j+1)\:mod\:n}' - x_{(j+1)\:mod\:n} + reduce_{j+1}, &
           u_{(j+1)\:mod\:n}' > 0 \: \wedge \: x_{(j+1) \:mod\:n} \geq reduce_j
     \end{cases}$ \\
  In the end, $r = reduce$ is such that $r = \frac{V - \sum\limits_{x \in S}x}{n - |S|}$ where
  $S = \{\text{flows } y \text{ from } s \text{ to } N^{+}(s) \text{ according to } maxFlow : y < r\}$. Also,
  $\sum\limits_{i=1}^{n}u_i' = \sum\limits_{i=1}^{n}\max{(0,x_i - r)}$. TOPROVE

old abs naive correctness proof
\begin{proof}[Proof of correctness for algorithm \ref{abs}]
   \begin{itemize}
      \item We will show that $\forall i \in [n] \: u_i' \leq x_i$. \\
      On line~\ref{abscapinit}, $\forall i \in [n] \: u_i' = x_i$. Subsequently $u_i'$ is modified on
      line~\ref{abscapzero}, where it becomes equal to 0 and on line~\ref{abscapreduce}, where it is assigned
      $x_i - reduce$. It holds that $x_i - reduce \leq x_i$ because initially $reduce = \frac{V}{n} \geq 0$ and
      subsequently $reduce$ is modified only on line~\ref{absreducemodify} where it is increased ($n > empty$ because of
      line~\ref{absifemptylessn} and $reduce > x_i$ because of line~\ref{absifflowlessreduce}, thus
      $\frac{reduce - x_i}{n - empty} > 0$). We see that $\forall i \in [n], u_i' \leq x_i$.
      \item We will show that $\sum\limits_{i=1}^{n}u_i' = F - V$. \\
      The variable $reduction$ keeps track of the total reduction that has happened and breaks the \textbf{while} loop
      when $reduction \geq V$. We will first show that $reduction = \sum\limits_{i=1}^{n}(x_i- u_i')$ at all times and
      then we will prove that $reduction = V$ at the end of the execution. Thus we will have proven that
      $\sum\limits_{i=1}^{n}u_i'= \sum\limits_{i=1}^{n}x_i - V = F - V$.
      \begin{itemize}
         \item On line~\ref{abscapinit}, $u_i' = x_i \Rightarrow \sum\limits_{i=1}^{n}(x_i- u_i') = 0$ and
         $reduction = 0$. \\
         On line~\ref{abscapzero}, $u_i'$ is reduced to 0 thus $\sum\limits_{i=1}^{n}(x_i- u_i')$ is increased by $u_i'$.
         Similarly, on line~\ref{absreductionincrease} $reduction$ is increased by $u_i'$, the same as the increase in
         $\sum\limits_{i=1}^{n}(x_i- u_i')$. \\
         On line~\ref{abscapreduce}, $u_i'$ is reduced by $u_i' - x_i + reduce$ thus $\sum\limits_{i=1}^{n}(x_i- u_i')$
         is increased by $u_i' - x_i + reduce$. On line~\ref{absreductionmodify}, $reduction$ is increased by
         $u_i' - x_i + reduce$, which is equal to the increase in $\sum\limits_{i=1}^{n}(x_i- u_i')$. \\
         We also have to note that neither $u_i'$ nor $reduction$ is modified in any other way from line~\ref{absloop}
         and on, thus we conclude that $reduction = \sum\limits_{i=1}^{n}(x_i- u_i')$ at all times.
         \item Suppose that $reduction_j > V$ on the line~\ref{absreturn}. Since $reduction_j$ exists, $reduction_{j-1} < V$.
         If $x_{j \: mod \: n} < reduce_{j-1}$ then $reduction_j = reduction_{j-1} + u_{j \: mod \:n}'$.
         Since $reduction_j > V$, $u_{j \: mod \:n}' > V - reduction_{j-1}$. TOCOMPLETE\\

      \end{itemize}
   \end{itemize}
\end{proof}

old abs naive complexity proof
\begin{proof}[Complexity of algorithm \ref{abs}]
   In the worst case scenario, each time we iterate over all capacities only the last non-zero capacity will become zero
   and every non-zero capacity must be recalculated. This means that every $n$ steps exactly 1 capacity becomes zero
   and eventually all capacities (maybe except for one) become zero. Thus we need $O(n^2)$ steps in the worst case.
\end{proof}

old abs naive Dinf min proof
\begin{proof}[Proof that algorithm \ref{abs} minimizes the $||\Delta_i||_\infty$ norm]
   Suppose that $U'$ is the result of an execution of algorithm \ref{abs} that does not minimize the $||\Delta_i||_\infty$
   norm. Suppose that $W$ is a valid solution that minimizes the $||\Delta_i||_\infty$ norm. Let $\delta$ be the minimum
   value of this norm. There exists $i \in [n]$ such that $x_i - w_i = \delta$ and $u_i' < w_i$. Because both $U'$
   and $W$ are valid solutions ($\sum\limits_{i=1}^{n}u_i' = \sum\limits_{i=1}^{n}w_i = F - V$), there must exist a set
   $S \subset U'$ such that $\forall u_j' \in S, u_j' > w_j$ TOCOMPLETE.
\end{proof}

Reduntant bibliography
     \bibitem{ddosattacks}
     Patrikakis C., Masikos M., Zouraraki O.: Distributed Denial of Service Attacks. The Internet Protocol Journal, Vol. 7,
     N. 4,
     \url{http://www.cisco.com/c/en/us/about/press/internet-protocol-journal/back-issues/table-contents-30/dos-attacks.html}
     (2004)
     \bibitem{ebayfees}
     Standard ebay selling fees, \url{http://pages.ebay.com/help/sell/fees.html}
     \bibitem{ebayguarantee}
     ebay money back guarantees, \url{http://pages.ebay.com/ebay-money-back-guarantee/questions.html}
     \bibitem{openbazaar}
     What is OpenBazaar, \url{https://blog.openbazaar.org/what-is-openbazaar/}
     \bibitem{multisig}
     Buterin V.: Bitcoin Multisig Wallet: The Future of Bitcoin. Bitcoin Magazine (2014),
     \url{https://bitcoinmagazine.com/articles/multisig-future-bitcoin-1394686504}
     \bibitem{bitcoinguide}
     Bitcoin Developer Guide, \url{https://bitcoin.org/en/developer-guide}
     \bibitem{multisigfraud}
     Can Bitcoin and Multisig Reduce Identity Theft and Fraud?,
     \url{https://blog.openbazaar.org/can-bitcoin-and-multisig-reduce-identity-theft-and-fraud/}
     Biryukov A., Khovratovich D., Pustogarov I.: Deanonymisation of clients in Bitcoin P2P network. arXiv:1405.7418 [cs.CR]
     (2014)
     \url{https://arxiv.org/pdf/1405.7418v3.pdf}
     \bibitem{toposort}
     Kahn Arthur B.: Topological sorting of large networks. Communications of the ACM Vol. 5, Issue 11, pp. 558-562, ACM,
     New York (1962)
