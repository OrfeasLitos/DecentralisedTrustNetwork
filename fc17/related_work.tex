\section{Related Work}
  The topic of trust has been repeatedly attacked with several approaches: Webs-of-trust can be used as a basis for trust as
  explored by Caronni \cite{wot}. PGP \cite{pgp} implements one and its transitive closure is explored by Pathfinder
  \cite{pathfinder}. A transitive web-of-trust for fighting spam is explored in Freenet \cite{freenet}. Mui et al. \cite{mui}
  and J\o{}sang et al. \cite{beta} propose two ways of calculating trust towards distant nodes. Vi\c{s}an et al. \cite{vpc}
  calculate trust in an objective, hierarchical way. Other systems require central trusted third parties, such as CA-based
  PKIs \cite{pki} and Bazaar \cite{bazaar}, or, in the case of Byzantine Fault Tolerance, authenticated membership
  \cite{byzantine}. While other trust systems attempt to be decentralized, they do not prove any Sybil resilience properties
  and hence may be Sybil attackable. Such systems are FIRE \cite{fire}, CORE \cite{core}, Gr\"unert et al. \cite{ghkkw} and
  Repantis et al. \cite{rk}. These systems define trust in a non-financial manner.

  We agree with the work of Gollmann \cite{badtrust} in that the meaning of trust should not be extrapolated. We have adopted
  their advice in our paper and urge our readers to adhere to the definitions of \textit{direct} and \textit{indirect} trust
  as they are used here.

  The Beaver marketplace \cite{beaver} includes a trust model that relies on fees to discourage Sybil attacks. We chose to
  avoid fees in our system and mitigate Sybil attacks in a different manner. Our motivating application for exploring trust
  in a decentralized setting is the OpenBazaar marketplace. Transitive financial trust for OpenBazaar has previously been
  explored by Zindros \cite{dionyziz}. That work however does not define trust as a monetary value. We are strongly inspired
  by Karlan et al. \cite{kmrs} who give a sociological justification for the central design choice of identifying trust with
  risk. We greatly appreciate the work in TrustDavis \cite{davis}, which proposes a financial trust system that exhibits
  transitive properties and in which trust is defined as lines-of-credit, similar to our system. We were able to extend their
  work by using the blockchain for automated proofs-of-risk, a feature not available to them at the time.

  Our conservative strategy and Transitive Game are very similar to the mechanism proposed by Fugger \cite{iou} which also
  illustrates financial trust transitivity and is used by Ripple \cite{ripple} and Stellar \cite{stellar}. IOUs in these
  correspond to reversed edges of trust in our system. The critical difference is that our denominations of trust are
  expressed in a global currency and that coins must pre-exist in order to be trusted and so there is no money-as-debt.
  Furthermore, we prove that trust and maximum flows are equivalent, a direction not explored in their paper, even though we
  believe it must hold for both our and their systems.
