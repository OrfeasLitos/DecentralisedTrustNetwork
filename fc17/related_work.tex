\section{Related Work}
  The topic of trust has been repeatedly attacked with several approaches: Purely cryptographic infrastructure where trust
  is rather binary and transitivity is limited to one step beyond actively trusted parties is explored in PGP \cite{pgp}. A
  transitive web-of-trust for fighting spam is explored in Freenet \cite{freenet}. Other systems require central trusted
  third parties, such as PKI \cite{pki} and Bazaar \cite{bazaar}, or, in the case of BFT, authenticated membership
  \cite{byzantine}. While other trust systems attempt to be decentralized, they do not prove any Sybil resilience properties
  and hence may be Sybil attackable. Such systems are FIRE \cite{fire}, CORE \cite{core} and others \cite{openrep,ghkkw,rk}.
  Other systems that define trust in a non-financial way are \cite{mui,beta,pace,vpc,sdt,wot,pathfinder}.

  Readers are urged to adhere to the definitions of direct and indirect trust as proposed in this work and not to extrapolate
  their meanings, as suggested by \cite{badtrust}.

  The Beaver marketplace \cite{beaver} includes a trust model that relies on
  fees to discourage Sybil attacks. We chose to avoid fees in our system and mitigate Sybil attacks in a different manner.
  Our motivating application for exploring trust in a decentralized setting is the OpenBazaar marketplace. Transitive
  financial trust for OpenBazaar has previously been explored by \cite{dionyziz}. That work however does not define trust
  as a monetary value. We are strongly inspired by \cite{kmrs}
  which gives a sociological justification for the central design choice of identifying trust with
  risk. We greatly appreciate the work in TrustDavis \cite{davis}, which proposes a financial trust system that exhibits
  transitive properties and in which trust is defined as lines-of-credit, similar to our system. We were able to extend
  their work by using the blockchain for automated proofs-of-risk, a feature not available to them at the time. Our
  conservative strategy and the Transitive Game are very similar to the mechanism proposed by the economic paper
  \cite{iou} which also illustrates financial trust transitivity. 

