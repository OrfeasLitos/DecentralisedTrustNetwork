\section{Introduction}
  Online marketplaces can be categorized as centralized and decentralized.
  Two examples of each category are \href{http://www.ebay.com}{ebay} and \href{https://openbazaar.org/}{OpenBazaar}.
  The common denominator of established online marketplaces is that the reputation of each vendor and client is typically
  expressed in the form of stars and user-generated reviews that are viewable by the whole network.

  The goal of "Trust Is Risk" is to offer a reputation system for decentralized marketplaces where the trust each user gives
  to the other users is quantifiable in monetary terms. The central assumption used throughout this paper is that trust is
  equivalent to risk, or the proposition that $Alice$'s \textit{trust} in another user $Charlie$ is defined to be the
  \textit{maximum sum of money} that $Alice$ can lose when $Charlie$ is free to choose any strategy he wants. To flesh out
  this concept, we will use \textit{lines of credit} as proposed by Sanchez \cite{loc}. $Alice$ joins the network by
  explicitly entrusting a certain amount of money to another user, say her friend, $Bob$ (see Fig.~\ref{fig:bottleneckA}
  and~\ref{fig:bottleneckB}). If $Bob$ has already entrusted an amount of money to a third user, $Charlie$, then $Alice$
  indirectly trusts $Charlie$ since if the latter wished to play unfairly, he could have already stolen the money entrusted to
  him by $Bob$. We will later see that $Alice$ can now engage in economic interaction with $Charlie$.

  To implement lines-of-credit, we use Bitcoin \cite{bitcoin}, a decentralized cryptocurrency that differs from conventional
  currencies in that it does not depend on trusted third parties. All transactions are public as they are recorded on a
  decentralized ledger, the blockchain. Each transaction takes some coins as input and produces some coins as output. If the
  output of a transaction is not connected to the input of another one, then this output belongs to the UTXO, the set of
  unspent transaction outputs. Intuitively, the UTXO contains all coins not yet spent.
  \medskip \ \\
  \subimport{common/figures/}{simpleexample.tikz} \smallskip \ \\
  We propose a new kind of wallet where coins are not exclusively owned, but are placed in shared accounts materialized
  through 1-of-2 multisigs, a bitcoin construction that permits any one of two pre-designated users to spend the coins
  contained within a shared account \cite{masteringbitcoin}. We will use the notation 1/$\{Alice, Bob\}$ to represent a
  1-of-2 multisig that can be spent by either $Alice$ or $Bob$. In this notation, the order of names is irrelevant, as
  either user can spend. However, the user who deposits the money initially into the shared account is relevant -- she is the
  one risking her money.

  Our approach changes the user experience in a subtle but drastic way. A user no more has to base her trust towards a
  store on stars or ratings which are not expressed in financial units. She can simply consult her wallet to decide whether
  the store is trustworthy and, if so, up to what value, denominated in bitcoin. This system works as follows: Initially
  $Alice$ migrates her funds from her private bitcoin wallet to 1-of-2 multisig addresses shared with friends she
  comfortably trusts. We call this direct trust. Our system is agnostic to the means players use to determine who is
  trustworthy for these direct 1-of-2 deposits. This dubious kind of trust is confined to the direct neighbourhood of each
  player; indirect trust towards unknown users is calculated by a deterministic algorithm. In comparison, systems with global
  ratings do not distinguish between neighbours and other users, thus offering dubious trust indications for everyone.

  Suppose that $Alice$ is viewing the item listings of vendor $Charlie$. Instead of $Charlie$'s stars, $Alice$ will see a
  positive value that is calculated by her wallet and represents the maximum monetary value that $Alice$ can safely pay to
  complete a purchase from $Charlie$. This value, known as indirect trust, is calculated in Theorem~\ref{trustflow} -- Trust
  Flow. Note that indirect trust towards a user is not global but subjective; each user views a personalized indirect trust
  based on the network topology. The indirect trust reported by our system maintains the following desired security property:
  If $Alice$ makes a purchase from $Charlie$, then she is exposed to no more risk than she was already taking willingly. The
  existing voluntary risk is exactly that which $Alice$ was taking by sharing her coins with her trusted friends. We prove
  this in Theorem~\ref{riskinv} -- Risk Invariance. Obviously it will not be safe for $Alice$ to buy anything from $Charlie$
  or any other vendor if she has not directly entrusted any value to any other user.

  We see that in Trust Is Risk the money is not invested at the time of purchase and directly to the vendor, but at an
  earlier point in time and only to parties that are trustworthy for out of band reasons. The fact that this system can
  function in a completely decentralized fashion will become clear in the following sections. We prove this in
  Theorem~\ref{sybil} -- Sybil Resilience.

  We make the design choice that an entity can express her trust maximally in terms of her available capital. Thus, an
  impoverished player cannot allocate much direct trust to her friends, no matter how trustworthy they are. On the other hand,
  a rich player may entrust a small fraction of her funds to a player that she does not find trustworthy to a great extent and
  still exhibit more direct trust than the impoverished player of the previous example. There is no upper limit to trust; each
  player is only limited by her funds. We thus take advantage of the following remarkable property of money: To normalise
  subjective human preferences into objective value.

  There are several incentives for a user to join this network. First, she has access to stores that would be inaccessible
  otherwise. Moreover, two friends can formalize their mutual trust by directly entrusting the same amount to each other. A
  large company that casually subcontracts other companies can express its trust towards them. A government can choose to
  directly entrust its citizens with money and confront them using a corresponding legal arsenal if they make irresponsible
  use of this trust. A bank can provide loans as outgoing and manage savings as incoming direct trust. Last but not least,
  the network can be viewed as a possible investment and speculation field since it constitutes a completely new area for
  financial activity.

  It is worth noting that the same physical person can maintain multiple pseudonymous identities in the same trust network
  and that multiple independent trust networks for different purposes can coexist. On the other hand, the same pseudonymous
  identity can be used to establish trust in different contexts.
