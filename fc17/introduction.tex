\section{Introduction}
   Modern online marketplaces can be roughly categorized as centralized and decentralized.
   Two major examples of each category are \href{http://www.ebay.com}{ebay} and \href{https://openbazaar.org/}{OpenBazaar}.
   The common denominator of established online marketplaces is that the reputation of each vendor and client is either
   expressed in the form of stars and user-generated reviews that are viewable by the whole network, or not expressed at
   all inside the marketplace and instead is entirely built on word-of-mouth or other out-of-band means.

   Our goal is to create a decentralized marketplace where the trust each user gives to the rest of the users is
   quantifiable, measurable and expressable in monetary terms. The central concept used throughout this paper is
   that trust is equivalent to risk, or the proposition that $Alice$'s \textit{trust} to another user $Bob$ is defined to
   be the \textit{maximum sum of money} that $Alice$ can lose when $Bob$ is free to choose any strategy he wants. To flesh
   out this concept, we will use \textit{lines of credit} as proposed by Washington Sanchez \cite{loc}, not to be confused
   with the synonymous financial product. Joining the network
   will be done by explicitly entrusting a certain amount of money to another user, say $Bob$. If $Bob$ has already
   entrusted an amount of money to a third user, $Charlie$, then we indirectly trust $Charlie$ since if the latter wished
   to play unfairly, he could have already stolen the money entrusted to him by $Bob$. Thus we can engage in economic
   interaction with $Charlie$. The currency used is Bitcoin \cite{bitcoin}. \medskip \ \\
   \subimport{common/figures/}{simpleexample.tikz} \smallskip \ \\
   We thus propose a new kind of wallet where coins are not stored locally, but are placed in 1-of-2 multisigs, a bitcoin
   construction that permits any one of two pre-designated users to spend the coins contained therein
   \cite{masteringbitcoin}. We will use the notation 1/$\{Alice, Bob\}$ to represent a 1-of-2 multisig that can be spent by
   either $Alice$ or $Bob$.

   Our approach changes the user experience in a subtle but drastic way. A user no more has to base her trust towards a
   store on stars, ratings or other dubious and non-quantifiable trust metrics. She can simply consult her wallet to
   decide whether the store is trustworthy and, if so, up to what value. This system works as follows: Initially $Alice$
   migrates her funds from P2PKH addresses in the UTXO \cite{masteringbitcoin} to 1-of-2 multisig addresses shared with
   friends she comfortably trusts. We call this direct trust. Our system is agnostic to the means players use to determine
   who is trustworthy for these direct 1-of-2 deposits.

   Suppose that $Alice$ is viewing the item listings of vendor $Charlie$. Instead of $Charlie$'s stars, $Alice$ will see a
   positive value that is calculated by her wallet and represents the maximum monetary value that $Alice$ can safely use to
   complete a purchase from $Charlie$. We examine exactly how this value is calculated in Trust Flow theorem
   (\ref{trustflow}). This monetary value reported by our system maintains the desired security property that, if $Alice$
   makes this purchase, then she is exposed to no more risk than she was willing to expose herself towards her friends.
   We prove this result in the Risk Invariance theorem (\ref{riskinv}). Obviously it will not be safe for $Alice$ to buy
   anything from $Charlie$ or any other vendor if she has entrusted no value to any other player.

   We see that in TrustIsRisk the money is not invested at the time of the purchase and directly to the vendor, but at an
   earlier point in time and only to parties that are trustworthy for out-of-band reasons. The fact that this system can
   function in a completely decentralized fashion will become clear in the following sections. We prove this result in the
   Sybil Resilience theorem (\ref{sybil}).

   There are several incentives for a user to join this network. First of all, she can have access to a store that is
   otherwise inaccessible. Moreover, two friends can formalize their mutual trust by entrusting the same amount to each
   other. A large company that casually subcontracts other companies to complete various tasks can express its trust
   towards them using this method. A government can choose to entrust its citizens with money and confront them using a
   corresponding legal arsenal if they make irresponsible use of this trust. A bank can provide loans as outgoing and
   manage savings as incoming trust and thus has a unique opportunity of expressing in a formal and absolute way its
   credence by publishing its incoming and outgoing trust. Last but not least, the network can be viewed as a possible
   field for investment and speculation since it constitutes a completely new area for financial activity.

   It is worth noting that the same physical person can maintain multiple pseudonymous identities in the same trust network
   and that multiple independent trust networks for different purposes can coexist. On the other hand, the same
   pseudonymous identity can be used to establish trust in different contexts.
