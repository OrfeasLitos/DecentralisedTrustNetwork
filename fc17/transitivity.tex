\section{Trust Transitivity}
  In this section we define some strategies and show the corresponding algorithms. Then we define the
  Transitive Game that represents the worst-case scenario for an honest player when another player decides to depart from
  the network with her money and all the money directly entrusted to her.
  \subimport{common/definitions/}{idlestrategy.tex}

  \subimport{common/algorithms/}{idlestrategycode.tex}
  The inputs and outputs are identical to those of \texttt{idleStrategy()} for the rest of the strategies, thus we avoid
  repeating them.
  \subimport{common/definitions/}{evilstrategy.tex}

  \subimport{common/algorithms/}{evilstrategycode.tex}

  \subimport{common/definitions/}{conservativestrategy.tex}

  \subimport{fc17/algorithms/}{conservativestrategycode.tex}
  \texttt{SelectSteal()} returns $y_v$ with $v \in N^{-}\left(A\right)_{j-1}$ such that
  \begin{equation}
  \label{stealrestriction}
     \sum\limits_{v \in N^{-}\left(A\right)_{j-1}}y_v = Dmg_{A, j} \enspace \wedge \enspace \forall v \in N^{-}\left(A\right)_{j-1},
     y_v \leq DTr_{v \rightarrow A, j-1} \enspace.
  \end{equation}
  Player $A$ can arbitrarily define how \texttt{SelectSteal()} distributes the $Steal\left(\right)$ actions
  each time she calls the function, as long as (\ref{stealrestriction}) is respected. 

  As we can see, the definition covers a multitude of options for the conservative player, since in case $0 < Dmg_{A,j}
  < in_{A,j-1}$ she can choose to distribute the $Steal\left(\right)$ actions in any way she chooses.

  The rationale behind this strategy arises from a real-world common situation. Suppose there are a client, an
  intermediary and a producer. The client entrusts some value to the intermediary so that the latter can buy the desired
  product from the producer and deliver it to the client. The intermediary in turn entrusts an equal value to the
  producer, who needs the value upfront to be able to complete the production process. However the producer eventually
  does not give the product neither reimburses the value, due to bankruptcy or decision to exit the market with an unfair
  benefit. The intermediary can choose either to reimburse the client and suffer the loss, or refuse to return the money
  and lose the client's trust. The latter choice for the intermediary is exactly the conservative strategy. It is used
  throughout this work as a strategy for all the intermediary players because it models effectively the worst-case
  scenario that a client can face after an evil player decides to steal everything she can and the rest of the players do
  not engage in evil activity.

  We continue with a possible evolution of the game, the Transitive Game. In turn 0, there is already a network in place. All
  players apart from $A$ and $B$ follow the conservative strategy. The set of players is not modified throughout the
  Transitive Game, thus we can refer to $\mathcal{V}_j$ as $\mathcal{V}$. Each conservative player can be in one of three
  states: Happy, Angry or Sad. Happy players have 0 loss, Angry players have positive loss and positive incoming direct trust
  (line~\ref{trstealaddwangry}), thus are able to replenish their loss at least in part and Sad players have positive loss,
  but 0 incoming direct trust (line~\ref{trstealaddwsad}), thus they cannot replenish the loss.
  \subimport{fc17/algorithms/}{transitivegame.tex}

  \noindent An example execution follows:

  \subimport{common/figures/}{transitivegameexample.tikz}

  \noindent Let $j_0$ be the first turn on which $B$ is chosen to play. Until then, all players will pass their turn since
  nothing has been stolen yet (see the Conservative World theorem in Appendix A \cite{trustisrisk} for a formal proof of this
  simple fact).  Moreover, let $v = Player(j)$. The Transitive Game generates turns:
  \begin{align}
     Turn_j = \bigcup\limits_{w \in N^{-}\left(v\right)_{j-1}}\{Steal\left(y_w,w\right)\} \enspace, & \mbox{ where} \\
     \sum\limits_{w \in N^{-}\left(v\right)_{j-1}}y_w = \min\left(in_{v, j-1}, Dmg_{v, j}\right) \enspace. &
  \end{align}
 
  \noindent We see that if $Dmg_{v, j} = 0$, then $Turn_j = \emptyset$. From the definition of $Dmg_{v,j}$ and knowing that no
  strategy in this case can increase any direct trust, we see that $Dmg_{v,j} \geq 0$. Also, it is $Loss_{v,j} \geq 0$ because
  if $Loss_{v,j} < 0$, then $v$ has stolen more value than she has been stolen, thus she would not be following the
  conservative strategy.
