\section{Further Research}

   %\subsection{Trust Transfer Algorithms}
      When $Alice$ makes a purchase from $Bob$, she has to reduce her outgoing direct trust in a manner such that the
      supposition (\ref{primetrust}) of Risk Invariance theorem is satisfied. How $Alice$ can recalculate her outgoing
      direct trust will be discussed in a future paper.

   %\subsection{Dynamic Setting}
      Our game is static. In a future dynamic setting, users should be able to play simultaneously, freely join, depart or
      disconnect temporarily from the network. Other types of multisigs, such as 1-of-3, can be explored for the
      implementation of multi-party direct trust.

   %\subsection{Zero knowledge}
      MaxFlow in our case needs complete network knowledge, which can lead to privacy issues through deanonymisation
      techniques \cite{deanonymisation}. Calculating the flows in zero knowledge remains an open question.

   %\subsection{Game Theoretic Analysis}
      Our game theoretic analysis is simple. An interesting analysis would involve modelling repeated purchases with the
      respective edge updates on the trust graph and treating trust on the network as part of the utility function.

   %\subsection{Implementation and Experimental Results}
      An implementation as a wallet on any blockchain of our financial game is most welcome. A simulation or actual
      implementation of TrustIsRisk, combined with analysis of the resulting dynamics can yield interesting experimental
      results. Subsequently, our trust network can be used in other applications, such as decentralized social networks
      \cite{synereo}.

