\section{The Trust Graph}
  We now engage in the formal description of the proposed system, accompanied by helpful examples.
  \subimport{common/definitions/}{gamegraph.tex}
  The nodes represent the players, the edges represent the existing direct trusts and the weights represent the amount of
  value attached to the corresponding direct trust. As we will see, the game evolves in turns. The subscript of the graph
  represents the corresponding turn.
  \subimport{common/definitions/}{players.tex}
  Each node has a corresponding non-negative number that represents its capital. A node's capital is the total value that
  the node possesses exclusively and nobody else can spend.
  \subimport{common/definitions/}{capital.tex}
  The capital is the value that exists in the game but is not shared with trusted parties. The capital of $A$ can be
  modified only during her turns, according to her actions. No capital can be added in the course of the game through
  external means.

  We also define a player's assets to be the sum of her capital and her total outgoing trust.

  \subimport{common/definitions/}{assets.tex}

  The formal definition of direct trust follows:
  \subimport{common/definitions/}{directtrust.tex}
  This definition agrees with the title of this paper and coincides with the intuition and sociological experimental results
  of \cite{kmrs} that the trust $Alice$ shows to $Bob$ in real-world social networks corresponds to the extent of danger in
  which $Alice$ is putting herself into in order to help $Bob$. An example graph with its corresponding transactions in the
  UTXO can be seen below.

  \subimport{common/figures/}{utxo.tikz}

  Any algorithm that has access to the graph $\mathcal{G}_j$ has implicitly access to all direct trusts of this graph.

  \subimport{common/definitions/}{neighbourhood.tex}
  \subimport{common/definitions/}{inouttrust.tex}
