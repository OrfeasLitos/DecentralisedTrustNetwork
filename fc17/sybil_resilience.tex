\section{Sybil Resilience}
  One of the primary aims of this system is to mitigate the danger for Sybil attacks \cite{sybilattack} whilst maintaining
  fully decentralized autonomy.

  Here we extend the definition of indirect trust to many players.
  \subimport{common/definitions/}{indirecttrustmultiplayer.tex}
  \noindent We now extend Trust Flow theorem~\ref{trustflow} to many players.
  \subimport{common/theorems/}{multiplayertrustflowtheorem.tex}
  \subimport{common/proofs/}{multiplayertrustflowproof.tex}
  \noindent We now define several useful notions to tackle the problem of Sybil attacks. Let Eve be a possible attacker.
  \subimport{common/definitions/}{corrupted.tex}
  \subimport{common/definitions/}{sybil.tex}
  \subimport{common/definitions/}{collusion.tex}
  \subimport{common/figures/}{collusion.tikz}
  \subimport{fc17/theorems/}{sybilrestheorem.tex}
  \subimport{fc17/proofsketches/}{sybilresproofsketch.tex}
  We have proven that controlling $|\mathcal{C}|$ is irrelevant for Eve, thus Sybil attacks are meaningless. We note that
  this theorem does not deliver reassurances against attacks involving deception techniques. More specifically, a malicious
  player can create several identities, use them legitimately to inspire others to deposit direct trust to these identities
  and then switch to the evil strategy, thus defrauding everyone that trusted the fabricated identities. These identities
  correspond to the corrupted set of players and not to the Sybil set because they have direct incoming trust from outside
  the collusion.

  In conclusion, we have successfully delivered our promise for a Sybil-resilient decentralized financial trust system with
  invariant risk for purchases.

