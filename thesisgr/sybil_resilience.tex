\section{\textlatin{Sybil} Αντίσταση}
  Ένας από τους κεντρικούς στόχους αυτού του συστήματος είναι να περιορίσει τον κίνδυνο για επιθέσεις \textlatin{Sybil}
  διατηρώντας ταυτόχρονα πλήρως αποκεντρωμένη αυτονομία.

  Εδώ επεκτείνουμε τον ορισμό της έμμεσης εμπιστοσύνης σε πολλούς παίκτες.
  \subimport{thesisgr/definitions/}{indirecttrustmultiplayer.tex}
  Τώρα επεκτείνουμε το θεώρημα Εμπιστοσύνης -- Ροής (\ref{trustflow}) σε πολλούς παίκτες.
  \subimport{thesisgr/theorems/}{multiplayertrustflowtheorem.tex}
  \subimport{thesisgr/proofs/}{multiplayertrustflowproof.tex}
  Ορίζουμε τώρα διάφορες χρήσιμες έννοιες για να αντιμετωπίσουμε το πρόβλημα των επιθέσεων \textlatin{Sybil}. Έστω μία πιθανή
  επιτιθέμενη, η \textlatin{Eve}.
  \subimport{thesisgr/definitions/}{corrupted.tex}
  \subimport{thesisgr/definitions/}{sybil.tex}
  \subimport{thesisgr/definitions/}{collusion.tex}
  \subimport{thesisgr/figures/}{collusion.tikz}
  \subimport{thesisgr/theorems/}{sybilrestheorem.tex}
  \subimport{thesisgr/proofsketches/}{sybilresproofsketch.tex}
  Αποδείξαμε ότι ο έλεγχος του $|\mathcal{C}|$ δεν αφορά την \textlatin{Eve}, συνεπώς οι επιθέσεις \textlatin{Sybil} δεν έχουν
  νόημα. Σημειώνουμε ότι αυτό το θεώρημα δεν προσφέρει διαβεβαιώσεις ενάντια σε επιθέσεις που συμπεριλαμβάνουν τεχνικές
  εξαπάτησης. Πιο συγκεκριμένα, μία κακεντρεχής παίκτης μπορεί να δημιουργήσει πολλές ταυτότητες, να τις χρησιμοποιήσει με
  δίκαιο τρόπο ούτως ώστε να πείσει άλλες να καταθέσουν άμεση εμπιστοσύνη σε αυτές τις ταυτότητες και μετά να μεταβεί στην
  κακιά στρατηγική, εξαπατώντας έτσι όλες όσες εμπιστεύθηκαν τις κατασκευασμένες ταυτότητες. Αυτές οι ταυτότητες αντιστοιχούν
  στο διεφθαρμένο σύνολο παικτών και όχι στο σύνολο \textlatin{Sybil} γιατί διαθέτουν εισερχόμενη άμεση εμπιστοσύνη από
  παίκτες έξω από τη συνεργασία.

  Συμπερασματικά, έχουμε δημιουργήσει με επιτυχία ένα αποκεντρωμένο σύστημα οικονομικής εμπιστοσύνης με αντίσταση σε επιθέσεις
  \textlatin{Sybil} και αμετάβλητο κίνδυνο για αγορές, όπως υποσχεθήκαμε.
