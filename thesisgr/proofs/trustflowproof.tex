\begin{proof}
  Από το Λήμμα~\ref{maxflowgame} υπάρχει εκτέλεση του Μεταβατικού Παιχνιδιού τέτοια ώστε $Loss_A \geq maxFlow\left(A,
  B\right)$. Αφού η $Tr_{A \rightarrow B}$ είναι η μέγιστη ζημία που μπορεί να έχει υποστεί η $A$ μετά τη σύγκλιση του
  Μεταβατικού Παιχνιδιού, βλέπουμε ότι
  \begin{equation}
  \label{trgeqmaxflow}
     Tr_{A \rightarrow B} \geq maxFlow\left(A, B\right) \enspace.
  \end{equation}
  Όμως κάποια εκτέλεση του Μετατβατικού Παιχνιδιού δίνει $Tr_{A \rightarrow B} = Loss_A$.
  Από το Λήμμα~\ref{gameflow}, αυτή η εκτέλεση αντιστοιχεί σε μία ροή. Συνεπώς
  \begin{equation}
  \label{trleqmaxflow}
     Tr_{A \rightarrow B} \leq maxFlow\left(A, B\right) \enspace.
  \end{equation}
  Το θεώρημα προκύπτει από το (\ref{trgeqmaxflow}) και το (\ref{trleqmaxflow}).
\end{proof}
