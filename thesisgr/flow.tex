\section{Ροή Εμπιστοσύνης}
  Μπορούμε τώρα να ορίσουμε την έμμεση εμπιστοσύνη από την $A$ στη $B$.
  \subimport{thesisgr/definitions/}{indirecttrust.tex}
  \noindent Είναι $Tr_{A \rightarrow B} \geq DTr_{A \rightarrow B}$. Το επόμενο θεώρημα δείχνει ότι η $Tr_{A \rightarrow B}$
  είναι πεπερασμένη.
  \subimport{thesisgr/theorems/}{convergencetheorem.tex}
  \subimport{thesisgr/proofsketches/}{convergenceproofsketch.tex}
  Πλήρεις αποδείξεις όλων των θεωρημάτων και λημμάτων υπάρχουν στο Παράρτημα.

  Στην περίπτωση ενός \texttt{\textlatin{TransitiveGame}(}$\mathcal{G}$\texttt{,}$A$\texttt{,}$B$\texttt{)}, χρησιμοποιούμε το
  συμβολισμό $Loss_A = Loss_{A, j}$, όπου $j$ είναι ένας γύρος στον οποίο το παιχνίδι έχει συγκλίνει. Είναι σημαντικό να
  σημειώσουμε ότι η $Loss_A$ δεν είναι η ίδια για επανειλημμένες εκτελέσεις αυτού του είδους παιχνιδιού, αφού η σειρά με την
  οποία επιλέγονται οι παίκτες μπορεί να διαφέρει ανάμεσα σε εκτελέσεις και οι συντηρητικές παίκτες έχουν το περιθώριο να
  επιλέξουν ποιες εισερχόμενες άμεσες εμπιστοσύνες θα κλέψουν και πόσο από την καθεμία.

  Έστω ένας κατευθυνόμενος γράφος με βάρη $G$. Θα μελετήσουμε τη μέγιστη ροή στο γράφο αυτό. Για μία εισαγωγή στο πρόβλημα
  μέγιστης ροής βλέπε \cite{clrs} σελ. 708. Θεωρώντας το βάρος κάθε ακμής ως τη χωρητικότητά της, μία απόδοση ροής $X =
  [x_{vw}]_{\mathcal{V} \times \mathcal{V}}$ με πηγή $A$ και καταβόθρα $B$ είναι έγκυρη όταν:
  \begin{equation}
  \label{flow1}
    \forall (v, w) \in \mathcal{E}, x_{vw} \leq c_{vw} \mbox{ και}
  \end{equation}
  \begin{equation}
  \label{flow2}
    \forall v \in \mathcal{V} \setminus \{A,B\}, \sum\limits_{w \in N^{+}(v)}x_{wv} = \sum\limits_{w \in N^{-}(v)}x_{vw}
    \enspace.
  \end{equation}
  Δεν υποθέτουμε συμμετρία κατεύθυνσης στην απόδοση $X$. Η τιμή ροής είναι $\sum\limits_{v \in N^{+}\left(A\right)}x_{Av}$, η
  οποία προκύπτει ίση με $\sum\limits_{v \in N^{-}\left(B\right)}x_{vB}$. Υπάρχει αλγόριθμος που επιστρέφει τη μέγιστη δυνατή
  ροή από την $A$ στη $B$, γνωστός ως $MaxFlow\left(A, B\right)$. Αυτός ο αλγόριθμος χρειάζεται πλήρη γνώση του γράφου. Η
  γρηγορότερη εκδοχή του έχει χρονική πολυπλοκότητα $O\left(|\mathcal{V}||\mathcal{E}|\right)$ \cite{maxflownm}. Η τιμή ροής
  του $MaxFlow\left(A, B\right)$ συμβολίζεται $maxFlow\left(A, B\right)$.

  Θα εισάγουμε τώρα δύο λήμματα που θα χρησιμοποιηθούν για την απόδειξη ενός από τα κεντρικά αποτελέσματα αυτής της εργασίας,
  το θεώρημα Εμπιστοσύνης -- Ροής.
  \subimport{thesisgr/lemmas/}{flowgamelemma.tex}
  \subimport{thesisgr/proofsketches/}{flowgameproofsketch.tex}
  \subimport{thesisgr/lemmas/}{gameflowlemma.tex}
  \subimport{thesisgr/proofsketches/}{gameflowproofsketch.tex}
  \subimport{thesisgr/theorems/}{trustflowtheorem.tex}
  \subimport{thesisgr/proofs/}{trustflowproof.tex}

  Ας σημειωθεί ότι η μέγιστη ροή είναι η ίδια στις ακόλουθες δύο περιπτώσεις: Αν μία παίκτης επιλέξει την κακιά στρατηγική και
  αν αυτή η παίκτης επιλέξει μία παραλλαγή της κακιάς στρατηγικής στην οποία δεν μηδενίζει την εξερχόμενη άμεση εμπιστοσύνη
  της.

  Επιπλέον δικαιολόγηση της μεταβατικότητας της εμπιστοσύνης με χρήση της μέγιστης ροής μπορεί να βρεθεί στην κοινωνιολογική
  εργασία \cite{kmrs} όπου η άμεση αντιστοίχιση των μέγιστων ροών και της εμπειρικής εμπιστοσύνης επαληθεύεται πειραματικά.

  Εδώ βλέπουμε ένα ακόμη σημαντικό θεώρημα που δίνει τη βάση για συναλλαγές αμετάβλητου κινδύνου μεταξύ διαφορετικών, πιθανώς
  αγνώστων, ατόμων.
  \subimport{thesisgr/theorems/}{riskinvtheorem.tex}
  \subimport{thesisgr/proofs/}{riskinvproof.tex}
  Είναι διαισθητικά προφανές ότι η $A$ μπορεί να μειώσει την εξερχόμενη άμεση εμπιστοσύνη με τρόπου που να επιτυγχάνει το
  (\ref{primetrust}), αφού το $maxFlow\left(A, B\right)$ είναι συνεχές ως προς τις εξερχόμενες άμεσες εμπιστοσύνες της $A$.
  Αφήνουμε αυτόν τον υπολογισμό ως μέρος μελλοντικής έρευνας.
