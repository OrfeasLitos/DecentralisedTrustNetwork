\section{Σχετικές Εργασίες}
  Το θέμα της εμπιστοσύνης έχει προσεγγιστεί επανειλημμένα από διάφορες οπτικές: Καθαρά κρυπτογραφική υποδομή στην οποία η
  εμπιστοσύνη είναι σχεδόν δυαδική και η μεταβατικότητα περιορίζεται σε ένα βήμα πέρα από άτομα που εμπιστεύεται κανείς
  ενεργητικά εξερευνάται στο \textlatin{PGP} \cite{pgp}. Ένας μεταβατικός ιστός εμπιστοσύνης για την αντιμετώπιση ανεπιθύμητης
  αλληλογραφίας μελετάται στο \textlatin{Freenet} \cite{freenet}. Άλλα συστήματα απαιτούν κεντρικούς αξιόπιστους τρίτους, όπως
  \textlatin{PKI} βασιζόμενα σε αρχές πιστοποίησης \cite{pki} και το \textlatin{Bazaar} \cite{bazaar}, ή, στην περίπτωση της
  Βυζαντινής ανοχής στα σφάλματα, πιστοποιημένη συμμετοχή \cite{byzantine}. Ενώ άλλα συστήματα εμπιστοσύνης επιχειρούν να
  είναι αποκεντρωμένα, δεν αποδεικνύουν καμία ιδιότητα αντίστασης σε \textlatin{Sybil} επιθέσεις και συνεπώς ίσως να δέχονται
  τέτοιες επιθέσεις. Τέτοια συστήματα είναι το \textlatin{FIRE} \cite{fire}, το \textlatin{CORE} \cite{core} και άλλα
  \cite{openrep,ghkkw,rk}. Άλλα συστήματα που ορίζουν την εμπιστοσύνη με έναν μη οικονομικό τρόπο είναι τα
  \cite{mui,beta,pace,vpc,sdt,wot,pathfinder}.

  Συμφωνούμε με την εργασία \cite{badtrust} στο ότι η σημασία της εμπιστοσύνης δεν πρέπει να προεκτείνεται απρόσεκτα. Έχουμε
  υιοθετήσει τις συμβουλές τους στην παρούσα εργασία και παροτρύνουμε τον αναγνώστη να παραμένει στους ορισμούς της
  \textit{άμεσης} και \textit{έμμεσης} εμπιστοσύνης.

  Η αγορά \textlatin{Beaver} \cite{beaver} περιλαμβάνει ένα μοντέλο εμπιστοσύνης που βασίζεται σε χρεώσεις για να αποθαρρύνει
  τις επιθέσεις \textlatin{Sybil}. Επιλέξαμε να αποφύγουμε τις χρεώσεις στο σύστημά μας και να αντιμετωπίσουμε τις επιθέσεις
  \textlatin{Sybil} με άλλο τρόπο. Η κινητήριος εφαρμογή για την έρευνα στο θέμα της εμπιστοσύνης σε ένα αποκεντρωμένο
  περιβάλλον είναι η αγορά \textlatin{OpenBazaar}. Η μεταβατική οικονομική εμπιστοσύνη για το \textlatin{OpenBazaar} έχει
  μελετηθεί παλαιότερα στο \cite{dionyziz}. Η εργασία αυτή ωστόσο δεν ορίζει την εμπιστοσύνη σαν οικονομική αξία. Έχουμε
  εμπνευστεί ισχυρά από το \cite{kmrs} το οποίο δίνει μια κοινωνιολογική επιβεβαίωση για την κεντρική σχεδιαστική επιλογή της
  ταύτησης της εμπιστοσύνης με τον κίνδυνο. Εκτιμούμε ιδιαιτέρως την εργασία στο \textlatin{TrustDavis} \cite{davis}, το οποίο
  προτείνει ένα σύστημα οικονομικής εμπιστοσύνης που εμφανίζει μεταβατικές ιδιότητες και στο οποίο η εμπιστοσύνη ορίζεται ως
  πιστωτικές γραμμές, όμοια με το δικό μας σύστημα. Μπορέσαμε να επεκτείνουμε την εργασία τους χρησιμοποιώντας το
  \textlatin{blockchain} για αυτόματες αποδείξεις κινδύνου, ένα εργαλείο που δεν είχαν στη διάθεσή τους τότε.

  Η συντηρητική μας στρατηγική και το Μεταβατικό Παιχνίδι είναι πολύ παρόμοια με το μηχανισμό που προτείνεται στην οικονομική
  εργασία \cite{iou} η οποία επίσης περιγράφει μεταβατικότητα οικονομικής εμπιστοσύνης και χρησιμοποιείται από το
  \textlatin{Ripple} \cite{ripple} και το \textlatin{Stellar} \cite{stellar}. Τα \textlatin{IOU} στις προαναφερθείσες εργασίες
  αντιστοιχούν σε ανεστραμμένες ακμές εμπιστοσύνης στο δικό μας σύστημα. Η κρίσιμη διαφορά είναι ότι η δική μας εμπιστοσύνη
  εκφράζεται με ένα ενιαίο νόμισμα και τα ότι τα νομίσματα πρέπει να προϋπάρχουν για να τα εμπιστευθεί κάποια σε κάποια άλλη,
  άρα δεν υπάρχει χρήμα ως χρέος. Επιπλέον, αποδεικνύουμε ότι η εμπιστοσύνη και οι μέγιστες ροές είναι ισοδύναμες, μία
  ανεξερεύνητη κατεύθυνση από τις εργασίες τους, παρ' όλο που πιστεύουμε ότι θα πρέπει να ισχύει και για τα δικά τους
  συστήματα.
