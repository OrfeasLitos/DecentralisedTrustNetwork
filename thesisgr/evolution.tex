\section{Η Εξέλιξη της Εμπιστοσύνης}
  \subimport{thesisgr/definitions/}{turns.tex}
  \subimport{common/}{turnexample.tex}
  Χρησιμοποιούμε $prev\left(j\right)$ και $next\left(j\right)$ για να δηλώσουμε τον προηγούμενο και τον επόμενο γύρο που
  παίχθηκε αντίστοιχα από τον $Player\left(j\right)$.
  \subimport{common/definitions/}{prevnextturn.tex}
  \subimport{common/definitions/}{damage.tex}
  \subimport{common/definitions/}{history.tex}
  Γνώση του αρχικού γράφου $\mathcal{G}_0$, τα αρχικά κεφάλαια όλων των παικτών και την ιστορία ισοδυναμούν με πλήρη κατανόηση
  της εξέλιξης του παιχνιδιού. Χτίζοντας στο παράδειγμα του σχήματος \ref{fig:utxo}, μπορούμε να δούμε το γράφο που προκύπτει
  όταν ο $D$ παίξει
  \begin{equation}
  \label{turnexample}
     Turn_1 = \{Steal\left(1, A\right), Add\left(4, C\right)\} \enspace.
  \end{equation}
  \subimport{common/figures/}{turnexample.tikz}

  Το \textlatin{Trust Is Risk} ελέγχεται από έναν αλγόριθμο που επιλέγει έναν παίκτη, λαμβάνει το γύρο που ο παίκτης αυτός
  επιθυμεί να παίξει και, αν ο γύρος του είναι έγκυρος, τον εκτελεί. Αυτά τα βήματα επαναλαμβάνονται επ' αόριστον. Θεωρούμε
  ότι οι παίκτες επιλέγονται με τέτοιο τρόπο που ένας παίκτης, μετά από τον γύρο του, τελικά θα ξαναπαίξει αργότερα.
  \subimport{thesisgr/algorithms/}{trustisriskgame.tex}

  Η \textlatin{\texttt{strategy[}$A$\texttt{]()}} προσφέρει στον παίκτη $A$ πλήρη γνώση του παιχνιδιού, εκτός από τα κεφάλαια των άλλων
  παικτών. Αυτή η παραδοχή μπορεί να μην είναι πάντα ρεαλιστική.

  Η \textlatin{\texttt{executeTurn()}} ελέγχει την εγκυρότητα του γύρου \texttt{\textlatin{Turn}} και τον αντικαθιστά με έναν
  κενό γύρο αν είναι άκυρος. Ακόλουθα, δημιουργεί ένα νέο γράφο $\mathcal{G}_j$ και ανανεώνει την ιστορία αναλόγως. Για τους
  αντίστοιχους ψευδοκώδικες, δείτε το Παράρτημα.
