\section{Λειτουργία}
  Θα ακολουθήσουμε τώρα τα βήματα της $Alice$ από τη σύνδεση με το δίκτυο μέχρι να ολοκληρώσει επιτυχώς μια αγορά. Ας
  υποθέσουμε ότι αρχικά όλα τα νομίσματά της, ας πούμε $10\bitcoin$, είναι αποθηκευμένα έτσι που αποκλειστικά εκείνη μπορεί να
  τα ξοδέψει.

  Δύο αξιόπιστοι φίλοι, ο $Bob$ και ο $Charlie$, την πείθουν να δοκιμάσει το \textlatin{Trust Is Risk}. Εγκαθιστά το πορτοφόλι
  \textlatin{Trust Is Risk} και μεταφέρει τα $10\bitcoin$ από το κανονικό \textlatin{bitcoin} πορτοφόλι της, εμπιστεύοντας
  $2\bitcoin$ στον $Bob$ και $5\bitcoin$ στον $Charlie$. Τώρα ελέγχει αποκλειστικά $3\bitcoin$ και διακινδυνεύει $7\bitcoin$
  με αντάλλαγμα το να είναι μέρος του δικτύου. Έχει πλήρη αλλά όχι αποκλειστική πρόσβαση στα $7\bitcoin$ που εμπιστεύθηκε
  στους φίλους της και αποκλειστική πρόσβαση στα υπόλοιπα $3\bitcoin$, που αθροίζονται στα $10\bitcoin$.

  Μερικές ημέρες αργότερα, ανακαλύπτει ένα διαδικτυακό κατάστημα παπουτσιών του $Dean$, ο οποίος είναι συνδεδεμένος επίσης στο
  \textlatin{Trust Is Risk}. Η $Alice$ βρίσκει ένα ζευγάρι παπούτσια που κοστίζει $1\bitcoin$ και ελέγχει την αξιοπιστία του
  $Dean$ μέσω του νέου της πορτοφολιού. Ας υποθέσουμε ότι ο $Dean$ προκύπτει αξιόπιστος μέχρι $4\bitcoin$. Αφού το $1\bitcoin$
  είναι λιγότερο από τα $4\bitcoin$, η $Alice$ πραγματοποιεί την αγορά μέσω του καινούριου της πορτοφολιού με σιγουριά.

  Τότε βλέπει στο πορτοφόλι της ότι τα αποκλειστικά της νομίσματα αυξήθηκαν στα $6\bitcoin$, τα νομίσματα που εμπιστεύεται
  στον $Bob$ και στον $Charlie$ μειώθηκαν στα $0.5\bitcoin$ και $2.5\bitcoin$ αντίστοιχα και ότι εμπιστεύεται τον $Dean$ με
  $1\bitcoin$, όσο και η αξία των παπουτσιών. Επίσης, η αγορά της είναι σημειωμένη ως <<σε εξέλιξη>>. Αν η $Alice$ ελέγξει την
  έμμεση εμπιστοσύνη της προς τον $Dean$, θα είναι και πάλι $4\bitcoin$. Στο παρασκήνιο, το πορτοφόλι της ανακατένειμε τα
  νομίσματα που εμπιστευόταν με τρόπο ώστε εκείνη να εμπιστεύεται άμεσα στον $Dean$ τόσα νομίσματα όσο κοστίζει το αγορασμένο
  προϊόν και η εμπιστοσύνη που εμφανίζει το πορτοφόλι να είναι ίση με την αρχική.

  Τελικά όλα πάνε καλά και τα παπούτσια φτάνουν στην $Alice$. Ο $Dean$ επιλέγει να εξαργυρώσει τα νομίσματα που του
  εμπιστεύθηκε η $Alice$ κι έτσι το πορτοφόλι της δε δείχνει ότι εμπιστεύεται κανένα νόμισμα στον $Dean$. Μέσω του πορτοφολιού
  της, σημειώνει την αγορά ως επιτυχή. Αυτό επιτρέπει στο σύστημα να αναπληρώσει τη μειωμένη εμπιστοσύνη προς τον $Bob$ και
  τον $Charlie$, θέτοντας τα νομίσματα άμεσης εμπιστοσύνης στα $2\bitcoin$ και στα $5\bitcoin$ αντίστοιχα και πάλι. Η $Alice$
  τώρα ελέγχει αποκλειστικά $2\bitcoin$. Συνεπώς τώρα μπορεί να χρησιμοποιήσει συνολικά $9\bitcoin$, γεγονός αναμενόμενο, αφού
  έπρεπε να πληρώσει $1\bitcoin$ για τα παπούτσια.
