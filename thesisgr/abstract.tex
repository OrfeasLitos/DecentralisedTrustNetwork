\begin{abstract}
  Κεντρικά συστήματα φήμης χρησιμοποιούν αστέρια και κριτικές και επομένως χρειάζονται απόκρυψη αλγορίθμων για να αποφεύγουν
  τον αθέμιτο χειρισμό. Σε αυτόνομα αποκεντρωμένα συστήματα ανοιχτού κώδικα αυτή η πολυτέλεια δεν είναι διαθέσιμη.
  Στο παρόν κατασκευάζουμε ένα δίκτυο φήμης για αποκεντρωμένες αγορές όπου η εμπιστοσύνη που δίνει ο κάθε χρήστης στους
  υπόλοιπους είναι μετρήσιμη και εκφράζεται με νομισματικούς όρους. Εισάγουμε ένα νέο μοντέλο για πορτοφόλια
  \textlatin{bitcoin} στα οποία τα νομίσματα κάθε χρήστη μοιράζονται σε αξιόπιστους συνεργάτες. Η άμεση εμπιστοσύνη ορίζεται
  χρησιμοποιώντας μοιραζόμενους λογαριασμούς μέσω των 1-από-2 \textlatin{multisig} του \textlatin{bitcoin}. Η έμμεση
  εμπιστοσύνη ορίζεται έπειτα με μεταβατικό τρόπο. Αυτό επιτρέπει να επιχειρηματολογούμε με αυστηρό παιγνιοθεωρητικό τρόπο ως
  προς την ανάλυση κινδύνου. Αποδεικνύουμε ότι ο κίνδυνος και οι μέγιστες ροές είναι ισοδύναμα στο μοντέλο μας και ότι το
  σύστημά μας είναι ανθεκτικό σε επιθέσεις \textlatin{Sybil}. Το σύστημά μας επιτρέπει τη λήψη σαφών οικονομικών αποφάσεων ως
  προς την υποκειμενική χρηματική ποσότητα με την οποία μπορεί ένας παίκτης να εμπιστευθεί μία ψευδώνυμη οντότητα. Μέσω
  ανακατανομής της άμεσης εμπιστοσύνης, ο κίνδυνος που διατρέχεται κατά την αγορά από έναν ψευδώνυμο πωλητή παραμένει
  αμετάβλητος.
  \begin{otherlanguage}{english}
  \keywords{\selectlanguage{greek}αποκεντρωμένο $\cdot$ εμπιστοσύνη $\cdot$ δίκτυο εμπιστοσύνης $\cdot$ γραμμές πίστωσης
            $\cdot$ εμπιστοσύνη ως κίνδυνος $\cdot$ ροή $\cdot$ φήμη $\cdot$ \\ \selectlanguage{english}decentralized $\cdot$
            trust $\cdot$ web-of-trust $\cdot$ bitcoin $\cdot$ multisig $\cdot$ line-of-credit $\cdot$ trust-as-risk $\cdot$
            flow $\cdot$ reputation}
  \end{otherlanguage}
\end{abstract}
\newpage
%\patchcmd{\abstract}{Abstract}{Summary}{}{}
\begin{otherlanguage}{english}
\begin{abstract}
  Centralized reputation systems use stars and reviews and thus require algorithm secrecy to avoid manipulation.
  In autonomous open source decentralized systems this luxury is not available.
  We create a reputation network for decentralized marketplaces where the trust each user gives to the rest of the users is
  quantifiable and expressed in monetary terms.
  We introduce a new model for bitcoin wallets in which user coins are split among trusted associates.
  Direct trust is defined using shared bitcoin accounts via bitcoin's 1-of-2 multisig.
  Indirect trust is subsequently defined transitively.
  This enables formal game theoretic arguments pertaining to risk analysis.
  We prove that risk and maximum flows are equivalent in our model and that our system is Sybil-resilient.
  Our system allows for concrete financial decisions on the subjective monetary amount a pseudonymous party can be trusted
  with.
  Through direct trust redistribution, the risk incurred from making a purchase from a pseudonymous vendor in this manner
  remains invariant.
\end{abstract}
\end{otherlanguage}
