\section{Μεταβατικότητα Εμπιστοσύνης}
  Στην ενότητα αυτή ορίζουμε μερικές στρατηγικές και δείχνουμε τους ανάλογους αλγορίθμους. Μετά ορίζουμε το Μεταβατικό
  Παιχνίδι (\textlatin{Transitive Game}) που αναπαριστά το σενάριο χειρότερης περίπτωσης για έναν τίμιο παίκτη όταν κάποιος
  άλλος παίκτης αποφασίζει να φύγει από το δίκτυο με τα χρήματά του και όλα τα χρήματα που άλλοι εμπιστεύονται άμεσα σε αυτόν.
  \subimport{thesisgr/definitions/}{idlestrategy.tex}

  \subimport{thesisgr/algorithms/}{idlestrategycode.tex}
  Οι είσοδοι και οι έξοδοι είναι πανομοιότυποι με αυτούς της \texttt{\textlatin{idleStrategy()}} στις υπόλοιπες στρατηγικές,
  συνεπώς αποφεύγουμε την επανάληψή τους.
  \subimport{thesisgr/definitions/}{evilstrategy.tex}

  \subimport{thesisgr/algorithms/}{evilstrategycode.tex}

  \subimport{thesisgr/definitions/}{conservativestrategy.tex}

  \subimport{thesisgr/algorithms/}{conservativestrategycode.tex}
  Η \texttt{\textlatin{SelectSteal()}} επιστρέφει $y_v$ με $v \in N^{-}\left(A\right)_{j-1}$ τέτοιο ώστε
  \begin{equation}
  \label{stealrestriction}
     \sum\limits_{v \in N^{-}\left(A\right)_{j-1}}y_v = Damage_{A, j} \wedge \forall v \in N^{-}\left(A\right)_{j-1},
     y_v \leq DTr_{v \rightarrow A, j-1} \enspace.
  \end{equation}
  Ο παίκτης $A$ μπορεί να ορίσει κατά βούληση πώς η \texttt{\textlatin{SelectSteal()}} θα κατανείμει τις πράξεις
  $Steal\left(\right)$ κάθε φορά που καλεί τη συνάρτηση, εφ' όσον ο περιορισμός (\ref{stealrestriction}) είναι σεβαστός.

  Όπως βλέπουμε, ο ορισμός καλύπτει μια πληθώρα επιλογών για τον συντηρητικό παίκτη, αφού στην περίπτωση που $0 < Damage_{A,j}
  < in_{A,j-1}$ μπορεί να επιλέξει να κατανείμει τις πράξεις $Steal\left(\right)$ όπως επιθυμεί.

  Ο συλλογισμός πίσω από αυτή τη στρατηγική προκύπτει από μια συνηθισμένη περίπτωση στον πραγματικό κόσμο. Έστω ένας πελάτης,
  ένας μεσάζοντας κι ένας παραγωγός. Ο πελάτης εμπιστεύεται κάποια αξία στο μεσάζοντα ώστε ο τελευταίος να μπορέσει να
  αγοράσει το επιθυμητό προϊόν από τον παραγωγό και να το παραδώσει στον πελάτη. Ο μεσάζοντας με τη σειρά του εμπιστεύεται ίση
  αξία στον παραγωγό, ο οποίος απαιτεί την προκαταβολή του ποσού για να μπορέσει να ολοκληρώσει τη διαδικασία παραγωγής.
  Ωστόσο, ο παραγωγός τελικά δε δίνει το προϊόν ούτε επιστρέφει το ποσό λόγω πτώχευσης ή επιλογής να φύγει από την αγορά με
  ένα άδικο όφελος. Ο μεσάζοντας τότε μπορεί να επιλέξει είτε να αποζημιώσει τον πελάτη και να υποστεί τη ζημία, ή να αρνηθεί
  την αποζημίωση και να χάσει την εμπιστοσύνη του πελάτη. Η τελευταία επιλογή για τον μεσάζοντα είναι ακριβώς η συντηρητική
  στρατηγική. Χρησιμοποιείται στη συνέχεια του παρόντος ως η στρατηγική για όλους τους ενδιάμεσους παίκτες γιατί μοντελοποιεί
  με επιτυχία το σενάριο χειρότερης περίπτωσης που ένας πελάτης μπορεί να αντιμετωπίσει αφού ένας κακός παίκτης αποφασίσει να
  κλέψει ό,τι μπορεί και οι υπόλοιποι παίκτες δεν εμπλέκονται σε κακή δράση.

  Συνεχίζουμε με μία δυνατή εξέλιξη του παιχνιδιού, το Μεταβατικό Παιχνίδι. Στο γύρο 0, υπάρχει ήδη ένα συγκεκριμένο δίκτυο.
  Όλοι οι παίκτες εκτός του $A$ και του $B$ ακολουθούν τη συντηρητική στρατηγική. Επιπλέον, το σύνολο των παικτών δε
  μεταβάλλεται κατά τη διάρκεια του Μεταβατικού Παιχνιδιού, συνεπώς μπορούμε να αναφερθούμε στο $\mathcal{V}_j$ για κάθε γύρο
  $j$ ως $\mathcal{V}$. Επίσης, κάθε συντηρητικός παίκτης μπορεί να βρίσκεται σε μία από τρεις καταστάσεις: Χαρούμενος
  (\textlatin{Happy}), Θυμωμένος (\textlatin{Angry}) ή Λυπημένος (\textlatin{Sad}). Οι Χαρούμενοι παίκτες έχουν ζημία 0, οι
  Θυμωμένοι παίκτες έχουν θετική ζημία και θετική εισερχόμενη άμεση εμπιστοσύνη, άρα μπορούν να αναπληρώσουν τη ζημία τους
  τουλάχιστον μερικώς και οι Λυπημένοι παίκτες έχουν θετική ζημία, αλλά 0 εισερχόμενη άμεση εμπιστοσύνη, άρα δεν μπορούν να
  αναπληρώσουν τη ζημία. Αυτές οι συμβάσεις θα ισχύουν όποτε χρησιμοποιούμε το Μεταβατικό Παιχνίδι.
  \subimport{thesisgr/algorithms/}{transitivegame.tex}

  Ένα παράδειγμα εκτέλεσης ακολουθεί:

  \subimport{thesisgr/figures/}{transitivegameexample.tikz}

  Έστω $j_0$ ο πρώτος γύρος στον οποίο ο $B$ επιλέγεται. Μέχρι τότε, όλοι οι παίκτες θα παίζουν <<πάσο>> αφού τίποτα δεν έχει
  κλαπεί ακόμα (βλέπε το Παράρτημα (Θεώρημα~\ref{conservativeworld}) για μια αυστηρή απόδειξη αυτού του απλού γεγονότος).
  Επιπλέον, έστω $v = Player(j)$ και $j' = prev\left(j\right)$. Το Μεταβατικό Παιχνίδι παράγει γύρους:
  \begin{equation}
     Turn_j = \bigcup\limits_{w \in N^{-}\left(v\right)_{j-1}}\{Steal\left(y_w,w\right)\} \enspace,
  \end{equation}
  όπου
  \begin{equation*}
     \sum\limits_{w \in N^{-}\left(v\right)_{j-1}}y_w = \min\left(in_{v, j-1}, Damage_{v, j}\right) \enspace.
  \end{equation*}
 
  Βλέπουμε ότι αν $Damage_{v, j} = 0$, τότε $Turn_j = \emptyset$.

  Από τον ορισμό του $Damage_{v,j}$ και γνωρίζοντας ότι καμία στρατηγική σε αυτή την περίπτωση δεν μπορεί να αυξήσει καμία
  άμεση εμπιστοσύνη, βλέπουμε ότι $Damage_{v,j} \geq 0$. Επίσης, είναι $Loss_{v,j} \geq 0$ γιατί αν $Loss_{v,j} < 0$, τότε ο
  $v$ θα είχε κλέψει περισσότερη αξία απ' ότι του έχει κλαπεί, συνεπώς δε θα ακολουθόύσε τη συντηρητική στρατηγική.
