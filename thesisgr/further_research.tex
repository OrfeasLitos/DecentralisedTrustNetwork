\section{Μελλοντική Έρευνα}

  %\subsection{Trust Transfer Algorithms}
    Όταν η $Alice$ πραγματοποιεί μία αγορά από τον $Bob$, η πρώτη πρέπει να μειώσει την εξερχόμενη άμεση εμπιστοσύνη της με
    τρόπο ώστε η προϋπόθεση (\ref{primetrust}) του θεωρήματος Αμετάβλητου Κινδύνου να ικανοποιείται. Το πώς η $Alice$ μπορεί
    να επανυπολογίσει την εξερχόμενη άμεση εμπιστοσύνη της θα συζητηθεί σε μελλοντική εργασία.

  %\subsection{Dynamic Setting}
    Το παιχνίδι μας είναι στατικό. Σε ένα μελλοντικό δυναμικό περιβάλλον, οι χρήστες πρέπει να μπορούν να παίζουν ταυτόχρονα,
    να συνδέονται, να αποχωρούν ή να αποσυνδέονται προσωρινά από το δίκτυο. Άλλα είδη \textlatin{multisig}, όπως 1-από-3,
    μποροόυν να ερευνηθούν για την υλοποίηση άμεσης εμπιστοσύνης πολλών παικτών.

  %\subsection{Zero knowledge}
    Ο αλγόριθμος \textlatin{MaxFlow} χρειάζεται πλήρη γνώση του δικτύου, κάτι που μπορεί να οδηγήσει σε προβλήματα
    ιδιωτικότητας μέσω τεχνικών αποανωνυμοποίησης \cite{deanonymisation}. Ο υπολογισμός των ροών με μηδενική γνώση παραμένει
    ένα ανοιχτό ερώτημα. Το \cite{silentwhispers} και ο κεντροποιημένος προκάτοχός του, το \textlatin{PrivPay} \cite{privpay},
    φαίνεται να προσφέρουν ανεκτίμητες ιδέες ως προς το πώς μπορεί να επιτευχθεί η ιδιωτικότητα.

  %\subsection{Game Theoretic Analysis}
    Η παιγνιοθεωρητική μας ανάλυση είναι απλή. Μία ενδιαφέρουσα ανάλυση θα περιλάμβανε τη μοντελοποίηση επαναλαμβανόμενων
    αγορών με τις σχετικές ανανεώσεις ακμών στο γράφο εμπιστοσύνης και την αντιμετώπιση της εμπιστοσύνης στο δίκτυο ως μέρος
    της συνάρτησης χρησιμότητας.

  %\subsection{Implementation and Experimental Results}
    Η υλοποίηση του οικονομικού μας παιχνιδιού ως πορτοφόλι σε οποιοδήποτε \textlatin{blockchain} θα ήταν ευπρόσδεκτη. Μία
    προσομοίωση ή πραγματική υλοποίηση του \textlatin{Trust Is Risk}, σε συνδυασμό με μία ανάλυση των δυναμικών που προκύπτουν
    μπορεί να προσφέρουν ενδιαφέροντα πειραματικά αποτελέσματα. Στη συνέχεια, το δίκτυο εμπιστοσύνης μπορεί να χρησιμοποιηθεί
    σε άλλες εφαρμογές, όπως αποκεντρωμένα κοινωνικά δίκτυα \cite{synereo}.
