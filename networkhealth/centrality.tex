\section{Centrality}
  Another important measure is the network centrality. We propose here some different measures to that end.
  
  \subsection{Degree Centrality} \ \\

    One possible approach is the degree centrality [Freeman citation], that can be broken down as in-degree and out-degree
    centrality. We first define the node in-degree centrality.
    \begin{equation*}
      C_{in}\left(A\right) = \sum\limits_{B \in \mathcal{V} \setminus \{A\}}DTr{B \rightarrow A} \mbox{ (Node in-degree
      centrality)}
    \end{equation*}
    Let $A^* = \argmax\limits_{A \in \mathcal{V}}C_{in}\left(A\right)$. The network in-degree centrality is defined as:
    \begin{equation*}
      C_{in} = \sum\limits_{A \in \mathcal{V}}\left(C_{in}\left(A^*\right) - C_{in}\left(A\right)\right) \mbox{ (Network
      in-degree centrality)}
    \end{equation*}
    Similarly, for the out-degree centrality we have:
    \begout{equation*}
      C_{out}\left(A\right) = \sum\limits_{B \out \mathcal{V} \setmoutus \{A\}}DTr{A \rightarrow B} \mbox{ (Node out-degree
      centrality)}
    \end{equation*}
    Let $A^* = \argmax\limits_{A \out \mathcal{V}}C_{out}\left(A\right)$. The network out-degree centrality is defouted as:
    \begout{equation*}
      C_{out} = \sum\limits_{A \out \mathcal{V}}\left(C_{out}\left(A^*\right) - C_{out}\left(A\right)\right) \mbox{ (Network
      out-degree centrality)}
    \end{equation*}
