Alternative transitive games are flows proof with pseudoflows
\begin{sepproof}[Transitive Games Are Flows Lemma (\ref{gameflow})] \ \\
   Let $\mathcal{H}$ be an execution of the Transitive Game on graph $\mathcal{G}$ and $j$ the convergence turn. We will
   show that a flow from $A$ to $B$ can be constructed such that $\sum\limits_{v \in N^{+}\left(A\right)}x_{Av} =
   Loss_{A, j}$. First, we construct a pseudoflow $X$ on $G$ as follows:
   \begin{equation*}
   \begin{gathered}
      \forall v, w \in \mathcal{V}, x_{vw} = \sum\limits_{\overset{i = 0}{Player\left(i\right) = w}}^jy_i \enspace,
      \mbox{ where}\\
      Steal(y_i, v) \in Turn_i
   \end{gathered}
   \end{equation*}
   The configuration described above is a pseudoflow \cite{amo} because
   \begin{equation*}
       \forall v,w \in \mathcal{V}, \sum\limits_{\overset{i = 0}{Player\left(i\right) = w}}^jy_i \leq
       DTr_{v \rightarrow w, 0} = c_{vw} \enspace.
   \end{equation*}
   By the definition of $X$, it holds that
   \begin{equation}
   \label{desiredoutgoingflow}
       \sum\limits_{v \in N^{+}\left(A\right)}x_{Av} = Loss_A \enspace.
   \end{equation}
   Suppose that $X$ contains a excess node $v$. In this node it is
   \begin{equation*}
      \sum\limits_{w \in N^{-}\left(v\right)}x_{wv} > \sum\limits_{w \in N^{+}\left(v\right)}x_{vw} \enspace.
   \end{equation*}
   By the definition of $X$ however, $v$ stole more than she was stolen, thus does not follow the conservative strategy.
   We have reached a contradiction, thus there exist no excess nodes in $X$.
