\section{Strategy Generalization}
  More specifically, apart from the capacity $c_e$, each edge $e \in \mathcal{E}$ is assigned a gain factor $\gamma_e \in
  \left[0, 1\right]$. 
  
  \subimport{common/definitions/}{extendedstrategy.tex}

  Given a graph with capacities, loss factors, a source $A$ and a sink $B$, the generalized MaxFlow algorithm yields a
  generalized flow that corresponds to the steal actions in the worst case scenario for $A$, given that $A$ follows the Idle
  strategy, $B$ follows the Evil strategy and all other players follow the Extended strategy [proof needed]. For the sake of
  example, consider the following two graphs:
  
  \subimport{generalizedMaxFlow/figures/}{simple1.tikz}

  \subimport{generalizedMaxFlow/figures/}{simple2.tikz}

  We will now attempt to obtain some insight as to how players behave on the two ends of the spectrum of the loss factor
  $\gamma$. Consider an edge $e = \left(Alice, Bob\right) \in \mathcal{E}$. If $\gamma_e = 0$, then $Bob$ will not steal any
  funds from $Alice$, no matter how big a loss he has suffered. On the other hand, if $\gamma_e = 1$, then $Bob$ can replenish
  any amount of stolen funds through stealing from $Alice$, given that she directly trusts him with sufficient funds.
  
  If $\forall v \in \mathcal{V}, \gamma_{\left(v, Alice\right)} = 0$, then $Alice$ is following the Idle strategy, whereas if
  $\forall v \in \mathcal{V}, \gamma_{\left(v, Alice\right)} = 1$, then $Alice$ is following the Conservative strategy. [proof
  needed] The incoming direct trusts to the Evil player should all have a gain factor equal to 1, since we consider that she
  steals all her incoming direct trust.
