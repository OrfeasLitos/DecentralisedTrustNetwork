\section{Introduction}
  In our previous work, we presented three distinct strategies for the players: the Idle, the Conservative and the Evil
  strategy. The process of determining the indirect trust from Alice to Bob involved assigning the Idle strategy to Alice, the
  Evil strategy to Bob and the Conservative strategy to all other players. The indirect trust from Alice to Bob would then be
  the worst case scenario for Alice when Bob initiates a "chain reaction" of steal actions. This value was proven to be equal
  to the maximum flow from Alice to Bob. Consider however the two following cases:

  \subimport{generalizedMaxFlow/figures/}{fewhops.tikz}

  \subimport{generalizedMaxFlow/figures/}{manyhops.tikz}

  One could argue that intuitively it should be $Tr_{\mathcal{G}_1, A \rightarrow B} > Tr_{\mathcal{G}_2, A \rightarrow B}$,
  since the longer chain of players connecting A and B in $\mathcal{G}_2$ introduces more uncertainty as to whether B is
  trustworthy. Nevertheless, according to our prior approach, it is $Tr_{\mathcal{G}_1, A \rightarrow B} = Tr_{\mathcal{G}_2,
  A \rightarrow B}$ since $maxFlow_{\mathcal{G}_1}\left(A, B\right) = maxFlow_{\mathcal{G}_2}\left(A, B\right)$.

  To mitigate this problem, we introduce a generalization of the previous approach that can take care of this observation. The
  mechanism that we propose is analogous to an attenuation factor that accounts for the number of hops, but is better suited
  for the ambience of maximum flow. Each edge $\left(v, w\right)$ of the graph is enhanced with an additional number called
  \textit{loss factor}, which intuitively represents the "leakage" ratio on this edge, or the percentage of the damage
  incurred by $w$ that is carried over to $v$ through this edge, in case $w$ is stolen some funds.
