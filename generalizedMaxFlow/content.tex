The generalized MaxFlow algorithm expects as input a graph with capacities as well as a \textit{gain factor} for each edge,
which in our case will be a number in $\left[0, 1\right]$. Intuitively, the gain factor of an edge $\left(v, w\right)$
represents the ratio of "leakage" this edge causes. In our case, it represents the percentage of funds that $v$ will try to
replenish when $w$ steals from her. For the sake of example, consider the following graph:

\begin{center}
\begin{tikzpicture}[>=latex,line join=bevel,]
%%
\begin{scope}
  \definecolor{strokecol}{rgb}{0.0,0.0,0.0};
  \pgfsetstrokecolor{strokecol}
  \draw (157.29bp,11.5bp) node {Fig.1: Alice trusts Charlie 10};
\end{scope}
  \node (Charlie) at (278.19bp,41.0bp) [draw,ellipse] {Charlie};
  \node (Bob) at (150.79bp,41.0bp) [draw,ellipse] {Bob};
  \node (Alice) at (29.897bp,41.0bp) [draw,ellipse] {Alice};
  \draw [->] (Alice) ..controls (76.038bp,41.0bp) and (96.132bp,41.0bp)  .. (Bob);
  \draw (91.795bp,48.5bp) node {c=10};
  \draw [->] (Bob) ..controls (193.52bp,41.0bp) and (213.47bp,41.0bp)  .. (Charlie);
  \draw (209.79bp,48.5bp) node {c=20};
%
\end{tikzpicture}
\end{center}

A gain factor of 0 means that $v$ will tolerate any amount of stolen funds by $w$ without trying to replenish them by stealing
others that directly trust her, whereas a gain factor of 1 means that $v$ will try to replenish any amount of stolen funds by
$w$. If the gain factor is 0 on edges $\left(v, w\right)$ for all $w \in \mathcal{V}$, then $v$ is following the Idle
strategy, whereas if the gain factor is 1 on edges $\left(v, w\right)$ for all $w \in \mathcal{V}$, then $v$ is following the
Conservative strategy.
