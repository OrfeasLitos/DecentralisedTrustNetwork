\begin{sepproof}{Proof of Lemma \ref{maxflowgame}: MaxFlows Are Transitive Games} \ \\
\label{maxflowgameproof}
   We suppose that the turn of $\mathcal{G}$ is 0. In other words, $\mathcal{G} = \mathcal{G}_0$. Let
   $X = \{x_{vw}\}_{\mathcal{V} \times \mathcal{V}}$ be the flows returned by $MaxFlow\left(A, E\right)$. For any graph
   $G$ there exists a $MaxFlow$ that is a DAG. We can easily prove this using the Flow Decomposition theorem
   \cite{amo}, which states that each flow can be seen as a finite set of paths from $A$ to $E$ and cycles, each
   having a certain flow. We execute $MaxFlow\left(A, E\right)$ and we apply the aforementioned theorem. The
   cycles do not influence the $maxFlow\left(A, E\right)$, thus we can remove these flows. The resulting flow is a
   $MaxFlow\left(A, E\right)$ without cycles, thus it is a DAG. Topologically sorting this DAG, we obtain a total order
   of its nodes such that $\forall$ nodes $v, w \in \mathcal{V} : v < w \Rightarrow x_{wv} = 0$ \cite{clrs}. Put
   differently, there is no flow from larger to smaller nodes. $E$ is maximum since it is the sink and thus has no
   outgoing flow to any node and $A$ is minimum since it is the source and thus has no incoming flow from any node. The
   desired execution of Transitive Game will choose players following the total order inversely, starting from player
   $E$. We observe that $\forall v \in \mathcal{V} \setminus \{A, E\}, \sum\limits_{w \in \mathcal{V}}x_{wv} =
   \sum\limits_{w \in \mathcal{V}}x_{vw} \leq maxFlow\left(A, E\right) \leq in_{E, 0}$. Player $E$ will follow a modified
   evil strategy where she steals value equal to her total incoming flow, not her total incoming trust. Let $j_2$ be the
   first turn when $A$ is chosen to play. We will show using strong induction that there exists a set of valid actions
   for each player according to their respective strategy such that at the end of each turn $j$ the corresponding player
   $v = Player\left(j\right)$ will have stolen value $x_{wv}$ from each in-neighbour $w$.

   Base case: In turn 1, $E$ steals value equal to $\sum\limits_{w \in \mathcal{V}}x_{wE}$, following the modified evil
   strategy.
   \begin{equation*}
      Turn_1 = \bigcup\limits_{v \in N^{-}\left(E\right)_0}\{Steal\left(x_{vE}, v\right)\}
   \end{equation*}

   Induction hypothesis: Let $k \in [j_2 - 2]$. We suppose that $\forall i \in [k]$, there exists a valid set of actions,
   $Turn_i$, performed by $v = Player\left(i\right)$ such that $v$ steals from each player $w$ value equal to $x_{wv}$.
   \begin{equation*}
      \forall i \in [k], Turn_i = \bigcup\limits_{w \in N^{-}\left(v\right)_{i-1}}\{Steal\left(x_{wv}, w\right)\}
   \end{equation*}

   Induction step: Let $j = k + 1, v = Player\left(j\right)$. Since all the players that are greater than $v$ in the
   total order have already played and all of them have stolen value equal to their incoming flow, we deduce that $v$ has
   been stolen value equal to $\sum\limits_{w \in N^{+}\left(v\right)_{j-1}}x_{vw}$. Since it is the first time $v$
   plays, $\forall w \in N^{-}\left(v\right)_{j-1}, DTr_{w \rightarrow v, j-1} = DTr_{w \rightarrow v, 0} \geq x_{wv}$, thus
   $v$ is able to choose the following turn:
   \begin{equation*}
      Turn_j = \bigcup\limits_{w \in N^{-}\left(v\right)_{j-1}}\{Steal\left(x_{wv}, w\right)\}
   \end{equation*}
   Moreover, this turn satisfies the conservative strategy since
   \begin{equation*}
      \sum\limits_{w \in N^{-}\left(v\right)_{j-1}}x_{wv} = \sum\limits_{w \in N^{+}\left(v\right)_{j-1}}x_{vw} \enspace.
   \end{equation*}
   Thus $Turn_j$ is a valid turn for the conservative player $v$.

   We have proven that in the end of turn $j_2 - 1$, player $E$ and all the conservative players will have stolen value
   exactly equal to their total incoming flow, thus $A$ will have been stolen value equal to her outgoing flow, which is
   $maxFlow(A, E)$. Since there remains no Angry player, $j_2$ is a convergence turn, thus $Loss_{A, j_2} = Loss_A$. We
   can also see that if $E$ had chosen the original evil strategy, the described actions would still be valid only by
   supplementing them with additional $Steal\left(\right)$ actions, thus $Loss_A$ would further increase. This proves the
   theorem.
\end{sepproof}
