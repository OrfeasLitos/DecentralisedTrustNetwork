    \begin{sepproof}{Proof of Theorem \ref{convergence}: Trust Convergence} \ \\
    \label{convergenceproof}
       First of all, after turn $j_0$ player $E$ will always pass her turn
       because she has already nullified her incoming and outgoing direct trusts in $Turn_{j_0}$, the evil strategy does not
       contain any case where direct trust is increased or where the evil player starts directly trusting another player and
       the other players do not follow a strategy in which they can choose to $Add\left(\right)$ trust to $E$. The same holds
       for player $A$ because she follows the idle strategy. As far as the rest of the players are concerned, consider the
       Transitive Game. As we can see from lines~\ref{trsteallossinit} and~\ref{trsteallossincrease}
       -~\ref{trsteallossdecrease}, it is
       \begin{equation*}
          \forall j, \sum\limits_{v \in \mathcal{V}_j}Loss_v = in_{E, j_0-1} \enspace.
       \end{equation*}
       In other words, the total loss is constant and equal to the total value stolen by $E$. Also, as we can see in
       lines~\ref{trstealsadinit} and~\ref{trstealtrueentersad}, which are the only lines where the $Sad$ set is modified,
       once a player enters the $Sad$ set, it is impossible to exit from this set. Also, we can see that players in $Sad
       \cup Happy$ always pass their turn. We will now show that eventually the $Angry$ set will be empty, or equivalently
       that eventually every player will pass their turn. Suppose that it is possible to have an infinite amount of turns
       in which players do not choose to pass. We know that the number of nodes is finite, thus this is possible only if
       \begin{equation*}
          \exists j': \forall j \geq j', |Angry_j \cup Happy_j| = c > 0 \wedge Angry_j \neq \emptyset \enspace.
       \end{equation*}
       This statement is valid because the total number of angry and happy players cannot increase because no player leaves
       the $Sad$ set and if it were to be decreased, it would eventually reach 0. Since $Angry_j \neq \emptyset$, a player
       $v$ that will not pass her turn will eventually be chosen to play. According to the Transitive Game, $v$ will either
       deplete her incoming trust and enter the $Sad$ set (line~\ref{trstealtrueentersad}), which is contradicting $|Angry_j
       \cup Happy_j| = c$, or will steal enough value to enter the $Happy$ set, that is $v$ will achieve $Loss_{v, j} = 0$.
       Suppose that she has stolen $m$ players. They, in their turn, will steal total value at least equal to the value
       stolen by $v$ (since they cannot go sad, as explained above). However, this means that, since the total value being
       stolen will never be reduced and the turns this will happen are infinite, the players must steal an infinite amount of
       value, which is impossible because the direct trusts are finite in number and in value. More precisely, let $j_1$ be
       a turn in which a conservative player is chosen and
       \begin{equation*}
          \forall j \in \mathbb{N}, DTr_j = \sum\limits_{w,w' \in \mathcal{V}}DTr_{w \rightarrow w', j} \enspace.
       \end{equation*}
       Also, without loss of generality, suppose that
       \begin{equation*}
          \forall j \geq j_1, out_{A, j} = out_{A, j_1} \enspace.
       \end{equation*}
       In $Turn_{j_1}$, $v$ steals
       \begin{equation*}
          St = \sum\limits_{i=1}^{m}y_i \enspace.
       \end{equation*}
       We will show using induction that
       \begin{equation*}
          \forall n \in \mathbb{N}, \exists j_n \in \mathbb{N} : DTr_{j_n} \leq DTr_{j_1-1} - nSt \enspace.
       \end{equation*}

       Base case: It holds that
       \begin{equation*}
          DTr_{j_1} = DTr_{j_1-1} - St \enspace.
       \end{equation*}
       Eventually there is a turn $j_2$ when every player in $N^{-}(v)_{j-1}$ will have played. Then it holds that
       \begin{equation*}
          DTr_{j_2} \leq DTr_{j_1} - St = DTr_{j_1-1} - 2St \enspace,
       \end{equation*}
       since all players in $N^{-}(v)_{j-1}$ follow the conservative strategy, except for $A$, who will not have been stolen
       anything due to the supposition.

       Induction hypothesis: Suppose that
       \begin{equation*}
          \exists k > 1 : j_k > j_{k-1} > j_1 \Rightarrow DTr_{j_k} \leq DTr_{j_{k-1}} - St \enspace.
       \end{equation*}

       Induction step: There exists a subset of the $Angry$ players, $S$, that have been stolen at least value $St$ in total
       between the turns $j_{k-1}$ and $j_k$, thus there exists a turn $j_{k+1}$ such that all players in $S$ will have
       played and thus
       \begin{equation*}
          DTr_{j_{k+1}} \leq DTr_{j_k} - St \enspace.
       \end{equation*}
       We have proven by induction that
       \begin{equation*}
          \forall n \in \mathbb{N}, \exists j_n \in \mathbb{N} : DTr_{j_n} \leq DTr_{j_1-1} - nSt \enspace.
       \end{equation*}
       However
       \begin{equation*}
          DTr_{j_1-1} \geq 0 \wedge St > 0 \enspace,
       \end{equation*}
       thus
       \begin{equation*}
          \exists n' \in \mathbb{N} : n'St > DTr_{j_1-1} \Rightarrow DTr_{j_{n'}} < 0 \enspace.
       \end{equation*}
       We have a contradiction because
       \begin{equation*}
          \forall w,w' \in \mathcal{V}, \forall j \in \mathbb{N}, DTr_{w \rightarrow w', j} \geq 0 \enspace,
       \end{equation*}
       thus eventually $Angry = \emptyset$ and everybody passes.
    \end{sepproof}
