\begin{sepproof}{Proof of Lemma \ref{maxflowdec}: maxFlow Monotonicity} \ \\
  Supppose that
  \begin{equation*}
    \exists \delta_1, \delta_2 : \delta_1 < \delta_2 \wedge maxFlow\left(\delta_1\right) \leq maxFlow\left(\delta_2\right)
    < F \enspace.
  \end{equation*}
  We choose $X'_1, X'_2$ such that
  \begin{equation*}
    \forall i \in [n], x'_{1, i} \leq x_i \wedge x'_{2, i} \leq x_i \enspace.
  \end{equation*}
  This is always possible because we can derive the maximum flows $X'_1$ and $X'_2$ starting with $X$ and reducing the flow
  along paths beginning from edges with capacities $c'_{1,i} < x_i$ and $c'_{2,i} < x_i$ respectively, until the flows become
  valid.

  Define $MinCut\left(\delta\right)$ as the minimum cut set of the $MaxFlow\left(\delta\right)$ configuration. Let
  \begin{equation*}
    S_j = \{i \in [n] : v_i \in N^{+}\left(A\right) \cap MinCut\left(\delta_j\right)\}, j \in \{1, 2\} \enspace.
  \end{equation*}
  It holds that $S_j \neq \emptyset$. Suppose that $S_j = \emptyset$. Since $F > F'_j$, there exists a path from $A$ to $B$
  on $G'_j$ with positive flow not used, thus $X'_j$ is not the maximum flow, which is a contradiction. Thus
  $S_j \neq \emptyset, j \in \{1, 2\} \enspace.$

%  (Then \begin{equation*}
%    MinCut\left(\delta_j\right) = MinCut\left(C\right) \Rightarrow maxFlow\left(\delta_j\right) = F, j \in \{1, 2\}
%    \enspace,
%  \end{equation*}
%  which is a contradiction. Thus $S_j \neq \emptyset$.)
%
  Moreover, it holds that $S_1 \subseteq S_2$, since $\forall i \in [n], c'_{2, i} \leq c'_{1, i}$. More precisely, it is
  \begin{equation*}
    \forall i \in [n] : c'_{2, i} > 0, c'_{2, i} < c'_{1, i} \enspace.
  \end{equation*}
  Every node in the $MinCut\left(\delta_j\right)$ is saturated, thus
  \begin{equation*}
    \forall i \in S_1, x'_{j, i} = c'_{j, i}, j \in \{1, 2\} \enspace.
  \end{equation*}
  Thus $\sum\limits_{i \in S_1} x'_{2, i} < \sum\limits_{i \in S_1}x'_{1, i}$ and, since $maxFlow(\delta_1) \leq
  maxFlow(\delta_2)$, we conclude that for $X'_1, X'_2$ it is $\sum\limits_{i \in [n] \setminus S_1}x'_{2, i} >
  \sum\limits_{i \in [n] \setminus S_1}x'_{1, i}$. However, since $x_{i,j}' \leq x_i, j \in \{1,2\}$, the configuration $X''$
  such that
  \begin{align*}
    \forall i \in S_1&, x''_i = x'_{1, i} \\
    \forall i \in [n] \setminus S_1&, x''_i = x'_{2, i}
  \end{align*}
  is a valid flow configuration for $C'_1$ and then
  \begin{equation*}
  \begin{gathered}
    F'_1 \geq \sum\limits_{i \in S_1}x''_i + \sum\limits_{i \in [n] \setminus S_1}x''_i = \sum\limits_{i \in S_1}x'_{1, i} +
    \sum\limits_{i \in [n] \setminus S_1}x'_{2, i} > maxFlow\left(\delta_1\right)
    \enspace,
  \end{gathered}
  \end{equation*}
  which is a contradiction because $F'_1 = maxFlow\left(\delta_1\right)$ by the hypothesis. Thus
  $maxFlow\left(\delta_1\right) > maxFlow\left(\delta_2\right)$ and, since $\delta_1, \delta_2$ were chosen arbitrarily with
  the restriction $\delta_1 < \delta_2$, we deduce that the proposition holds.
\end{sepproof}
