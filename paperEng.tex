\documentclass[11pt]{article}
\usepackage[a4paper,top=2cm,bottom=2cm,left=1.5cm,right=1.5cm]{geometry}
\usepackage{graphicx}
\usepackage[linesnumbered,ruled,noend]{algorithm2e}
%\usepackage{algorithm}
\usepackage{lipsum}
%\usepackage{algpseudocode}
\usepackage[utf8]{inputenc}
\usepackage[english]{babel}
\usepackage[normalem]{ulem}
\usepackage{amsmath}
\usepackage{mathtools}
\usepackage{amsthm}
\usepackage{caption}
\usepackage{subcaption}
\usepackage{amssymb}
\usepackage{fancyvrb}
\usepackage{fancyhdr}
\usepackage{lastpage}
\usepackage{hyperref}
\usepackage{courier}
\usepackage{listings}

\newtheorem{theorem}{Theorem}[section]
\theoremstyle{definition}
\newtheorem{definition}{Definition}[section]
\theoremstyle{corollary}
\newtheorem{corollary}{Corollary}[section]
\theoremstyle{lemma}
\newtheorem{lemma}{Lemma}[section]

\lstset{frame=tb,
%  language=Java,
  aboveskip=3mm,
  belowskip=3mm,
  showstringspaces=false,
  columns=flexible,
  basicstyle={\small\ttfamily},
  numbers=none,
  numberstyle=\tiny\color{gray},
  keywordstyle=\color{blue},
  commentstyle=\color{dkgreen},
  stringstyle=\color{mauve},
  breaklines=true,
  breakatwhitespace=true,
  tabsize=3
}

% PDF bookmarks
\usepackage{color,hyperref}
\definecolor{darkblue}{rgb}{0.0,0.0,0.3}
\hypersetup{colorlinks,breaklinks,
    linkcolor=darkblue,urlcolor=darkblue,
    anchorcolor=darkblue,citecolor=darkblue}

\pagestyle{fancy}{ %
    \fancyhf{} % remove everything
        \renewcommand{\headrulewidth}{0pt} % remove lines as well
        \renewcommand{\footrulewidth}{0.5pt}
        \rhead{\leftmark}
        \lhead{Decentralized financial reputation with multisig for lines-of-credit}
        \rfoot{Page \thepage\ of \pageref{LastPage}}}

%\newlength\myindent
%\setlength\myindent{1em}
%\newcommand{%
%  \begingroup
%  \setlength{\itemindent}{\myindent}
%  \addtolength{\algorithmicindent}{\myindent}
%}
%\newcommand{\endgroup}
\makeatletter
\def\blfootnotemark{\xdef\@thefnmark{}\@footnotemark}
\makeatother
\makeatletter
\def\blfootnote{\xdef\@thefnmark{}\@footnotetext}
\makeatother
 
\begin{document}
  \begin{centering}
     \Large{\textbf{Trust Is Risk: Introducing a decentralized platform for financial trust}}
  \end{centering}
  \ \\ \ \\
  \hspace*{\fill}
  \begin{minipage}[t]{7cm}
     \begin{flushleft}
        Orfeas Stefanos Thyfronitis Litos \\
        \textit{National Technical University of Athens} \\
        \texttt{orfeas.litos@hotmail.com}
     \end{flushleft}
  \end{minipage}
  \hfill
  \begin{minipage}[t]{7cm}
     \begin{flushright}
        Dionysis Zindros\textsuperscript{\textdagger}\blfootnotemark \\
        \textit{University of Athens} \\
        \texttt{dionyziz@di.uoa.gr}
     \end{flushright}
  \end{minipage}
  \hspace*{\fill} \\ \ \\
  \blfootnote{\textsuperscript{\textdagger}Research supported by ERC project CODAMODA, project \#259152}
%  \textdagger Research supported by ERC project CODAMODA, project \#259152.

  \section{Abstract}
  Reputation in centralized systems typically uses stars and review-based
  trust. These systems require extensive manual intervention and secrecy to
  avoid manipulation. In decentralized systems this luxury is not available
  as the reputation system should be autonomous and open source. Previous
  peer-to-peer reputation systems define trust abstractly and do not allow for
  financial arguments pertaining to reputation. We propose a concrete
  sybil-resilient decentralized reputation system in which direct trust is
  defined as lines-of-credit using bitcoin's 1-of-2 multisig. We introduce a new
  model for bitcoin wallets in which user coins are split among trusted friends.
  Indirect trust is subsequently defined using a transitive property. This
  enables formal game theoretic arguments pertaining to risk analysis. Using our
  reputation model, we define financial risk and prove that risk and max flows
  are equivalent. We then propose several algorithms for the redistribution of
  trust so that a decision can be made on whether an anonymous third party can
  be indirectly trusted. In such a setting, the risk incurred by making a
  purchase from an anonymous vendor remains invariant. Finally, we prove the
  correctness of our algorithms and provide optimality arguments for various
  norms.

  \section{Introduction}

  \section{Keywords}
      decentralized, trust, web-of-trust, bitcoin, multisig, line-of-credit, trust-as-risk, flow  

  \section{Key points}

  \section{Definitions}
      \begin{definition}[Graph] \ \\
         TrustIsRisk is represented by a sequence of wheighted directed graphs $(\mathcal{G}_j)$ where $\mathcal{G}_j =
         (\mathcal{V}_j, \mathcal{E}_j), j \in \mathbb{N}$. Members of $\mathcal{E}_j$ are tuples of two nodes from
         $\mathcal{V}_j$. More formally, $e \in \mathcal{E}_j \Rightarrow \exists A,B \in \mathcal{V}_j : e = (A,B)$.
         Also, since the graphs are wheighted, there exists a sequence of functions $(c_j)$ with $c_j : \mathcal{E}_j
         \rightarrow \mathbb{R}^{+}$.
      \end{definition}
      \begin{definition}[Players] \ \\
         The set $\mathcal{V}_j = V(\mathcal{G}_j)$ is the set of all players in the network, otherwise understood as the
         set of all pseudonymous identities.
      \end{definition}
      \begin{definition}[Capital of $A$, $Cap_A$] \ \\
        Total amount of value that exists in P2PKH in the UTXO and can be spent by $A$. We also define $Cap_{A,j}$ as
        the total amount of value that exists in P2PKH in the UTXO and can be spent by $A$ during turn $j$.
      \end{definition}
      \begin{definition}[Direct Trust from $A$ to $B$ after turn $j$, $DTr_{A \rightarrow B, j}$] \ \\
         Total amount of value that exists in 1/$\{A,B\}$ multisigs in the UTXO in the end of turn $j$, where the money is
         deposited by $A$.
         $$DTr_{A \rightarrow B, j} =
            \begin{cases}
               c_j(A, B), & if (A, B) \in \mathcal{E}_j \\
               0, & if (A, B) \notin \mathcal{E}_j
            \end{cases}$$
         A function or algorithm that has access to the graph $\mathcal{G}_j$ has implicitly access to all direct trusts
         of this graph. The exception are the oracles, which in this case have access only to their incoming and outgoing
         direct trusts.
      \end{definition}
      \begin{definition}[(In/Out) Neighbourhood of $A$ on turn $j$, $N^{+}(A)_j, N^{-}(A)_j, N(A)_j$] \ 
         \begin{enumerate}
            \item Let $N^{+}(A)_j$ be the set of players $B$ that $A$ directly trusts with any positive value at the end of
               turn $j$. More formally, $N^{+}(A)_j = \{B \in \mathcal{V}_j : DTr_{A \rightarrow B, j} > 0\}$. $N^{+}(A)_j$
               is called out neighbourhood of $A$ on turn $j$. Let also $S \subset \mathcal{V}_j$. $N^{+}(S)_j =
               \bigcup\limits_{A \in S}N^{+}(A)_j$.
            \item Let $N^{-}(A)_j$ be the set of players $B$ that directly trust $A$ with any positive value at the end of
               turn $j$. More formally, $N^{-}(A)_j = \{B \in \mathcal{V}_j : DTr_{B \rightarrow A, j} > 0\}$. $N^{-}(A)_j$
               is called in neighbourhood of $A$ on turn $j$. Let also $S \subset \mathcal{V}_j$. $N^{-}(S)_j =
               \bigcup\limits_{A \in S}N^{-}(A)_j$.
            \item Let $N(A)_j$ be the set of players $B$ that either directly trust or are directly trusted by $A$ with
               any positive value at the end of turn $j$. More formally, $N(A)_j = N^{+}(A)_j \cup N^{-}(A)_j$. $N(A)_j$ is
               called neighbourhood of $A$ on turn $j$. Let also $S \subset \mathcal{V}_j$.
               $N(S)_j = N^{+}(S)_j \cup N^{-}(S)_j$.
            \item Let $N(A)_{j,i}$ (respectively $N^{+}(A)_{j,i}, N^{-}(A)_{j,i}, N(S)_{j,i}, N^{+}(S)_{j,i}, N^{-}(S)_{j,i},
               S \subset \mathcal{V}_j$) be the $i$-th element of set $N(A)_j$ (respectively of $N^{+}(A)_j, N^{-}(A)_j,
               N(S)_j, N^{+}(S)_j, N^{-}(S)_j$), according to an arbitrary but fixed enumeration of the set players.
         \end{enumerate}
      \end{definition}
      \begin{definition}[Total incoming/outgoing trust of $A$ in turn $j$, $in_{A, j}, out_{A, j}$] \ \\
         $$in_{A, j} = \sum\limits_{v \in N^{-}(A)_j}DTr_{v \rightarrow A, j}$$
         $$out_{A, j} = \sum\limits_{v \in N^{+}(A)_j}DTr_{A \rightarrow v, j}$$
      \end{definition}
%      \begin{definition}[$B$ steals $x$ from $A$] \ \\
%         $B$ steals value $x$ from $A$ when $B$ reduces the $DTr_{A \rightarrow B}$ by $x$ and increases $Cap_B$ by $x$.
%         This makes sense when $x \leq DTr_{A \rightarrow B}$.
%      \end{definition}
      \begin{definition}[Turns] \ \\
         The game we are describing is turn-based. Let $DTr_{B \rightarrow A, j}$ be $B$'s direct trust to $A$ in turn $j$.
         In each turn $j$ exactly one player $A \in \mathcal{V}, A = Player(j)$, chooses an action (according to a certain
         strategy) that can be one of the following, or a finite combination thereof:
         \begin{enumerate}
            \item Steal value $y_B, 0 \leq y_B \leq DTr_{B \rightarrow A, j-1}$ from $B \in N^{-}(A)$.
            $DTr_{B \rightarrow A, j} = DTr_{B \rightarrow A, j-1} - y_B$. ($Steal(y_B, B)$)
            \item Add value $y_B, -DTr_{A \rightarrow B, j-1} \leq y_B$ to $B \in \mathcal{V}$.
            $DTr_{A \rightarrow B, j} = DTr_{A \rightarrow B, j-1} + y_B$. When $y_B < 0$, we say that $A$ reduces her trust
            to $B$ by $-y_B$, when $y_B > 0$, we say that $A$ increases her trust to $B$ by $y_B$.
            If $DTr_{A \rightarrow B, j-1} = 0$, then we say that $A$ starts directly trusting $B$. ($Add(y_B, B)$)
         \end{enumerate}
         If player $A$ chooses no action in her turn, we say that she passes her turn. Also, let $Y_{st}, Y_{add}$ be the
         total value to be stolen and added respectively by $A$ in her turn, $j$. For a turn to be feasible, it must hold
         that $Y_{add} - Y_{st} \leq Cap_{A, j-1}$. We set $Cap_{A, j} = Cap_{A, j-1} + Y_{st} - Y_{add}$. Moreover, player
         $A$ is not allowed to choose two actions of the same kind against the same player in the same turn. \\
         The set of actions a player makes in turn $j$ is $Turn_j$. Examples:
         \begin{itemize}
            \item $Turn_{j_1} = \emptyset$
            \item $Turn_{j_2} = \{Steal(y, B), Add(w, B)\}$ (given that $DTr_{B \rightarrow A, j_2 - 1} \leq y \wedge
            -DTr_{A \rightarrow B, j_2 - 1} \leq w \wedge y - w \leq Cap_{A, j_2-1}$, where $A = Player(j_2)$)
            \item $Turn_{j_3} = \{Steal(x, B), Add(y, C), Add(w, D)\}$ (given that $DTr_{B \rightarrow A, j_3 - 1} \leq x
            \wedge -DTr_{A \rightarrow C, j_3-1} \leq y \wedge -DTr_{A \rightarrow D, j_3 - 1} \leq w \wedge
            x - y - w \leq Cap_{A, j_3-1}$, where $A = Player(j_3)$)
            \item $Turn_{j_4} = \{Steal(x, B), Steal(y, B)\}$ is not a valid turn because it contains two $Steal()$ actions
            against the same player. If $x + y \leq DTr_{B \rightarrow A}$, the correct alternative would be $Turn_{j_4} =
            \{Steal(x+y, B)\}$, where $A = Player(j_4)$.
         \end{itemize}
      \end{definition}
      \begin{definition}[Previous/Next turn of a player] \ \\
         Let $j \in \mathbb{N}$ a turn with $Player(j) = A$. We define $prev(j), next(j)$ as the previous and next turn
         that $A$ is chosen to play respectively. If $j$ is the first turn that $A$ plays, $prev(j) = 0$. More formally,
         $$prev(j) = \begin{cases}
            \max{\{k \in \mathbb{N} : k < j \wedge Player(k) = A\}}, & \{k \in \mathbb{N} : k < j \wedge Player(k) = A\}
            \neq \emptyset \\
            0, & \{k \in \mathbb{N} : k < j \wedge Player(k) = A\} = \emptyset
         \end{cases}$$
         $$next(j) = \min{\{k \in \mathbb{N} : k > j \wedge Player(k) = A\}}$$
         $next(j)$ is always well defined with the assumption that eventually everybody plays.
      \end{definition}
      \begin{definition}[$A$ is stolen $x$] \ \\
         Let $j, j'$ be two consecutive turns of $A$ ($next(j) = j'$). We say that $A$ has been stolen a value $x$ between
         $j$ and $j'$ if $out_{A,j} - out_{A,j'} = x > 0$. If turns are not specified, we implicitly refer to the current
         and the previous turns.
      \end{definition}
      \begin{definition}[History] \ \\
         We define History, $\mathcal{H} = (\mathcal{H}_j)$, as the sequence of all the tuples containing the sets of actions
         and the corresponding player. $\mathcal{H}_j = (Player(j), Turn_j)$.
      \end{definition}
      \begin{definition}[Conservative strategy] \ \\
         A player $A$ is said to follow the conservative strategy in turn $j$ if for any value $x$ that has been stolen from
         her since the previous turn she played, she substitutes it in her turn by stealing from others that trust her value
         equal to $\min{(x,in_{A,j})}$ and she takes no other action.
         More formally, let $j' = prev(j), Damage_j = out_{A,j'} - out_{A,j-1}$. If $Strategy(A) = Conservative$, then
         $\forall j \in \mathbb{N}: Player(j) = A$ it is $$Turn_j =
         \begin{cases}
            \emptyset, & Damage_j \leq 0 \\
            \bigcup\limits_{i=1}^{k}\{Steal(y_i,v_i)\}, & Damage_j > 0, N^{-}(A)_j = \{v_1,...,v_k\}
         \end{cases}$$
         In the second case, it is $\sum\limits_{i=1}^{k}y_i = \min(in_{A,j-1}, Damage_j)$. \\
%         If $j$ is the first turn in which $A$ plays, $j'$ is not well defined. In this case, we choose $Turn_j = \emptyset$,
%         except if it is otherwise denoted in some special cases. \\
      \end{definition}
      As we can see, the definition covers a multitude of options for the conservative player, since in case $0 < Damage_j <
      in_{A,j-1}$ she can choose to distribute the $Steal(s)()$ in any way she chooses, as long as $\forall i, y_i \leq
      DTr_{N^{-}(A)_{j,i} \rightarrow A, j-1} \wedge \sum\limits_{i=1}^{|N^{-}(A)_j|}y_i = Damage_j$.
      The oracle remembers $PrevOutTrust = out_{A, j'}$ for  $j' = prev(j)$ and can observe all incoming and outgoing direct
      trusts of player $A$, $\forall v \in N^{-}(A)_{j-1}, DTr_{v \rightarrow A, j-1}, \forall v \in N^{+}(A)_{j-1},
      DTr_{A \rightarrow v, j-1}$. We note that $N(A)_{j-1} = N(A)_j$. \\
      \begin{algorithm}[H]
         \label{conservativeoracle}
         \SetKwInOut{Input}{Input}
         \SetKwInOut{Output}{Output}
         \SetKwFunction{SelectSteal}{SelectSteal}
         \Input{previous graph $\mathcal{G}_{j-1}$}
         \Output{$Turn_j$}
         \caption{Conservative Oracle}
         $\mathcal{O}_{cons}(\mathcal{G}_{j-1})$ : \\ {
            $NewOutTrust \gets \sum\limits_{v \in N^{+}(A)_{j-1}}DTr_{A \rightarrow v, j-1}$ \\
            $NewInTrust \gets \sum\limits_{v \in N^{-}(A)_{j-1}}DTr_{v \rightarrow A, j-1}$ \\
            $Damage \gets PrevOutTrust - NewOutTrust$ \\
            \If{$Damage > 0$}{
               \If{$Damage \geq NewInTrust$}{
                  $Turn_j \gets \emptyset$ \\
                  \For{$v \in N^{-}(A)_{j-1}$}{
                     $Turn_j \gets Turn_j \cup \{Steal(DTr_{v \rightarrow A, j-1}, v)\}$}}
               \Else{
                  $(y_1,...,y_{|N^{-}(A)_{j-1}|}) \gets$ \SelectSteal{$DTr_{N^{-}(A)_{j-1,1} \rightarrow A, j-1}$,$...$,
                    $DTr_{N^{-}(A)_{j-1,|N^{-}(A)_{j-1}|} \rightarrow A, j-1}$,$Damage$} \\
                  $Turn_j \gets \emptyset$ \\
                  \For{$i \gets 1$ to $|N^{-}(A)_{j-1}|$}{
                     $Turn_j \gets Turn_j \cup \{Steal(y_i, N^{-}(A)_{j-1,i})\}$}}}
            \Else{$Turn_j \gets \emptyset$}
            \Return{$Turn_j$}}
      \end{algorithm}
      \texttt{SelectSteal()} returns $y_i, i \in [|N^{-}(A)_j|] : \forall i, y_i \leq DTr_{N^{-}(A)_{j,i} \rightarrow A},
      \sum\limits_{i=1}^{|N^{-}(A)_j|}y_i = Damage$. 
      \begin{definition}[Idle strategy] \ \\
         A player $A$ is said to follow the idle strategy if she passes in her turn. More formally, if $Strategy(A) =
         Idle$, then $\forall j \in \mathbb{N} : Player(j) = A$ it is $Turn_j = \emptyset$.
      \end{definition}
      \begin{algorithm}[H]
         \label{idleoracle}
         \SetKwInOut{Input}{Input}
         \SetKwInOut{Output}{Output}
         \Input{previous graph $\mathcal{G}_{j-1}$}
         \Output{$Turn_j$}
         \caption{Idle Oracle}
         $\mathcal{O}_{idle}(\mathcal{G}_{j-1})$ : \\ {
            \Return{$\emptyset$}}
      \end{algorithm}
      \begin{definition}[Evil strategy] \ \\
         A player $A$ is said to follow the evil strategy if she steals value $y_B = DTr_{B \rightarrow A, j-1} \:
         \forall \: B \in N^{-}(A)_j$ (steals all incoming direct trust) and reduces her trust to $C$ by
         $DTr_{A \rightarrow C, j-1} \: \forall \: C \in N^{+}(A)_j$ (nullifies her outgoing direct trust) in her turn.
         More formally, if $Strategy(A) = Evil$, then $\forall j \in \mathbb{N} : Player(j) = A$ it is $Turn_j =
         \{Steal(y_1,N^{-}(A)_{j,1}),...,Steal(y_m,N^{-}(A)_{j,m}), Add(w_1,N^{+}(A)_{j,1}),...,Add(w_l,N^{+}(A)_{j,l})\}$ 
         where $m = |N^{-}(A)_j|, l = |N^{+}(A)_j|, \forall i \in [m], y_i = DTr_{N^{-}(A)_{j,i} \rightarrow A, j-1},
         \forall i \in [l], w_i = -DTr_{A \rightarrow N^{+}(A)_{j,i},j-1}$. We note again that $N(A)_{j-1} = N(A)_j$.
      \end{definition}
      \begin{algorithm}[H]
         \label{eviloracle}
         \SetKwInOut{Input}{Input}
         \SetKwInOut{Output}{Output}
         \Input{previous graph $\mathcal{G}_{j-1}$}
         \Output{$Turn_j$}
         \caption{Evil Oracle}
         $\mathcal{O}_{evil}(\mathcal{G}_{j-1})$ : \\ {
            $Turn_j \gets \emptyset$ \\
            \For{$v \in N^{-}(A)_{j-1}$}{
               $Turn_j \gets Turn_j \cup \{Steal(DTr_{v \rightarrow A, j-1}, v)\}$}
            \For{$w \in N^{+}(A)_{j-1}$}{
               $Turn_j \gets Turn_j \cup \{Add(-DTr_{A \rightarrow v, j-1}, w)\}$}
            \Return{$Turn_j$}}
      \end{algorithm}
      \begin{definition}[Indirect trust from $A \in \mathcal{V}_j$ to $B \in \mathcal{V}_j$, $Tr_{A \rightarrow B, j}$] \ \\
         Maximum possible value that can be stolen from $A$ if $B$ follows the evil strategy, $A$ follows the idle strategy
         and everyone else ($\mathcal{V} \setminus \{A,B\}$) follows the conservative strategy. More formally,
         $$Tr_{A \rightarrow B, j} = \max\limits_{j' : j' > j, configurations}{[out_{A,j} - out_{A,j'}]}$$ where
         $Strategy(A) = Idle, Strategy(B) = Evil, \forall C \in \mathcal{V} \setminus \{A,B\}, Strategy(C) = Conservative$.
      \end{definition}
      \begin{definition}[Indirect trust from $A \in \mathcal{V}_j$ to $S \subset \mathcal{V}_j$, $Tr_{A \rightarrow S, j}$]
         \ \\Maximum possible value that can be stolen from $A$ if all players in $S$ follow the evil strategy, $A$ follows
         the idle strategy and everyone else ($\mathcal{V} \setminus (S \cup \{A\})$) follows the conservative strategy. More
         formally, $$Tr_{A \rightarrow S, j} = \max\limits_{j' : j' > j, configurations}{[out_{A,j} - out_{A,j'}]}$$ where
         $Strategy(A) = Idle, \forall E \in S, Strategy(E) = Evil,
         \forall C \in \mathcal{V} \setminus \{A,E\}, Strategy(C) = Conservative$.
      \end{definition}
      \begin{definition}[Trust Reduction] \ \\
         Let $A, B \in \mathcal{V}, x_i$ flow to $N^{+}(A)_i$ resulting from $maxFlow(A,B), u_i =
         DTr_{A \rightarrow N^{+}(A)_i,j-1}, u_i' = DTr_{A \rightarrow N^{+}(A)_i,j},$ \\ $i \in [|N^{+}(A)|],
         j \in \mathbb{N}$.
         \begin{enumerate}
            \item The Trust Reduction on neighbour $i, \delta_i$ is defined as $\delta_i = u_i - u_i'$.
            \item The Flow Reduction on neigbour $i, \Delta_i$ is defined as $\Delta_i = x_i - u_i'$.
         \end{enumerate}
         We will also use the standard notation for 1-norm and $\infty$-norm, that is:
         \begin{enumerate}
            \item $||\delta_i||_1 = \sum\limits_{i \in N^{+}(A)}\delta_i$
            \item $||\delta_i||_\infty = \max\limits_{i \in N^{+}(A)}\delta_i$.
         \end{enumerate}
      \end{definition}
      \begin{definition}[Restricted Flow] \ \\
         Let $A, B \in \mathcal{V}, i \in [|N^{+}(A)|]$.
         \begin{enumerate}
            \item Let $F_{A_i \rightarrow B}$ be the flow from $A$ to $N^{+}(A)_i$ as calculated by the $maxFlow(A,B)$
               ($x_i'$) when $u_i' = u_i,$ \\ $u_k' = 0 \:\forall k \in [|N^{+}(A)|] \wedge k \neq i$.
            \item Let $S \subset N^{+}(A)$. Let $F_{A_S \rightarrow B}$ be the sum of flows from $A$ to $S$ as
               calculated by the $maxFlow(A,B)$ ($\sum\limits_{i=1}^{|S|}x_i'$) when $u_C' = u_C \: \forall C \in S,
               u_D' = 0 \: \forall D \in N^{+}(A) \setminus S$.
         \end{enumerate}
      \end{definition}
      \begin{definition}[Collusion] \ \\
         A collusion of players $S \subset \mathcal{V}$ is a set of players that is entirely controlled by a single entity in
         the physical world. From a game theoretic point of view, other players ($v \in \mathcal{V} \setminus S$) perceive
         the collusion as independent players with a distinct strategy each, whereas in reality they are all subject to a
         single strategy dictated by the controlling entity.
      \end{definition}
  \section{Theorems-Algorithms}
    The following algorithm has read access to direct trusts in $\mathcal{G}_{j-1}$ and write access to direct trusts in
    $\mathcal{G}_j$.
    \begin{algorithm}[H]
       \label{executeturn}
       \SetKwInOut{Input}{Input}
       \SetKwInOut{Output}{Output}
       \SetKwFunction{validateTurn}{validateTurn}
       \SetKwFunction{commitTurn}{commitTurn}
       \SetKwFunction{executeTurn}{executeTurn}
       \Input{player $A$, old graph $\mathcal{G}_{j-1}$, old capital $Cap_{A, j-1}$, $ProvisionalTurn$}
       \Output{new graph $\mathcal{G}_j$, new capital $Cap_{A, j}$, new history $\mathcal{H}_j$}
       \caption{Execute Turn}
       \executeTurn{$A$, $\mathcal{G}_{j-1}$, $Cap_{A, j-1}$, $ProvisionalTurn$} : \\ {
          $(Turn_j, NewCap) \gets$ \validateTurn{$A$, $\mathcal{G}_{j-1}$, $Cap_{A, j-1}$, $ProvisionalTurn$} \\
          \Return{\commitTurn{$A$, $mathcal{G}_{j-1}$, $NewCap$, $Turn_j$}}} 
    \end{algorithm}
    \begin{algorithm}[H]
       \label{validateturn}
       \SetKwInOut{Input}{Input}
       \SetKwInOut{Output}{Output}
       \SetKwFunction{validateTurn}{validateTurn}
       \Input{player $A$, old graph $\mathcal{G}_{j-1}$, old capital $Cap_{A, j-1}$, $ProvisionalTurn$}
       \Output{$Turn_j$, new capital $Cap_{A, j}$}
       \caption{Validate Turn}
       \validateTurn{$A$, $\mathcal{G}_{j-1}$, $Cap_{A, j-1}$, $ProvisionalTurn$} : { \\
          $Y_{st} \gets 0$ \\
          $Y_{add} \gets 0$ \\
          \For{$action \in ProvisionalTurn$}{
             $action$ \textbf{match do} \\ {
                \textbf{case} $Steal(y,w)$ \textbf{do} \\ {
                   \If{$y > DTr_{w \rightarrow A,j-1} \vee y < 0$}{
                      \Return{$\emptyset$, $Cap_{A, j-1}$}}
                   \Else{$Y_{st} \gets Y_{st} + y$}}
                \textbf{case} $Add(y,w)$ \textbf{do} \\ {
                   \If{$y < -DTr_{A \rightarrow w,j-1}$}{
                      \Return{$\emptyset$, $Cap_{A, j-1}$}}
                   \Else{$Y_{add} \gets Y_{add} + y$}}}}
          \If{$Y_{add} - Y_{st} > Cap_{A, j-1}$}{
             \Return{$\emptyset$, $Cap_{A, j-1}$}}
          \Else{\Return{$ProvisionalTurn$, $Cap_{A, j-1} + Y_{st} - Y_{add}$}}}
    \end{algorithm}
    \begin{algorithm}[H]
       \label{committurn}
       \SetKwInOut{Input}{Input}
       \SetKwInOut{Output}{Output}
       \SetKwFunction{commitTurn}{commitTurn}
       \Input{player $A$, old graph $\mathcal{G}_{j-1}$, old capital $Cap_{A, j-1}$, $ProvisionalTurn$}
       \Output{new graph $\mathcal{G}_j$, new capital $Cap_{A, j}$, new history $\mathcal{H}_j$}
       \caption{Commit Turn}
       \commitTurn{$A$, $\mathcal{G}_{j-1}$, $Cap_{A, j-1}$, $Turn_j$} : { \\
          \For{$(v, w) \in \mathcal{E}_j$}{
                $DTr_{v \rightarrow w, j} \gets DTr_{v \rightarrow w, j-1}$}
          \For{$action \in Turn_j$}{
             $action$ \textbf{match do} \\ {
               \textbf{case} $Steal(y,w)$ \textbf{do} \\ {
                  $DTr_{w \rightarrow A, j} \gets DTr_{w \rightarrow A, j-1} - y$} \\
               \textbf{case} $Add(y,w)$ \textbf{do} \\ {
                  $DTr_{A \rightarrow w, j} \gets DTr_{A \rightarrow w, j} + y$}}}
          $Cap_{A, j} \gets NewCap$ \\
          $\mathcal{H}_j \gets (Player(j), Turn_j)$} \\
          \Return{$\mathcal{G}_j$, $Cap_{A, j}$, $\mathcal{H}_j$}
    \end{algorithm}
    \begin{algorithm}[H]
       \label{trustisriskgame}
%       \SetKwInOut{Input}{Input}
%       \SetKwInOut{Output}{Output}
       \SetKwFunction{executeTurn}{executeTurn}
       \caption{TrustIsRisk Game}
       $j \gets 0$ \\
       \While{True}{
          $j \gets j + 1$ \\
          $v \overset{\$}{\gets} \mathcal{V}_j$ \\
          $ProvisionalTurn \gets \mathcal{O}_v(\mathcal{G}_{j-1})$ \\
          $(G_j, Cap_{v, j}, H_j) \gets$ \executeTurn{$v$, $\mathcal{G}_{j-1}$, $Cap_{v, j-1}$, $ProvisionalTurn$}}
    \end{algorithm}

    On turn $0$, there is already a network in place. \\
    \begin{algorithm}[H]
       \label{transitivesteal}
       \SetKwInOut{Input}{Input}
       \SetKwInOut{Output}{Output}
       \SetKwFunction{executeTurn}{executeTurn}
       \Input{$A$ idle player, $E$ evil player}
       \Output{$\mathcal{H}$ history}
       \caption{Transitive Steal}
       $Angry \gets \emptyset$ \\
       $Happy \gets \emptyset$ \\
       $Sad \gets \emptyset$ \\
       \For{$v \in \mathcal{V}_{0} \setminus \{E\}$}{
          $Loss_v \gets 0$ \\
          \If{$v \neq A$}{
             $Happy \gets Happy \cup \{v\}$}}
       $j \gets 0$ \\
       \While{True}
          {$j \gets j + 1$ \\
           $v \overset{\$}{\gets} \mathcal{V}_j \setminus\{A\}$ \\
           $Turn_j \gets \mathcal{O}_v(\mathcal{G}_{j-1})$ \\ %\Comment{$\mathcal{O}_v = \mathcal{O}_{idle}$} \\
           \executeTurn{$\mathcal{G}_{j-1}$, $Cap_{v, j-1}$, $Turn_j$} \\
           \For{$action \in Turn_j$}{
              $action$ \textbf{match do}{ \\
                 \textbf{case} $Steal(y, w)$ \textbf{do}{ \\
                    $exchange \gets y$ \\
                    $Loss_w \gets Loss_w + exchange$ \\
                    \If{$v \neq E$}{
                       $Loss_v \gets Loss_v - exchange$}
                    \If{$w \neq A$}{
                       $Happy \gets Happy \setminus \{w\}$ \\
                       \If{$in_{w, j} = 0$}
                          {$Sad \gets Sad \cup \{w\}$}
                       \Else{$Angry \gets Angry \cup \{w\}$}}}}}
           $Angry \gets Angry \setminus \{v\}$ \\
           \If{$in_{v, j} = 0 \wedge Loss_v > 0$}{
               $Sad \gets Sad \cup \{v\}$}
           \If{$Loss_v = 0$}{
               $Happy \gets Happy \cup \{v\}$}}
    \end{algorithm}
    Let $j_0$ be the first turn on which $E$ is chosen to play. Until then, according to theorem \ref{conservativeworld},
    all players will pass their turn.
    Given that $Damage_{v,j} = out_{v,j'} - out_{v,j}$ where $j' = prev(j)$, the algorithm generates turns:
    $$Turn_j =
      \begin{cases}
         \emptyset, & Damage_{v,j-1} = 0 \\
         \bigcup\limits_{i=1}^{k}\{Steal(y_i,v_i)\}, & Damage_j > 0, N^{-}(A)_j = \{v_1,...,v_k\}
      \end{cases}$$
    In the second case, it is $\sum\limits_{i=1}^{k}y_i = \min(in_{v, j-1}, Damage_{v, j-1})$. From the
    definition of $Damage_{v,j}$ and knowing that no strategy in this case can increase any direct trust, it is obvious
    that $Damage_{v,j} \geq 0$. Also, we can see that $Loss_{v,j} \geq 0$
    because if $Loss_{v,j} < 0$, then $v$ has stolen more value than she has been stolen, thus she would not be following the
    conservative strategy.
    \begin{lemma}[$Loss$ equivalent to $Damage$] \ \\
       Let $j \in \mathbb{N}, v \in \mathcal{V}_j \setminus \{A, E\}, v = Player(j)$. Then $\min(in_{v, j}, Loss_{v, j}) = 
       \min(in_{v, j}, Damage_{v, j})$.
    \end{lemma}
    \begin{proof} \ 
       $j \in \mathbb{N}: v = Player(j)$.
       \begin{itemize}
          \item $v \in Happy_{j-1}$. Then
          \begin{enumerate}
             \item $v \in Happy_j$ because $Turn_{j} = \emptyset$,
             \item $Loss_{v, j} = 0$ because otherwise $v \notin Happy_j$,
             \item $Damage_{v, j} = 0$, or else any reduction in direct trust to $v$ would increase equally
             $Loss_{v, j}$ (line 18), which cannot be decreased again but during an Angry player's turn (line 20).
             \item $in_{v, j} \geq 0$
          \end{enumerate}
          Thus $\min(in_{v, j}, Damage_{v,j}) = \min(in_{v, j}, Loss_{v,j}) = 0$.
          \item $v \in Sad_{j-1}$. Then
          \begin{enumerate}
             \item $v \in Sad_j$ because $Turn_{j} = \emptyset$, 
             \item $in_{v, j} = 0$ (lines 28-29),
             \item $Damage_{v, j} \geq 0 \wedge Loss_{v, j} \geq 0$.
          \end{enumerate}
          Thus $\min(in_{v, j}, Damage_{v,j}) = \min(in_{v, j}, Loss_{v,j}) = 0$.
          \item $v \in Angry_{j-1} \wedge v \in Happy_j$. Then the same argument as in the first case holds, if
          we ignore the argument (1).
          \item $v \in Angry_j \wedge v \in Sad_j$. Then the same argument as in the second case holds, if 
          we ignore the argument (1).
       \end{itemize}
    \end{proof}

    \begin{theorem}[Trust convergence theorem] \ \\
    \label{convergence}
       Let $A,E \in \mathcal{V} : Strategy(A) = Idle$, $Strategy(E) = Evil$, $\forall C \in \mathcal{V} \setminus \{A,E\},
       Strategy(C) = Conservative$ and $j_0 \in \mathbb{N} : Player(j_0) = E$. Given that all players eventually play, there
       exists a turn $j' > j_0 : \forall j \geq j', Turn_j = \emptyset$.
    \end{theorem}
    \begin{proof}
       First of all, $\forall j > j_0 : Player(j) = E \Rightarrow Turn_j = \emptyset$ because $E$ has already nullified his
       incoming and outgoing direct trusts in $Turn_{j_0}$, the evil strategy does not contain any case where direct trust is
       increased or where the evil player starts directly trusting another player and the other players do not follow a
       strategy in which they can choose to $Add()$ trust to $E$, thus player $E$ can do nothing $\forall j > j_0$. Also
       $\forall j > j_0 : Player(j) = A \Rightarrow Turn_j = \emptyset$ because of the idle strategy that $A$ follows. As far
       as the rest of the players are concerned, consider the algorithm \ref{transitivesteal}, which is a variation of the
       TrustIsRisk Game. \\
       As we can see from lines 5 and 18-20, $\forall j, \sum\limits_{v \in \mathcal{V}_j}Loss_v = in_{E, j_0-1}$, that is
       the total loss is constant and equal to the total value stolen by $E$. Also, as we can see in lines 3 and 29, which
       are the only lines where the $Sad$ set is modified, once a player enters the $Sad$ set, it is impossible to exit from
       this set. Also, we can see that players in $Sad \cup Happy$ always pass their turn. We will now show that eventually
       the $Angry$ set will be empty, or equivalently that eventually every player will pass their turn. Suppose that it is
       possible to have an infinite amount of turns that players do not choose to pass. We know that the number of nodes is
       finite, thus this is possible only if $\exists j_1: \forall j \geq j_1, |Angry_j \cup Happy_j| = c > 0 \wedge Angry_j
       \neq \emptyset$ (the total number of angry and happy players cannot increase because no player leaves the $Sad$ set
       and if it were to be decreased, it would eventually reach 0). Since $Angry_j \neq \emptyset$, a player $v$ that will
       not pass her turn will eventually be chosen to play. According to algorithm \ref{transitivesteal}, $v$ will either
       deplete her incoming trust and enter the $Sad$ set (line 29), which is contradicting $|Angry_j \cup Happy_j| = c$,
       or will steal enough value to enter the $Happy$ set, that is $v$ will achieve $Loss_{v, j} = 0$. Suppose that she has
       stolen $m$ players. They, in their turn, will steal total value at least equal to the value stolen by $v$ (since they
       cannot go sad, as explained above). However, this means that, since the total value being stolen will never be reduced
       and the turns this will happen are infinite, the players must steal an infinite amount of value, which is impossible
       because the direct trusts are finite in number and in value. More precisely, let $\forall j \in \mathbb{N}, DTr_j =
       \sum\limits_{w,w' \in \mathcal{V}}DTr_{w \rightarrow w', j}$. Also, without loss of generality, suppose that $\forall
       j \geq j_1, out_{A, j} = out_{A, j_1}$. \\ In $Turn_{j_1}$, $v$ steals $St_{j_1} = \sum\limits_{i=1}^{m}y_i$. Thus
       $DTr_{j_1} = DTr_{j_1-1} - St_{j_1}$. Eventually there is a turn $j_2$ when every player in $N^{-}(v)$ will have
       played. Then $S_{j_2} \leq DTr_{j_1} - St_{j_1} = DTr_{j_1-1} - 2St_{j_1}$, since all players in $N^{-}(v)$ follow the
       conservative strategy, except maybe for $A$, who will not have been stolen anything due to the supposition. \\
       Suppose that $\exists k > 1 : j_k > j_{k-1} > j_1 \Rightarrow DTr_{j_k} \leq DTr_{j_{k-1}} - St_{j_1}$. Then there
       exists a subset of the $Angry$ players, $S$, that have been stolen at least value $St_{j_1}$ in total between the
       turns $j_{k-1}$ and $j_k$, thus there exists a turn $j_{k+1}$ such that all players in $S$ will have played and thus
       $DTr_{j_{k+1}} \leq DTr_{j_k} - St_{j_1}$. We have proven by induction that $\forall n \in \mathbb{N}, \exists j_n \in
       \mathbb{N} : DTr_{j_n} \leq DTr_{j_1-1} - nSt_{j_1}$. However $DTr_{j_1-1}, St_{j_1} \in \mathbb{N}$, thus $\exists n'
       \in \mathbb{N} : n'St_{j_1} > DTr_{j_1-1} \Rightarrow DTr_{j_n'} < 0$. We have a contradiction because $\forall w,w'
       \in \mathcal{V}, \forall j \in \mathbb{N}, DTr_{w \rightarrow w', j} \geq 0$, thus eventually $Angry = \emptyset$ and
       everybody passes.
    \end{proof}

    \begin{theorem}[Saturation theorem] \ \\ 
    \label{saturation}
       Let $s$ source, $n = |N^{+}(s)|, x_i, i \in [n]$, flows to $s$'s neighbours as calculated by the
       maxFlow algorithm, $u_i'$ new direct trusts to the $n$ neighbours and $x_i'$ new flows to the neighbours
       as calculated by the maxFlow algorithm with the new direct trusts, $u_i'$. It holds that
       $\forall i \in [n], u_i' \leq x_i \Rightarrow x_i' = u_i'$.
    \end{theorem}
    \begin{proof} \ 
       $\forall i \in [n], x_i' > u_i'$ is impossible because a flow cannot be higher than its
       corresponding capacity. Thus $\forall i \in [n], x_i' \leq u_i'$. (1) \\
       In the initial configuration of $u_i$ and according to the flow problem setting, a combination of flows
       $y_i$ such that $\forall i \in [n], y_i = u_i'$ is a valid, albeit not necessarily maximum,
       configuration with a flow $\sum\limits_{i=1}^{n}y_i$. Suppose that $\exists k \in [n] : x_k'
       < u_k'$ as calculated by the maxFlow algorithm with the new direct trusts, $u_i'$. Then for the new
       maxFlow $F'$ it holds that $F' = \sum\limits_{i=1}^{n}x_i' < \sum\limits_{i=1}^{n}y_i$ since $x_k' < y_k$
       and (1) which is impossible because the configuration $\forall i \in [n], x_i' = y_i$ is valid since 
       $\forall i \in [n], y_i = u_i'$ and also has a higher flow, thus the maxFlow algorithm will
       prefer the configuration with the higher flow. Thus we deduce that $\forall i \in [n], x_i' = u_i'$.
    \end{proof}

    \begin{theorem}[Trust flow theorem - TOCHECK] \ \\
    \label{trustflow}
       $Tr_{A \rightarrow B} = MaxFlow_{A \rightarrow B}$ (Treating direct trusts as capacities)
    \end{theorem}
    \begin{proof} \ \\
       Suppose that the flow graph $FG$ is composed of $V(FG)$ nodes and $E(FG)$ edges. Each edge $e_{vw}$ has a
       corresponding capacity $u_{vw}$ which is constant and a corresponding flow $x_{vw}$ which can change depending on the
       flow assignment $X = [x_{vw}]_{V(FG) \times V(FG)}$ we choose. In flow context, for an assignment $X$ to be valid, two
       properties must hold:
       \begin{enumerate}
          \item $\forall e_{vw} \in E(FG), x_{vw} \leq u_{vw}$
          \item $\forall v \in V(FG) \setminus \{A,B\}, \sum\limits_{w \in N^{+}(v)}x_{wv} =
                \sum\limits_{w \in N^{-}(v)}x_{vw}$
       \end{enumerate} (p.709 Introduction to algorithms (CLRS), third edition)
       First we will show that the MaxFlow can be a result of a valid execution of \ref{transitivesteal} and afterwards we
       will show that each valid execution of algorithm \ref{transitivesteal} corresponds to a valid flow from $A$ to $B$.
       Thus we will have proven that $Tr_{A \rightarrow B} = MaxFlow_{A \rightarrow B}$.
       \begin{itemize}
          \item We will first show that there exists an execution of algorithm \ref{transitivesteal} such that $Loss_A =
          maxFlow_{A \rightarrow B}$. Let $X$ be the flows as returned by an execution of the $maxFlow_{A \rightarrow B}$
          algorithm on $\mathcal{G}_0$. It is known that all flows are DAGs [citation needed] and that all DAGs are a partial
          order of their nodes based on the partial ordering $x_{vw} > 0 \Rightarrow v < w$ [citation needed]. From this
          partial order, we can create a total order with an algorithm such as topoSort [citation needed]. The maximum
          element of the total order is a node that does not have any outgoing flow. It is obvious that removing any node
          from a DAG cannot create a cycle, thus the graph that remains after removing a node from a DAG is also a DAG, thus
          it has a total order as well, which can be chosen to be the same total order as before removing the node, except
          for the removed node. If the removed node was maximum or minimum, the new total order is obvious. \\
          Player $B$ is the maximum node in turn 0 because she is the sink of the MaxFlow algorithm, thus she is the first to
          be chosen to play and steals all her incoming and outgoing trust. $\forall v \in N^{-}(B)_0, x_{vB} \leq
          DTr_{v \rightarrow B, 0}$ and $\sum\limits_{v \in N^{-}(B)_0}x_{vB} = maxFlow_{A \rightarrow B}$. The graph
          $FG_1 = FG_0 \setminus \{B\}$ is also a DAG and corresponds to the previous total order if we remove
          the maximum element, $B$. \\
          Suppose that $\forall j \in [k], k > 0$, the player $v$ corresponding to the maximum element is chosen to play for
          the first time, that $\forall w \in N^{-}(v)_{j-1} (= N^{-}(v)_0), x_{wv} \leq y$ where $Steal(y,w) \in Turn_k
          \wedge \sum\limits_{w \in N^{-}(v)_0}x_{wv} = \sum\limits_{w \in N^{+}(v)_0}x_{vw}$. \\
          For $j = k+1$, $Player(k+1) = v'$ corresponds to the maximum element of the previous total order with the element
          $v$ removed and it is the first time player $v'$ plays, since $v > v'$ in all previous steps thus $v'$ was not
          maximum. It also holds that $\forall w \in N^{-}(v)_0, x_{wv'} \leq DTr_{w \rightarrow v', 0}$ since the
          $x_{wv'}$ are chosen by the maxFlow algorithm with corresponding capacities the direct trusts and, since
          $\sum\limits_{w \in N^{-}(v')_0}x_{wv'} = \sum\limits_{w \in N^{+}(v')_0}x_{v'w}$ and player $v'$ has already been
          stolen value equal to $\sum\limits_{w \in N^{+}(v')_0}x_{v'w}$ (since she has no outgoing flow in turn $j$), player
          $v'$ can choose to steal from each player $w \in N^{-}(v')_0$ value at least equal to $x_{wv'}$ without violating
          the conservative strategy. \\
          We have proven using induction that if the algorithm chooses only maximum nodes, after exactly $|V(FG)| - 1$
          turns (we do not count idle player $A$) every evil and conservative player will have stolen at least value equal
          to the flow passing through them and player $A$ will have been stolen value exactly equal to
          $maxFlow_{A \rightarrow B} \Rightarrow Loss_A = maxFlow_{A \rightarrow B}$.
          \item We will now show that for any valid execution of algorithm \ref{transitivesteal} there exists at least one
          valid flow from $A$ to $B$, $X$, such that $Loss_A = \sum\limits_{v \in N^{+}(A)}x_{Av}$. Let $j$ be a turn where
          \ref{transitivesteal} has converged ($j$ exists, according to theorem \ref{convergence}). Then $Loss_{A, j} =
          out_{A, 0} - out_{A, j}$. We create a new graph $FG'$ such that $V(FG') = V(FG), E(FG') = E(FG), \forall (v, w) 
          \in E(FG'), c'(v,w) = DTr_{v \rightarrow w, 0} - DTr_{v \rightarrow w, j}$. We execute the $MaxFlow_{A \rightarrow
          B}$ algorithm on $FG'$ and we get a flow $X'$. We will show that flow $X=X'$ is also valid for the initial graph
          $FG$ and that $\sum\limits_{v \in N{+}(A)}x_{Av} = Loss_{A,j}$.
          \begin{itemize}
             \item The flow $X$ is obviously valid for the initial graph because $\forall (v,w) \in E(FG), c(v,w) = DTr_{v
             \rightarrow w, 0} \geq DTr_{v \rightarrow w, 0} - DTr_{v \rightarrow w, j} = c'(v,w) \geq x_{vw}$ and it
             already holds that $\forall v \in V(FG'), \sum\limits_{w \in N^{-}(v)}x'_{wv} = \sum\limits_{w \in N^{+}(v)}
             x'_{vw}$, thus it also holds for the flows of $X$.
             \item We can easily see that $Loss_{A,j} \geq \sum\limits_{v \in N{+}(A)}x_{Av}$ because $Loss_{A,j} =
             out_{A,0} - out_{A,j} = \sum\limits_{v \in N^{+}(A)}c'(A,v)$. To show that $Loss_{A,j} \leq
             \sum\limits_{v \in N{+}(A)}x_{Av}$, we first suppose that $Loss_{A,j} > \sum\limits_{v \in N{+}(A)}x_{Av}$. We
             will now prove that there exists a residual path from $A$ to $B$. $Loss_{A,j} = \sum\limits_{v \in N^{+}(A)}
             (DTr_{A \rightarrow v, 0} - DTr_{A \rightarrow v, j}) = \sum\limits_{v \in N^{+}(A)}c_{Av}$. From the
             supposition we can see that $\sum\limits_{v \in N^{+}(A)}c_{Av} > \sum\limits_{v \in N^{+}(A)}x_{Av}
             \Rightarrow \exists v \in N^{+}(A) : c_{Av} > x_{Av}$. \\
             Since $\forall v \in V(FG) \setminus \{A,B\}, \sum\limits_{w \in N^{-}(v)}c_{wv} \overset{conservative}{=}
             \sum\limits_{w \in N^{+}(v)}c_{vw} \wedge \sum\limits_{w \in N^{-}(v)}x_{wv} \overset{flow}{=}
             \sum\limits_{w \in N^{+}(v)}x_{vw}$, it holds that $\forall v \in V(FG) \setminus \{A,B\}, \sum\limits_{w \in
             N^{-}(v)}(c_{wv} - x_{wv}) = \sum\limits_{w \in N^{+}(v)}(c_{vw} - x_{vw})$. \\
             We will now show that $\forall v \in V(FG) \setminus
             \{A,B\}, (\exists w \in V(FG) : c_{wv} > x_{wv} \Rightarrow \exists u \in V(FG) : c_{vu} > x_{vu})$. Suppose
             that the previous statement is false. Then it would hold that $\exists v \in V(FG) \setminus \{A,B\} :
             (\exists w \in V(FG) : c_{wv} > x_{wv} \wedge \forall u \in V(FG), c_{vu} = x_{vu})$ (1). But then we have
             $\sum\limits_{w \in V(FG)}c_{wv} \overset{(1)}{>} \sum\limits_{w \in V(FG)}x_{wv} \overset{flow}{=}
             \sum\limits_{w \in V(FG)}x_{vw} \overset{(1)}{=} \sum\limits_{w \in V(FG)}c_{vw} \overset{conservative}{=}
             \sum\limits_{w \in V(FG)}c_{wv} \Rightarrow \sum\limits_{w \in V(FG)}c_{wv} > \sum\limits_{w \in V(FG)}c_{wv}$
             which is a contradiction. Thus we showed that $\forall v \in V(FG) \setminus \{A,B\}, (\exists w \in V(FG) :
             c_{wv} > x_{wv} \Rightarrow \exists u \in V(FG) : c_{vu} > x_{vu})$. \\
%             The flow graph that resulted from
%             $MaxFlow_{A \rightarrow B}$ is a DAG, thus there exists a corresponding total ordering, as we saw before.
%             Obviously $A = v_0$ and $B = v_{|V(FG)|}$. When an element $v_k$ is in the $k$-th position in this total
%             ordering, it has incoming flow only from smaller elements and outgoing flow only to bigger elements, that is
%             $\forall l < k, x_{kl} = 0 \wedge \forall m > k, x_{mk} = 0$.
%             Thus the previous result can be rewritten this way: $\forall k \in [|V(FG)|],
%             (\exists l < k : c_{v_lv_k} > x_{v_lv_k} \Rightarrow \exists m > k : c_{v_kv_m} > x_{v_kv_m})$.
             Thus, the supposition $Loss_{A, j} > \sum\limits_{v \in N^{+}(A)}x_{Av}$ combined with the previous result shows
             that there exists a residual path from $A$ to $B$ since we can start from $A$ and find a series of sequential
             edges that all have flows smaller than the corresponding capacities and eventually reach $B$ in at most
             $|E(FG)|$ steps, thus $X'$ is not a maximum flow, which is a contradiction. Thus $Loss_{A,j} \leq
             \sum\limits_{v \in N{+}(A)}x_{Av}$ and, since also $Loss_{A,j} \geq \sum\limits_{v \in N{+}(A)}x_{Av}$, we
             deduce that $Loss_{A,j} = \sum\limits_{v \in N{+}(A)}x_{Av}$.
          \end{itemize}
       \end{itemize}
       We finally conclude that $Tr_{A \rightarrow B} = MaxFlow_{A \rightarrow B}$.
    \end{proof}
%2nd bullet
%          \item We will now show that for any valid execution of algorithm \ref{transitivesteal} there exists at least one
%          valid flow from $A$ to $B$, $X$, such that $Loss_A = \sum\limits_{v \in N^{+}(A)}x_{Av}$. Let $j$ be a turn where
%          \ref{transitivesteal} has converged ($j$ exists, according to theorem \ref{convergence}). Then $Loss_{A, j} =
%          out_{A, 0} - out_{A, j}$. Let $\forall v \in N^{+}(A)_0, x_{Av} = DTr_{A \rightarrow v, 0} - DTr_{A \rightarrow
%          v, j}$. For any conservative player $v \in N^{+}(A)_0$, let $\forall w \in N^{+}(v)_0, x_{vw} \leq DTr_{v
%          \rightarrow w, 0} - DTr_{v \rightarrow w, j}, \sum\limits_{w \in N^{+}(A)_0}x_{vw} = x_{Av}$. This is possible
%          because $v$ is conservative, thus the value she stole from $A$ must have been stolen previously from her. More
%          generally, $\forall v \in \mathcal{V}_0 \setminus \{A,B\}, \forall w \in N^{+}(v)_0, x_{vw} \leq
%          DTr_{v \rightarrow w, 0} - DTr_{v \rightarrow w, j}, \sum\limits_{w \in N^{+}(v)_0}x_{vw} = \sum\limits_{w \in
%          N^{-}(v)_0}x_{wv}$. Since the graph we build is a DAG in every step, which corresponds to a partial order, there
%          always exists a total order that we can get using an algorithm such as topoSort [citation needed]. Thus, by
%          choosing to calculate the outgoing flows only of the minimum element of this total order, it is possible to create
%          a valid flow network from $A$ to $B$ in exactly $|V(FG)| - 1$ iterations of the above steps.
%OLD
%          \item The flow to $A$ is the flow that results from the following process: After the execution of
%          \ref{transitivesteal}, for each sad player iteratively replenish the $DTr$ stolen from the sad player by the one
%          that stole from her (if multiple players stole from the sad player, then replenish all the stolen $DTr$). Repeat
%          the process until the evil player replenishes the initially stolen $DTr$. This is always possible because if there
%          is no player who stole from each one who is replenished, then the $Steal()$ she did in the first place would not be
%          according to the conservative strategy. Also this process will end with the evil player replenishing $DTr$ equal
%          to the sum of $DTr$ that was stolen from sad players because the conservative players cannot avoid replenishing,
%          or else they do not follow the conservative strategy. The $DTr$ stolen from $A$ will not be replenished, since
%          the player(s) that have stolen from $A$ will not replenish the stolen value and, inductively, this value will not
%          be replenished. Thus $A$ will have been stolen the exact same value that the modified evil player has stolen,
%          $\forall w,v \in V(FG), DTr_{v \rightarrow w} \geq x_{vw}$ (1st requirement for flows) and there would be no node
%          that gets more flow than it pushes, except for $A$ and $B$ (2nd requirement for flows), thus it is a valid flow.
%          \item Let $X$ be the flows as returned by an execution of the $maxFlow$ algorithm. The evil player can steal
%          the values denoted by $X$ and every other player can steal exactly as much as the $X$ flows denote, since they
%          have the 1st property and thus are stealable in any strategy and also hold the 2nd property, thus they comply with
%          the conservative strategy. More concretely, $\forall v,w \in V(FG), DTr_{v \rightarrow w}' = x_{vw}$. Then the two
%          properties of flows hold:
%          \begin{itemize}
%             \item $\forall v,w \in V(FG),x_{vw} \leq DTr_{v \rightarrow w}$ and thus any set of strategies that include only
%             $Steal()$ actions such that $\sum\limits_{y : Steal(y,w) \in Turn_j, Player(j) = v}y = DTr_{v \rightarrow w} -
%             x_{vw}$ is feasible.
%             \item $\forall v \in V(FG) \setminus \{A,B\}, \sum\limits_{w \in N^{+}(v)}x_{wv} =
%             \sum\limits_{w \in N^{-}(v)}x_{vw}$ thus $\forall v \in V(FG) \setminus \{A,B\}, Strategy(v) = Conservative$.
%          \end{itemize}
%             
%       \end{itemize}
%       Thus the maximum value $A$ can lose if $B$ is evil is $Tr_{A \rightarrow B} = maxFlow_{A \rightarrow B}$.
%\ \\ OLDER
%       \begin{enumerate}
%	   \item We will show that $Tr_{A \rightarrow B} \leq MaxFlow_{A \rightarrow B}$.
%          We know that $MaxFlow_{A \rightarrow B} = MinCut_{A \rightarrow B}$. We will show that, if everybody except
%          A and B follows the conservative strategy,  $Tr_{A \rightarrow B} \leq MinCut_{A \rightarrow B}$. Suppose that in
%          round $i$ all the members of the MinCut, $P$, have stolen the maximum value they can from members that belong
%          in the MaxFlow graph and nobody in the partition in which $A$ belongs has stolen yet any value. Let the total
%          stolen value from the MinCut members be $St$. It is obvious that $St_i \leq MinCut_{A \rightarrow B}$, because
%          otherwise there would exist $u \in P$ that doesn't follow the conservative strategy, since they stole more than they
%          were stolen from. The same argument holds for any round $i' > i$ because in each round an conservative player can
%          steal only up to the value she has been stolen. It is also impossible that the $St$ increase further due to
%          stolen value from members of the partition of $B$ since members of $P$ disconnect the two partitions and have
%          already played their turns, thus $\forall i' > i, St_{i'} \leq St_i$. There exists a round, $k$, when all the
%          conservative players stop stealing, so in the worst case $A$ will have been stolen
%          $Tr_{A \rightarrow B} = St_k \leq MinCut_{A \rightarrow B} = MaxFlow_{A \rightarrow B}$.
%          \item We can see that $Tr_{A \rightarrow B} \geq MaxFlow_{A \rightarrow B}$ because the strategy where each
%          one of the non-idle players steals value equal to the incoming flows from their respective friends is a valid
%          strategy that does not contradict with the conservative strategy, since for every conservative player $w$ it holds that
%          $\sum\limits_{v \in N^{-}(w)}x_{vw} = \sum\limits_{v \in N^{+}(w)}x_{wv}$ and according to the strategy each
%          conservative player will have been stolen value equal to $\sum\limits_{v \in N^{+}(w)}x_{wv}$. More concretely,
%          let $Player(j) = B$ and $Player(j+d) = C :$
%       \end{enumerate}
%       Combining the two results, we see that $Tr_{A \rightarrow B} = MaxFlow_{A \rightarrow B}$.
%        OLD PROOF START
%        \begin{enumerate}
%           \item $Tr_{A \rightarrow B} \geq MaxFlow_{A \rightarrow B}$ because by the definition of $Tr_{A \rightarrow B}$,
%           B leaves taking with him all the incoming trust, so there is no trust flowing towards him after leaving.
%           $Tr_{A \rightarrow B} < MaxFlow_{A \rightarrow B}$ would imply that after B left, there would still remain trust
%           flowing from A to B.
%           \item $Tr_{A \rightarrow B} \leq MaxFlow_{A \rightarrow B}$ \\
%           Suppose that $Tr_{A \rightarrow B} > MaxFlow_{A \rightarrow B}$ (1). Then, using the min cut - max flow theorem we
%           see that there is a set of capacities $U= \{u_1,...,u_n\}$ with flows $X = \{x_1,...,x_n\}$ such that
%           $\sum\limits_{i=1}^{n}{x_i} = MaxFlow_{A \rightarrow B}$ and, if severed $(\forall i \in [n] \: u_i' = 0)$
%           the flow from A to B would be $0$, or, put differently, there would be no directed trust path from A to B. No
%           strategy followed by B could reduce the value of A, so our supposition (1) cannot be true.
%        \end{enumerate}
%        OLD PROOF END

    \begin{theorem}[Conservative world theorem] \ \\
    \label{conservativeworld}
       If everybody follows the conservative strategy, nobody steals any amount from anybody.
    \end{theorem}
    \begin{proof} \ \\
       Suppose that there exists a subseries of History, $(Turn_{j_k})$, where $Turn_{j_k} = \{Steal(y_1,B_1),...,
       Steal(y_m,B_m)\}$. This subseries must have an initial element, $Turn_{j_1}$. However, $Player(j_1)$ follows the conservative
       strategy, thus somebody must have stolen from her as well, so $Player(j_1)$ cannot be the initial element. We have a
       contradiction, thus there cannot exist a series of stealing actions when everybody is conservative.
    \end{proof}

    \begin{theorem}[Trust transfer theorem (flow terminology)] \ \\
    \label{trusttransfer}
       Let $s$ source, $t$ sink, $n = N^{+}(s)$ \\
       $X = \{x_1, ..., x_n\}$ outgoing flows from $s$, \\
       $U = \{u_1, ..., u_n\}$ outgoing capacities from $s$, \\
       $V$ the value to be transferred. \\
       Nodes apart from $s$, $t$ follow the conservative strategy. \\
       Obviously $maxFlow = F = \sum\limits_{i=1}^{n}{x_i}$.
       {\em \begin{lstlisting}
            /                      ....                     \
           / x_s1/u_s1                         x_1t/u_1t     \
          /                                                   \
         /                                                     \
        / x_s2/u_s2                               x_2t/u_2t     \
       s-------------              ....          ------------t
        \      .                                           .    /
         \     .                                           .   /
          \    .                                           .  /
           \ x_sn/u_sn             ....        x_mt/u_mt     /
            \                                               /
       \end{lstlisting}}
       We create a new graph where
       \begin{enumerate}
         \item  $\sum\limits_{i=1}^{n}{u_i'} = F - V$
         \item $\forall i \in [n] \: u_i' \leq x_i$
       \end{enumerate}
 
       It holds that $maxFlow' = F' = F - V$.
    \end{theorem}
    \begin{proof} \
        From theorem \ref{saturation} we can see that $x_i' = u_i'$. It holds that $F' = \sum\limits_{i=1}^nx_i' =
        \sum\limits_{i=1}^nu_i' = F - V$.
    \end{proof}

%    \begin{corollary}[Requirement for $\sum\limits_{i=1}^{n}{u_{s, i}'} = F - V$, $u_{s, i}' \leq x_{s, i}$] \ \\
%       In the setting of \ref{trusttransfer}, it is impossible to have $maxFlow' = F - V$ if
%       $\sum\limits_{i=1}^{n}{u_{s, i}'} > F - V \wedge \forall i \in [n],u_{s, i}' \leq x_{s, i}$.
%    \end{corollary}
%    \begin{proof}
%       Due to \ref{trusttransfer}, $maxFlow' = F - V$ if $\sum\limits_{i=1}^{n}{u_{s, i}'} = F - V
%       \wedge \forall i \in [n], u_{s, i}' \leq x_{s, i}$. If we create new capacities such that
%       $\forall i \in [n], u_{s,i}'' \leq x_{s,i}$, then obviously $maxFlow'' = \sum\limits_{i=1}^{n}{u_{s,i}''}$. If
%       additionally $\sum\limits_{i=1}^{n}{u_{s,i}''} > F - V$, then $maxFlow'' > F - V$.
%    \end{proof}

    \begin{lemma}[Flow limit lemma] \ \\
       \label{flowlimit}
       It is impossible for the outgoing flow $x_i$ from $A$ to an out neighbour of $A$ to be greater than
       $F_{A_i \rightarrow B}$. More formally, $x_i \leq F_{A_i \rightarrow B}$.
    \end{lemma}
    \begin{proof}
       Suppose a configuration where $\exists i : x_i > F_{A_i \rightarrow B}$. If we reduce the capacities $u_k, k \neq i$
       the flow that passes from $i$ in no case has to be reduced. Thus we can set $\forall k \neq i, u_k' = 0$ and $u_i' =
       u_i$. Then $\forall k \neq i,x_k' = 0, x_i' = x_i$ is a valid configuration and thus by definition $F_{A_i \rightarrow
       B} = x_i' = x_i > F_{A_i \rightarrow B}$, which is a contradiction. Thus $\forall i \in [|N^{+}(A)|], x_i \leq
       F_{A_i \rightarrow B}$.
    \end{proof}

    \begin{theorem}[Trust-saving Theorem] \ \\
    \label{trustsave}
       A configuration $U' : u_i' = F_{A_i \rightarrow B}$ for some $i \in [|N^{+}(A)|]$ can yield the same $maxFlow$ with a
       configuration $U'' : u_i'' = u_i, \forall k \in [|N^{+}(A)|], k \neq i, u_k'' = u_k'$.
    \end{theorem}
    \begin{proof}
       We know that $x_i \leq F_{A_i \rightarrow B}$ (lemma \ref{flowlimit}), thus we can see that any increase in $u_i'$
       beyond $F_{A_i \rightarrow B}$ will not influence $x_i$ and subsequently will not incur any change on the rest of the
       flows.
    \end{proof}

    \begin{theorem}[Invariable trust reduction with naive algorithms] \ \\
    \label{invariability}
       Let $A$ source, $n = |N^{+}(A)|$ and $u_i'$ new direct trusts. If $\forall i \in [n],u_i' \leq x_i$,
       Trust Reduction $||\delta_i||_1$ is independent of $x_i, u_i' \:\: \forall$ valid configurations of $x_i$
    \end{theorem}
    \begin{proof} 
       Since $\forall i \in [n],u_i' \leq x_i$ it is (according to \ref{saturation}) $x_i' = u_i'$, thus
       $\delta_i = u_i - x_i'$. We know that $\sum\limits_{i=1}^{n}x_i' = F - V$, so we have $||\delta_i||_1 =
       \sum\limits_{i=1}^{n}\delta_i = \sum\limits_{i=1}^{n}(u_i - x_i') = \sum\limits_{i=1}^{n}u_i - F + V$ independent
       from $x_i', u_i'$
    \end{proof}

    \begin{theorem}[Dependence impossibility theorem] \ \\
    \label{independence}
       ${\partial x_k \over \partial x_i} = 0$ with $x_i$ the flow from MaxFlow $\Rightarrow
         \forall x_i' \leq x_i, {\partial x_k \over \partial x_i} = 0$ ceteris paribus
    \end{theorem}
    \begin{proof}
       TODO
    \end{proof}
    Note: The maxFlow is the same in the following two cases: When a player chooses the evil strategy and when the same
    player chooses a variation of the evil strategy where she does not nullify her outgoing direct trust.

    \begin{theorem}[Trust to multiple players] \ \\
    \label{trustmany}
       Let $S \subset \mathcal{V}, T$ auxiliary player such that $\forall B \in S, DTr_{B \rightarrow T} = \infty$.
       It holds that $\forall A \in \mathcal{V} \setminus S, Tr_{A \rightarrow S} = maxFlow(A, T)$.
    \end{theorem}       
    \begin{proof}
       If $T$ chooses the evil strategy and all players in $S$ play according to the conservative strategy, they will have to steal
       all their incoming direct trust, thus they will act in a way identical to following the evil strategy as far as
       maxFlow is concerned, thus, by \ref{trustflow}, $Tr_{A \rightarrow T} = maxFlow(A, T) = Tr_{A \rightarrow S}$.
    \end{proof}

    One of the primary aims of this system is to mitigate the danger for sybil attacks whilst maintaining fully decentralized
    autonomy. Let Eve be a possible attacker. Since participation in the network does not require any kind of registration,
    Eve can create any number of players. We will call the set of these players $\mathcal{C}$, or Sybil set. Moreover, Eve
    can invest any amount she chooses, thus she can arbitrarily set the direct trusts of any player $C \in \mathcal{C}$ to
    any player $P \in \mathcal{V}$ ($DTr_{C \rightarrow P}$) and can also steal all incoming direct trust to these players.
    Additionally, we give Eve a set of players $B \in \mathcal{B}$ that she has corrupted (the corrupted set), so she fully
    controls their direct trusts to any player $P \in \mathcal{V}$ ($DTr_{B \rightarrow P}$) and can also steal all incoming
    direct trust to these players. The players $B \in \mathcal{B}$ are considered to be legitimate before the corruption,
    thus they can be directly trusted by any player $P \in \mathcal{V}$ ($DTr_{P \rightarrow B} \geq 0$). However, players
    $C \in \mathcal{C}$ can be trusted only by players $D \in \mathcal{B} \cup \mathcal{C}$ ($DTr_{D \rightarrow C} \geq 0$)
    and not by players $A \in \mathcal{V} \setminus (\mathcal{B} \cup \mathcal{C})$ ($DTr_{A \rightarrow C} = 0$).
    \begin{lemma}[$\forall S \subset \mathcal{V}, v \in S \Rightarrow \forall w \in \mathcal{V}, (v,w) \notin MinCut$] \ \\
    \label{mincutmany}
       Let $S \subset \mathcal{V}$. When calculating $maxFlow_{A \rightarrow S}$, it is impossible to have an edge $(v,w) \in
       MinCut : v \in S$.
    \end{lemma}
    \begin{proof}
       Let $T$ be the auxiliary node ($\forall v \in S, c_{vT} = \infty$). Since $out_A < \infty$, $maxFlow_{A \rightarrow S}
       < \infty$. All edges in the $MinCut$ are saturated, thus $\nexists v \in S : (v,T) \in MinCut$, or else
       $maxFlow_{A \rightarrow S} = \infty$. Suppose that $\exists v \in S, w \in \mathcal{V} : (v,w) \in MinCut$. Then this
       edge must be saturated, that is $x_{vw} = c_{vw} > 0$. However, there exists an alternative flow configuration $X'$
       where $\forall (u,u') \in \mathcal{E} \setminus \{(v,w), (v,T)\}, x_{u,u'}' = x_{u,u'}, x_{vw}' = 0, x_{vT}' = x_{vT}
       + x_{vw}$, which is valid because $\sum\limits_{w \in N^{+}(v)}x_{vw} = \sum\limits_{w \in N^{+}(v)}x_{vw}' \wedge
       c_{vT} = \infty \Rightarrow x_{vT}' \leq c_{vT}$ and $X'$ is maximum as well because it carries exactly the same flow
       as $X$. Thus $(v,w) \notin MinCut$.
    \end{proof}
    \begin{theorem}[Sybil resilience] \ \\
    \label{sybil}
       Let $\mathcal{B} \cup \mathcal{C} \subset \mathcal{V} (\mathcal{B} \cap \mathcal{C} = \emptyset)$ be a collusion of
       players who are controlled by an adversary, Eve. Eve also controls the number of players in the Sybil set
       $\mathcal{C}, |\mathcal{C}|$, but players $C \in \mathcal{C}$ are not directly trusted by players outside the
       collusion, contrary to players $B \in \mathcal{B}$, the corrupted set, who may be directly trusted by any player in
       $\mathcal{V}$. It holds that $Tr_{A \rightarrow \mathcal{B}} = Tr_{A \rightarrow \mathcal{B} \cup \mathcal{C}}$.
    \end{theorem}
    \begin{proof}
       Suppose that there exist $|\mathcal{B} \cup \mathcal{C}|$ consecutive turns during which all the colluding players
       choose actions according to the evil strategy. More formally, suppose that $\exists j: \forall d \in [|\mathcal{B}
       \cup \mathcal{C}|], Player(j+d) \in \mathcal{B} \cup \mathcal{C} \wedge \forall d_1, d_2 \in [|\mathcal{B} \cup
       \mathcal{C}|], d_1 \neq d_2, Player(j + d_1) \neq Player(j + d_2) \wedge \forall d \in [|\mathcal{B} \cup
       \mathcal{C}|], Strategy(Player(j+d)) = Evil$. Let $T$ be an auxiliary player such that $\forall B \in \mathcal{B},
       DTr_{B \rightarrow T} = \infty$ and $T'$ be another auxiliary player such that $\forall D \in \mathcal{B} \cup
       \mathcal{C}, DTr_{D \rightarrow T'} = \infty$. According to \ref{trustmany}, $Tr_{A \rightarrow \mathcal{B}} =
       maxFlow(A, T), Tr_{A \rightarrow \mathcal{B} \cup \mathcal{C}} = maxFlow(A, T')$. Consider the partition of
       $\mathcal{V}, \mathcal{P} = \{\mathcal{B} \cup \mathcal{C}, \mathcal{V} \setminus (\mathcal{B} \cup \mathcal{C})\} =
       \{P_1, P_2\}$. The edges from $P_2$ to $P_1$ will carry a flow $X_P, X_{P'}$ and the edges inside of $P_1$ will carry
       a flow $X_T, X_{T'}$ from the calculation of $maxFlow(A, T), maxFlow(A, T')$ respectively. $maxFlow(A, T) \leq
       maxFlow(A, T')$ because the maximal configuration of $X_T$ can be part of a valid configuration of $X_{T'}$ since
       edges in $\mathcal{B}$ are edges in $\mathcal{B} \cup \mathcal{C}$. Both maxFlows are not infinite, because $out_A <
       \infty$ and according to \ref{mincutmany} all the edges in both $MinCut$s have their starting point outside
       $\mathcal{B} \cup \mathcal{C}$. However, then in both cases the minCut is the same, thus $maxFlow(A, T) =
       maxFlow(A, T')$. MISSING Thus we conclude that $Tr_{A \rightarrow \mathcal{B}} =
       Tr_{A \rightarrow \mathcal{B} \cup \mathcal{C}}$.
    \end{proof}
    We have proven that controlling $|\mathcal{C}|$ is irrelevant for Eve, thus Sybil attacks are meaningless. \\

    Here we show three naive algorithms for calculating new direct trusts so as to maintain invariable risk when paying
    a trusted party. To prove the correctness of the algorithms, it suffices to prove that $\forall i \in [n] \:
    u_i' \leq x_i$ and that $\sum\limits_{i=1}^{n}u_i' = F - V$ where $F = \sum\limits_{i=1}^{n}x_i$. \\
    \begin{algorithm}[H]
       \label{fcfs}
       \SetKwInOut{Input}{Input}
       \SetKwInOut{Output}{Output}
       \Input{$x_i$ flows, $n = |N^{+}(s)|$, $V$ value}
       \Output{$u_i'$ capacities}
       \caption{First-come, first-served trust transfer}
       $F \gets \sum\limits_{i=1}^{n}x_i$ \\
       \If{$F < V$}{\Return $\bot$}
       $F_{cur} \gets F$ \\
       \For{$i \gets 1$ to $n$}
          {$u_i' \gets x_i$} 
       $i \gets 1$ \\
       \While{$F_{cur} > F - V$}
          {$reduce \gets \min{(x_i, F_{cur} - F + V)}$ \\
           $F_{cur} \gets F_{cur} - reduce$ \\
           $u_i' \gets x_i - reduce$ \\
           $i \gets i + 1$}
       \Return $U' = \bigcup\limits_{k=1}^{n}\{u_k'\}$
    \end{algorithm}
    \begin{proof}[Proof of correctness for algorithm \ref{fcfs}] \
       \begin{itemize}
          \item We will show that $\forall i \in [n] \: u_i' \leq x_i$. \\
          Let $i \in [n]$. In line 6 we can see that $u_i' = x_i$ and the only other occurence of $u_i'$
          is in line 11 where it is never increased $(reduce \geq 0)$, thus we see that, when returned, $u_i' \leq x_i$.
          \item We will show that $\sum\limits_{i=1}^{n}u_i' = F - V$. \\
          $F_{cur,0} = F$ \\
          If $F_{cur,i} \geq F - V$, then $F_{cur,i+1}$ does not exist because the \emph{while} loop breaks after calculating
          $F_{cur,i}$. \\
          Else $F_{cur,i+1} = F_{cur,i} - \min{(x_{i+1}, F_{cur,i} - F + V)}$. \\
          If for some $i, \min{(x_{i+1}, F_{cur,i} - F + V)} = F_{cur,i} - F + V$, then $F_{cur,i+1} = F - V$, so if
          $F_{cur,i+1}$ exists, then $\forall k < i, F_{cur,k} = F_{cur,k-1} - x_k \Rightarrow F_{cur,i} =
          F - \sum\limits_{k=1}^{i}x_k$ \\
          Furthermore, if $F_{cur,i+1} = F - V$ then $u_{i+1}' = x_{i+1} - F_{cur,i} + F - V =
          x_i - F + \sum\limits_{k=1}^{i-1}x_k + F - V = \sum\limits_{k=1}^{i}x_k - V$, $\forall k \leq i, u_k' = 0$
          and $\forall k > i+1, u_k' = x_k$. \\
          In total, we have $\sum\limits_{k=1}^{n}u_k' = \sum\limits_{k=1}^{i}x_k - V + \sum\limits_{k=i+1}^{n}x_k =
          \sum\limits_{k=1}^{n}x_k - V \Rightarrow \sum\limits_{k=1}^{n}u_k' = F - V$.
       \end{itemize}
    \end{proof}
    \begin{proof}[Complexity of algorithm \ref{fcfs}] \ \\
       First we will prove that on line 13 $i \leq n+1$. Suppose that $i > n+1$ on line 13. This means that $F_{cur,n}$
       exists and $F_{cur,n} = F - \sum\limits_{i=1}^{n}x_i = 0 \leq F - V$ since, according to the condition on line 2,
       $F - V \geq 0$. This means however that the \emph{while} loop on line 8 will break, thus $F_{cur,n+1}$ cannot exist
       and $i = n + 1$ on line 13, which is a contradiction, thus $i \leq n+1$ on line 13. Since $i$ is incremented by 1
       on every iteration of the \emph{while} loop (line 12), the complexity of the \emph{while} loop is $O(n)$ in the
       worst case. The complexity of lines 2-4 and 7 is $O(1)$ and the complexity of lines 1, 5-6 and 13 is $O(n)$, thus
       the total complexity of algorithm \ref{fcfs} is $O(n)$.
    \end{proof}

    \begin{algorithm}[H]
       \label{abs}
       \SetKwInOut{Input}{Input}
       \SetKwInOut{Output}{Output}
       \Input{$x_i$ flows, $n = |N^{+}(s)|$, $V$ value}
       \Output{$u_i'$ capacities}
       \caption{Absolute equality trust transfer ($||\Delta_i||_\infty$ minimizer)}
       $F \gets \sum\limits_{i=1}^{n}x_i$ \\
       \If{$F < V$}{\Return $\bot$}
       \For{$i \gets 1$ to $n$}
          {$u_i' \gets x_i$}
       $reduce \gets {V \over n}$ \\
       $reduction \gets 0$ \\
       $empty \gets 0$ \\
       $i \gets 0$ \\
       \While{$reduction < V$}
          {\If{$u_i' > 0$}{\If{$x_i < reduce$}
                {$empty \gets empty + 1$ \\
                 \If{$empty < n$}
                    {$reduce \gets reduce + \frac{reduce - x_i}{n - empty}$}
                 $reduction \gets reduction + u_i'$ \\
                 $u_i' \gets 0$ \\}
           \ElseIf{$x_i \geq reduce$}{$reduction \gets reduction + u_i' - (x_i - reduce)$ \\
                 $u_i' \gets x_i - reduce$}}
           $i \gets (i + 1) mod \:n$}
       \Return $U' = \bigcup\limits_{k=1}^{n}\{u_k'\}$
    \end{algorithm}
       \noindent
       We will start by showing some results useful for the following proofs. Let $j$ be the number of iterations of the
       \textbf{while} loop for the rest of the proofs for algorithm \ref{abs} (think of $i$ from line 20 without the
       $mod\:n$).\\
       First we will show that $empty \leq n$. $empty$ is only modified on line 12 where it is incremented by 1. This
       happens only when $u_i' > 0$ (line 11), which is assigned the value 0 on line 16. We can see that the
       incrementation of $empty$ can happen at most $n$ times because $|U'| = n$. Since $empty_0 = 0$, $empty \leq n$
       at all times of the execution. \\
       Next we will derive the recursive formulas for the various variables. \\
       $empty_0 = 0$ \\
       $empty_{j+1} = 
          \begin{cases}
             empty_j, & u_{(j+1)\:mod\:n}' = 0 \\
             empty_j+1, & u_{(j+1)\:mod\:n}' > 0 \: \wedge \: x_{(j+1) \:mod\:n} < reduce_j \\
             empty_j, & u_{(j+1)\:mod\:n}' > 0 \: \wedge \: x_{(j+1) \:mod\:n} \geq reduce_j
          \end{cases}$ \\ \ \\
       $reduce_0 = \frac{V}{n}$ \\
       $reduce_{j+1} =
          \begin{cases}
             reduce_j, & u_{(j+1)\:mod\:n}' = 0 \\
             reduce_j + \frac{reduce_j-x_{(j+1)\:mod\:n}}{n-empty_{j+1}}, & u_{(j+1)\:mod\:n}' > 0 \: \wedge \:
                x_{(j+1) \:mod\:n} < reduce_j \\
             reduce_j, & u_{(j+1)\:mod\:n}' > 0 \: \wedge \: x_{(j+1) \:mod\:n} \geq reduce_j
          \end{cases}$ \\ \ \\
       $reduction_0 = 0$ \\
       $reduction_{j+1} =
          \begin{cases}
             reduction_j, & u_{(j+1)\:mod\:n}' = 0 \\
             reduction_j + u_{(j+1)\:mod\:n}', & u_{(j+1)\:mod\:n}' > 0 \: \wedge \: x_{(j+1) \:mod\:n} < reduce_j \\
             reduction_j + u_{(j+1)\:mod\:n}' - x_{(j+1)\:mod\:n} + reduce_{j+1}, &
                u_{(j+1)\:mod\:n}' > 0 \: \wedge \: x_{(j+1) \:mod\:n} \geq reduce_j
          \end{cases}$ \\
       In the end, $r = reduce$ is such that $r = \frac{V - \sum\limits_{x \in S}x}{n - |S|}$ where
       $S = \{\text{flows } y \text{ from } s \text{ to } N^{+}(s) \text{ according to } maxFlow : y < r\}$. Also,
       $\sum\limits_{i=1}^{n}u_i' = \sum\limits_{i=1}^{n}\max{(0,x_i - r)}$. TOPROVE
    \begin{proof}[Proof of correctness for algorithm \ref{abs}] \
       \begin{itemize}
          \item We will show that $\forall i \in [n] \: u_i' \leq x_i$. \\
          On line 9, $\forall i \in [n] \: u_i' = x_i$. Subsequently $u_i'$ is modified on line 16, where
          it becomes equal to 0 and on line 19, where it is assigned $x_i - reduce$. It holds that $x_i - reduce \leq x_i$
          because initially $reduce = \frac{V}{n} \geq 0$ and subsequently $reduce$ is modified only on line 14 where it
          is increased ($n > empty$ because of line 13 and $reduce > x_i$ because of line 11, thus
          $\frac{reduce - x_i}{n - empty} > 0$). We see that $\forall i \in [n], u_i' \leq x_i$.
          \item We will show that $\sum\limits_{i=1}^{n}u_i' = F - V$. \\
          The variable $reduction$ keeps track of the total reduction that has happened and breaks the \textbf{while} loop
          when $reduction \geq V$. We will first show that $reduction = \sum\limits_{i=1}^{n}(x_i- u_i')$ at all times and
          then we will prove that $reduction = V$ at the end of the execution. Thus we will have proven that
          $\sum\limits_{i=1}^{n}u_i'= \sum\limits_{i=1}^{n}x_i - V = F - V$.
          \begin{itemize}
             \item On line 9, $u_i' = x_i \Rightarrow \sum\limits_{i=1}^{n}(x_i- u_i') = 0$ and $reduction = 0$. \\
             On line 16, $u_i'$ is reduced to 0 thus $\sum\limits_{i=1}^{n}(x_i- u_i')$ is increased by $u_i'$.
             Similarly, on line 15 $reduction$ is increased by $u_i'$, the same as the increase in
             $\sum\limits_{i=1}^{n}(x_i- u_i')$. \\
             On line 19, $u_i'$ is reduced by $u_i' - x_i + reduce$ thus $\sum\limits_{i=1}^{n}(x_i- u_i')$ is increased
             by $u_i' - x_i + reduce$. On line 18, $reduction$ is increased by $u_i' - x_i + reduce$, which is equal
             to the increase in $\sum\limits_{i=1}^{n}(x_i- u_i')$. \\
             We also have to note that neither $u_i'$ nor $reduction$ is modified in any other way from line 10 and on,
             thus we conclude that $reduction = \sum\limits_{i=1}^{n}(x_i- u_i')$ at all times.
             \item Suppose that $reduction_j > V$ on the line 21. Since $reduction_j$ exists, $reduction_{j-1} < V$.
             If $x_{j \: mod \: n} < reduce_{j-1}$ then $reduction_j = reduction_{j-1} + u_{j \: mod \:n}'$.
             Since $reduction_j > V$, $u_{j \: mod \:n}' > V - reduction_{j-1}$. TOCOMPLETE\\
             
          \end{itemize}
       \end{itemize}
    \end{proof}
    \begin{proof}[Complexity of algorithm \ref{abs}] \ \\
       In the worst case scenario, each time we iterate over all capacities only the last non-zero capacity will become zero
       and every non-zero capacity must be recalculated. This means that every $n$ steps exactly 1 capacity becomes zero
       and eventually all capacities (maybe except for one) become zero. Thus we need $O(n^2)$ steps in the worst case.
    \end{proof}
    A variation of this algorithm using a Fibonacci heap with complexity $O(n)$ can be created, but that is part of 
    further research.
    \begin{proof}[Proof that algorithm \ref{abs} minimizes the $||\Delta_i||_\infty$ norm] \ \\
       Suppose that $U'$ is the result of an execution of algorithm \ref{abs} that does not minimize the $||\Delta_i||_\infty$
       norm. Suppose that $W$ is a valid solution that minimizes the $||\Delta_i||_\infty$ norm. Let $\delta$ be the minimum
       value of this norm. There exists $i \in [n]$ such that $x_i - w_i = \delta$ and $u_i' < w_i$. Because both $U'$
       and $W$ are valid solutions ($\sum\limits_{i=1}^{n}u_i' = \sum\limits_{i=1}^{n}w_i = F - V$), there must exist a set
       $S \subset U'$ such that $\forall u_j' \in S, u_j' > w_j$ TOCOMPLETE.
    \end{proof}

    \begin{algorithm}[H]
       \label{prop}
       \SetKwInOut{Input}{Input}
       \SetKwInOut{Output}{Output}
       \Input{$x_i$ flows, $n = |N^{+}(s)|$, $V$ value}
       \Output{$u_i'$ capacities}
       \caption{Proportional equality trust transfer}
       $F \gets \sum\limits_{i=1}^{n}x_i$ \\
       \If{$F < V$}{\Return $\bot$}
       \For{$i \gets 1$ to $n$}
          {$u_i' \gets x_i - \frac{V}{F} x_i$}
       \Return $U' = \bigcup\limits_{k=1}^{n}\{u_k'\}$
    \end{algorithm} \ \\
    \begin{proof}[Proof of correctness for algorithm \ref{prop}] \
       \begin{itemize}
          \item We will show that $\forall i \in [n] \: u_i' \leq x_i$. \\
          According to line 5, which is the only line where $u_i'$ is changed, $u_i' = x_i - \frac{V}{F}x_i \leq x_i$
          since $x_i, V, F > 0$ and $V \leq F$.
          \item We will show that $\sum\limits_{i=1}^{n}u_i' = F - V$. \\
          With $F = \sum\limits_{i=1}^{n}x_i$, on line 6 it holds that $\sum\limits_{i=1}^{n}u_i' = \sum\limits_{i=1}^{n}
          (x_i - \frac{V}{F}x_i) = \sum\limits_{i=1}^{n}x_i - \frac{V}{F}\sum\limits_{i=1}^{n}x_i = F - V$.
       \end{itemize}
    \end{proof}
    \begin{proof}[Complexity of algorithm \ref{prop}] \ \\
       The complexity of lines 1, 4-5 and 6 is $O(n)$ and the complexity of lines 2-3 is $O(1)$, thus the total complexity
       of algorithm \ref{prop} is $O(n)$.
    \end{proof}

    Naive algorithms result in $u_i' \leq x_i$, thus according to \ref{invariability}, $||\delta_i||_1$ is invariable for
    any of the possible solutions $U'$, which is not necessarily the minimum (usually it will be the maximum). The following
    algorithms concentrate on minimizing two $\delta_i$ norms, $||\delta_i||_\infty$ and $||\delta_i||_1$. \\
    \begin{algorithm}[H]
       \label{dinf}
       \SetKwInOut{Input}{Input}
       \SetKwInOut{Output}{Output}
       \SetKwFunction{BinSearch}{BinSearch}
       \Input{$X = \{x_i\}$ flows, $n = |N^{+}(s)|$, $V$ value, $\epsilon_1$, $\epsilon_2$}
       \Output{$u_i'$ capacities}
       \caption{$||\delta_i||_\infty$ minimizer}
       \If{$\epsilon_1 < 0 \vee \epsilon_2 < 0$}
          {\Return $\bot$}
       $F \gets \sum\limits_{i=1}^{n}x_i$ \\
       \If{$F < V$}
          {\Return $\bot$}
       $\delta_{max} \gets \max\limits_{i \in [n]}\{u_i\}$ \\
       $\delta^* \gets$ \BinSearch{0,$\delta_{max}$,F-V,n,X,$\epsilon_1$,$\epsilon_2$} \\
       \For{$i \gets 1$ to $n$}
          {$u_i' \gets \max{(u_i - \delta^*, 0)}$}
       \Return $U' = \bigcup\limits_{k=1}^{n}\{u_k'\}$
    \end{algorithm}
    Since trust should be considered as a continuous unit and binary search dissects the possible interval for the solution
    on each recursive call, inclusion of the $\epsilon$-parameters in \texttt{BinSearch} is necessary for the algorithm to
    complete in a finite number of steps. \\
    \begin{algorithm}[H]
       \label{binsearch}
       \SetKwFunction{BinSearch}{BinSearch}
       \SetKwInOut{Input}{Input}
       \SetKwInOut{Output}{Output}
       \Input{$bot$, $top$, $F'$, $n$, $X$, $\epsilon_1$, $\epsilon_2$}
       \Output{$\delta^*$}
       \caption*{\textbf{function} \texttt{BinSearch} }
       \If{$bot = top$}{\Return $bot$}
       \Else{
          \For{$i \gets 1$ to $n$}
              {$u_i' \gets \max{(0,u_i - \frac{top + bot}{2})}$}
          \If{$maxFlow < F' - \epsilon_1$}
	     {\Return \BinSearch{$bot$, $\frac{top+bot}{2}$,$F'$,$n$,$X$,$\epsilon_1$,$\epsilon_2$}}
          \ElseIf{$maxFlow > F' + \epsilon_2$}
	     {\Return \BinSearch{$\frac{top+bot}{2}$, $top$,$F'$,$n$,$X$.$\epsilon_1$,$\epsilon_2$}}
          \Else
             {\Return $\frac{top + bot}{2}$}
       }
    \end{algorithm}
    \begin{proof}[Proof that $maxFlow(\delta)$ is strictly decreasing for $\delta: maxflow(\delta) < F$] \ \\
       Let $maxFlow(\delta)$ be the $maxFlow$ with $\forall i \in [n], u_i' = max(0, u_i - \delta)$.
       We will prove that the function $maxFlow(\delta)$ is strictly decreasing for all $\delta \leq \max\limits_{i \in
       [n]}\{u_i\}$ such that $maxFlow(\delta) < F$. \\
       Supppose that $\exists \delta_1, \delta_2 : \delta_1 < \delta_2 \wedge maxFlow(\delta_1) \leq maxFlow(\delta_2) < F$.
       We will work with configurations of $x_{i,j}'$ such that $x_{i,j}' \leq x_i, j \in \{1,2\}$. \\
       Let $S_j = \{i \in N^{+}(s) : i \in MinCut_j\}$. It holds that $S_1 \neq \emptyset$ because otherwise $MinCut_1 =
       MinCut_{\delta = 0}$ which is a contradiction because then $maxFlow(\delta_1) = F$. Moreover, it holds that
       $S_1 \subseteq S_2$, since $\forall u_{i,2}' > 0,u_{i,2}' < u_{i,1}'$. Every node in the $MinCut_j$ is saturated, thus
       $\forall i \in S_1, x_{i,j}' = u_{i,j}'$. Thus $\sum\limits_{i \in S_1} x_{i,2} < \sum\limits_{i \in S_1}x_{i,1}$ and,
       since $maxFlow(\delta_1) \leq maxFlow(\delta_2)$, we conclude that for the same configurations,
       $\sum\limits_{i \in N^{+}(s) \setminus S_1} x_{i,2} > \sum\limits_{i \in N^{+}(s) \setminus S_1}x_{i,1}$.
       However, since $x_{i,j}' \leq x_i, j \in \{1,2\}$, the configuration
       $[x_{i,1}'' = x_{i,2}', i \in N^{+}(s) \setminus S_1], [x_{i,1}'' = x_{i,1}', i \in S_1]$ is valid for
       $\delta = \delta_1$ and then $\sum\limits_{i \in S_1}x_{i,1}'' + \sum\limits_{i \in N^{+}(s) \setminus S_1}x_{i,1}'' =
       \sum\limits_{i \in S_1}x_{i,1}' + \sum\limits_{i \in N^{+}(s) \setminus S_1}x_{i,2}' > maxFlow(\delta_1)$,
       contradiction. Thus $maxFlow(\delta)$ is strictly decreasing.
    \end{proof}
       We can see that if $V > 0, F' = F - V < F$ thus if $\delta \in (0, \max\limits_{i \in [n]}\{u_i\}]:
       maxFlow(\delta)= F' \Rightarrow \delta = \min||\delta_i||_\infty : maxFlow(||\delta_i||_\infty) = F'$.
       
    \begin{proof}[Proof of correctness for function \ref{binsearch}] \ \\
       Supposing that $[F' - \epsilon_1, F' + \epsilon_2] \subset [maxFlow(top),maxFlow(bot)]$, or equivalently
       $maxFlow(top) \leq F' - \epsilon_1 \wedge maxFlow(bot) \geq F' + \epsilon_2$, we will prove that
       $maxFlow(\delta^*) \in [F' - \epsilon_1, F' + \epsilon_2]$. \\
       First of all, we should note that if an invocation of \texttt{BinSearch} returns without calling
       \texttt{BinSearch} again (line 2 or 11), its return value will be equal to the return value of the initial invocation
       of \texttt{BinSearch}, as we can see on lines 7 and 9, where the return value of the invoked \texttt{BinSearch}
       is returned without any modification. The case where \texttt{BinSearch} is called again is analyzed next:
       \begin{itemize}
          \item If $maxFlow(\frac{top+bot}{2}) < F' - \epsilon_1 < F'$ (line 6) then, since $maxFlow(\delta)$ is strictly
          decreasing, $\delta^* \in [bot,\frac{top+bot}{2})$. As we see on line 7, the interval $(\frac{top+bot}{2}, top]$
          is discarded when the next \texttt{BinSearch} is called. Since $F' + \epsilon_2 \leq maxFlow(bot)$, we have
          $[F' - \epsilon_1, F' + \epsilon_2] \subset [maxFlow(\frac{top+bot}{2}), maxFlow(bot)]$ and
          the length of the available interval is divided by 2.
          \item Similarly, if $maxFlow(\frac{top+bot}{2}) > F' + \epsilon_2 > F'$ (line 8) then $\delta^* \in
          (\frac{top+bot}{2}, top]$. According to line 9, the interval $[bot, \frac{top+bot}{2})$ is discarded when the next
          \texttt{BinSearch} is called. Since $F'- \epsilon_1 \geq maxFlow(top)$, we have $[F' - \epsilon_1, F' + \epsilon_2]
          \subset (maxFlow(top),$ $maxFlow(\frac{top+bot}{2})]$ and the length of the available interval is divided by 2.
       \end{itemize}
       As we saw, $[F' - \epsilon_1, F' + \epsilon_2] \subset [maxFlow(top),maxFlow(bot)]$ in every recursive call and
       $top - bot$ is divided by 2 in every call. From topology we know that $A \subset B \Rightarrow |A| < |B|$, so the
       recursive calls cannot continue infinitely. $|[F' - \epsilon_1, F' + \epsilon_2]| = \epsilon_1 + \epsilon_2$. Let
       $bot_0, top_0$ the input values given to the initial invocation of \texttt{BinSearch}, $bot_j,top_j$ the input
       values given to the $j$-th recursive call of \texttt{BinSearch} and $len_j =|[bot_j, top_j]| = top_j - bot_j$. We have
       $\forall j > 0, len_j = top_j - bot_j = \frac{top_{j-1} - bot_{j-1}}{2} \Rightarrow \forall j >0, len_j =
       \frac{top_0 - bot_0}{2^j}$. We understand that in the worst case $len_j = \epsilon_1 + \epsilon_2 \Rightarrow
       2^j = \frac{top_0-bot_0}{\epsilon_1 + \epsilon_2} \Rightarrow j = \log_2(\frac{top_0-bot_0}{\epsilon_1+\epsilon_2})$.
       Also, as we saw earlier, $\delta^*$ is always in the available interval, thus $maxFlow(\delta^*) \in [F' - \epsilon_1,
       F' + \epsilon_2]$.
%       We will show that the output of \ref{binsearch}, $\delta \in [bot_0, top_0]$, is such that
%\subset [0, \max\limits_{i \in [n]}\{u_i\}]
%       $\sum\limits_{i=1}^{n}\max{(u_i - \delta, 0)} = F' = F - V$. \\
%       We can easily see that $\delta_1 < \delta_2 \Rightarrow \sum\limits_{i=1}^{n}\max{(u_i - \delta_1, 0)} >
%       \sum\limits_{i=1}^{n}\max{(u_i - \delta_2, 0)} \Rightarrow maxFlow_{\delta_1} \geq maxFlow_{\delta_2} (1)$.
%       The recursive function starts backtracking either on line 11, where $maxFlow = F'$, or on line 2 where $bot_j=top_j$.
%       In the latter case, it is $round(\frac{bot_{j-1}+top_{j-1}}{2}) = bot_j$ and we have either $bot_{j-1} = bot_j$ or
%       $top_{j-1} = top_j$.
%       \begin{itemize}
%          \item $bot_{j-1} = bot_j \Rightarrow round(\frac{bot_j + top_{j-1}}{2}) = bot_j \xRightarrow{top_{j-1} > bot_j}
%          bot_j < \frac{bot_j + top_{j-1}}{2} < bot_j + 0.5 \Rightarrow bot_j < top_{j-1} < bot_j + 1$ impossible.
%          \item $top_{j-1} = top_j \Rightarrow round(\frac{bot_{j-1} + top_j}{2}) = top_j \xRightarrow{top_j > bot_{j-1}}
%          top_j - 0.5 \leq \frac{bot_{j-1} + top_j}{2} < top_j \Rightarrow top_j - 1 \leq bot_{j-1} < top_j
%          \Rightarrow bot_{j-1} = top_j - 1$. In this case $round(\frac{bot_{j-1} + top_{j-1}}{2}) =
%          round(\frac{top_j - 1 + top_j}{2}) = round(top_j - 0.5) = top_j \Rightarrow
%          maxFlow_{\frac{bot_{j-1} + top_{j-1}}{2}} = maxFlow_{bot_j}$. Since $bot_j$ exists, \\
%          $maxFlow_{\frac{bot_{j-1} + top_{j-1}}{2}} \neq F'$.
%       \end{itemize}
%$bot_{j-1} = bot_j - 1 \wedge top_{j-1} = top_j$ or $bot_{j-1} = bot_j \wedge
%       top_{j-1} = top_j + 1$.
%       \begin{itemize}
%          \item If $bot_{j-1} = bot_j - 1 \wedge top_{j-1} = top_j, maxFlow_{\frac{top_{j-1}+bot_{j-1}}{2}} >F',
%          (1) \Rightarrow maxFlow_{bot_j} \leq maxFlow_{\frac{top_{j-1}+bot_{j-1}}{2}}$
%       \end{itemize}
%$\forall \delta': 0 \leq \delta' < bot_j, maxFlow > F'$ because $\exists i \in [0,j):bot_i \leq \delta' \leq top_i$
    \end{proof}
    \begin{proof}[Complexity of function \ref{binsearch}] \ \\
       Lines 1-2 have complexity $O(1)$, lines 4-5 have complexity $O(n)$, lines 6-11 have complexity
       $O(maxFlow) + O(BinSearch)$. As we saw in the proof of correctness for function \ref{binsearch}, we need at most
       $\log_2(\frac{top - bot}{\epsilon_1 + \epsilon_2})$ recursive calls of \texttt{BinSearch}. Thus the function
       \ref{binsearch} has worst-case complexity $O((maxFlow + n)\log_2(\frac{top - bot}{\epsilon_1 + \epsilon_2}))$.
    \end{proof}
    \begin{proof}[Proof of correctness for algorithm \ref{dinf}] \ \\
       We will show that $maxFlow \in [F - V - \epsilon_1, F - V + \epsilon_2]$, with $u_i'$ decided by algorithm
       \ref{dinf}. \\
       Obviously $maxFlow(0) = F, maxFlow(\max\limits_{i \in [n]}\{u_i\}) = 0$, thus $\delta^* \in
       \max\limits_{i \in [n]}\{u_i\}$. According to the proof of correctness for function \ref{binsearch},
       we can directly see that $maxFlow(\delta^*) \in [F - V - \epsilon_1, F - V + \epsilon_2]$, given that
       $\epsilon_1, \epsilon_2$ are chosen so that $F - V - \epsilon_1 \geq 0, F - V + \epsilon_2 \leq F$, so as to satisfy
       the condition $[F' - \epsilon_1, F' + \epsilon_2] \subset [maxFlow(top),maxFlow(bot)]$.
    \end{proof}
    \begin{proof}[Complexity of algorithm \ref{dinf}] \ \\
       The complexity of lines 1,2 and 4-6 is $O(n)$ and the complexity of line 3 is $O(BinSearch) = O((maxFlow + n)
       \log_2(\frac{\delta_{max}}{\epsilon_1 + \epsilon_2}))$, thus the total complexity of algorithm \ref{dinf} is
       $O((maxFlow + n)\log_2(\frac{\delta_{max}}{\epsilon_1 + \epsilon_2}))$.
    \end{proof}

    However, we need to minimize $\sum\limits_{i=1}^{n}(u_i-u_i') = ||\delta_i||_1$.

  \section{Related Work}

  \section{Further Research}

  While our trust network can form a basis for risk-invariant transactions in
  the anonymous and decentralized setting, more research is required to achieve
  other desirable properties. Some directions for future research are outlined
  below.

  \subsection{Zero knowledge}

  Our network evaluates indirect trust by computing the max flow in the graph
  of lines-of-credit. In order to do that, complete information about the
  network is required. However, disclosing the network topology may be
  undesirable, as it subverts the identity of the participants even when
  participants are treated pseudonymously, as deanonymization techniques can be
  used. To avoid such issues, exploring the ability to calculate flows in a
  zero knowledge fashion may be desirable. However, performing network queries
  in zero knowledge may allow an adversary to extract topological information.
  More research is required to establish how flows can be calculated
  effectively in zero knowledge and what bounds exist in regards to information
  revealed in such fashion.

  \section{References}

\end{document}
