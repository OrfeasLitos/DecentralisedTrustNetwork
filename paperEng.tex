\documentclass[11pt]{llncs}
\usepackage[pdftex]{graphicx}
\usepackage[linesnumbered,ruled,noend]{algorithm2e}
\usepackage[english]{babel}
\usepackage{amsmath}
\let\proof\relax
\let\endproof\relax
\usepackage{amsthm}
\usepackage{mathtools}
\usepackage{amssymb}
\usepackage{lmodern}
\usepackage[T1]{fontenc}
\usepackage{listings}
\usepackage{graphviz}
\usepackage{dot2texi}
\usepackage{tikz}
\usepackage{pgfplots}
\usetikzlibrary{shapes,arrows}
%\usepackage[utf8]{inputenc}
\usepackage{hyperref}
%\usepackage{chngcntr}

\lstset{frame=tb,
  showstringspaces=false,
  columns=flexible,
  basicstyle=\ttfamily,
  keywordstyle=\color{blue},
  commentstyle=\color{dkgreen},
  stringstyle=\color{mauve},
  breaklines=true,
  breakatwhitespace=true,
  tabsize=3,
  escapeinside={(*@}{@*)},
  frame=n
}
\lstdefinestyle{numbers}{numbers=left, stepnumber=1, numberstyle=\tiny, numbersep=10pt}
\let\origthelstnumber\thelstnumber
\makeatletter
\newcommand*\Suppressnumber{%
  \lst@AddToHook{OnNewLine}{%
    \let\thelstnumber\relax%
     \advance\c@lstnumber-\@ne\relax%
    }%
}

\newcommand*\Reactivatenumber{%
  \lst@AddToHook{OnNewLine}{%
   \let\thelstnumber\origthelstnumber%
   \advance\c@lstnumber\@ne\relax}%
}

\def\bitcoin{%
  \leavevmode
  \vtop{\offinterlineskip %\bfseries
    \setbox0=\hbox{B}%
    \setbox2=\hbox to\wd0{\hfil\hskip-.03em
    \vrule height .3ex width .15ex\hskip .08em
    \vrule height .3ex width .15ex\hfil}
    \vbox{\copy2\box0}\box2}}

%\newcounter{proofctr}
%\newcommand{\proofctr}[2][]{%
%    \proof[#1]{#2}%
%    \refstepcounter{proofctr}%
%}
%
%\counterwithout{proofctr}{section}
%\renewenvironment{proof}{(\arabic{proof})}

\theoremstyle{definition}
\newtheorem{sepproof}{Proof}

\pagestyle{plain}

\newenvironment{proofsketch}{\textit{Proof Sketch.}}{\qed \ \\}

\bibliographystyle{splncs}

% PDF bookmarks
\usepackage{color,hyperref}
\definecolor{darkblue}{rgb}{0.0,0.0,0.3}
\hypersetup{colorlinks,breaklinks,
    linkcolor=darkblue,urlcolor=darkblue,
    anchorcolor=darkblue,citecolor=darkblue}

\begin{document}
\title{Trust Is Risk: \\ A Decentralized Financial Trust Platform}
\author{Orfeas Stefanos Thyfronitis Litos\inst{1} \and Dionysis Zindros\inst{2} \fnmsep
                                               \thanks{Research supported by ERC project CODAMODA, project \#259152}}
\institute{National Technical University of Athens \and National and Kapodistrian University of Athens \\
           \email{orfeas.litos@hotmail.com}, \email{dionyziz@di.uoa.gr}}
\maketitle
\thispagestyle{plain}
  \begin{abstract}
     Reputation in centralized systems uses stars and review-based trust. Such systems require manual intervention and
     secrecy to avoid manipulation. In autonomous and open source decentralized systems this luxury is not available.
     Previous peer-to-peer reputation systems do not allow for financial arguments pertaining to reputation. We propose a
     concrete sybil-resilient decentralized reputation system in which direct trust is defined as lines-of-credit using
     bitcoin's 1-of-2 multisig. We introduce a new model for bitcoin wallets in which user coins are split among trusted
     associates. Indirect trust is subsequently defined transitively. This enables formal game theoretic arguments pertaining
     to risk analysis. We prove that risk and max flows are equivalent in our model. Our system allows for concrete financial
     decisions on the monetary amount a pseudonymous party can be trusted with. Through algorithmic trust redistribution, the
     risk incurred from making a purchase from a pseudonymous party in this manner remains invariant.
%     \keywords{decentralized $\cdot$ trust $\cdot$ web-of-trust $\cdot$ bitcoin $\cdot$ multisig $\cdot$ line-of-credit
%               $\cdot$ trust-as-risk $\cdot$ flow $\cdot$ reputation}
  \end{abstract}

  \section{Introduction}
     Modern online marketplaces can be roughly categorized as centralized and decentralized.
     Two major examples of each category are \href{http://www.ebay.com}{ebay} and \href{https://openbazaar.org/}{OpenBazaar}.
     The common denominator of established online marketplaces is that the reputation of each vendor and client is either
     expressed in the form of stars and user-generated reviews that are viewable by the whole network, or not expressed at
     all inside the marketplace and instead is entirely built on word-of-mouth or other out-of-band means.
%     \href{http://www.ebay.com}{ebay} is
%     centralized and as such it is vulnerable to ddos attacks \cite{ddosattacks} and can be considered as a single point of
%     failure, it charges fees for the use of its services \cite{ebayfees} and maintains a private database of personal data,
%     but it can give money-back guarantees \cite{ebayguarantee} since it is run by a single company that has a financial
%     advantage in keeping its clients satisfied. On the other hand, \href{https://openbazaar.org/}{OpenBazaar} is a
%     decentralized platform built on bitcoin \cite{bitcoin}, where individual stores or its \href{https://duosear.ch}{search
%     engine} are vulnerable to ddos attacks \cite{ddosattacks}, but not the platform as a whole. Additionally, it does not
%     charge fees for its usage \cite{openbazaar} and there is no central agent recording all the transactions alongside with
%     private data \cite{openbazaar} but it is possible for a buyer or a seller colluding with a  moderator to scam the third
%     party and there exists no central authority able to verify the truth of her claim and reimburse her
%     \cite{multisigfraud}. Even though trust (or distrust) should be directed to each store individually, it is very likely
%     that the whole platform will be discarded as untrusted by the scammed party.

     Our goal is to create a decentralized marketplace where the trust each user gives to the rest of the users is
     quantifiable, measurable and expressable in monetary terms. The central concept used throughout this paper is
     that trust is equivalent to risk, or the proposition that $Alice$'s \textit{trust} to another user $Bob$ is defined to
     be the \textit{maximum sum of money} that $Alice$ can lose when $Bob$ is free to choose any strategy he wants. To flesh
     out this concept, we will use \textit{lines of credit} as proposed by Washington Sanchez \cite{loc}. Joining the network
     will be done by explicitly entrusting a certain amount of money to another user, say $Bob$. If $Bob$ has already
     entrusted an amount of money to a third user, $Charlie$, then we indirectly trust $Charlie$ since if the latter wished
     to play unfairly, he could have already stolen the money entrusted to him by $Bob$. Thus we can engage in economic
     interaction with $Charlie$.
     \begin{center}
        \begin{tikzpicture}
        \begin{dot2tex}[dot,tikz,options=-s -tmath,scale=0.57]
           digraph G {
             #size="0.01,0.01?"
             rankdir=LR
             subgraph cluster0 {
                labelloc="c"
                label="Ex.1:\\ Alice\\ trusts\\ Charlie\\ 10"
                Alice1 -> Bob1 [label="10"]
                Bob1 -> Charlie1 [label="20"]
                Alice1 [label="Alice"]
                Bob1 [label="Bob"]
                Charlie1 [label="Charlie"]
             }
             subgraph cluster1 {
                labelloc="t"
                label="Ex.2:\\ Alice\\ trusts\\ Charlie\\ 5"
                Alice2 -> Bob2 [label="10"]
                Bob2 -> Charlie2 [label="5"]
                Alice2 [label="Alice"]
                Bob2 [label="Bob"]
                Charlie2 [label="Charlie"]
             }
             Charlie1 -> Alice2 [style=invisible]
           }
        \end{dot2tex}
        \end{tikzpicture}
     \end{center} \ \\

     We thus propose a new kind of wallet where coins are not stored locally, but are placed in 1-of-2 multisigs, a bitcoin
     construction that permits any one of two pre-designated users to spend the coins contained therein \cite{multisig}. We
     will use the notation 1/$\{Alice, Bob\}$ to represent a 1-of-2 multisig that can be spent by either $Alice$ or $Bob$.

     Our approach changes the user experience in a subtle but drastic way. A user no more has to base her trust towards a
     store on stars, ratings or other dubious and non-quantifiable trust metrics. She can simply consult her wallet to
     decide whether the store is trustworthy and, if so, up to what value. This system works as follows: Initially $Alice$
     migrates her funds from P2PKH addresses in the UTXO \cite{bitcoinguide} to 1-of-2 multisig addreses shared with friends
     she comfortably trusts. We call this direct trust. Our system is agnostic to the means players use to determine who is
     trustworthy for these direct 1-of-2 deposits.

     Suppose that $Alice$ is viewing the item listings of vendor $Charlie$. Instead of $Charlie$'s stars, $Alice$ will see a
     positive value that is calculated by her wallet and represents the maximum monetary value that $Alice$ can safely use to
     complete a purchase from $Charlie$. We examine exactly how this value is calculated in Trust Flow Theorem
     (\ref{trustflow}). This monetary value reported by our system maintains the desired security property that, if $Alice$
     makes this purchase, then she is exposed to no more risk than she was willing to expose herself towards her friends.
     We prove this result in the Risk Invariance Theorem (\ref{riskinv}). Obviously it will not be safe for $Alice$ to buy
     anything from $Charlie$ or any other vendor if she has entrusted no value to any other player.

     We see that in TrustIsRisk the money is not invested at the time of the purchase and directly to the vendor, but at an
     earlier point in time and only to parties that are trustworthy for out-of-band reasons. The fact that this system can
     function in a completely decentralized fashion will become clear in the following sections. We prove this result in the
     Sybil Resilience Theorem (\ref{sybil}).

%     check the trust \textit{flowing} from her to the store (which will be a number
%     expressed in bitcoins) and if this number exceeds the price of the product, she is safe to complete the transaction
%     after modifying her direct trust towards her friends in an appropriate way, a process that can be handled by one of
%     several algorithms that we propose, or in any custom way the user chooses to. On the other side, there is no guarantee
%     that the store will complete is part of the transaction and no central authority can reimburse the money. However, it is
%     possible for the defrauded user to make up for her loss by in turn stealing from other users that trust her, an action
%     that will tautologically reduce their trust to her.
     
     There are several incentives for a user to join this network. First of all, she can have access to a store that is
     otherwise unaccessible. Moreover, two friends can formalize their mutual trust by entrusting the same amount to each
     other. A large company that casually subcontracts other companies to complete various tasks can express its trust
     towards them using this method. A government can choose to entrust its citizens with money and confront them using a
     corresponding legal arsenal if they make irresponsible use of this trust. A bank can provide loans as outgoing and
     manage savings as incoming trust and thus has a unique opportunity of expressing in a formal and absolute way its
     credence by publishing its incoming and outgoing trust. Last but not least, the network can be viewed as a possible
     field for investment and speculation since it constitutes a completely new area for financial activity.

     It is worth noting that the same physical person can maintain multiple pseudonymous identities in the same trust network
     and that multiple independent trust networks for different purposes can coexist. On the other hand, the same
     pseudonymous identity can be used to establish trust in different contexts.
  \section{The Trust Graph}
%     The state of the game at any given moment can be represented by a directed weighted graph where each node represents a
%     pseudonymous identity that is considered to correspond to exactly one user and each edge represents a direct trust, the
%     weight of which is the value which the head entrusts the tail with. The weights can be only positive numbers. If a
%     direct trust drops to 0, the corresponding edge does not exist anymore.
     We now engage in the formal description of the proposed system, accompanied by helpful examples.
     \begin{definition}[Graph]
        TrustIsRisk is represented by a sequence of directed weighted graphs $\left(\mathcal{G}_j\right)$ where $\mathcal{G}_j
        = \left(\mathcal{V}_j, \mathcal{E}_j\right), j \in \mathbb{N}$.
%        Members of $\mathcal{E}_j$ are tuples of two nodes from
%        $\mathcal{V}_j$. More formally, $e \in \mathcal{E}_j \Rightarrow \exists A,B \in \mathcal{V}_j : e = \left(A,B\right)$.
        Also, since the graphs are weighted, there exists a sequence of functions $\left(c_j\right)$ with $c_j : \mathcal{E}_j
        \rightarrow \mathbb{R}^{+}$.
     \end{definition}
     The nodes represent the players, the edges represent the existing direct trusts and the weights represent the amount of
     value attatched to the corresponding direct trust. As we will see, the game evolves in turns. The subscript of the graph
     represents the corresponding turn.
     \begin{definition}[Players]
        The set $\mathcal{V}_j = V\left(\mathcal{G}_j\right)$ is the set of all players in the network, otherwise understood
        as the set of all pseudonymous identities.
     \end{definition}
     Each node has a corresponding non-negative number that represents its capital. A node's capital is the total value that
     the node posesses exclusively and nobody else can spend.
     \begin{definition}[Capital]
        The capital of $A$ at the end of turn $j$, $Cap_{A, j}$, is defined as the total value that exists in P2PKH in the
        UTXO and can be spent by $A$ at the end of turn $j$. If the turn is obvious, we may omit the subscript $j$.
     \end{definition}
     A rational player would like to maximize her capital in the long term. The formal definition of direct trust follows:
     \begin{definition}[Direct Trust]
        Direct trust from $A$ to $B$ at the end of turn $j$, $DTr_{A \rightarrow B, j}$, is defined as the total amount of
        value that exists in 1/$\{A,B\}$ multisigs in the UTXO in the end of turn $j$, where the money is deposited by $A$.
        \begin{equation}
           DTr_{A \rightarrow B, j} =
              \begin{cases}
                 c_j\left(A, B\right), & if \left(A, B\right) \in \mathcal{E}_j \\
                 0, & if \left(A, B\right) \notin \mathcal{E}_j
              \end{cases}
        \end{equation}
     \end{definition}
     Any algorithm that has access to the graph $\mathcal{G}_j$ has implicitly access to all direct trusts of this graph.
     \begin{center}
     \begin{tikzpicture}
     \begin{dot2tex}[dot,tikz,options=-s -tmath,scale=0.6]
        digraph G  {
           rankdir=LR;
           edge [arrowsize=0.6];
           node [shape=circle];
           compound=true;
           subgraph cluster0 {
              color=none;
              label = "TrustIsRisk\\ game\\ graph";
              A -> C [label="5BTC"];
              A -> D [label="6BTC"];
              C -> B [label="3BTC"];
              C -> E [label="10BTC"];
              D -> B [label="2BTC"];
           }
           J1 [style=invisible];
           J2 [style=invisible];
           J1 -> J2 [dir=both];
           subgraph cluster1 {
              rank=same;
              label="Bitcoin\\ UTXO";
              Z1 [style=invisible];
              a1 [label="5BTC"];
              Y1 [style=invisible];
              a1 -> Y1 [label="1/\\{A,C\\}"];
              Z1 -> a1 [label="A"];
              Z2 [style=invisible];
              a2 [label="6BTC"];
              Y2 [style=invisible];
              a2 -> Y2 [label="1/\\{A,D\\}"];
              Z2 -> a2 [label="A"];
              Z3 [style=invisible];
              a3 [label="10BTC"];
              Y3 [style=invisible];
              a3 -> Y3 [label="1/\\{C,E\\}"];
              Z3 -> a3 [label="C"];
              Z4 [style=invisible];
              a4 [label="3BTC"];
              Y4 [style=invisible];
              a4 -> Y4 [label="1/\\{C,B\\}"];
              Z4 -> a4 [label="C"];
              Z5 [style=invisible];
              a5 [label="2BTC"];
              Y5 [style=invisible];
              a5 -> Y5 [label="1/\\{D,B\\}"];
              Z5 -> a5 [label="D"];
           }
           B  -> J1 [style=invisible]
           J2 -> Z3 [style=invisible]
        }
     \end{dot2tex}
     \end{tikzpicture}
     \end{center} \ \\
%     \begin{center}
%     \begin{tikzpicture}
%     \begin{dot2tex}[dot,tikz,options=-s -tmath,scale=0.5]
%digraph G {
%  subgraph cluster_level1 {
%    label="Level 1"
%    subgraph cluster_room1 {
%        label="Room 1"
%        a -> b
%    }
%    subgraph cluster_room2 {
%        label="Room 2"
%        c -> d
%    }
%  }
%  subgraph cluster_level2 {
%    label="Level 2"
%    subgraph cluster_room2_1 {
%        label="Room 1"
%        e -> f
%    }
%    subgraph cluster_room2_2 {
%        label="Room 2"
%        g -> h
%    }
%  }
%  {edge[style=invis]
%    {b d} -> {g e}
%  }
%}
%     \end{dot2tex}
%     \end{tikzpicture}
%     \end{center}
     We use the notation $N^{+}(A)$ to refer to the nodes directly trusted by $A$ and $N^{-}(A)$ for the nodes that directly
     trust $A$. We also use the notation $in_{A, j}, out_{A, j}$ to refer to the total incoming and outgoing direct trust
     respectively. For a reference of common definitions, see Appendix. An example graph with its corresponding transactions
     in the UTXO can be seen below. [Image]
  \section{Evolution of Trust}
     \begin{definition}[Turns]
        The game we are describing is turn-based. In each turn $j$ exactly one player $A \in \mathcal{V}, A =
        Player\left(j\right)$, chooses an action (according to a certain strategy) that can be one of the following, or a
        finite combination thereof:
        \begin{enumerate}
           \item Steal value $y_B$ from $B \in N^{-}\left(A\right)_{j-1}$, where
           $0 \leq y_B \leq DTr_{B \rightarrow A, j-1}$. Then it is:
           \begin{equation}
              DTr_{B \rightarrow A, j} = DTr_{B \rightarrow A, j-1} - y_B \:\: \left(Steal\left(y_B, B\right)\right)
           \end{equation}
           \item Add value $y_B$ to $B \in \mathcal{V}$, where $-DTr_{A \rightarrow B, j-1} \leq y_B$. Then it is:
           \begin{equation}
              DTr_{A \rightarrow B, j} = DTr_{A \rightarrow B, j-1} + y_B \:\: \left(Add\left(y_B, B\right)\right)
           \end{equation}
           When $y_B < 0$, we say that $A$ reduces her trust to $B$ by $-y_B$, when $y_B > 0$, we say that $A$ increases her
           trust to $B$ by $y_B$. If $DTr_{A \rightarrow B, j-1} = 0$, then we say that $A$ starts directly trusting $B$.
        \end{enumerate}
        If player $A$ chooses no action in her turn, we say that she passes her turn. Also, let $Y_{st}, Y_{add}$ be the
        total value to be stolen and added respectively by $A$ in her turn, $j$. For a turn to be feasible, it must hold
        that
        \begin{equation}
           Y_{add} - Y_{st} \leq Cap_{A, j-1} \enspace.
        \end{equation}
        Capital is updated in every turn:
        \begin{equation}
           Cap_{A, j} = Cap_{A, j-1} + Y_{st} - Y_{add} \enspace.
        \end{equation}
        Moreover, player $A$ is not allowed to choose two actions of the same kind against the same player in the same turn.
        \\ The set of actions a player makes in turn $j$ is represented with $Turn_j$. The new graph that emerges by applying
        the actions on $\mathcal{G}_{j-1}$ is $\mathcal{G}_j$.
     \end{definition}
     The only requirement we impose to the choice of the player of each turn is that eventually everybody will play in some
     turn. We will use $prev\left(j\right)$ and $next\left(j\right)$ to denote the previous and the next turn that
     is played by $Player(j)$ respectively. A formal definition can be found in the Appendix.
     \begin{definition}[Damage]
        Let $j$ be a turn such that $Player\left(j\right) = A$.
        \begin{equation}
           Damage_{A, j} = out_{A, prev\left(j\right)} - out_{A, j-1}
        \end{equation}
        We say that $A$ has been stolen value $Damage_{A, j}$ between $prev\left(j\right)$ and $j$ if $Damage_{A, j} > 0$.
        If turns are not specified, we implicitly refer to the current and the previous turns.
     \end{definition}
     \begin{definition}[History]
        We define History, $\mathcal{H} = \left(\mathcal{H}_j\right)$, as the sequence of all the tuples containing the sets
        of actions and the corresponding player.
        \begin{equation}
           \mathcal{H}_j = \left(Player\left(j\right), Turn_j\right)
        \end{equation}
     \end{definition}
     Knowledge of the initial graph $\mathcal{G}_0$ and the history amount to full comprehension of the evolution of the
     game. Building on the previous example, we can see the resulting graph when $D$ plays
     \begin{equation}
        Turn_1 = \{Steal\left(1, B\right), Add\left(4, C\right)\} \enspace.
     \end{equation}
     [Image]
     In its initial form TrustIsRisk is controlled by an algorithm that chooses a player at random, receives the turn that
     this player wishes to play and, if this turn is valid, executes it. These steps are repeated indefinitely.
     \Suppressnumber
     \begin{lstlisting}[label=trustisriskgame, style=numbers]
TrustIsRisk Game (*@\Reactivatenumber@*)
j = 0
while (True)
  j = j + 1
  (*@$v \overset{\$}{\gets} \mathcal{V}_j$@*)
  ProvisionalTurn = (*@$v$@*)Oracle((*@$\mathcal{G}_{j-1}$@*), (*@$v$@*), (*@$\mathcal{H}$@*))
  ((*@$G_j$@*), (*@$Cap_{v, j}$@*), (*@$H_j$@*)) = executeTurn((*@$\mathcal{G}_{j-1}$@*), (*@$v$@*), (*@$Cap_{v, j-1}$@*),
      ProvisionalTurn)
    \end{lstlisting}
    The following algorithm has read access to direct trusts in $\mathcal{G}_{j-1}$ and write access to direct trusts in
    $\mathcal{G}_j$.
    \Suppressnumber
    \begin{lstlisting}[label=executeturn, style=numbers]
Execute Turn
Input : old graph (*@$\mathcal{G}_{j-1}$@*), player (*@$A \in \mathcal{V}_{j-1}$@*), old capital (*@$Cap_{A, j-1}$@*), ProvisionalTurn
Output : new graph (*@$\mathcal{G}_j$@*), new capital (*@$Cap_{A, j}$@*), new history (*@$\mathcal{H}_j$ \Reactivatenumber@*)
executeTurn((*@$\mathcal{G}_{j-1}$@*), (*@$A$@*), (*@$Cap_{A, j-1}$@*), ProvisionalTurn) :
  ((*@$Turn_j$@*), NewCap) = validateTurn((*@$\mathcal{G}_{j-1}$@*), (*@$A$@*), (*@$Cap_{A, j-1}$@*), ProvisionalTurn)
  return(commitTurn((*@$\mathcal{G}_{j-1}$@*), (*@$A$@*), (*@$Turn_j$@*), NewCap))
    \end{lstlisting}
    \Suppressnumber
    \begin{lstlisting}[label=validateturn, style=numbers]
Validate Turn
Input : old graph (*@$\mathcal{G}_{j-1}$@*), player (*@$A \in \mathcal{V}_{j-1}$@*), old capital (*@$Cap_{A, j-1}$@*), ProvisionalTurn
Output : (*@$Turn_j$@*), new capital (*@$Cap_{A, j}$@*) (*@\Reactivatenumber@*)
validateTurn((*@$\mathcal{G}_{j-1}$@*), (*@$A$@*), (*@$Cap_{A, j-1}$@*), ProvisionalTurn) :
  (*@$Y_{st}$@*) = 0
  (*@$Y_{add}$@*) = 0
  for (action (*@$\in$@*) ProvisionalTurn)
    action match do
      case (*@$Steal($@*)y(*@$,w)$@*) do
        if (y > (*@$DTr_{w \rightarrow A,j-1}$@*) || y < 0)
          return((*@$\emptyset$@*), (*@$Cap_{A, j-1}$@*))
        else
          (*@$Y_{st}$@*) = (*@$Y_{st}$@*) + y
      case (*@$Add($@*)y(*@$,w)$@*) do
        if (y < -(*@$DTr_{A \rightarrow w,j-1}$@*))
          return((*@$\emptyset$@*), (*@$Cap_{A, j-1}$@*))
        else
          (*@$Y_{add}$@*) = (*@$Y_{add}$@*) + y
  if ((*@$Y_{add}$@*) - (*@$Y_{st}$@*) > (*@$Cap_{A, j-1}$@*))
    return((*@$\emptyset$@*), (*@$Cap_{A, j-1}$@*))
  else
    return(ProvisionalTurn, (*@$Cap_{A, j-1} + Y_{st} - Y_{add}$@*))
    \end{lstlisting}
    \Suppressnumber
    \begin{lstlisting}[label=committurn, style=numbers]
Commit Turn
Input : player old graph (*@$\mathcal{G}_{j-1}$@*), (*@$A \in \mathcal{V}_{j-1}$@*), (*@$Turn_j$@*), NewCap
Output : new graph (*@$\mathcal{G}_j$@*), new capital (*@$Cap_{A, j}$@*), new history (*@$\mathcal{H}_j$ \Reactivatenumber@*)
commitTurn((*@$\mathcal{G}_{j-1}$@*), (*@$A$@*), (*@$Turn_j$@*), NewCap) :
  for (((*@$v$@*), (*@$w$@*)) (*@$\in \mathcal{E}_j$@*))
    (*@$DTr_{v \rightarrow w, j}$@*) = (*@$DTr_{v \rightarrow w, j-1}$@*)
  for (action (*@$\in Turn_j$@*))
    action match do
      case (*@$Steal($@*)y,(*@$,w)$@*) do
        (*@$DTr_{w \rightarrow A, j}$@*) = (*@$DTr_{w \rightarrow A, j-1} - y$@*)
      case (*@$Add($@*)y(*@$,w)$@*) do
        (*@$DTr_{A \rightarrow w, j}$@*) = (*@$DTr_{A \rightarrow w, j-1} + y$@*)
  (*@$Cap_{A, j}$@*) = NewCap
  (*@$\mathcal{H}_j$@*) = ((*@$A$@*), (*@$Turn_j$@*))
  return((*@$\mathcal{G}_j$@*), (*@$Cap_{A, j}$@*), (*@$\mathcal{H}_j$@*))
    \end{lstlisting}
  \section{Trust Transitivity}
     \begin{definition}[Idle Strategy]
        A player $A$ is said to follow the idle strategy if she passes in her turn. 
     \end{definition}
     \Suppressnumber
     \begin{lstlisting}[label=idleoracle, style=numbers]
Idle Oracle
Input : previous graph (*@$\mathcal{G}_{j-1}$@*), player (*@$A$@*), history (*@$\mathcal{H}$@*)
Output : (*@$Turn_j$@*)
idleOracle((*@$\mathcal{G}_{j-1}$@*), (*@$A$@*), (*@$\mathcal{H}$@*)) : (*@\Reactivatenumber@*)
  return((*@$\emptyset$@*))
     \end{lstlisting}
     \begin{definition}[Evil Strategy]
        A player $A$ is said to follow the evil strategy if she steals all incoming direct trust and nullifies her outgoing
        direct trust in her turn.
     \end{definition}
     \Suppressnumber
     \begin{lstlisting}[label=eviloracle, style=numbers]
Evil Oracle
Input : previous graph (*@$\mathcal{G}_{j-1}$@*), player (*@$A$@*), history (*@$\mathcal{H}$@*)
Output : (*@$Turn_j$@*)
evilOracle((*@$\mathcal{G}_{j-1}$@*), (*@$A$@*), (*@$\mathcal{H}$@*)) : (*@\Reactivatenumber@*)
  Steals = (*@$\bigcup\limits_{v \in N^{-}\left(A\right)_{j-1}}\{Steal(DTr_{v \rightarrow A, j-1}, v)\}$@*)
  Adds = (*@$\bigcup\limits_{v \in N^{+}\left(A\right)_{j-1}}\{Add(-DTr_{A \rightarrow v, j-1}, v)\}$@*)
  (*@$Turn_j$@*) = Steals(*@$\: \cup \:$@*)Adds
  return((*@$Turn_j$@*))
     \end{lstlisting}
     \begin{definition}[Conservative Strategy]
        Let $j$ be the current turn and $x$ the value that has been stolen from player $A$ since the previous turn she
        played. Player $A$ is said to follow the conservative strategy if she replenishes her lost value by stealing
        from others that trust her as much as she can up to $x$ and she takes no other action.
     \end{definition}
     \Suppressnumber
     \begin{lstlisting}[label=conservativeoracle, style=numbers]
Conservative Oracle
Input : previous graph (*@$\mathcal{G}_{j-1}$@*), player (*@$A$@*), history (*@$\mathcal{H}$@*)
Output : (*@$Turn_j$@*)
consOracle((*@$\mathcal{G}_{j-1}$@*), (*@$A$@*), (*@$\mathcal{H}$@*)) : (*@\Reactivatenumber@*)
  Damage = (*@$out_{A, prev\left(j\right)}$@*) - (*@$out_{A, j-1}$@*)
  if (Damage > 0)
    if (Damage >= (*@$in_{A, j-1}$@*))
      (*@$Turn_j$@*) = (*@$\bigcup\limits_{v \in N^{-}\left(A\right)_{j-1}}\{Steal\left(DTr_{v \rightarrow A, j-1}, v\right)\}$@*)
    else
      (*@$y$@*) = SelectSteal((*@$G_j$@*), (*@$A$@*), Damage)    #(*@$y$@*) = (*@$\{y_v : v \in N^{-}\left(A\right)_{j-1}\}$@*)
      (*@$Turn_j$@*) = (*@$\bigcup\limits_{v \in N^{-}\left(A\right)_{j-1}}\{Steal\left(y_v, v\right)\}$@*)
  else
    (*@$Turn_j$@*) = (*@$\emptyset$@*)
  return((*@$Turn_j$@*))
     \end{lstlisting}
     \texttt{SelectSteal()} returns $y_v$ with $v \in N^{-}\left(A\right)_{j-1}$ such that
     \begin{equation}
        \sum\limits_{v \in N^{-}\left(A\right)_{j-1}}y_v = Damage_{A, j} \wedge \forall v \in N^{-}\left(A\right)_{j-1},
        y_v \leq DTr_{v \rightarrow A, j-1} \enspace.
     \end{equation}
     Each conservative player can arbitrarily define how \texttt{SelectSteal()} distributes the $Steal\left(\right)$ actions
     each time she calls the function, as long as the above restriction is respected. As we can see, the definition covers a
     multitude of options for the conservative player, since in case $0 < Damage_{A,j} < in_{A,j-1}$ she can choose to
     distribute the $Steal\left(\right)$ actions in any way she chooses.

     The rationale behind this strategy arises from a real-world common situation. Suppose there are a client, an
     intermediary and a producer. The client entrusts some value to the intermediary so that the latter can buy the desired
     product from the producer and deliver it to the client. The intermediary in turn entrusts an equal value to the
     producer, who needs the value upfront to be able to complete the production process. However the producer eventually
     does not give the product neither reimburses the value, due to bankruptcy or decision to exit the market with an unfair
     benefit. The intermediary can choose either to reimburse the client and suffer the loss, or refuse to return the money
     and lose the client's trust. The latter choice for the intermediary is exactly the conservative strategy. It is used
     throughout this work as a strategy for all the intermediary players because it models effectively the worst-case
     scenario that a client can face after an evil player decides to steal everything she can and the rest of the players do
     not engage in evil activity.

     We continue with a very useful possible evolution of the game, the Transitive Game. In turn 0, there is already a network
     in place. Also, all players apart from $A$ and $E$ follow the conservative strategy. Furthermore, the set of players is
     not modified throughout the Transitive Game, thus we can refer to $\mathcal{V}_j$ for any turn $j$ as $\mathcal{V}$.
     These conventions will hold whenever we use the Transitive Game.
     \Suppressnumber
     \begin{lstlisting}[label=transitivegame, style=numbers]
Transitive Game
Input : graph (*@$\mathcal{G}_0$@*), (*@$A \in \mathcal{V}$@*) idle player, (*@$E \in \mathcal{V}$@*) evil player
Output : history (*@$\mathcal{H}$@*) (*@\Reactivatenumber@*)
Angry = (*@$\emptyset$@*)
Happy = (*@$\mathcal{V} \setminus \{A, E\}$@*)
Sad = (*@$\emptyset$@*) (*@\label{trstealsadinit}@*)
for ((*@$v \in \mathcal{V} \setminus \{E\}$@*))
  (*@$Loss_v$@*) = 0 (*@\label{trsteallossinit}@*)
j = 0
while (True)
  j = j + 1
  (*@$v \overset{\$}{\gets} \mathcal{V} \setminus \{A\}$@*)
  (*@$Turn_j$@*) = (*@$v$@*)Oracle((*@$\mathcal{G}_{j-1}$@*), (*@$v$@*), (*@$\mathcal{H}$@*))
  executeTurn((*@$\mathcal{G}_{j-1}$@*), (*@$Cap_{v, j-1}$@*), (*@$Turn_j$@*))
  for (action (*@$\in Turn_j$@*))
    action match do
      case (*@$Steal($@*)y(*@$,w)$@*) do
        exchange = y
        (*@$Loss_w$@*) = (*@$Loss_w$@*) + exchange (*@\label{trsteallossincrease}@*)
        if ((*@$v$@*) != (*@$E$@*))
          (*@$Loss_v$@*) = (*@$Loss_v$@*) - exchange (*@\label{trsteallossdecrease}@*)
        if ((*@$w$@*) != (*@$A$@*))
          Happy = Happy(*@$\:\setminus\: \{w\}$@*)
          if ((*@$in_{w, j}$@*) == 0)
            Sad = Sad(*@$\:\cup\: \{w\}$@*)
          else
            Angry = Angry(*@$\:\cup\: \{w\}$@*)
  if ((*@$v$@*) != (*@$E$@*))
    Angry = Angry(*@$\:\setminus\: \{v\}$@*)
    if ((*@$Loss_v$@*) > 0) (*@\label{trstealifentersad}@*) 
      Sad = Sad(*@$\:\cup\: \{v\}$@*)                  #(*@$in_{v, j}$@*) should be zero (*@\label{trstealtrueentersad}@*)
    if ((*@$Loss_v$@*) == 0)
      Happy = Happy(*@$\:\cup\: \{v\}$@*)
     \end{lstlisting}
     Let $j_0$ be the first turn on which $E$ is chosen to play. Until then, all players will pass their turn since nothing
     has been stolen yet (see Appendix (Theorem \ref{conservativeworld}) for a formal proof of this simple fact). Moreover,
     let $v = Player(j)$ and $j' = prev\left(j\right)$. Given that
     \begin{equation}
        Damage_{v,j} = out_{v, j'} - out_{v, j-1} \enspace,
     \end{equation}
     the algorithm generates turns:
     \begin{equation}
        Turn_j =
          \begin{cases}
             \emptyset, & Damage_{v,j} = 0 \\
             \bigcup\limits_{w \in N^{-}\left(v\right)_{j-1}}\{Steal\left(y_w,w\right)\}, & Damage_{v, j} > 0
          \end{cases} \enspace.
     \end{equation}
     In the second case, it is
     \begin{equation}
        \sum\limits_{w \in N^{-}\left(v\right)_{j-1}}y_w = \min\left(in_{v, j-1}, Damage_{v, j}\right) \enspace.
     \end{equation}
     From the definition of $Damage_{v,j}$ and knowing that no strategy in this case can increase any direct trust, it is
     obvious that $Damage_{v,j} \geq 0$. Also, we can see that $Loss_{v,j} \geq 0$ because if $Loss_{v,j} < 0$, then $v$ has
     stolen more value than she has been stolen, thus she would not be following the conservative strategy.
  \section{Trust Flow}
     \begin{definition}[Indirect Trust]
        The indirect trust from $A$ to $B$ after turn $j$ is defined as the maximum possible value that can be stolen from
        $A$ if $B$ follows the evil strategy, $A$ follows the idle strategy and everyone else ($\mathcal{V} \setminus
        \{A,B\}$) follows the conservative strategy. More formally, if
        \begin{equation}
        \begin{gathered}
           Strategy\left(A\right) = Idle \wedge Strategy\left(B\right) = Evil \wedge \\
           \wedge \forall v \in \mathcal{V} \setminus \{A,B\}, Strategy\left(v\right) = Conservative
        \end{gathered}
        \end{equation}
        and $choices$ are the different actions between which the conservative players can choose, then
        \begin{equation}
           Tr_{A \rightarrow B, j} = \max\limits_{j' : j' > j, choices}{\left[out_{A,j} - out_{A,j'}\right]}
        \end{equation}
     \end{definition}
    \begin{theorem}[Trust Convergence Theorem] \ \\
       \label{convergence}
       Consider a Transitive Game and let $j_0$ be the first turn that $E$ is chosen to play. Given that all players
       eventually play, there exists a turn $j' > j_0$ such that
       \begin{equation}
          \forall j \geq j', Turn_j = \emptyset \enspace.
       \end{equation}
    \end{theorem}
    \begin{proofsketch}
       If the game didn't converge, the $Steal\left(\right)$ actions would continue forever without reduction of the amount
       stolen over time, thus they would reach infinity. However this is impossible, since there exists only finite total
       trust. For the complete proof, see Appendix (Proof \ref{convergenceproof}).
    \end{proofsketch}
    In games where there exists one evil $E$, one idle player $A$ and the rest of the players are conservative, we define
    $Loss_A = Loss_{A, j}$, where $j$ is a turn that the game has converged. It is important to note that $Loss_A$ is not
    the same for repeated executions of this kind of game, since the order in which players are chosen may differ between
    executions and the conservative players are free to choose which incoming trusts they will steal and how much from each.

    Let $G$ be a weighted directed graph. According to \cite{clrs} p. 709, if we consider each edge's capacity as its weight
    ($\forall e \in E(G), c_e = c(e)$), we say that a flow assignment $X = [x_{vw}]_{V(FG) \times V(FG)}$ with a source $A$
    and a sink $B$ is valid when:
    \begin{equation}
    \label{flow1}
       \forall (v, w) \in E(FG), x_{vw} \leq c_{vw}
    \end{equation}
    and
    \begin{equation}
    \label{flow2}
       \forall v \in V(FG) \setminus \{A,B\}, \sum\limits_{w \in N^{+}(v)}x_{wv} = \sum\limits_{w \in N^{-}(v)}x_{vw}
       \enspace.
    \end{equation}
    \begin{lemma}[MaxFlows Are Transitive Games] \ \\
       \label{maxflowgame}
       Let $\mathcal{G}_{j_0}$ be a game graph at a specific turn $j_0$, let $A, E \in \mathcal{V}_{j_0}$ and
       $maxFlow\left(A, E\right)$ the maximum flow from $A$ to $E$ executed on $\mathcal{G}_{j_0}$. There exists an execution
       of \texttt{TransitiveGame(}$\mathcal{G}_{j_0}, A, E$\texttt{)} such that
       \begin{equation}
          maxFlow\left(A, E\right) \leq Loss_A \enspace.
       \end{equation}
    \end{lemma}
    \begin{proofsketch}
       The desired execution of \texttt{TransitiveGame()} will contain all flows from the $MaxFlow\left(A, E\right)$ as
       equivalent $Steal\left(\right)$ actions. The players will play in turns, moving from $E$ back to $A$. Each player will
       steal from his predecessors as much as was stolen from her. The flows and the conservative strategy share the property
       that the total input is equal to the total output. For the complete proof, see Appendix
       (Proof \ref{maxflowgameproof}).
    \end{proofsketch}
    \begin{lemma}[Transitive Games Are Flows] \ \\
       \label{gameflow}
       Let $\mathcal{H} = $\texttt{TransitiveGame(}$\mathcal{G}, A, E$\texttt{)} for some game graph $\mathcal{G}$ and $A,
       E \in \mathcal{V}$. There exists a valid flow
       $X = \{x_{wv}\}_{\mathcal{V} \times \mathcal{V}}$ on $\mathcal{G}_0$ such that
       \begin{equation}
          \sum\limits_{v \in \mathcal{V}}x_{Av} = Loss_A \enspace.
       \end{equation}
    \end{lemma}
    \begin{proofsketch}
       If we remove the sad players from the game, the $Steal\left(\right)$ actions that remain constitute a valid flow from
       $A$ to $E$. For the complete proof, see Appendix (Proof \ref{gameflowproof}).
    \end{proofsketch}
    \begin{theorem}[Trust Flow Theorem] \ \\
       \label{trustflow}
       Let $\mathcal{G}$ be a game graph and $A, B \in \mathcal{V}$. It holds that
       \begin{equation}
          Tr_{A \rightarrow B} = maxFlow\left(A, B\right) \enspace.
       \end{equation}
    \end{theorem}
    \begin{proof} \ \\
       From lemma \ref{maxflowgame} we see that there exists an execution of the Transitive Game such that
       \begin{equation}
          Loss_A = maxFlow\left(A, B\right) \enspace.
       \end{equation}
       Since $Tr_{A \rightarrow B}$ is the maximum loss that $A$ can suffer after the convergence of the Transitive Game, we
       see that
       \begin{equation}
       \label{trgeqmaxflow}
          Tr_{A \rightarrow B} \geq maxFlow\left(A, B\right) \enspace.
       \end{equation}
       Moreover, there exists an execution of the Transitive Game such that
       \begin{equation}
          Tr_{A \rightarrow B} = Loss_A \enspace.
       \end{equation}
       From lemma \ref{gameflow}, this execution corresponds to a flow. Thus
       \begin{equation}
       \label{trleqmaxflow}
          Tr_{A \rightarrow B} \leq maxFlow\left(A, B\right) \enspace.
       \end{equation}
       The theorem follows from (\ref{trgeqmaxflow}) and (\ref{trleqmaxflow}).
    \end{proof}
%       Without loss of generality, we can suppose that the turn in which we are interested is 0 ($\mathcal{G} =
%       \mathcal{G}_0$). First we will show that the $MaxFlow$ can be a result of a valid execution of algorithm
%       \ref{transitivegame} and afterwards we will show that each valid execution of algorithm \ref{transitivegame}
%       corresponds to a valid flow from $A$ to $B$. Thus we will have proven that $Tr_{A \rightarrow B} = maxFlow(A, B)$.
%       \begin{itemize}
%          \item We will first show that there exists an execution of algorithm \ref{transitivegame} such that $Loss_A =
%          maxFlow(A, B)$. Let $X$ be the flows as returned by an execution of the $MaxFlow(A, B)$ algorithm on $\mathcal{G}$.
%          It is known that all flows are DAGs [citation needed] and that all DAGs are a partial order of their nodes based on
%          the partial ordering $x_{vw} \leq 0 \Rightarrow v < w$ [citation needed]. From this partial order, we can create a
%          total order with an algorithm such as topoSort \cite{toposort}. The maximum element of the total order is a node
%          that does not have any outgoing flow. Removing any node from a DAG cannot create a cycle, thus the graph that
%          remains after removing a node from a DAG is also a DAG, thus it has a total order as well, which can be chosen to
%          be the same total order as before removing the node, except for the removed node. If the removed node was maximum
%          or minimum, the new total order is obvious. We will prove our claim using induction. \\
%          \begin{itemize}
%             \item Player $B$ is the maximum node in turn 0 because she is the sink of the MaxFlow algorithm, thus she is the
%             first to be chosen to play and steals all her incoming and outgoing trust. $\forall v \in N^{-}(B)_0, x_{vB}
%             \leq DTr_{v \rightarrow B, 0}$ and $\sum\limits_{v \in N^{-}(B)_0}x_{vB} = maxFlow(A, B)$. The graph
%             $\mathcal{G}_1 = \mathcal{G}_0 \setminus \{B\}$ is also a DAG and corresponds to the previous total order if we
%             remove the maximum element, $B$.
%             \item Suppose that $\forall j \in [k], k > 0$, the player $v$ corresponding to the maximum element is chosen to
%             play for the first time, that $\forall w \in N^{-}(v)_{j-1} (= N^{-}(v)_0), x_{wv} \leq y$ where $Steal(y,w) \in
%             Turn_j \wedge \sum\limits_{w \in N^{-}(v)_0}x_{wv} = \sum\limits_{w \in N^{+}(v)_0}x_{vw}$.
%             \item For $j = k+1$, $Player(k+1) = v'$ corresponds to the maximum element of the previous total order with the
%             element $v$ removed and it is the first time player $v'$ plays, since $v > v'$ in all previous steps thus $v'$
%             was not maximum. It also holds that $\forall w \in N^{-}(v')_0, x_{wv'} \leq DTr_{w \rightarrow v', 0}$ since
%             the $x_{wv'}$ are chosen by the maxFlow algorithm with corresponding capacities the direct trusts and, since
%             $\sum\limits_{w \in N^{-}(v')_0}x_{wv'} = \sum\limits_{w \in N^{+}(v')_0}x_{v'w}$ and player $v'$ has already
%             been stolen value equal to $\sum\limits_{w \in N^{+}(v')_0}x_{v'w}$ (since she has no outgoing flow in turn
%             $j$), player $v'$ can choose to steal from each player $w \in N^{-}(v')_0$ value at least equal to $x_{wv'}$
%             without violating the conservative strategy.
%          \end{itemize}
%          We have proven using induction that if the algorithm chooses only maximum nodes, after exactly $|V(\mathcal{G}_0)|-
%          1$ turns (we do not count idle player $A$) every player except for $A$ will have stolen at least value equal to the
%          flow passing through them and player $A$ will have been stolen value exactly equal to $maxFlow(A, B) \Rightarrow
%          Loss_A = maxFlow(A, B)$.
%          \item We will now show that for any valid execution of algorithm \ref{transitivegame} there exists at least one
%          valid flow from $A$ to $B$, such that $Loss_A = \sum\limits_{v \in N^{+}(A)_0}x_{Av}$. Let $j$ be a turn where
%          \ref{transitivegame} has converged ($j$ exists, according to theorem \ref{convergence}). Then $Loss_{A, j} =
%          out_{A, 0} - out_{A, j}$. We create a new graph $\mathcal{G}'$ such that $V(\mathcal{G}') = V(\mathcal{G}) \cup
%          \{T\}, E(\mathcal{G}') = E(\mathcal{G}) \cup \{(T, v) : v \in Sad_j\} \cup (T, A), \forall (v, w) \in E(\mathcal{G}),
%          c'_{vw} = DTr_{v \rightarrow w, 0} - DTr_{v \rightarrow w, j}, \forall v \in Sad_j, c'_{Tv} = c'_{TA} = \infty$
%          ($T$ is an auxiliary source that trusts infinitely $A$ and all the $Sad$ nodes). We execute the $MaxFlow(T, B)$
%          algorithm on $\mathcal{G}'$ and we get a flow $X' : \forall v,w \in V(\mathcal{G}), x'_{vw} = c'_{vw}$ (all the
%          edges, except for the auxiliary ones, saturated). Thus $\sum\limits_{v \in N^{+}(A)_0}x'_{Av} = Loss_{A, j}$.
%          If we create a new graph $\mathcal{G}''$ with $V(\mathcal{G}'') = V(\mathcal{G}'), E(\mathcal{G}'') =
%          E(\mathcal{G}'), \forall v \in Sad_j, c''_{Tv} = 0, c''_{TA} = \infty, \forall v,w \in V(\mathcal{G}), c''_{vw} =
%          c_{vw}$ (the auxiliary node trusts only $A$) and execute the $MaxFlow(T, B) = X''$ algorithm on $\mathcal{G}''$,
%          $\sum\limits_{v \in N^{+}(A)_0}x''_{Av} = \sum\limits_{v \in N^{+}(A)_0}x'_{Av}$ (the outgoing flow from
%          $A$ will remain the same as in $\mathcal{G}'$) since no capacity accesible from $A$ has been modified (the only
%          changed capacities are those that begin from $T$ and there is no incoming flow to $T$) thus $\sum\limits_{v \in
%          N^{+}(A)_0}x''_{Av} \geq \sum\limits_{v \in N^{+}(A)_0}x'_{Av}$ and $\sum\limits_{v \in N^{+}(A)_0}x'_{Av} =
%          \sum\limits_{v \in N^{+}(A)_0}c'_{Av} = \sum\limits_{v \in N^{+}(A)_0}c''_{Av}$ (the outgoing edges from $A$ were
%          already saturated) thus $\sum\limits_{v \in N^{+}(A)_0}x''_{Av} \leq \sum\limits_{v \in N^{+}(A)_0}x'_{Av}$. Thus
%          the resulting flow is equal to $Loss_{A, j}$, or $\sum\limits_{v \in N^{+}(A)_0}x''_{Av} = \sum\limits_{v \in
%          N^{+}(A)_0}c'_{Av} = Loss_{A, j}$.
%       \end{itemize}
%       We finally conclude that $Tr_{A \rightarrow B} = MaxFlow(A, B)$.
%          \begin{itemize}
%             \item The flow $X$ is obviously valid for the initial graph because $\forall (v,w) \in E(FG), c(v,w) = DTr_{v
%             \rightarrow w, 0} \geq DTr_{v \rightarrow w, 0} - DTr_{v \rightarrow w, j} = c'(v,w) \geq x_{vw}$ and it
%             already holds that $\forall v \in V(FG'), \sum\limits_{w \in N^{-}(v)}x'_{wv} = \sum\limits_{w \in N^{+}(v)}
%             x'_{vw}$, thus it also holds for the flows of $X$.
%             \item We can easily see that $Loss_{A,j} \geq \sum\limits_{v \in N{+}(A)}x_{Av}$ because $Loss_{A,j} =
%             out_{A,0} - out_{A,j} = \sum\limits_{v \in N^{+}(A)}c'(A,v)$. To show that $Loss_{A,j} \leq
%             \sum\limits_{v \in N{+}(A)}x_{Av}$, we first suppose that $Loss_{A,j} > \sum\limits_{v \in N{+}(A)}x_{Av}$. We
%             will now prove that there exists a residual path from $A$ to $B$. $Loss_{A,j} = \sum\limits_{v \in N^{+}(A)}
%             (DTr_{A \rightarrow v, 0} - DTr_{A \rightarrow v, j}) = \sum\limits_{v \in N^{+}(A)}c_{Av}$. From the
%             supposition we can see that $\sum\limits_{v \in N^{+}(A)}c_{Av} > \sum\limits_{v \in N^{+}(A)}x_{Av}
%             \Rightarrow \exists v \in N^{+}(A) : c_{Av} > x_{Av}$. \\
%             Since $\forall v \in V(FG) \setminus \{A,B\}, \sum\limits_{w \in N^{-}(v)}c_{wv} \overset{conservative}{=}
%             \sum\limits_{w \in N^{+}(v)}c_{vw} \wedge \sum\limits_{w \in N^{-}(v)}x_{wv} \overset{flow}{=}
%             \sum\limits_{w \in N^{+}(v)}x_{vw}$, it holds that $\forall v \in V(FG) \setminus \{A,B\}, \sum\limits_{w \in
%             N^{-}(v)}(c_{wv} - x_{wv}) = \sum\limits_{w \in N^{+}(v)}(c_{vw} - x_{vw})$. \\
%             We will now show that $\forall v \in V(FG) \setminus
%             \{A,B\}, (\exists w \in V(FG) : c_{wv} > x_{wv} \Rightarrow \exists u \in V(FG) : c_{vu} > x_{vu})$. Suppose
%             that the previous statement is false. Then it would hold that $\exists v \in V(FG) \setminus \{A,B\} :
%             (\exists w \in V(FG) : c_{wv} > x_{wv} \wedge \forall u \in V(FG), c_{vu} = x_{vu})$ (1). But then we have
%             $\sum\limits_{w \in V(FG)}c_{wv} \overset{(1)}{>} \sum\limits_{w \in V(FG)}x_{wv} \overset{flow}{=}
%             \sum\limits_{w \in V(FG)}x_{vw} \overset{(1)}{=} \sum\limits_{w \in V(FG)}c_{vw} \overset{conservative}{=}
%             \sum\limits_{w \in V(FG)}c_{wv} \Rightarrow \sum\limits_{w \in V(FG)}c_{wv} > \sum\limits_{w \in V(FG)}c_{wv}$
%             which is a contradiction. Thus we showed that $\forall v \in V(FG) \setminus \{A,B\}, (\exists w \in V(FG) :
%             c_{wv} > x_{wv} \Rightarrow \exists u \in V(FG) : c_{vu} > x_{vu})$. \\
%             The flow graph that resulted from
%             $MaxFlow(A, B)$ is a DAG, thus there exists a corresponding total ordering, as we saw before.
%             Obviously $A = v_0$ and $B = v_{|V(FG)|}$. When an element $v_k$ is in the $k$-th position in this total
%             ordering, it has incoming flow only from smaller elements and outgoing flow only to bigger elements, that is
%             $\forall l < k, x_{kl} = 0 \wedge \forall m > k, x_{mk} = 0$.
%             Thus the previous result can be rewritten this way: $\forall k \in [|V(FG)|],
%             (\exists l < k : c_{v_lv_k} > x_{v_lv_k} \Rightarrow \exists m > k : c_{v_kv_m} > x_{v_kv_m})$.
%             Thus, the supposition $Loss_{A, j} > \sum\limits_{v \in N^{+}(A)}x_{Av}$ combined with the previous result shows
%             that there exists a residual path from $A$ to $B$ since we can start from $A$ and find a series of sequential
%             edges that all have flows smaller than the corresponding capacities and eventually reach $B$ in at most
%             $|E(FG)|$ steps, thus $X'$ is not a maximum flow, which is a contradiction. Thus $Loss_{A,j} \leq
%             \sum\limits_{v \in N{+}(A)}x_{Av}$ and, since also $Loss_{A,j} \geq \sum\limits_{v \in N{+}(A)}x_{Av}$, we
%             deduce that $Loss_{A,j} = \sum\limits_{v \in N{+}(A)}x_{Av}$.
%          \end{itemize}
%2nd bullet
%          \item We will now show that for any valid execution of algorithm \ref{transitivegame} there exists at least one
%          valid flow from $A$ to $B$, $X$, such that $Loss_A = \sum\limits_{v \in N^{+}(A)}x_{Av}$. Let $j$ be a turn where
%          \ref{transitivegame} has converged ($j$ exists, according to theorem \ref{convergence}). Then $Loss_{A, j} =
%          out_{A, 0} - out_{A, j}$. Let $\forall v \in N^{+}(A)_0, x_{Av} = DTr_{A \rightarrow v, 0} - DTr_{A \rightarrow
%          v, j}$. For any conservative player $v \in N^{+}(A)_0$, let $\forall w \in N^{+}(v)_0, x_{vw} \leq DTr_{v
%          \rightarrow w, 0} - DTr_{v \rightarrow w, j}, \sum\limits_{w \in N^{+}(A)_0}x_{vw} = x_{Av}$. This is possible
%          because $v$ is conservative, thus the value she stole from $A$ must have been stolen previously from her. More
%          generally, $\forall v \in \mathcal{V}_0 \setminus \{A,B\}, \forall w \in N^{+}(v)_0, x_{vw} \leq
%          DTr_{v \rightarrow w, 0} - DTr_{v \rightarrow w, j}, \sum\limits_{w \in N^{+}(v)_0}x_{vw} = \sum\limits_{w \in
%          N^{-}(v)_0}x_{wv}$. Since the graph we build is a DAG in every step, which corresponds to a partial order, there
%          always exists a total order that we can get using an algorithm such as topoSort [citation needed]. Thus, by
%          choosing to calculate the outgoing flows only of the minimum element of this total order, it is possible to create
%          a valid flow network from $A$ to $B$ in exactly $|V(FG)| - 1$ iterations of the above steps.
%OLD
%          \item The flow to $A$ is the flow that results from the following process: After the execution of
%          \ref{transitivegame}, for each sad player iteratively replenish the $DTr$ stolen from the sad player by the one
%          that stole from her (if multiple players stole from the sad player, then replenish all the stolen $DTr$). Repeat
%          the process until the evil player replenishes the initially stolen $DTr$. This is always possible because if there
%          is no player who stole from each one who is replenished, then the $Steal()$ she did in the first place would not be
%          according to the conservative strategy. Also this process will end with the evil player replenishing $DTr$ equal
%          to the sum of $DTr$ that was stolen from sad players because the conservative players cannot avoid replenishing,
%          or else they do not follow the conservative strategy. The $DTr$ stolen from $A$ will not be replenished, since
%          the player(s) that have stolen from $A$ will not replenish the stolen value and, inductively, this value will not
%          be replenished. Thus $A$ will have been stolen the exact same value that the modified evil player has stolen,
%          $\forall w,v \in V(FG), DTr_{v \rightarrow w} \geq x_{vw}$ (1st requirement for flows) and there would be no node
%          that gets more flow than it pushes, except for $A$ and $B$ (2nd requirement for flows), thus it is a valid flow.
%          \item Let $X$ be the flows as returned by an execution of the $maxFlow$ algorithm. The evil player can steal
%          the values denoted by $X$ and every other player can steal exactly as much as the $X$ flows denote, since they
%          have the 1st property and thus are stealable in any strategy and also hold the 2nd property, thus they comply with
%          the conservative strategy. More concretely, $\forall v,w \in V(FG), DTr_{v \rightarrow w}' = x_{vw}$. Then the two
%          properties of flows hold:
%          \begin{itemize}
%             \item $\forall v,w \in V(FG),x_{vw} \leq DTr_{v \rightarrow w}$ and thus any set of strategies that include only
%             $Steal()$ actions such that $\sum\limits_{y : Steal(y,w) \in Turn_j, Player(j) = v}y = DTr_{v \rightarrow w} -
%             x_{vw}$ is feasible.
%             \item $\forall v \in V(FG) \setminus \{A,B\}, \sum\limits_{w \in N^{+}(v)}x_{wv} =
%             \sum\limits_{w \in N^{-}(v)}x_{vw}$ thus $\forall v \in V(FG) \setminus \{A,B\}, Strategy(v) = Conservative$.
%          \end{itemize}
%             
%       \end{itemize}
%       Thus the maximum value $A$ can lose if $B$ is evil is $Tr_{A \rightarrow B} = maxFlow(A, B)$.
%\ \\ OLDER
%       \begin{enumerate}
%	   \item We will show that $Tr_{A \rightarrow B} \leq MaxFlow(A, B)$.
%          We know that $MaxFlow(A, B) = MinCut_{A \rightarrow B}$. We will show that, if everybody except
%          A and B follows the conservative strategy,  $Tr_{A \rightarrow B} \leq MinCut_{A \rightarrow B}$. Suppose that in
%          round $i$ all the members of the MinCut, $P$, have stolen the maximum value they can from members that belong
%          in the MaxFlow graph and nobody in the partition in which $A$ belongs has stolen yet any value. Let the total
%          stolen value from the MinCut members be $St$. It is obvious that $St_i \leq MinCut_{A \rightarrow B}$, because
%          otherwise there would exist $u \in P$ that doesn't follow the conservative strategy, since they stole more than they
%          were stolen from. The same argument holds for any round $i' > i$ because in each round an conservative player can
%          steal only up to the value she has been stolen. It is also impossible that the $St$ increase further due to
%          stolen value from members of the partition of $B$ since members of $P$ disconnect the two partitions and have
%          already played their turns, thus $\forall i' > i, St_{i'} \leq St_i$. There exists a round, $k$, when all the
%          conservative players stop stealing, so in the worst case $A$ will have been stolen
%          $Tr_{A \rightarrow B} = St_k \leq MinCut_{A \rightarrow B} = MaxFlow(A, B)$.
%          \item We can see that $Tr_{A \rightarrow B} \geq MaxFlow(A, B)$ because the strategy where each
%          one of the non-idle players steals value equal to the incoming flows from their respective friends is a valid
%          strategy that does not contradict with the conservative strategy, since for every conservative player $w$ it holds that
%          $\sum\limits_{v \in N^{-}(w)}x_{vw} = \sum\limits_{v \in N^{+}(w)}x_{wv}$ and according to the strategy each
%          conservative player will have been stolen value equal to $\sum\limits_{v \in N^{+}(w)}x_{wv}$. More concretely,
%          let $Player(j) = B$ and $Player(j+d) = C :$
%       \end{enumerate}
%       Combining the two results, we see that $Tr_{A \rightarrow B} = MaxFlow(A, B)$.
%        OLD PROOF START
%        \begin{enumerate}
%           \item $Tr_{A \rightarrow B} \geq MaxFlow(A, B)$ because by the definition of $Tr_{A \rightarrow B}$,
%           B leaves taking with him all the incoming trust, so there is no trust flowing towards him after leaving.
%           $Tr_{A \rightarrow B} < MaxFlow(A, B)$ would imply that after B left, there would still remain trust
%           flowing from A to B.
%           \item $Tr_{A \rightarrow B} \leq MaxFlow(A, B)$ \\
%           Suppose that $Tr_{A \rightarrow B} > MaxFlow(A, B)$ (1). Then, using the min cut - max flow theorem we
%           see that there is a set of capacities $U= \{u_1,\dots,u_n\}$ with flows $X = \{x_1,\dots,x_n\}$ such that
%           $\sum\limits_{i=1}^{n}{x_i} = MaxFlow(A, B)$ and, if severed $(\forall i \in [n] \: u_i' = 0)$
%           the flow from A to B would be $0$, or, put differently, there would be no directed trust path from A to B. No
%           strategy followed by B could reduce the value of A, so our supposition (1) cannot be true.
%        \end{enumerate}
%        OLD PROOF END


     Note: The maxFlow is the same in the following two cases: When a player chooses the evil strategy and when the same
     player chooses a variation of the evil strategy where she does not nullify her outgoing direct trust.
     \begin{theorem}[Risk Invariance]
     \label{riskinv}
        Let $\mathcal{G}$ game graph, $A, B \in \mathcal{V}$ and $V$ the desired value to be transferred from $A$ to $B$,
        with $V \leq Tr_{A \rightarrow B}$. Let also $\mathcal{G}'$ such that
        \begin{align}
           \mathcal{V}' &= \mathcal{V} \\
%           DTr'_{A \rightarrow B} &= DTr_{A \rightarrow B} \\
           \forall v \in \mathcal{V}' \setminus \{A\}, \forall w \in \mathcal{V}'&, DTr'_{v \rightarrow w} =
           DTr_{v \rightarrow w} \enspace.
        \end{align}
        Furthermore, suppose that there exists an assignment for the outgoing trust of $A, DTr'_{A \rightarrow v}$, such that
        \begin{equation}
        \label{primetrust}
           Tr'_{A \rightarrow B} = Tr_{A \rightarrow B} - V \enspace.
        \end{equation}
        Let another game graph, $\mathcal{G}''$, be identical to $\mathcal{G}'$ except for the following change:
        \begin{equation}
           DTr''_{A \rightarrow B} = DTr'_{A \rightarrow B} + V \enspace.
        \end{equation}
        It then holds that
        \begin{equation}
           Tr''_{A \rightarrow B} = Tr_{A \rightarrow B} \enspace.
        \end{equation}
     \end{theorem}
     \begin{proofsketch}
        The two graphs $\mathcal{G}'$ and $\mathcal{G}''$ differ only on the weight of the edge $\left(A, B\right)$, which is
        larger by $V$ in $\mathcal{G}''$. Thus the two $MaxFlow$s will choose the same flow, except for $\left(A, B\right)$,
        where it will be $x''_{AB} = x'_{AB} + V$.
     \end{proofsketch}
     \begin{sepproof} (Risk Invariance Theorem (\ref{riskinv}))
        Let
        \begin{align}
           \forall v,w \in \mathcal{V}', c'_{vw} &= DTr'_{v \rightarrow w} \mbox{ and} \\
           \forall v,w \in \mathcal{V}'', c''_{vw} &= DTr''_{v \rightarrow w} \enspace.
        \end{align}
        Then
        \begin{equation}
        \label{ccompare}
           \forall v, w \in \mathcal{V}, c'_{vw} \leq c''_{vw}
        \end{equation}
        and any valid flow on $\mathcal{G}'$ is a valid flow on $\mathcal{G}''$ as well. Furthermore, any
        $MaxFlow\left(A, B\right)$ chooses $x_{AB} = c_{AB}$, thus
        \begin{equation}
        \label{xcompare}
           x''_{AB} = x'_{AB} + V \enspace.
        \end{equation}
        From (\ref{ccompare}) and (\ref{xcompare}) we see that
        \begin{equation}
        \label{doublebigger}
           maxFlow_{\mathcal{G}''}\left(A, B\right) \geq maxFlow_{\mathcal{G}'}\left(A, B\right) + V \enspace.
        \end{equation}
        Now suppose that
        \begin{equation}
        \label{mfsupposition}
           maxFlow_{\mathcal{G}''}\left(A, B\right) > maxFlow_{\mathcal{G}'}\left(A, B\right) + V \enspace.
        \end{equation}
        Then 
        \begin{equation}
           \sum\limits_{v \in {N''}^{-}\left(B\right) \setminus \{A\}}x''_{vB} > \sum\limits_{v \in {N'}^{-}\left(B\right)
           \setminus \{A\}}x'_{vB} \enspace.
        \end{equation}
        However, it holds that
        \begin{equation}
        \label{cequal}
           \forall e \in \mathcal{V} \setminus \{\left(A, B\right)\}, c'_e = c''_e \enspace,
        \end{equation}
        and $x_{AB}$ flows directly from $A$ to $B$ without adding to the incoming or outgoing flow of any intermediate node,
        thus $MaxFlow_{\mathcal{G}''}$ can choose
        \begin{equation}
           \forall e \in \mathcal{V} \setminus \{\left(A, B\right)\}, x''_e = x'_e
        \end{equation}
        and thus, by contradiction with (\ref{mfsupposition}), it holds that
        \begin{equation}
        \label{singlebigger}
           maxFlow_{\mathcal{G}''}\left(A, B\right) \leq maxFlow_{\mathcal{G}'}\left(A, B\right) + V \enspace.
        \end{equation}
        From (\ref{doublebigger}) and (\ref{singlebigger}) we get
        \begin{equation}
        \label{mfequal}
           maxFlow_{\mathcal{G}''}\left(A, B\right) = maxFlow_{\mathcal{G}'}\left(A, B\right) + V \enspace.
        \end{equation}
        Finally, it holds that
        \begin{equation}
        \begin{gathered}
           Tr''_{A \rightarrow B} = maxFlow_{\mathcal{G}''}\left(A, B\right) \overset{\left(\ref{mfequal}\right)}{=} \\
           = maxFlow_{\mathcal{G}'}\left(A, B\right) + V = Tr'_{A \rightarrow B} + V
           \overset{\left(\ref{primetrust}\right)}{=} Tr_{A \rightarrow B} \enspace.
        \end{gathered}
        \end{equation}
        The proposition is proved.
     \end{sepproof}
  \section{Sybil Resilience}
     One of the primary aims of this system is to mitigate the danger for sybil attacks [citation needed] whilst maintaining
     fully decentralized autonomy.
     \begin{definition}[Indirect Trust to Multiple Players]
        The indirect trust from player $A$ to a set of players, $S \subset \mathcal{V}$ is defined as the maximum possible
        value that can be stolen from $A$ if all players in $S$ follow the evil strategy, $A$ follows the idle strategy and
        everyone else ($\mathcal{V} \setminus \left(S \cup \{A\}\right)$) follows the conservative strategy. More formally,
        if $S \subset \mathcal{V}$,
        \begin{equation}
        \begin{gathered}
           Strategy\left(A\right) = Idle \wedge \forall E \in S, Strategy\left(E\right) = Evil \wedge \\
           \wedge \forall v \in \mathcal{V} \setminus \left(S \cup \{A\}\right), Strategy\left(v\right) = Conservative
        \end{gathered}
        \end{equation}
        and $choices$ are the different actions between which the conservative players can choose, then
        \begin{equation}
           Tr_{A \rightarrow S, j} = \max\limits_{j' : j' > j, configurations}{\left[out_{A,j} - out_{A,j'}\right]}
        \end{equation}
     \end{definition}
    \begin{theorem}[Multi-Player Trust Flow] \ \\
       \label{trustmany}
       Let $S \subset \mathcal{V}$ and $T$ auxiliary player such that
       \begin{equation}
          \forall B \in S, DTr_{B \rightarrow T} = \infty \enspace.
       \end{equation}
       It holds that
       \begin{equation}
          \forall A \in \mathcal{V} \setminus S, Tr_{A \rightarrow S} = maxFlow\left(A, T\right) \enspace.
       \end{equation}
    \end{theorem}       
    \begin{proof}
       If $T$ chooses the evil strategy and all players in $S$ play according to the conservative strategy, they will have to
       steal all their incoming direct trust, thus they will act in a way identical to following the evil strategy as far as
       $MaxFlow$ is concerned. The theorem follows thus from the Trust Flow Theorem.
%       \begin{equation}
%          \forall A \in \mathcal{V} \setminus S, Tr_{A \rightarrow T} = maxFlow\left(A, T\right) = Tr_{A \rightarrow S}
%          \enspace.
%       \end{equation}
    \end{proof}
     Let Eve be a possible attacker.
     \begin{definition}[Corrupted Set]
        Let $\mathcal{G}$ be a game graph and let Eve have a set of players $\mathcal{B} \subset \mathcal{V}$ corrupted, so
        that she fully controls their outgoing direct trusts to any player in $\mathcal{V}$ and can also steal all incoming
        direct trust to players in $\mathcal{B}$. We call this the corrupted set. The players $\mathcal{B}$ are considered to
        be legitimate before the corruption, thus they may be directly trusted by any player in $\mathcal{V}$.
     \end{definition}
     \begin{definition}[Sybil Set]
        Let $\mathcal{G}$ be a game graph. Since participation in the network does not require any kind of registration, Eve
        can create any number of players. We will call the set of these players $\mathcal{C}$, or Sybil set. Moreover, Eve
        %can invest any amount she chooses, thus she
        can arbitrarily set the direct trusts of any player in $\mathcal{C}$ to any player and can also steal all
        incoming direct trust to players in $\mathcal{C}$. However, players $\mathcal{C}$ can be directly trusted only by
        players $\mathcal{B} \cup \mathcal{C}$ but not by players $\mathcal{V} \setminus (\mathcal{B} \cup \mathcal{C})$,
        where $\mathcal{B}$ is a set of players corrupted by Eve.
     \end{definition}
     \begin{definition}[Collusion]
        Let $\mathcal{G}$ be a game graph. Let $\mathcal{B} \subset \mathcal{V}$ be a corrupted set and $\mathcal{C} \subset
        \mathcal{V}$ be a Sybil set, both controlled by Eve. The tuple $\left(\mathcal{B}, \mathcal{C}\right)$ is called a
        collusion and is entirely controlled by a single entity in the physical world. From a game theoretic point of view,
        players $\mathcal{V} \setminus (\mathcal{B} \cup \mathcal{C})$ perceive the collusion as independent players with a
        distinct strategy each, whereas in reality they are all subject to a single strategy dictated by the controlling
        entity, Eve.
     \end{definition}
%    In short,
%    \begin{equation}
%      \forall D \in \mathcal{B} \cup \mathcal{C}, P \in \mathcal{V}, \mbox{ Eve controls } DTr_{D \rightarrow P}, \mbox{ can
%      steal } DTr_{P \rightarrow D}
%    \end{equation}
%    and can change $|\mathcal{C}|$.
%    Furthermore, the following restriction is true:
%    \begin{equation}
%       \forall C \in \mathcal{C}, A \in \mathcal{V} \setminus \left(\mathcal{B} \cup \mathcal{C}\right), DTr_{A \rightarrow
%       C} = 0 \enspace.
%    \end{equation}
    \begin{theorem}[Sybil Resilience] \ \\
       \label{sybil}
%       Let $\mathcal{B} \cup \mathcal{C} \subset \mathcal{V}$ with $\mathcal{B} \cap \mathcal{C} = \emptyset$ be a collusion
%       of players who are controlled by an adversary, Eve. Eve also controls the number of players in the Sybil set,
%       $|\mathcal{C}|$. Players in $\mathcal{C}$ are not directly trusted by players outside the collusion,
%       contrary to players in $\mathcal{B}$, the corrupted set, who may be directly trusted by any player in
%       $\mathcal{V}$. 
       Let $\mathcal{G}$ be a game graph and $\left(\mathcal{B}, \mathcal{C}\right)$ be a collusion of players on
       $\mathcal{G}$. It holds that
       \begin{equation}
          Tr_{A \rightarrow \mathcal{B} \cup \mathcal{C}} = Tr_{A \rightarrow \mathcal{B}} \enspace.
       \end{equation}
    \end{theorem}
    \begin{proofsketch}
       The incoming trust to $\mathcal{B} \cup \mathcal{C}$ cannot be higher than the incoming trust to $\mathcal{B}$ since
       $\mathcal{C}$ has no incoming trust from players outside the collusion. For the complete proof, see
       Appendix (Proof \ref{sybilproof}).
    \end{proofsketch}
    We have proven that controlling $|\mathcal{C}|$ is irrelevant for Eve, thus Sybil attacks are meaningless.

  \section{Related Work}

  \section{Further Research}
     While our trust network can form a basis for risk-invariant transactions in the anonymous and decentralized setting,
     more research is required to achieve other desirable properties. Some directions for future research are outlined below.

  \subsection{Zero knowledge}
     Our network evaluates indirect trust by computing the max flow in the graph of lines-of-credit. In order to do that,
     complete information about the network is required. However, disclosing the network topology may be undesirable, as it
     subverts the identity of the participants even when participants are treated pseudonymously, as deanonymisation
     techniques can be used \cite{deanonymisation}. To avoid such issues, exploring the ability to calculate flows in a zero
     knowledge fashion may be desirable. However, performing network queries in zero knowledge may allow an adversary to
     extract topological information. More research is required to establish how flows can be calculated effectively in zero
     knowledge and what bounds exist in regards to information revealed in such fashion.

     The current description of TrustIsRisk refers to a static setting where the game evolves in turns. In each turn only one
     user changes the state of the network. In the dynamic setting, the users should be able to play simultaneously, freely
     join, leave and disconnect temporarily from the network.

  \section*{Appendix}
    \subsection{Common Notation}
     \begin{definition}[Assets]
        Sum of $A$'s capital and outgoing trust.
        \begin{equation}
           As_{A, j} = Cap_{A, j} + out_{A, j}
        \end{equation}
        If the turn we refer to is obvious, it is possible to omit $j$.
     \end{definition}
     \begin{definition}[Neighbourhood]
        \label{neighbourhood}
        \begin{enumerate}
           \item Let $N^{+}\left(A\right)_j$ be the set of players $B$ that $A$ directly trusts with any positive value at
              the end of turn $j$. More formally,
              \begin{equation}
                 N^{+}\left(A\right)_j = \{B \in \mathcal{V}_j : DTr_{A \rightarrow B, j} > 0\} \enspace.
              \end{equation}
              $N^{+}\left(A\right)_j$ is called out neighbourhood of $A$ on turn $j$. Let $S \subset \mathcal{V}_j$.
              \begin{equation}
                 N^{+}\left(S\right)_j = \bigcup\limits_{A \in S}N^{+}\left(A\right)_j
              \end{equation}
           \item Let $N^{-}\left(A\right)_j$ be the set of players $B$ that directly trust $A$ with any positive value at the
              end of turn $j$. More formally,
              \begin{equation}
                 N^{-}\left(A\right)_j = \{B \in \mathcal{V}_j : DTr_{B \rightarrow A, j} > 0\} \enspace.
              \end{equation}
              $N^{-}\left(A\right)_j$ is called in neighbourhood of $A$ on turn $j$. Let $S \subset \mathcal{V}_j$.
              \begin{equation}
                 N^{-}\left(S\right)_j = \bigcup\limits_{A \in S}N^{-}\left(A\right)_j
              \end{equation}
           \item Let $N\left(A\right)_j$ be the set of players $B$ that either directly trust or are directly trusted by $A$
              with any positive value at the end of turn $j$. More formally,
              \begin{equation}
                 N\left(A\right)_j = N^{+}\left(A\right)_j \cup N^{-}\left(A\right)_j \enspace.
              \end{equation}
              $N\left(A\right)_j$ is called neighbourhood of $A$ on turn $j$. Let $S \subset \mathcal{V}_j$.
              \begin{equation}
                 N\left(S\right)_j = N^{+}\left(S\right)_j \cup N^{-}\left(S\right)_j
              \end{equation}
%           \item Let $S \subset \mathcal{V}_j$. Let $N\left(A\right)_{j,i}$ (respectively $N^{+}\left(A\right)_{j,i},
%              N^{-}\left(A\right)_{j,i}, N\left(S\right)_{j,i},$ $N^{+}\left(S\right)_{j,i}, N^{-}\left(S\right)_{j,i}$) be
%              the $i$-th element of set $N\left(A\right)_j$ (respectively of $N^{+}\left(A\right)_j,  N^{-}\left(A\right)_j,
%              N\left(S\right)_j, N^{+}\left(S\right)_j, N^{-}\left(S\right)_j$), according to an arbitrary but fixed
%              enumeration of the set players.
        \end{enumerate}
     \end{definition}
     \begin{definition}[Total Incoming/Outgoing Trust]
     \label{inouttrust}
        \begin{equation}
           in_{A, j} = \sum\limits_{v \in N^{-}\left(A\right)_j}DTr_{v \rightarrow A, j}
        \end{equation}
        \begin{equation}
           out_{A, j} = \sum\limits_{v \in N^{+}\left(A\right)_j}DTr_{A \rightarrow v, j}
        \end{equation}
     \end{definition}
     
     Let $A = Player(j)$. $Turn_j$ Examples:
     \begin{enumerate}
        \item \begin{equation}
           Turn_j = \emptyset
        \end{equation}
        \item \begin{equation}
           Turn_j = \{Steal\left(y, B\right), Add\left(w, B\right)\} \enspace,
        \end{equation}
        given that
        \begin{equation}
           DTr_{B \rightarrow A, j_2 - 1} \leq y \wedge -DTr_{A \rightarrow B, j_2 - 1} \leq w \wedge y - w \leq
           Cap_{A, j_2-1} \enspace.
        \end{equation}
        \item \begin{equation}
           Turn_j = \{Steal\left(x, B\right), Add\left(y, C\right), Add\left(w, D\right)\} \enspace,
        \end{equation}
        given that
        \begin{equation}
        \begin{gathered}
           DTr_{B \rightarrow A, j_3 - 1} \leq x \wedge -DTr_{A \rightarrow C, j_3-1} \leq y \wedge \\
           \wedge -DTr_{A \rightarrow D, j_3 - 1} \leq w \wedge x - y - w \leq Cap_{A, j_3-1} \enspace.
        \end{gathered}
        \end{equation}
        \item \begin{equation}
           Turn_j = \{Steal\left(x, B\right), Steal\left(y, B\right)\}
        \end{equation}
        is not a valid turn because it contains two $Steal\left(\right)$ actions against the same player. If
        \begin{equation}
           x + y \leq DTr_{B \rightarrow A} \enspace,
        \end{equation}
        the correct alternative would be
        \begin{equation}
           Turn_j = \{Steal\left(x+y, B\right)\} \enspace.
        \end{equation}
     \end{enumerate}
     \begin{definition}[Previous/Next Turn]
        Let $j \in \mathbb{N}$ a turn with $Player\left(j\right)$ $= A$. We define $prev\left(j\right), next\left(j\right)$
        as the previous and next turn that $A$ is chosen to play respectively. If $j$ is the first turn that $A$ plays,
        $prev\left(j\right) = 0$. More formally, if
        \begin{equation}
           P = \{k \in \mathbb{N} : k < j \wedge Player\left(k\right) = A\} \mbox{ and}
        \end{equation}
        \begin{equation}
           N = \{k \in \mathbb{N} : k > j \wedge Player\left(k\right) = A\} \enspace,
        \end{equation}
        then we define $prev\left(j\right), next\left(j\right)$ as follows:
        \begin{equation}
           prev\left(j\right) = \begin{cases}
              \max{P}, & P \neq \emptyset \\
              0, & P = \emptyset
           \end{cases}
        \end{equation}
        \begin{equation}
           next\left(j\right) = \min{N}
        \end{equation}
        $next\left(j\right)$ is always well defined with the assumption that eventually everybody plays.
     \end{definition}
    \subsection{Proofs, Lemmas and Theorems}
    \begin{lemma}[$Loss$ Equivalent to $Damage$] \ \\
       Let $j \in \mathbb{N}, v \in \mathcal{V}_j \setminus \{A, E\}, v = Player\left(j\right)$. It holds that
       \begin{equation}
          \min\left(in_{v, j}, Loss_{v, j}\right) = \min\left(in_{v, j}, Damage_{v, j}\right) \enspace.
       \end{equation}
    \end{lemma}
    \begin{proof}
       \begin{itemize}
          \item Let $v \in Happy_{j-1}$. Then
          \begin{enumerate}
             \item $v \in Happy_j$ because $Turn_{j} = \emptyset$,
             \item $Loss_{v, j} = 0$ because otherwise $v \notin Happy_j$,
             \item $Damage_{v, j} = 0$, or else any reduction in direct trust to $v$ would increase equally
             $Loss_{v, j}$ (line~\ref{trsteallossincrease}), which cannot be decreased again but during an Angry player's turn
             (line~\ref{trsteallossdecrease}).
             \item $in_{v, j} \geq 0$
          \end{enumerate}
          Thus
          \begin{equation}
             v \in Happy_{j-1} \Rightarrow \min\left(in_{v, j}, Damage_{v,j}\right) = \min\left(in_{v, j}, Loss_{v,j}\right)
             = 0 \enspace.
          \end{equation}
          \item Let $v \in Sad_{j-1}$. Then
          \begin{enumerate}
             \item $v \in Sad_j$ because $Turn_{j} = \emptyset$, 
             \item $in_{v, j} = 0$ (lines~\ref{trstealifentersad} -~\ref{trstealtrueentersad}),
             \item $Damage_{v, j} \geq 0 \wedge Loss_{v, j} \geq 0$.
          \end{enumerate}
          Thus
          \begin{equation}
             v \in Sad_{j-1} \Rightarrow \min\left(in_{v, j}, Damage_{v,j}\right) = \min\left(in_{v, j}, Loss_{v,j}\right) =
             0 \enspace.
          \end{equation}
          \item Let $v \in Angry_{j-1} \wedge v \in Happy_j$. Then the same argument as in the first case holds, if
          we ignore the argument (1).
          \item Let $v \in Angry_{j-1} \wedge v \in Sad_j$. Then the same argument as in the second case holds, if 
          we ignore the argument (1).
       \end{itemize}
       Thus the theorem holds in every case.
    \end{proof}

    \begin{sepproof}[Trust Convergence Theorem (\ref{convergence})] \ \\
    \label{convergenceproof}
       First of all,
       \begin{equation}
          \forall j > j_0 : Player\left(j\right) = E \Rightarrow Turn_j = \emptyset
       \end{equation}
        because $E$ has already nullified his incoming and outgoing direct trusts in $Turn_{j_0}$, the evil strategy does not
       contain any case where direct trust is increased or where the evil player starts directly trusting another player and
       the other players do not follow a strategy in which they can choose to $Add\left(\right)$ trust to $E$, thus player
       $E$ can do nothing $\forall j > j_0$. We also see that
       \begin{equation}
          \forall j > j_0 : Player(j) = A \Rightarrow Turn_j = \emptyset
       \end{equation}
       because of the idle strategy that $A$ follows. As far as the rest of the players are concerned, consider the
       Transitive Game. As we can see from lines~\ref{trsteallossinit} and~\ref{trsteallossincrease}
       -~\ref{trsteallossdecrease}, it is
       \begin{equation}
          \forall j, \sum\limits_{v \in \mathcal{V}_j}Loss_v = in_{E, j_0-1} \enspace.
       \end{equation}
       In other words, the total loss is constant and equal to the total value stolen by $E$. Also, as we can see in
       lines~\ref{trstealsadinit} and~\ref{trstealtrueentersad}, which are the only lines where the $Sad$ set is modified,
       once a player enters the $Sad$ set, it is impossible to exit from this set. Also, we can see that players in $Sad
       \cup Happy$ always pass their turn. We will now show that eventually the $Angry$ set will be empty, or equivalently
       that eventually every player will pass their turn. Suppose that it is possible to have an infinite amount of turns
       in which players do not choose to pass. We know that the number of nodes is finite, thus this is possible only if
       \begin{equation}
          \exists j': \forall j \geq j', |Angry_j \cup Happy_j| = c > 0 \wedge Angry_j \neq \emptyset \enspace.
       \end{equation}
       This statement is valid because the total number of angry and happy players cannot increase because no player leaves
       the $Sad$ set and if it were to be decreased, it would eventually reach 0. Since $Angry_j \neq \emptyset$, a player
       $v$ that will not pass her turn will eventually be chosen to play. According to the Transitive Game, $v$ will either
       deplete her incoming trust and enter the $Sad$ set (line~\ref{trstealtrueentersad}), which is contradicting $|Angry_j
       \cup Happy_j| = c$, or will steal enough value to enter the $Happy$ set, that is $v$ will achieve $Loss_{v, j} = 0$.
       Suppose that she has stolen $m$ players. They, in their turn, will steal total value at least equal to the value
       stolen by $v$ (since they cannot go sad, as explained above). However, this means that, since the total value being
       stolen will never be reduced and the turns this will happen are infinite, the players must steal an infinite amount of
       value, which is impossible because the direct trusts are finite in number and in value. More precisely, let $j_1$ be
       a turn in which a conservative player is chosen and
       \begin{equation}
          \forall j \in \mathbb{N}, DTr_j = \sum\limits_{w,w' \in \mathcal{V}}DTr_{w \rightarrow w', j} \enspace.
       \end{equation}
       Also, without loss of generality, suppose that
       \begin{equation}
          \forall j \geq j_1, out_{A, j} = out_{A, j_1} \enspace.
       \end{equation}
       In $Turn_{j_1}$, $v$ steals
       \begin{equation}
          St = \sum\limits_{i=1}^{m}y_i \enspace.
       \end{equation}
       We will show using induction that
       \begin{equation}
          \forall n \in \mathbb{N}, \exists j_n \in \mathbb{N} : DTr_{j_n} \leq DTr_{j_1-1} - nSt \enspace.
       \end{equation}

       Base case: It holds that
       \begin{equation}
          DTr_{j_1} = DTr_{j_1-1} - St \enspace.
       \end{equation}
       Eventually there is a turn $j_2$ when every player in $N^{-}(v)_{j-1}$ will have played. Then it holds that
       \begin{equation}
          DTr_{j_2} \leq DTr_{j_1} - St = DTr_{j_1-1} - 2St \enspace,
       \end{equation}
       since all players in $N^{-}(v)_{j-1}$ follow the conservative strategy, except for $A$, who will not have been stolen
       anything due to the supposition.

       Induction hypothesis: Suppose that
       \begin{equation}
          \exists k > 1 : j_k > j_{k-1} > j_1 \Rightarrow DTr_{j_k} \leq DTr_{j_{k-1}} - St \enspace.
       \end{equation}

       Induction step: There exists a subset of the $Angry$ players, $S$, that have been stolen at least value $St$ in total
       between the turns $j_{k-1}$ and $j_k$, thus there exists a turn $j_{k+1}$ such that all players in $S$ will have
       played and thus
       \begin{equation}
          DTr_{j_{k+1}} \leq DTr_{j_k} - St \enspace.
       \end{equation}
       We have proven by induction that
       \begin{equation}
          \forall n \in \mathbb{N}, \exists j_n \in \mathbb{N} : DTr_{j_n} \leq DTr_{j_1-1} - nSt \enspace.
       \end{equation}
       However
       \begin{equation}
          DTr_{j_1-1} \geq 0 \wedge St > 0 \enspace,
       \end{equation}
       thus
       \begin{equation}
          \exists n' \in \mathbb{N} : n'St > DTr_{j_1-1} \Rightarrow DTr_{j_{n'}} < 0 \enspace.
       \end{equation}
       We have a contradiction because
       \begin{equation}
          \forall w,w' \in \mathcal{V}, \forall j \in \mathbb{N}, DTr_{w \rightarrow w', j} \geq 0 \enspace,
       \end{equation}
       thus eventually $Angry = \emptyset$ and everybody passes. \qed
    \end{sepproof}

    \begin{sepproof}[MaxFlows Are Transitive Games Lemma (\ref{maxflowgame})] \ \\
    \label{maxflowgameproof}
       Without loss of generality, we suppose that the interesting turn is 0. In other words, $j_0 = 0$. Let $X =
       \{x_{vw}\}_{\mathcal{V} \times \mathcal{V}}$ be the flows returned by the execution of the $MaxFlow\left(A, B\right)$
       algorithm on $\mathcal{G}_0$. It is known that for any directed weighted graph $G$ there exists a $MaxFlow$ over $G$
       that is a DAG [citation needed]. We also know that we can apply the topological sort algorithm to any DAG and obtain a
       total ordering of its nodes with the following property: $\forall$ nodes $v, w$, it holds that
       $v < w \Rightarrow x_{wv} = 0$ [citation needed]. Put differently, there is no flow from larger to smaller nodes. We
       execute the topological sort on $X$ and obtain a total order of the nodes, such that $B$ is the maximum and
       $A$ is the minimum node. $B$ is maximum since it is the sink and thus has no outgoing flow to any node and $A$ is
       minimum since it is the source and thus has no incoming flow from any node. The desired execution of algorithm
       \ref{transitivegame} will choose players following the total order, starting from player $B$. We observe that
       $\forall v \in \mathcal{V} \setminus \{A, B\}, \sum\limits_{w \in \mathcal{V}}x_{wv} = \sum\limits_{w \in
       \mathcal{V}}x_{vw} \leq maxFlow\left(A, B\right) \leq in_{B, 0}$. Player $B$ will follow a modified evil strategy
       where she steals value equal to her total incoming flow, not her total incoming trust. Let $j_2$ be the first turn
       when $A$ is chosen to play. We will show using strong induction that there exists a set of valid actions for each
       player according to their respective strategy such that at the end of each turn $j$ the corresponding player $v =
       Player\left(j\right)$ will have stolen value $x_{wv}$ from each in neighbour $w$.

       Base case: In turn 1, $B$ steals value equal to $\sum\limits_{w \in \mathcal{V}}x_{wB}$, following the modified evil
       strategy.
       \begin{equation}
          Turn_1 = \bigcup\limits_{v \in N^{-}\left(B\right)_0}\{Steal\left(x_{vB}, v\right)\}
       \end{equation}

       Induction hypothesis: Let $k \in [j_2 - 2]$. We suppose that $\forall j \in [k]$, there exists a valid set of actions,
       $Turn_j$, performed by $v = Player\left(j\right)$ such that $v$ steals from each player $w$ value equal to $x_{wv}$.
       \begin{equation}
          \forall j \in [k], Turn_j = \bigcup\limits_{w \in N^{-}\left(v\right)_{j-1}}\{Steal\left(x_{wv}, w\right)\}
       \end{equation}

       Induction step: Let $j = k + 1, v = Player\left(j\right)$. Since all the players that are greater than $v$ in the
       total order have already played and all of them have stolen value equal to their incoming flow, we deduce that $v$ has
       been stolen value equal to $\sum\limits_{w \in N^{+}\left(v\right)_{j-1}}x_{vw}$. Since it is the first time $v$
       plays, $\forall w \in N^{-}\left(v\right)_{j-1}, DTr_{w \rightarrow v, j-1} = DTr_{w \rightarrow v, 0} \geq x_{wv}$, thus
       $v$ is able to choose the following turn:
       \begin{equation}
          Turn_j = \bigcup\limits_{w \in N^{-}\left(v\right)_{j-1}}\{Steal\left(x_{wv}, w\right)\}
       \end{equation}
       Moreover, this turn satisfies the conservative strategy since
       \begin{equation}
          \sum\limits_{w \in N^{-}\left(v\right)_{j-1}}x_{wv} = \sum\limits_{w \in N^{+}\left(v\right)_{j-1}}x_{vw} \enspace.
       \end{equation}
       Thus $Turn_j$ is a valid turn for the conservative player $v$.

       We have proven that in the end of turn $j_2 - 1$, player $B$ and all the conservative players will have stolen value
       exactly equal to their total incoming flow, thus $A$ will have been stolen value equal to her outgoing flow, which is
       $maxFlow(A, B)$. Since there remains no Angry player, it is obvious that $j_2$ is a turn that Transitive Game has
       converged thus $Loss_{A, j_2} = Loss_A$. It is also obvious that if $B$ had chosen the original evil strategy, the
       described actions would still be valid only by supplementing them with additional $Steal\left(\right)$ actions, thus
       $Loss_A$ would further increase. This proves the theorem.
%       We have proven that there exists a valid execution of the Transitive Game where
%       \begin{equation}
%          Loss_A \geq maxFlow\left(A, B\right) \enspace.
%       \end{equation}
       \qed
    \end{sepproof}

    \begin{sepproof}[Transitive Games Are Flows Lemma (\ref{gameflow})] \ \\
    \label{gameflowproof}
       Let $Sad, Happy, Angry$ be as defined in the Transitive Game. Let $\mathcal{G}'$ be a directed weighted graph based on
       $\mathcal{G}$ with an auxiliary source. Let also $j_1$ be a turn when the Transitive Game has converged. More
       precisely, $\mathcal{G}'$ is defined as follows:
       \begin{equation}
          \mathcal{V}' = \mathcal{V} \cup \{T\}
       \end{equation}
       \begin{equation}
          \mathcal{E}' = \mathcal{E} \cup \{(T, A)\} \cup \{(T, v) : v \in Sad_{j_1}\}
       \end{equation}
       \begin{equation}
          \forall (v, w) \in \mathcal{E}, c'_{vw} = DTr_{v \rightarrow w, 0} - DTr_{v \rightarrow w, j_1}
       \end{equation}
       \begin{equation}
          \forall v \in Sad_{j_1}, c'_{Tv} = c'_{TA} = \infty
       \end{equation}
       \begin{center}
       \begin{tikzpicture}
       \begin{dot2tex}
          digraph G {
           rankdir=LR;
           edge [arrowsize=0.6];
           node [shape=circle];
           compound=true;
           T -> {S1 A S2} [label=<&#xe2889e;>];
          }
       \end{dot2tex}
       \end{tikzpicture}
       \end{center}
%#           subgraph cluster0 {
%#              color=none;
%#              label = "TrustIsRisk\\ game\\ graph";
%#              A -> C [label="5BTC"];
%#              A -> D [label="6BTC"];
%#              C -> B [label="3BTC"];
%#              C -> E [label="10BTC"];
%#              D -> B [label="2BTC"];
%#           }
%#           J1 [style=invisible];
%#           J2 [style=invisible];
%#           J1 -> J2 [dir=both];
%#           subgraph cluster1 {
%#              rank=same;
%#              label="Bitcoin\\ UTXO";
%#              Z1 [style=invisible];
%#              a1 [label="5BTC"];
%#              Y1 [style=invisible];
%#              a1 -> Y1 [label="1/\\{B,C\\}"];
%#              Z1 -> a1 [label="A"];
%#              Z2 [style=invisible];
%#              a2 [label="6BTC"];
%#              Y2 [style=invisible];
%#              a2 -> Y2 [label="1/\\{B,D\\}"];
%#              Z2 -> a2 [label="A"];
%#              Z3 [style=invisible];
%#              a3 [label="10BTC"];
%#              Y3 [style=invisible];
%#              a3 -> Y3 [label="1/\\{C,E\\}"];
%#              Z3 -> a3 [label="C"];
%#              Z4 [style=invisible];
%#              a4 [label="3BTC"];
%#              Y4 [style=invisible];
%#              a4 -> Y4 [label="1/\\{C,B\\}"];
%#              Z4 -> a4 [label="C"];
%#              Z5 [style=invisible];
%#              a5 [label="2BTC"];
%#              Y5 [style=invisible];
%#              a5 -> Y5 [label="1/\\{D,B\\}"];
%#              Z5 -> a5 [label="D"];
%#           }
%#           B  -> J1 [style=invisible]
%#           J2 -> Z3 [style=invisible]
       We observe that $\forall v \in \mathcal{V},$
       \begin{equation}
       \label{gameflowin}
       \begin{gathered}
          \sum\limits_{w \in N^{'-}\left(v\right) \setminus \{T\}}c'_{wv} = \\
          = \sum\limits_{w \in N^{'-}\left(v\right) \setminus \{T\}}\left(DTr_{w \rightarrow v, 0} -
          DTr_{w \rightarrow v, j_1}\right) = \\
          = \sum\limits_{w \in N^{'-}\left(v\right) \setminus \{T\}}DTr_{w \rightarrow v, 0} -
          \sum\limits_{w \in N^{'-}\left(v\right) \setminus \{T\}}DTr_{w \rightarrow v, j-1} =  \\
          = in_{v, 0} - in_{v, j_1}
       \end{gathered}
       \end{equation}
       and
       \begin{equation}
       \label{gameflowout}
       \begin{gathered}
          \sum\limits_{w \in N^{'+}\left(v\right) \setminus \{T\}}c'_{vw} = \\
          = \sum\limits_{w \in N^{'+}\left(v\right) \setminus \{T\}}\left(DTr_{v \rightarrow w, 0} -
          DTr_{v \rightarrow w, j_1}\right) = \\
          = \sum\limits_{w \in N^{'+}\left(v\right) \setminus \{T\}}DTr_{v \rightarrow w, 0} -
          \sum\limits_{w \in N^{'+}\left(v\right) \setminus \{T\}}DTr_{v \rightarrow w, j-1} = \\
          = out_{v, 0} - out_{v, j_1} \enspace.
       \end{gathered}
       \end{equation}
       We can suppose that
       \begin{equation}
       \label{Aincoming}
          \forall j \in \mathbb{N}, in_{A, j} = 0 \enspace,
       \end{equation}
       since if we find a valid flow under this assumption, the flow will still be valid for the original graph. \\
       Next we try to calculate $MaxFlow\left(T, B\right) = X'$ on graph $\mathcal{G}'$. We observe that a flow in which it
       holds that $\forall v, w \in \mathcal{V}, x'_{vw} = c'_{vw}$ can be valid for the following reasons:
       \begin{itemize}
          \item $\forall v,w \in \mathcal{V}, x'_{vw} \leq c'_{vw}$ (Capacity flow requirement (\ref{flow1}) $\forall e \in
          \mathcal{E}$)
          \item Since $\forall v \in Sad_{j_1} \cup \{A\}, c'_{Tv} = \infty$, requirement (\ref{flow1}) holds for any flow
          $x'_{Tv} \geq 0$.
          \item Let $v \in \mathcal{V}' \setminus \left(Sad_{j_1} \cup \{T, A, B\}\right)$. According to the conservative
          strategy and since $v \notin Sad_{j_1},$ it holds that
          \begin{equation}
             out_{v, 0} - out_{v, j_1} = in_{v, 0} - in_{v, j_1} \enspace.
          \end{equation}
          Combining this observation with (\ref{gameflowin}) and (\ref{gameflowout}), we have that
          \begin{equation}
             \sum\limits_{w \in \mathcal{V}'}c'_{vw} = \sum\limits_{w \in \mathcal{V}'}c'_{wv} \enspace.
          \end{equation}
          (Flow Conservation requirement (\ref{flow2}) $\forall v \in \mathcal{V}' \setminus \left(Sad_{j_1}
          \cup \{T, A, B\}\right)$)
          \item Let $v \in Sad_{j_1}$. Since $v$ is sad, we know that
          \begin{equation}
             out_{v, 0} - out_{v, j_1} > in_{v, 0} - in_{v, j_1} \enspace.
          \end{equation}
          Since $c'_{Tv} = \infty$, we can set
          \begin{equation}
             x'_{Tv} = \left(out_{v, 0} - out_{v, j_1}\right) - \left(in_{v, 0} - in_{v, j_1}\right) \enspace.
          \end{equation}
          In this way, we have
          \begin{equation}
             \sum\limits_{w \in \mathcal{V}'}x'_{vw} = out_{v, 0} - out_{v, j_1} \mbox{ and}
          \end{equation}
          \begin{equation}
          \begin{gathered}
             \sum\limits_{w \in \mathcal{V}'}x'_{wv} = \sum\limits_{w \in \mathcal{V}' \setminus \{T\}}c'_{wv} + x'_{Tv} =
             in_{v, 0} - in_{v, j_1} + \\ + (out_{v, 0} - out_{v, j_1}) - (in_{v, 0} - in_{v, j_1}) = out_{v, 0} -
             out_{v, j_1} \enspace.
          \end{gathered}
          \end{equation}
          thus
          \begin{equation}
             \sum\limits_{w \in \mathcal{V}'}x'_{vw} = \sum\limits_{w \in \mathcal{V}'}x'_{wv} \enspace.
          \end{equation}
          (Requirement \ref{flow2} $\forall v \in Sad_{j_1}$)
          \item We set
          \begin{equation}
             x'_{TA} = \sum\limits_{v \in \mathcal{V}'}x'_{Av} \enspace,
          \end{equation}
          thus from (\ref{Aincoming}) we have
          \begin{equation}
             \sum\limits_{v \in \mathcal{V}'}x'_{vA} = \sum\limits_{v \in \mathcal{V}'}x'_{Av} \enspace.
          \end{equation}
          (Requirement \ref{flow2} for $A$)
       \end{itemize}
       We saw that for all nodes, the necessary properties for a flow to be valid hold and thus $X'$ is a valid flow for
       $\mathcal{G}$. Moreover, this flow is equal to $maxFlow(T, B)$ because all incoming flows to $B$ are saturated.
       Also we observe that
       \begin{equation}
       \label{xprimeequalloss}
          \sum\limits_{v \in \mathcal{V}'}x'_{Av} = \sum\limits_{v \in \mathcal{V}'}c'_{Av} = out_{A, 0} - out_{A, j_1} =
          Loss_A \enspace.
       \end{equation}
       We define another graph, $\mathcal{G}''$, based on $\mathcal{G}'$.
       \begin{equation}
          \mathcal{V}'' = \mathcal{V}'
       \end{equation}
       \begin{equation}
          E(\mathcal{G}'') = E(\mathcal{G}') \setminus \{(T, v) : v \in Sad_j\}
       \end{equation}
       \begin{equation}
          \forall e \in E(\mathcal{G}''), c''_e = c'_e
       \end{equation}
       If we execute the algorithm $MaxFlow(T, B)$ on the graph $\mathcal{G}''$, we will obtain a flow $X''$ in which
       \begin{equation}
          \sum\limits_{v \in \mathcal{V}''}x''_{Tv} = x''_{TA} = \sum\limits_{v \in \mathcal{V}''}x''_{Av} \enspace.
       \end{equation}
       The outgoing flow from $A$ in $X''$ will remain the same as in $X'$ for two reasons: \\
       No capacity reachable by $A$ is modified and $T$ has no incoming flow, thus
       \begin{equation}
          \sum\limits_{v \in \mathcal{V}''}x''_{Av} \geq \sum\limits_{v \in \mathcal{V}'}x'_{Av} \mbox{[citation needed] and}
       \end{equation}
       \begin{equation}
          \begin{rcases}
             \sum\limits_{v \in \mathcal{V}''}c''_{Av} = \sum\limits_{v \in \mathcal{V}'}c'_{Av} = \sum\limits_{v \in
             \mathcal{V}'}x'_{Av} \\
             \sum\limits_{v \in \mathcal{V}''}c''_{Av} \geq \sum\limits_{v \in \mathcal{V}''}x''_{Av}
          \end{rcases}
          \Rightarrow \sum\limits_{v \in \mathcal{V}''}x''_{Av} \leq \sum\limits_{v \in \mathcal{V}'}x'_{Av}
       \end{equation}
       Thus we conclude that
       \begin{equation}
       \label{primeequaldoubleprime}
          \sum\limits_{v \in \mathcal{V}''}x''_{Av} = \sum\limits_{v \in \mathcal{V}'}x'_{Av} \enspace.
       \end{equation}
       Let $X = X'' \setminus \{(T, A)\}$. Observe that
       \begin{equation}
          \sum\limits_{v \in \mathcal{V}''}x''_{Av} = \sum\limits_{v \in \mathcal{V}}x_{Av} \enspace.
       \end{equation}
       This flow is valid on graph $\mathcal{G}$ because
       \begin{equation}
          \forall e \in \mathcal{E}, c_e \geq c''_e \enspace.
       \end{equation}
%       We will now prove that
%       \begin{equation}
%          X = MaxFlow_{\mathcal{G}}\left(A, B\right) \enspace.
%       \end{equation}
%       \begin{itemize}
%          \item Let $X^A = MaxFlow_{\mathcal{G}}$. If we suppose that
%          \begin{equation}
%             maxFlow_{\mathcal{G}}\left(A, B\right) > \sum\limits_{v \in \mathcal{V}''}x''_{Av} \enspace,
%          \end{equation}
%          then we can set
%          \begin{equation}
%             X^T = X \cup \{(T, A)\} \mbox{ with}
%          \end{equation}
%          \begin{equation}
%             \forall v, w \in \mathcal{V}'', x^T_{vw} = x^A_{vw} \mbox{ and}
%          \end{equation}
%          \begin{equation}
%             x^T_{TA} = \sum\limits_{v \in \mathcal{V}''}x^A_{Av} \enspace.
%          \end{equation}
%          Since
%          \begin{equation}
%             \sum\limits_{v \in \mathcal{V}''}x^A_{Av} = maxFlow_{\mathcal{G}}\left(A, B\right) \enspace,
%          \end{equation}
%          we see that
%          \begin{equation}
%             \sum\limits_{v \in \mathcal{V}''}x^T_{Tv} = x^T_{TA} > x''_{TA} \enspace,
%          \end{equation}
%          thus $X''$ is not $MaxFlow_{\mathcal{G}}\left(T, B\right)$, which is a contradiction. Thus
%          \begin{equation}
%             maxFlow_{\mathcal{G}}\left(A, B\right) \leq \sum\limits_{v \in \mathcal{V}''}x''_{Av} \enspace,
%          \end{equation}
%          therefore
%          \begin{equation}
%             maxFlow_{\mathcal{G}}\left(A, B\right) \leq \sum\limits_{v \in \mathcal{V}}x_{Av} \enspace.
%          \end{equation}
%          \item If we suppose that
%          \begin{equation}
%             maxFlow_{\mathcal{G}}\left(A, B\right) < \sum\limits_{v \in \mathcal{V}''}x''_{Av} \enspace,
%          \end{equation}
%          we can likewise choose $X^A$ such that
%          \begin{equation}
%             \forall v, w \in \mathcal{V}'' \setminus \{T\}, x^A_{vw} = x''_{vw} \enspace,
%          \end{equation}
%          thus
%          \begin{equation}
%             \sum\limits_{v \in \mathcal{V}'' \setminus \{T\}}x^A_{Av} = \sum\limits_{v \in \mathcal{V}''}x''_{Av} \enspace.
%          \end{equation}
%          We deduce that
%          \begin{equation}
%             \sum\limits_{v \in \mathcal{V}'' \setminus \{T\}}x^A_{Av} > maxFlow_{\mathcal{G}}\left(A, B\right) \enspace,
%          \end{equation}
%          which is a contradiction. Thus
%          \begin{equation}
%             maxFlow_{\mathcal{G}}\left(A, B\right) \geq \sum\limits_{v \in \mathcal{V}''}x''_{Av} \enspace,
%          \end{equation}
%          therefore
%          \begin{equation}
%             maxFlow_{\mathcal{G}}\left(A, B\right) \geq \sum\limits_{v \in \mathcal{V}}x_{Av} \enspace.
%          \end{equation}
%       \end{itemize}
       Thus there exists a valid flow for each execution of the Transitive Game such that
       \begin{equation}
          \sum\limits_{v \in \mathcal{V}}x_{Av} = \sum\limits_{v \in \mathcal{V}''}x''_{Av}
          \overset{\left(\ref{primeequaldoubleprime}\right)}{=} \sum\limits_{v \in \mathcal{V}'}x'_{Av}
          \overset{\left(\ref{xprimeequalloss}\right)}{=} Loss_{A, j_1} \enspace,
       \end{equation}
       which is the flow $X$. \qed
    \end{sepproof}

    \begin{theorem}[Conservative World Theorem] \ \\
       \label{conservativeworld}
       If everybody follows the conservative strategy, nobody steals any amount from anybody.
    \end{theorem}
    \begin{proofsketch}
       If everybody is conservative, nobody can initiate the chain of steals.
    \end{proofsketch}
    \begin{proof} \ \\
       Suppose that we are interested in graphs $\mathcal{G}_j$. Let $(j_k)$ an increasing sequence of positive integers,
       \begin{equation}
       \begin{gathered}
          \mbox{let } S_{j_k} \subseteq N^{-}\left(Player\left(j_k\right)\right)_{j_k-1} \mbox{ and} \\
          \mbox{let } \forall v \in S_{j_k}, y_{v, j_k} > 0\enspace.
       \end{gathered}
       \end{equation}
       Suppose that there exists a subseries of History, $(Turn_{j_k})$, where
       \begin{equation}
          Turn_{j_k} = \bigcup\limits_{v \in S_{j_k}}\{Steal(y_{v, j_k},v)\} \enspace,
       \end{equation}
       This subseries must have an initial element, $Turn_{j_1}$. However, $Player(j_1)$ follows the conservative strategy,
       thus somebody must have stolen from her as well, so $Player(j_1)$ cannot be the initial element. We have a
       contradiction, thus the theorem holds.
    \end{proof}

%    \begin{lemma}[No Evil Edges in the $MinCut$] \ \\
%       \label{mincutmany}
%       Let $S \subset \mathcal{V}, A \notin S$. When calculating $MaxFlow\left(A, S\right)$, it is impossible to have an edge
%       $\left(v, w\right) \in MinCut\left(A, S\right) : v \in S$.
%    \end{lemma}
%    \begin{proof}[No Evil Edges in the $MinCut$ Lemma (\ref{mincutmany}]
%    \label{mincutmanyproof}
%       Let $T$ be the auxiliary node. It is
%       \begin{equation}
%          \forall v \in S, c_{vT} = \infty \enspace.
%       \end{equation}
%       We can see that $out_A < \infty$ and thus
%       \begin{equation}
%       \label{maxflowbounded}
%          maxFlow\left(A, S\right) < \infty \enspace.
%       \end{equation}
%       Since all edges in the $MinCut$ are saturated and due to (\ref{maxflowbounded}), we have
%       \begin{equation}
%          \nexists v \in S : \left(v, T\right) \in MinCut \enspace.
%       \end{equation}
%       Suppose that
%       \begin{equation}
%          \exists v \in S, w \in \mathcal{V} : \left(v, w\right) \in MinCut \enspace.
%       \end{equation}
%       Then this edge must be saturated, that is $x_{vw} = c_{vw} > 0$. However, there exists an alternative flow
%       configuration $X'$ where
%       \begin{equation}
%       \begin{gathered}
%          \forall \left(u, u'\right) \in \mathcal{E} \setminus \{\left(v, w\right), \left(v, T\right)\}, x_{u,u'}' =
%          x_{u,u'} \enspace, \\
%          x_{vw}' = 0 \mbox{ and} \\
%          x_{vT}' = x_{vT} + x_{vw} \enspace,
%       \end{gathered}
%       \end{equation}
%       which is valid because
%       \begin{equation}
%          \begin{rcases}
%             \sum\limits_{w \in N^{+}\left(v\right)}x_{vw} = \sum\limits_{w \in N^{+}\left(v\right)}x_{vw}' \\
%             c_{vT} = \infty
%          \end{rcases}
%          \Rightarrow x_{vT}' \leq c_{vT}
%       \end{equation}
%       and $X'$ is maximum as well because it carries exactly the same flow as $X$. Thus
%       \begin{equation}
%          \left(v, w\right) \notin MinCut \enspace.
%       \end{equation}
%       \qed
%    \end{proof}

    \begin{sepproof}[Sybil Resilience Theorem (\ref{sybil})] \ \\
    \label{sybilproof}
       Let $\mathcal{G}_1$ be a game graph defined as follows:
       \begin{equation}
          \mathcal{V}_1 = \mathcal{V} \cup \{T_1\} \enspace,
       \end{equation}
       \begin{equation}
          \mathcal{E}_1 = \mathcal{E} \cup \{(v, T_1) : v \in \mathcal{B} \cup \mathcal{C}\} \enspace,
       \end{equation}
       \begin{equation}
          \forall v,w \in \mathcal{V}_1 \setminus \{T_1\}, DTr^1_{v \rightarrow w} = DTr_{v \rightarrow w} \enspace,
       \end{equation}
       \begin{equation}
          \forall v \in \mathcal{B} \cup \mathcal{C}, DTr^1_{v \rightarrow T_1} = \infty \enspace,
       \end{equation}
       where $DTr_{v \rightarrow w}$ is the direct trust from $v$ to $w$ in $\mathcal{G}$ and $DTr^1_{v \rightarrow w}$ is
       the direct trust from $v$ to $w$ in $\mathcal{G}_1$. \\
       Let also $\mathcal{G}_2$ be the induced graph that results from $\mathcal{G}_1$ if we remove the Sybil set,
       $\mathcal{C}$. We rename $T_1$ to $T_2$ to facilitate comprehension. (Image) \\
%       The colluding players follow the evil strategy. Suppose that until some turn $j$ only good players are chosen to play.
%       Starting from turn $j$, suppose that there exist $|\mathcal{B} \cup \mathcal{C}|$ consecutive turns for the first game
%       and $|\mathcal{B}|$ consecutive turns for the second game during which all the colluding players are chosen to play.
%       More formally, suppose that
%       \begin{equation}
%       \begin{gathered}
%          \exists j \in \mathbb{N} : \forall d_1 \in [|\mathcal{B} \cup \mathcal{C}|], Player(j+d) \in \mathcal{B} \cup
%          \mathcal{C} \wedge \\
%          \wedge \forall d_1, d_2 \in [|\mathcal{B} \cup \mathcal{C}|], d_1 \neq d_2, Player(j + d_1) \neq Player(j + d_2)
%          \wedge \\
%          \wedge \forall d \in [|\mathcal{B} \cup \mathcal{C}|], Strategy(Player(j+d)) = Evil \wedge \\
%          \wedge \forall j' \in [j] Player(j') \notin \mathcal{B} \cup \mathcal{C}
%       \end{gathered}
%       \end{equation}
%       for the first and likewise for the second game. \\
       According to /heorem (\ref{trustmany}),
       \begin{equation}
       \label{trmaxflow}
          Tr_{A \rightarrow \mathcal{B} \cup \mathcal{C}} = maxFlow_1\left(A, T_1\right) \wedge
          Tr_{A \rightarrow \mathcal{B}} = maxFlow_2\left(A, T_2\right) \enspace.
       \end{equation}
%       From lemma (\ref{mincutmany}), we know that
%       \begin{equation}
%          \forall \left(v, w\right) \in MinCut_1\left(A, T_1\right), v \notin \mathcal{B} \cup \mathcal{C} \wedge \forall
%          \left(v, w\right) \in MinCut_2\left(A, T_2\right), v \notin \mathcal{B}
%       \end{equation}
%       and thus
%       \begin{equation}
%       \begin{gathered}
%          e \in MinCut_1 \Rightarrow e \in \mathcal{E}_2 \wedge \\
%          \wedge e \in MinCut_2 \Rightarrow e \in \mathcal{E}_1 \wedge \\
%          \wedge \forall e \in MinCut_1 \cup MinCut_2, c_1(e) = c_2(e) \enspace.
%       \end{gathered}
%       \end{equation}
       We will show that the $MaxFlow$ of each of the two graphs can be used to construct a valid flow of equal value for the
       other graph. The flow $X_1 = MaxFlow\left(A, T_1\right)$ can be used to construct a valid flow of equal value for the
       second graph if we set
       \begin{align}
          \forall v \in \mathcal{V}_2 \setminus \mathcal{B}, \forall w \in \mathcal{V}_2&, x_{vw,2} = x_{vw,1} \enspace, \\
          \forall v \in \mathcal{B}&, x_{vT_2,2} = \sum\limits_{w \in N^{+}_1\left(v\right)}x_{vw,1} \enspace, \\
          \forall v,w \in \mathcal{B}&, x_{vw,2} = 0 \enspace.
       \end{align}
       Therefore
       \begin{equation}
          maxFlow_1\left(A, T_1\right) \leq maxFlow_2\left(A, T_2\right)
       \end{equation}
       Likewise, the flow $X_2 = MaxFlow(A, T_2)$ is a valid flow for $\mathcal{G}_1$ because $\mathcal{G}_2$ is an induced
       subgraph of $\mathcal{G}_1$. Therefore
       \begin{equation}
          maxFlow_1\left(A, T_1\right) \geq maxFlow_2\left(A, T_2\right)
       \end{equation}
%       used to construct a valid flow of equal value for the first case if we set
%       \begin{align}
%          \forall v \in \mathcal{V}_1 \setminus \left(\mathcal{B} \cup \mathcal{C}\right), \forall w \in \mathcal{V}_1&,
%          x_{vw,1} = x_{vw,2} \enspace, \\
%          \forall v \in \mathcal{B}&, x_{vT_1,1} = \sum\limits_{w \in N^{+}(v)}x_{vw,2} \enspace, \\
%          \forall v \in \mathcal{C}, \forall w \in \mathcal{V}_1&, x_{vw,1} = 0 \enspace.
%       \end{align}
%       Observe that
%       \begin{equation}
%          \forall v \in \mathcal{V}_1 \setminus \left(\mathcal{B} \cup \mathcal{C}\right), \forall w \in \mathcal{C},
%          x_{vw, 1} = 0 \enspace.
%       \end{equation}
%       From these two observations, we deduce that there exists a function, say $F_2(X_1)$, that transforms
%       the $MaxFlow_1$ of the first graph into a valid flow for the second graph that has the same amount of flow as
%       $MaxFlow_1$ and there also exists a similar function $F_1(X_2)$ that transforms the $MaxFlow_2$ of the second graph
%       into a valid flow for the first graph that has the same amount of flow as $MaxFlow_2$. Suppose that
%       \begin{equation}
%         maxFlow_1 < maxFlow_2 \enspace.
%       \end{equation}
%       Then
%       \begin{equation}
%          F_1(MaxFlow_2) > maxFlow_1 \enspace,
%       \end{equation}
%       which is a contradiction. Likewise, suppose that
%       \begin{equation}
%          maxFlow_1 > maxFlow_2 \enspace.
%       \end{equation}
%       Then
%       \begin{equation}
%          F_2(MaxFlow_1) > maxFlow_2 \enspace,
%       \end{equation}
       We conclude that
       \begin{equation}
       \label{eqmaxflows}
          maxFlow\left(A, T_1\right) = maxFlow\left(A, T_2\right) \enspace,
       \end{equation}
       thus from (\ref{trmaxflow}) and (\ref{eqmaxflows}) the theorem holds.
%       \begin{equation}
%          Tr_{A \rightarrow \mathcal{B}} = Tr_{A \rightarrow \mathcal{B} \cup \mathcal{C}} \enspace.
%       \end{equation}
       \qed
    \end{sepproof}

  \begin{thebibliography}{20}
     \bibitem{ddosattacks}
     Patrikakis C., Masikos M., Zouraraki O.: Distributed Denial of Service Attacks. The Internet Protocol Journal, Vol. 7,
     N. 4,
     \url{http://www.cisco.com/c/en/us/about/press/internet-protocol-journal/back-issues/table-contents-30/dos-attacks.html}
     (2004)
     \bibitem{ebayfees}
     Standard ebay selling fees, \url{http://pages.ebay.com/help/sell/fees.html}
     \bibitem{ebayguarantee}
     ebay money back guarantees, \url{http://pages.ebay.com/ebay-money-back-guarantee/questions.html}
     \bibitem{openbazaar}
     What is OpenBazaar, \url{https://blog.openbazaar.org/what-is-openbazaar/}
     \bibitem{multisigfraud}
     Can Bitcoin and Multisig Reduce Identity Theft and Fraud?,
     \url{https://blog.openbazaar.org/can-bitcoin-and-multisig-reduce-identity-theft-and-fraud/}
     \bibitem{bitcoin}
     Nakamoto S.: Bitcoin: A Peer-to-Peer Electronic Cash System (2008)
     \bibitem{bitcoinguide}
     Bitcoin Developer Guide, \url{https://bitcoin.org/en/developer-guide}
     \bibitem{toposort}
     Kahn Arthu.r B.: Topological sorting of large networks. Communications of the ACM Vol. 5, Issue 11, pp. 558-562, ACM,
     New York (1962)
     \bibitem{clrs}
     Cormen T. H., Leiserson C. E., Rivest R. L., Stein C.: Introduction to Algorithms (3rd ed.). MIT Press and McGraw-Hill
     (2009) [1990]
     \bibitem{dionyziz}
     Zindros D. S.: Trust in decentralized anonymous marketplaces (2015)
     \bibitem{multisig}
     Buterin V.: Bitcoin Multisig Wallet: The Future of Bitcoin. Bitcoin Magazine (2014),
     \url{https://bitcoinmagazine.com/articles/multisig-future-bitcoin-1394686504}
     \bibitem{loc}
     Sanchez W.: Lines of Credit (2016) \url{https://gist.github.com/drwasho/2c40b91e169f55988618#part-3-web-of-credit}
     \bibitem{deanonymisation}
     Narayanan A., Shmatikov V.: De-anonymizing Social Networks. SP '09 Proceedings of the 2009 30th IEEE Symposium on
     Security and Privacy, pp. 173-187, 10.1109/SP.2009.22 (2009)
%     Biryukov A., Khovratovich D., Pustogarov I.: Deanonymisation of clients in Bitcoin P2P network. arXiv:1405.7418 [cs.CR]
%     (2014)
%     \url{https://arxiv.org/pdf/1405.7418v3.pdf}
  \end{thebibliography}

\end{document}
