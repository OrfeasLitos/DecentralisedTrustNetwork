\subsection{Centrality}
  Another important measure is the network centrality. Intuitively, a network that resembles a mesh is preferable over a
  network with a central hub that acts as an intermediary between all other pairs of nodes. This is because in the latter case
  Trust Is Risk does not offer any realistic improvement over the existing centralized marketplaces model (e.g. ebay).
  Nevertheless, the expected indirect trust measure is not always able to distinguish between the two. We thus propose here
  some different measures to that end.
  
  \subsubsection{Degree Centrality} \ \\

    One possible approach is the degree centrality \cite{freeman}, that can be broken down as in-degree and out-degree
    centrality. We first define the node in-degree centrality.
    \begin{equation*}
      C^d_{in}\left(A\right) = \sum\limits_{B \in \mathcal{V} \setminus \{A\}}DTr_{B \rightarrow A}\footnote{Maybe indirect
      trust is more intuitive here than direct trust.} \mbox{ (Node in-degree centrality)}
    \end{equation*}
    Let $A^* = \argmax\limits_{A \in \mathcal{V}}C^d_{in}\left(A\right)$. The network in-degree centrality is defined as:
    \begin{equation*}
      C^d_{in} = \sum\limits_{A \in \mathcal{V}}\left(C^d_{in}\left(A^*\right) - C^d_{in}\left(A\right)\right) \mbox{ (Network
      in-degree centrality)}
    \end{equation*}
    Similarly, for the out-degree centrality we have:
    \addtocounter{footnote}{-1}
    \begin{equation*}
      C^d_{out}\left(A\right) = \sum\limits_{B \in \mathcal{V} \setminus \{A\}}DTr_{A \rightarrow B}\footnotemark \mbox{ (Node
      out-degree centrality)}
    \end{equation*}
    Let $A^* = \argmax\limits_{A \in \mathcal{V}}C^d_{out}\left(A\right)$. The network out-degree centrality is defined as:
    \begin{equation*}
      C^d_{out} = \sum\limits_{A \in \mathcal{V}}\left(C^d_{out}\left(A^*\right) - C^d_{out}\left(A\right)\right) \mbox{
      (Network out-degree centrality)}
    \end{equation*}

    A problem of these centrality measures is that their unit is the currency used (i.e. Bitcoin) and thus may not always have
    an intuitive meaning. We would thus like to have a measure of \textit{centalization} that has no units and can take values
    in the interval $\left[0,1\right]$, with 0 corresponding to a network of no centralization (all nodes are equal, e.g.
    cycle) and 1 to a network with the maximum centralization possible (there is one central vital node for all, e.g. star). A
    centralization measure that achieves this target is proposed in [Freeman citation]. Here we use the following modified
    form. For a graph $\mathcal{G}$, the in- and out-centralization are defined as:
    \begin{align*}
      Cn^d_{in} &= \frac{C^d_{in}}{\max C^d_{in}} \mbox{ (in-degree centralization)} \\
      Cn^d_{out} &= \frac{C^d_{out}}{\max C^d_{out}} \mbox{ (out-degree centralization)} \enspace,
    \end{align*}
    where $\max C^d_{in}$ is defined as the maximum in-degree centrality for any graph with the same number of nodes for which
    the maximum direct\footnote{Maybe indirect trust is more intuitive here than direct trust.} trust is equal to the maximum
    \addtocounter{footnote}{-1} direct\footnotemark \ trust of $\mathcal{G}$; $\max C^d_{out}$ is defined equivalently.

  \subsubsection{maxFlow Centrality} \ \\

    An alternative measure of centrality for a player $A \in \mathcal{V}$ of a network $\mathcal{G}$ can be defined as the
    impact that the removal of $A$ would have on the indirect trust between the rest of the players. More specifically, let
    $\mathcal{G}' = \mathcal{G} \setminus \{A\}$. Then it is:
    \begin{equation*}
      C^{mF}\left(A\right) = \sum\limits_{B,C \in \mathcal{V}'}\left(Tr_{\mathcal{G}, B \rightarrow C} - Tr_{\mathcal{G}', B
      \rightarrow C}\right) \mbox{ (Node maxFlow centrality)} \enspace.
    \end{equation*}
    We can now follow the same steps as previously for the relevant network definitions. Let $A^*$ be the player with the
    maximum maxFlow centrality:
    \begin{equation*}
      A^* = \argmax\limits_{A \in \mathcal{V}}C^{mF}\left(A\right) \enspace.
    \end{equation*}
    Then the network maxFlow centrality is defined as follows:
    \begin{equation*}
      C^{mF} = \sum\limits_{A \in \mathcal{V}}\left(C^{mF}\left(A^*\right) - C^{mF}\left(A\right)\right)
    \end{equation*}
    and the centralization:
    \begin{equation*}
      Cn^{mF} = \frac{C^{mF}}{\max C^{mF}} \mbox{ (Network maxFlow centralization)} \enspace,
    \end{equation*}
    where $\max C^{mF}$ is the maximum centrality for any graph with the same number of nodes for which the maximum
    direct\footnote{Maybe indirect trust is more intuitive here than direct trust.} trust is equal to the maximum
    \addtocounter{footnote}{-1} direct\footnotemark \ trust of $\mathcal{G}$.

    \noindent\hrulefill
    \newpage
