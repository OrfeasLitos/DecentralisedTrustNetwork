\section{Trust Redistribution Algorithms} \label{riskinvalgs}
  In order to make use of the Risk Invariance Theorem in \cite{trustisrisk}, we have to find an algorithm that, given a
  network $\mathcal{G}$, a source $A \in \mathcal{V}$, a sink $B \in \mathcal{V}$ and a desired reduction $V$ to be applied
  upon the indirect trust $Tr_{\mathcal{G}, A \rightarrow B}$, it will return a network $\mathcal{G}'$ with modified outgoing
  direct trusts from $A$, such that $Tr_{\mathcal{G}', A \rightarrow B} = Tr_{\mathcal{G}, A \rightarrow B} - V$. We denote
  the $n$ outgoing direct trusts from $A$ with $c_i$ for $\mathcal{G}$ and $c'_i$ for $\mathcal{G}', i \in \left[n\right]$. We
  refer to the direct trusts as capacities because of the flow-related ambience. $C$ is the capacity configruation of
  $\mathcal{G}$ and $C'$ that of $\mathcal{G}'$. Likewise $X$ and $X'$ represent possible maximum flow configurations for the
  respective networks. Finally let $F = Tr_{\mathcal{G}, A \rightarrow B}$ and $F' = Tr_{\mathcal{G}', A \rightarrow B}$.

  There exist several different approaches for such an algorithm. Here we describe a linear program that achieves the desired
  reduction in $Tr_{A \rightarrow B}$ and minimizes the norm $||\delta_i||_1 = \sum\limits_{i=1}^{n}\left(c_i-c'_i\right)$ as
  well. Next we see the formulation of the problem in linear program form, along with a breakdown of each relevant matrix and
  vector.
  \subimport{common/lp/}{lpprimal.tex}
  We would like to find a solution that, except for maximizing the flow, also minimizes $||\delta_i||_1$ at the same time.
  More precisely, we would like to optimize
  \begin{equation*}
    \min{\sum\limits_{v \in \mathcal{V}}\left(c_{Av} - c'_{Av}\right)}
  \end{equation*}
  as well.
  Since we wish to optimize with regards to two objective functions, we approach the problem as follows: Initially, we ignore
  the minimization and derive the dual of the previous problem with respect to the maximisation. We then substitute the two
  problems' optimisations with an additional constraint that equates the two objective functions. Due to the Strong Duality
  theorem of linear programming \cite{amo}, this equality can be achieved only by the common optimal solution of the
  two problems. Next we treat the combination of constraints and variables of the primal and the dual problem, along with the
  newly introduced constraint and the previously ignored $||\delta_i||_1$ minimisation as a new linear problem. For every
  $j \in \left[n\right]$, the solution to this problem contains a $c'_{1j}$. These capacities will comprise the new
  configuration that player $A$ requires.

  We will now describe the dual problem in detail.
  \subimport{common/lp/}{lpdual.tex}
  Everything is now in place to define the linear problem whose solution $C'$ yields $maxFlow\left(C'\right) = F - V$ and
  minimizes $||\delta_i||_\infty$. The constraints are (\ref{lp:primal:constraints}) and (\ref{lp:dual:constraints})
  supplemented with a constraint that equates the two problems' optimality functions:
  \begin{equation*}
    \sum\limits_{j \in \left[n\right]}f'_{1j} = \sum\limits_{j \in \left[n\right]}c_{1j}y_{c1j} +
    \sum\limits_{i \in \left[n\right]}\sum\limits_{j \in \left[n\right]}c_{ij}y_{fcij} + \left(F - V\right)y_F \enspace.
  \end{equation*}
  The desired optimisation is
  \begin{equation*}
    \min{\sum\limits_{j \in \left[n\right]}\left(c_{1j} - c'_{1j}\right)} \enspace.
  \end{equation*}
  The final linear program consists of $2n^2 + 2n - 1$ variables and $2n^2 + 2n$ constraints. There exist a variety of
  algorithms that solve linear programs, such as simplex and ellipsoid algorithm. The ellipsoid algorithm requires polynomial
  time in the worst-case scenario, but for practical purposes algorithms of the simplex category seem to exhibit better
  behavior \cite{it}.
