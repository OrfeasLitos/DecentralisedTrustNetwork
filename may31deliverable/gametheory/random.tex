\subsection{Random Roles}
  All players follow the same strategy, according to which each player is permitted to freely add or steal direct trust from
  other players. After $R$ rounds exactly two players are selected at random (these choices follow the uniform distribution).
  One is dubbed seller and the other buyer. The seller offers a good that costs $C$, which she values at $C - l$ and the buyer
  values at $C + l$. The values $R, C$ and $l$, as well as the uniform distribution with which the buyer and seller are
  chosen, are common knowledge from the beginning of the game. The exchange completes if and only if $Tr_{Buyer \rightarrow
  Seller} \geq C$. In this game, Fig~\ref{fig:game} would be augmented by an additional level at the bottom, where Nature
  chooses the two transacting players. There are three variants of the game, each with a different utility for the players
  (the first two versions have two subvariants each).

  \subsubsection{Hoarders} \ \\

    If player $A$ is not chosen to be either buyer or seller, then her utility is equal to $Cap_{A, R}$. Intuitively players
    do not attach any value to having (incoming or outgoing) direct trust at the end of the game. If the buyer and the seller
    do not manage to complete the exchange, the buyer's utility is $Cap_{Buyer, R}$. If on the other hand they manage to
    exchange the good, then the buyer's utility is $Cap_{Buyer, R} + l$. Intuitively these utilities signify that the buyer
    uses her preexisting capital to buy. As for the seller there exist two subvariants for her utility:
    \begin{enumerate}
      \item If the exchange is eventually not completed, the seller's utility is $Cap_{Seller, R} - l$. If on the other hand
      the exchange takes place, the seller's utility is $Cap_{Seller, R}$. Intuitively, the seller is first obliged to buy the
      good from the environment at the cost of $C$.

      \item If the exchange is eventually not completed, the seller's utility is $Cap_{Seller, R} + C - l$. If the exchange
      takes place, the seller's utility is $Cap_{Seller, R} + C$. Intuitively, the seller is handed the good for free by the
      environment.
    \end{enumerate}
    
  \subsubsection{Sharers} \ \\

    If player $A$ is not chosen to be either buyer or seller, then her utility is equal to 
    \begin{equation*}
      \sum\limits_{\substack{B \in \mathcal{V} \\ B \neq A}}\left(DTr_{A \rightarrow B, R} + DTr_{B \rightarrow A, R}\right) +
      Cap_{A, R} \enspace.
    \end{equation*}
    Intuitively, players attach equal value to all the funds they can directly spend, regardless of whether others can spend
    them as well. If the buyer and the seller do not manage to complete the exchange, the buyer's utility is
    \begin{equation*}
      \sum\limits_{\substack{B \in \mathcal{V} \\ B \neq Buyer}}\left(DTr_{Buyer \rightarrow B, R} + DTr_{B \rightarrow Buyer,
      R}\right) + Cap_{Buyer, R} \enspace.
    \end{equation*}
    If on the other hand they manage to exchange the good, then the buyer's utility is 
    \begin{equation*}
      \sum\limits_{\substack{B \in \mathcal{V} \\ B \neq Buyer}}\left(DTr_{Buyer \rightarrow B, R} + DTr_{B \rightarrow Buyer,
      R}\right) + Cap_{Buyer, R} + l \enspace.
    \end{equation*}
    Intuitively these utilities signify that the buyer uses her preexisting accessible funds to buy. As for the seller there
    exist two subvariants for her utility:
    \begin{enumerate}
      \item If the exchange is not completed, the seller's utility is 
      \begin{equation*}
        \sum\limits_{\substack{B \in \mathcal{V} \\ B \neq Seller}}\left(DTr_{Seller \rightarrow B, R} + DTr_{B \rightarrow
	Seller, R}\right) + Cap_{Seller, R} - l \enspace.
      \end{equation*}
      If the exchange takes place, the seller's utility is
      \begin{equation*}
        \sum\limits_{\substack{B \in \mathcal{V} \\ B \neq Seller}}\left(DTr_{Seller \rightarrow B, R} + DTr_{B \rightarrow
	Seller, R}\right) + Cap_{Seller, R} \enspace.
      \end{equation*}
      Intuitively, the seller is first obliged to buy the good from the environment at the cost of $C$.

      \item If the exchange is not completed, the seller's utility is
      \begin{equation*}
        \sum\limits_{\substack{B \in \mathcal{V} \\ B \neq Seller}}\left(DTr_{Seller \rightarrow B, R} + DTr_{B \rightarrow
	Seller, R}\right) + Cap_{Seller, R} + C - l \enspace.
      \end{equation*}
      If the exchange takes place, the seller's utility is
      \begin{equation*}
        \sum\limits_{\substack{B \in \mathcal{V} \\ B \neq Seller}}\left(DTr_{Seller \rightarrow B, R} + DTr_{B \rightarrow
	Seller, R}\right) + Cap_{Seller, R} + C \enspace.
      \end{equation*}
      Intuitively, the seller is handed the good for free by the environment.
    \end{enumerate}

  \subsubsection{Materialists} \ \\

    If player $A$ is not chosen to be either buyer or seller, then her utility is 0. If the buyer and the seller do not
    manage to complete the exchange, their utility is 0 as well. If on the other hand they manage to exchange the good, then
    the utility is $l$ for both of them. Intuitively these utilities signify that in this game there is gain only for those
    who exchange goods and the gain is exactly the difference between the objective value and the subjective value that the
    relevant parties perceive.
