\subsection{Buyers, Sellers, Middlemen}
  Buyers only desire to buy as many products as possible and do not have incoming trust. Sellers only desire to sell as many
  products as possible in the highest possible price and do not have outgoing trust. Middlemen desire to accumulate capital
  and are allowed to have both incoming and outgoing trust. Their utility stems only from in-game factors. A more thorough
  description follows.

  \subsubsection{Buyers} \ \\

    Buyers initially are provided only with outgoing direct trust towards sellers and middlemen, no capital or incoming direct
    trust. They cannot reallocate their outgoing direct trust, only complete purchases of the product from sellers that are
    trustworthy enough.
    
    Buyers always want to buy the cheapest products provided. To pay for a product, they use the linear program defined in
    thesis.pdf to aquire the funds needed and directly entrust the necessary part of these funds to the seller. The rest of
    the funds are kept as capital. Sellers are supposed to complete their part of the exchange within a constant amount of
    rounds, $r$. If the product has arrived after $r$ rounds, during the turn following the arrival of the product the
    remaining funds are reallocated to the players from whom they were taken in proportional fashion. The way they are
    reallocated ensures that
    \begin{equation*}
      \forall B \in \mathcal{S}_1, \frac{DTr_{Buyer \rightarrow B, i}}{\sum\limits_{C \in \mathcal{S}_1}DTr_{Buyer \rightarrow
      C, i}} = \frac{DTr_{Buyer \rightarrow B, i + r}}{\sum\limits_{C \in \mathcal{S}_1}DTr_{Buyer \rightarrow C, i + r}}
      \enspace,
    \end{equation*}
    where $\mathcal{S}_1$ is the set of players from whom the buyer reduced her direct trust to initiate the purchase (this
    set may also contain the seller).

    If the product does not arrive after $r$ rounds, the direct trust to the seller is withdrawn (if it is still available)
    and all the funds involved (both the price and the surplus funds removed because of the linear program) are reallocated as
    direct trusts towards the rest of the players in a proportional fashion. The way they are reallocated ensures that
    \begin{equation*}
      \forall B \in \mathcal{S}_2, \frac{DTr_{Buyer \rightarrow B, i}}{\sum\limits_{C \in \mathcal{S}_2}DTr_{Buyer \rightarrow
      C, i}} = \frac{DTr_{Buyer \rightarrow B, i + r}}{\sum\limits_{C \in \mathcal{S}_2}DTr_{Buyer \rightarrow C, i + r}}
      \enspace,
    \end{equation*}
    where $\mathcal{S}_2 = \left(\mathcal{V} \setminus \mathcal{S}_1\right) \setminus{\{Buyer, Seller\}}$.

    The buyers' utilities are straightforward: They are equal to the amount of products they managed to buy throughout the
    game. This explains why buyers always prefer cheaper products: This way they are able to save funds so that they are
    hopefully able to buy more products.

  \subsubsection{Sellers} \ \\

    Sellers initially are provided only with incoming direct trust from buyers and middlemen, no capital or outgoing direct
    trust. They also possess a limited supply of products; the quantity of the products each seller initially has is decided
    randomly by the environment, much like the direct trusts. These products are useless to the sellers, they only want to
    sell them. The amount of products each seller has is common knowledge.
    
    During each round, each seller decides how many products and for what price (common for all products) she will make
    available for purchase during the next round. Obviously a seller cannot offer more products than she owns. Furthermore, if
    a buyer initiated a transaction during the previous round, the buyer must ship the product (reducing the amount of owned
    products by one). The seller may choose to convert any amount of incoming direct trust that she obtained through selling
    products into capital. The seller is not allowed to convert into capital any incoming direct trust that did not stem from
    a successful exchange.

    A seller's utility is the total amount of incoming direct trust and capital at the end of the game that stems solely from
    successfully completed purchases.

  \subsubsection{Middlemen} \ \\

    Middlemen initially have incoming direct trust from buyers and other middlemen, outgoing direct trust to sellers and other
    middlemen and some capital. They do not own or desire any products, they simply want to maximize the amount of capital
    they own at the end of the game.

    Middlemen have a very permissive strategy: They are allowed any combination of moves that do not violate the basic
    constraints described in the first section.

    The utility of a middleman is simply the capital the middleman possesses at the end of the game.
