\subsection{Proofs, Lemmas and Theorems}
  \subimport{common/lemmas/}{losseqdamagelemma.tex}
  \subimport{common/proofs/}{losseqdamageproof.tex}

  \subimport{common/sepproofs/}{convergencesepproof.tex}

  \subimport{common/sepproofs/}{flowgamesepproof.tex}

  \subimport{common/sepproofs/}{gameflowsepproof.tex}

  \subimport{common/theorems/}{consworldtheorem.tex}
  \subimport{common/proofs/}{consworldproof.tex}

  \subimport{common/sepproofs/}{sybilressepproof.tex}

  \subimport{thesis/sepproofs/}{saturationsepproof.tex}

  \subimport{thesis/sepproofs/}{maxflowcontinuitysepproof.tex}

  \subimport{thesis/sepproofs/}{fcfscorrectness.tex}

  \subimport{thesis/sepproofs/}{fcfscomplexity.tex}

  Note that we choose to calculate the complexity of the \texttt{length()} function as $O\left(n\right)$. Whereas this
  complexity is implementation-dependent, even with the most naive approaches it cannot be higher than $O\left(n\right)$,
  thus we use this worst-case complexity in our analysis to cover all cases. This approach is implicitly used in all
  subsequent complexity analyses, but since all complexities are greater than $O\left(n\right)$, this approach does not
  influence them.

  \ \\

  \subimport{thesis/sepproofs/}{abscorrectness.tex}

  \subimport{thesis/sepproofs/}{abscomplexity.tex}

  \subimport{thesis/sepproofs/}{absDinfnormminproof.tex}

  \subimport{thesis/sepproofs/}{propcorrectness.tex}

  \subimport{thesis/sepproofs/}{propcomplexity.tex}

  \subimport{thesis/sepproofs/}{maxflowmonotonicitysepproof.tex}

  We will now prove that \texttt{BinSearch} returns the desired $\delta^*$ when we provide it with an appropriate interval as
  input.

  \ \\

  \subimport{thesis/sepproofs/}{dinfbinsearchcorrectness.tex}

  \subimport{thesis/sepproofs/}{dinfbinsearchcomplexity.tex}

  \subimport{thesis/sepproofs/}{dinfmincorrectness.tex}

  \subimport{thesis/sepproofs/}{dinfmincomplexity.tex}
