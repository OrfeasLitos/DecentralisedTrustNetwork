    \subsection{Common Notation}
     \begin{definition}[Neighbourhood]
        \label{neighbourhood}
        \begin{enumerate}
           \item Let $N^{+}\left(A\right)_j$ be the set of players $B$ that $A$ directly trusts with any positive value at
              the end of turn $j$. More formally,
              \begin{equation}
                 N^{+}\left(A\right)_j = \{B \in \mathcal{V}_j : DTr_{A \rightarrow B, j} > 0\} \enspace.
              \end{equation}
              $N^{+}\left(A\right)_j$ is called out neighbourhood of $A$ on turn $j$. Let $S \subseteq \mathcal{V}_j$. Then
              \begin{equation}
                 N^{+}\left(S\right)_j = \bigcup\limits_{A \in S}N^{+}\left(A\right)_j \enspace.
              \end{equation}
           \item Let $N^{-}\left(A\right)_j$ be the set of players $B$ that directly trust $A$ with any positive value at the
              end of turn $j$. More formally,
              \begin{equation}
                 N^{-}\left(A\right)_j = \{B \in \mathcal{V}_j : DTr_{B \rightarrow A, j} > 0\} \enspace.
              \end{equation}
              $N^{-}\left(A\right)_j$ is called in neighbourhood of $A$ on turn $j$. Let $S \subseteq \mathcal{V}_j$. Then
              \begin{equation}
                 N^{-}\left(S\right)_j = \bigcup\limits_{A \in S}N^{-}\left(A\right)_j \enspace.
              \end{equation}
           \item Let $N\left(A\right)_j$ be the set of players $B$ that either directly trust or are directly trusted by $A$
              with any positive value at the end of turn $j$. More formally,
              \begin{equation}
                 N\left(A\right)_j = N^{+}\left(A\right)_j \cup N^{-}\left(A\right)_j \enspace.
              \end{equation}
              $N\left(A\right)_j$ is called neighbourhood of $A$ on turn $j$. Let $S \subset \mathcal{V}_j$. Then
              \begin{equation}
                 N\left(S\right)_j = N^{+}\left(S\right)_j \cup N^{-}\left(S\right)_j \enspace.
              \end{equation}
        \end{enumerate}
     \end{definition}
     \begin{definition}[Total Incoming/Outgoing Trust]
     \label{inouttrust}
        \begin{equation}
           in_{A, j} = \sum\limits_{v \in N^{-}\left(A\right)_j}DTr_{v \rightarrow A, j}
        \end{equation}
        \begin{equation}
           out_{A, j} = \sum\limits_{v \in N^{+}\left(A\right)_j}DTr_{A \rightarrow v, j}
        \end{equation}
     \end{definition}
     \newpage 
     Here we add some concrete $Turn_j$ examples. Let $A = Player(j)$.
     \begin{enumerate}
        \item \begin{equation*}
           Turn_j = \emptyset
        \end{equation*}
        \item \begin{equation*}
           Turn_j = \{Steal\left(y, B\right), Add\left(w, B\right)\} \enspace,
        \end{equation*}
        given that
        \begin{equation*}
           0 \leq y \leq DTr_{B \rightarrow A, j-1} \wedge -DTr_{A \rightarrow B, j-1} \leq w \wedge w - y \leq
           Cap_{A, j-1} \enspace.
        \end{equation*}
        \item \begin{equation*}
           Turn_j = \{Steal\left(x, B\right), Add\left(y, C\right), Add\left(w, D\right)\} \enspace,
        \end{equation*}
        given that
        \begin{equation*}
        \begin{gathered}
           0 \leq x \leq DTr_{B \rightarrow A, j-1} \wedge -DTr_{A \rightarrow C, j-1} \leq y \: \wedge \\
           \wedge -DTr_{A \rightarrow D, j-1} \leq w \wedge y + w - x \leq Cap_{A, j-1} \enspace.
        \end{gathered}
        \end{equation*}
        \item \begin{equation*}
           Turn_j = \{Steal\left(x, B\right), Steal\left(y, B\right)\}
        \end{equation*}
        is not a valid turn because it contains two $Steal\left(\right)$ actions against the same player. If
        \begin{equation*}
           0 \leq x \wedge 0 \leq y \wedge x + y \leq DTr_{B \rightarrow A, j-1} \enspace,
        \end{equation*}
        the correct alternative would be
        \begin{equation*}
           Turn_j = \{Steal\left(x+y, B\right)\} \enspace.
        \end{equation*}
     \end{enumerate}
     \begin{definition}[Previous/Next Turn]
        Let $j \in \mathbb{N}$ a turn with $Player\left(j\right)$ $= A$. We define $prev\left(j\right), next\left(j\right)$
        as the previous and next turn that $A$ is chosen to play respectively. If $j$ is the first turn that $A$ plays,
        $prev\left(j\right) = 0$. More formally, if
        \begin{equation*}
           P = \{k \in \mathbb{N} : k < j \wedge Player\left(k\right) = A\} \mbox{ and}
        \end{equation*}
        \begin{equation*}
           N = \{k \in \mathbb{N} : k > j \wedge Player\left(k\right) = A\} \enspace,
        \end{equation*}
        then we define $prev\left(j\right), next\left(j\right)$ as follows:
        \begin{equation}
           prev\left(j\right) = \begin{cases}
              \max{P}, & P \neq \emptyset \\
              0, & P = \emptyset
           \end{cases}
        \end{equation}
        \begin{equation}
           next\left(j\right) = \min{N}
        \end{equation}
        $next\left(j\right)$ is always well defined with the assumption that after each turn eventually everybody plays.
     \end{definition}
