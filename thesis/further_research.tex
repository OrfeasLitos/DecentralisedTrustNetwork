\section{Further Research}

  While our trust network can form a basis for risk-invariant transactions in a pseudonymous and decentralized setting, more
  research is required to achieve other desirable properties. Some directions for future research are outlined below.

%\subsection{Trust Transfer Algorithms}
  If $Alice$ trusts $Bob$ enough to make a purchase from him, she should not directly pay him the value of the good because
  then she will increase her trust towards $Bob$. She first has to reduce her outgoing direct trust in a manner such that the
  supposition (\ref{primetrust}) of Risk Invariance theorem is satisfied. The methods $Alice$ can use to recalculate her
  outgoing trust will be discussed in a future paper.

%\subsection{Dynamic Setting}
  The current description of TrustIsRisk refers to a static setting where the game evolves in turns. In each turn only one
  user changes the state of the network and the game is controlled by a central algorithm, the TrustIsRisk Game. In the
  dynamic setting, users should be able to play simultaneously, freely join, depart or disconnect temporarily from the
  network.

%\subsection{Zero knowledge}
  Our network evaluates indirect trust by computing the max flow in the graph of lines-of-credit. In order to do that,
  complete information about the network is required. However, disclosing the network topology may be undesirable, as
  it subverts the identity of the participants even when participants are treated pseudonymously, as deanonymisation
  techniques can be used \cite{deanonymisation}. To avoid such issues, exploring the ability to calculate flows in a
  zero knowledge fashion may be desirable. However, performing network queries in zero knowledge may allow an adversary
  to extract topological information. More research is required to establish how flows can be calculated effectively in
  zero knowledge and what bounds exist in regards to information revealed in such fashion.

%\subsection{Game Theoretic Analysis}
  Our game theoretic analysis is simple. An interesting analysis would involve modelling repeated purchases with the
  respective edge updates on the trust graph and treating trust on the network as part of the utility function.

%\subsection{Implementation}
  We are proposing a concrete financial game and not a theoretical concept. Thus its implementation as a wallet on any
  blockchain will be most welcome.

%\subsection{Experimental Results}
  A simulation or actual implementation of TrustIsRisk, combined with analysis of the resulting dynamics can yield
  interesting experimental results. Subsequently, our trust network can be used in other applications, such as decentralized
  social networks \cite{synereo}.

%\subsection{Alternative Multisigs}
  1-of-2 multisigs correspond intuitively to simple directed weighted graphs. However it can be interesting to explore
  the trust relations that can arise by using other types of multisig, such as 1-of-3, as vessel for multi-party trust
  schemes. Our results do not necessarily hold for other multisigs and the simple relations now represented by directed
  weighted graphs have to be revised under a new kind of representation.
