\section{Trust Flow}
  We can now define the indirect trust, or simply trust, from $A$ to $B$.
  \subimport{common/definitions/}{indirecttrust.tex}
  \noindent It is $Tr_{A \rightarrow B} \geq DTr_{A \rightarrow B}$. The next theorem shows that
  $Tr_{A \rightarrow B}$ is finite.
  \subimport{common/theorems/}{convergencetheorem.tex}
  \subimport{common/proofsketches/}{convergenceproofsketch.tex}
  Full proofs of all theorems and lemmas can be found in the Appendix.

  In the setting of \texttt{TransitiveGame(}$\mathcal{G}$\texttt{,}$A$\texttt{,}$E$\texttt{)}, we make use of the notation
  $Loss_A = Loss_{A, j}$, where $j$ is a turn that the game has converged. It is important to note that $Loss_A$ is
  not the same for repeated executions of this kind of game, since the order in which players are chosen may differ between
  executions and the conservative players are free to choose which incoming trusts they will steal and how much from each.

  Let $G$ be a weighted directed graph. We will investigate the maximum flow on this graph. For an introduction to the
  maximum flow problem see \cite{clrs} p. 708. Considering each edge's capacity as its weight, a flow assignment
  $X = [x_{vw}]_{V \times V}$ with a source $A$ and a sink $B$ is valid when:
  \begin{equation}
  \label{flow1}
     \forall (v, w) \in E, x_{vw} \leq c_{vw} \mbox{ and}
  \end{equation}
  \begin{equation}
  \label{flow2}
     \forall v \in V \setminus \{A,B\}, \sum\limits_{w \in N^{+}(v)}x_{wv} = \sum\limits_{w \in N^{-}(v)}x_{vw}
     \enspace.
  \end{equation}
  We do not suppose any skew symmetry in $X$. The flow value is $\sum\limits_{v \in N^{+}\left(A\right)}x_{Av}$, which is
  proven to be equal to $\sum\limits_{v \in N^{-}\left(B\right)}x_{vB}$. There exists an algorithm that returns the maximum
  possible flow from $A$ to $B$, namely $MaxFlow\left(A, B\right)$. This algorithm evidently needs full knowledge of the
  graph. The fastest version of this algorithm runs in $O\left(|V||E|\right)$ time \cite{maxflownm}. We refer to the flow
  value of $MaxFlow\left(A, B\right)$ as $maxFlow\left(A, B\right)$.

  We will now introduce two lemmas that will be used to prove the one of the central results of this work, the Trust Flow
  theorem.
  \subimport{common/lemmas/}{flowgamelemma.tex}
  \subimport{common/proofsketches/}{flowgameproofsketch.tex}
  \subimport{common/lemmas/}{gameflowlemma.tex}
  \subimport{common/proofsketches/}{gameflowproofsketch.tex}
  \subimport{common/theorems/}{trustflowtheorem.tex}
  \subimport{common/proofs/}{trustflowproof.tex}

   Note that the maxFlow is the same in the following two cases: If a player chooses the evil strategy and if that player
   chooses a variation of the evil strategy where she does not nullify her outgoing direct trust.

   Here we see another important theorem that gives the basis for risk-invariant transactions between different, possibly
   unknown, parties.
   \subimport{common/theorems/}{riskinvtheorem.tex}
   \subimport{common/proofs/}{riskinvproof.tex}
   It is intuitively obvious that it is possible for $A$ to reduce her outgoing direct trust in a manner that achieves
   (\ref{primetrust}), since $maxFlow\left(A, B\right)$ is continuous with respect to $A$'s outgoing direct trusts. We
   leave this calculation as part of further research.
