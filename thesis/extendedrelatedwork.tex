\section{Related Work}
  Trust is a wide topic that exhibits very interesting properties and can be defined in several, often competing manners.
  Here we will present briefly several alternative approaches that have been followed in pursuit of a satisfactory model of
  trust and another tightly related and equally elusive concept, reputation.

  The topic of trust has been repeatedly attacked with several approaches: Purely cryptographic infrastructure where trust is
  rather binary and transitivity is not possible is explored in PGP \cite{pgp}. A transitive web-of-trust for fighting spam
  is explored in Freenet \cite{freenet}. Other systems require central trusted third parties, such as PKI \cite{pki} and
  Bazaar \cite{bazaar}, or, in the case of BFT, authenticated membership \cite{byzantine}.

  Mui and Halberstadt \cite{mui} have proposed an elaborate model based on the triptych "trust, reciprocity, reputation",
  where reciprocal actions of an agent $A$ generate a corresponding reputation, which in turn influences other agents' trust
  to $A$. Trusting $A$ inspires other agents to reciprocate, thus completing the cycle. In this model, actions are limited to
  $cooperate$ and $defect$, reciprocity and reputation between two agents are real numbers in $\left[0, 1\right]$, the latter
  also depending on the context of interest. Lastly trust is derived as a mean value based on the agent's reputation and the
  known history. The variables are connected using the Beta distribution from statistics.

  This model has little resemblance with Trust Is Risk not only in the formalities, but mainly in the approach taken. Trust
  Is Risk proposes a new financial game, whereas \cite{mui} attempts to model and predict all kinds of conceivable trust.
  Trust Is Risk does not use statistics nor scales trust to $\left[0, 1\right]$ and thus can provide strong results, such as
  the Risk Invariance theorem.

  One relevant proposal is \cite{beta} which proposes a set of definitions and mathematical manipulations pertaining to
  trust, essentially providing the core for other integrated trust and reputation systems. Once more the Beta distribution is
  used to model the expected behavior of others, a fact that results in two major drawbacks. First of all, each agent's
  actions are confined to exactly two options, a constraint not applicable to real-world applications. Secondly, expecting
  people to act according to a certain distribution function is inviting them to trick and circumvent this assumption for
  personal gain. Furthermore, Sybil attacks are not addressed. Lastly, the system proposed has a centralized structure,
  however its core components could also function in a decentralized manner.

  FIRE \cite{fire} constitutes another attempt to tackle trust, this time in a practical setting. FIRE aims to create a
  decentralized rating system for services provided. It essentially calculates trust as "the sum of all the available ratings
  weighted by the rating relevance and normalized to the range of $\left[-1, 1\right]$." This setup needs two very disputable
  assumptions: Firstly that "[a]gents are willing to share their experiences with others" and secondly that "[a]gents are
  honest in exchanging information with one another." One side effect of the above assumptions is that FIRE is susceptible to
  Sybil attacks. Trust Is Risk does not make these assumptions, but can function even when each player follows any strategy
  she desires.

  CORE \cite{core} proposes a reputation protocol for MANET to avoid non collaboration of nodes. It uses the terms trust and
  reputation almost interchangeably. It is designed in a manner such that reputation in several different settings can be
  expressed through it, for example it can be applied and improve network speeds on the DSR Route Discovery function and the
  Packet Forwarding function.

  No results relating to Sybil attacks are derived, thus Sybil attacks may be possible, albeit pointless, since this protocol
  refers to setups without other incentives other than reducing the time for completing various jobs. The trust model
  required in such settings is considerably simpler and rigorous results are not as necessary for the system to function at a
  satisfactory level and to improve on similar systems without reputation.

  \cite{ghkkw} delivers a protocol that claims to cover the reputation needs of MAS, or Multi Agent Systems. The setting of
  the problem is an open environment where each agent can freely come and go, retaining her reputation in the process.
  The context of the reputation should be decided according to the needs of each particular distributed task. The reputation
  data itself is also distributed with the use of a DHT and agents are uniquely identifiable. The system keeps track of all
  transactions to avoid ratings that do not correspond to a transaction. Furthermore, all transactions must be rated.
  Platform reputation is also implemented to differentiate between end users and software platforms built for users. Each
  agent's credentials must be signed by at least one more agent apart from the original one. A layered implementation is used
  to abstract from the communication details and the backend complexity to the frontend interface.

  Since no hard restrictions are placed on reputation form, scale and metric, the system retains a context agnostic stance
  that puts the burden of avoiding whitewashing and other attacks on those that will implement and use \cite{ghkkw} as a
  reputation mechanism for specific distributed tasks. Some simple attacks are discussed and specific countermeasures are
  proposed. For example, agent cooperation is rather fruitless since each pair of agents can have only one rating and this
  single rating does not weigh much. However, Sybil attacks are not even discussed and there are no privacy considerations:
  all ratings are public and traceable. Staying true to the context agnostic nature of this project, no concrete definitions
  of reputation or trust are given.

  The same problem is attacked in \cite{rk}, where a middleware trust management system consisting of several protocols is
  introduced to provide a decentralized, self-organized way for reducing or even eliminating malicious and dishonest behavior
  by peers. No user intervention is one of the design goals, another one being resilience against peers' collusions. Trust
  from $Alice$ to $Bob$ is defined as the probability with which $Bob$ will act in $Alice$'s desired manner, as calculated by
  $Alice$. $Bob$'s reputation is defined as the sum of all ratings he has received, one rating for each object he has
  offered. This definition results in a unique, global reputation for each peer. $Bob$'s reputation is stored by all his
  neighbors. For $Bob$ to be deemed trustworthy by $Alice$ as far as a certain object is concerned, $Bob$'s reputation must
  exceed a trust threshold chosen by $Alice$ specifically for this kind of object. A protocol similar to IP is used for
  routing object requests. A certain peer reputation is itself assigned more confidence when more peers respond with this
  particular reputation. This policy encourages peers to remain connected and makes dishonest reputation reporting riskier.

  Several simple attacks such as reputation altering are explored and mitigated with various measures. One attack that is
  only partially avoided is the situation where $Eve$ chooses to have only colluding neighbors that report falsely high
  reputation values for $Eve$. The confidence value proposed earlier does not mitigate this attack, since $Eve$ is in
  principle not discouraged from creating an arbitrarily large number of fake peers, thus reinforcing her forged positive
  reputation even more. SybilGuard \cite{sybilguard} is proposed as a possible remedy for this type of attacks.

  Pace \cite{pace} promises to provide a trust management model that can be readily incorporated in decentralized
  applications that use event-based software architectures.

  \cite{wot} is a formal attempt to quantify trust over identity. In this work, trust from $Alice$ to $Bob$ is represented by
  a percentage that expresses the level of confidence $Alice$ has that $Bob$ will only sign public PGP keys whose
  corresponding private keys are rightly and uniquely owned by the person stated in the public key. This is a special use
  case for trust, much like financial trust which is a largely unrelated kind of trust. This type of trust has some
  transitive properties that are derived from statistical models and do not have a connection with the concept of flow. One
  drawback of such a design is that there is no single indirect trust measure and as a consequence there are no strong
  results such as the Trust Flow theorem, but each user must experiment and choose one or more trust metrics of her liking
  based on subjective factors.

  A practical solution for determining chains of trust over PGP is \cite{pathfinder}. This project has a simple web interface
  that takes a start and an end public key and returns trust paths from the first to the second consisting of a chain of PGP
  signatures of public keys freely available on public keyservers. It is up to the user to decide how much trust a particular
  chain inspires. This project shows how a previously end-to-end construction with no transitive properties such as PGP can
  be used as basis for a network with transitive properties.

  Yet another proposal is Open Reputation \cite{openrep}. This project provides a general framework for rating generation and
  dissemination in the era of IoT. Reputers attribute reputation to reputees and make them available through their own
  reputation. Reputees are rated with certain reputes over predefined virtues. For example, the reputer $Alice$ can create
  a repute of 98 over the virtue "speed" that is defined as an integer between 1 and 100 for a reputee washing machine she
  bought. $Alice$ will publish this repute along with her own reputation.

  \cite{openrep} Does not provide any kind of Sybil resilience, nor does it diverge from the often revisited theme where
  reputation is projected on an arbitrary bounded scale. This model may serve well under the assumption of honest reputers,
  but can be tricked if selfish motives require so.

  \cite{sdt} design an ambitious framework that claims to cover all needs of decentralized trust models based on reputation.
  It consists of the 4C's: Content, Communication, Computation and Counteraction. Content refers to the agents' network
  structure (hierarchical, nested, etc.), the representation of the reputation (discrete, continuous, etc.), the context of
  the reputation (financial, medical, etc.) and the period of validity of the reputation. Communication refers to the
  protocol used to collect and transmit information (hierarchical or ad-hoc, etc.), to the allowed hop count of information
  and to the actual content of the messages exchanged, possibly of many different types (informational, revocation,
  confirmation, etc.) Computation engages with the mathematical and design details of trust derivation, such as whether a
  simple average of recommendations is used, how to combine external information with personal experience and how the period
  of validity influences the computation. Lastly counteraction expresses the particular model's method of feedback
  dissemination, with two methods being proposed, namely active and passive dissemination. An XML specification is proposed
  for all the aforementioned aspects of the desired trust and reputation system.

  It is currently unclear whether our model can be expressed in the terms of \cite{sdt}. More importantly, there is little
  insight into whether the task of expressing it in such terms will be a valuable asset for Trust Is Risk.

  \cite{kmrs} applies the MaxFlow algorithm into real-world situations of informal contract enforcement and money borrowing
  schemes. Their approach combines the algorithmic and sociological aspects of the issue in a productive way and their
  results constitute a strong confirmation of the validity of our assumption that trust is risk. More precisely, they show
  that money borrowing between residents in an area of Peru can be correctly predicted by deriving direct trust from the time
  residents spend together and calculating indirect trust with the MaxFlow algorithm. Their results show that our central
  design choice corresponds to real-world trust dynamics. The one important difference of their model with ours is that
  the graphs used in \cite{kmrs} are directionless because of the way direct trust is derived, thus making their case a
  special case of the Trust Is Risk graph, where all direct trust is obligatorily mutual and equal.

  In \cite{jgs} it is stated that "willingness to take risks may be one of the few characteristics common to all trust
  situations" and \cite{mds} cites the same passage, adding "Trust is not taking risk \textit{per se}, but rather it is a
  \textit{willingness} to take risk." These observations corroborate our choice to define trust as risk.

  \cite{mds} proposes a concrete model for trust that incorporates several notions. For example, trust from agent $A$ to
  agent $B$ is a factor of $A$'s Propensity and $B$'s Ability, Benevolence and Integrity. Once more, the target of \cite{mds}
  is different from ours in that \cite{mds} attempts to explain the inner workings of trust, whereas we define trust as risk
  and build a financial game atop of this assumption.

  \cite{jib} constitutes a thorough and highly informative overview of trust and reputation systems up to the time of
  writing. Flow models are briefly discussed, however our case is not covered.

  Bartercast \cite{bartercast} uses the maximum flow algorithm in an innovative way to calculate trust towards unknown
  BitTorrent peers. MaxFlow usage there closely resembles to ours, however the abscence of a blockchain leaves room for
  sybil attacks. Nevertheless, simulation results show that freeriders that selfishly take advantage of the network obtain a
  progressively worse reputation, a fact that strengthens our reliance on MaxFlow as a suitable algorithm for trust
  calculation.

  Bazaar \cite{bazaar} proposes an encanhement to existing centralized marketplaces where subjective trust is calculated
  using the MaxFlow algorithm. The bootstrapping process of the network is extremely similar to how players join the Trust Is
  Risk network. However, their approach contains a weak point in the way new trust is calculated after each transaction.
  Furthermore, trust between parties is commutative because non-directed graphs are used. This may be viewed as a crucial
  restriction for Bazaar.  Nevertheless, inclusion of this system as an additional fraud detection system would probably
  decrease fraud cases and as a result insurance fees would diminish and customer satisfaction would increase.

  Beaver \cite{beaver} proposes an integrated decentralized marketplace solution that provides all the functionality of eBay
  or other centralized marketplaces. Up to one public review per transaction is permitted and user ratings are globally
  calculated and not subjective. On the downside, ad-hoc fees must be attached to several reputation generating actions to
  deter fraudulent merchants from arbitrarily improving their ratings through Sybil or other attacks. Our system promises to
  automate and integrate several comparable parts of Beaver in a more intuitive system with less hand-tuned parameters and
  arbitrary fees that, while discouraging fraudulent action, they also reduce vendors' and customers' desire to participate.

  A very different direction is chosen by \cite{iou}, where an economy based on personal IOUs is proposed. According to this
  scheme, a payment from $Alice$ to $Bob$ can be completed by $Alice$ offering some of her IOUs. If $Bob$ trusts her, that is
  a valuable enough payment for him. Otherwise they can find a chain of trust, comprised by other intermediary agents, the
  first of which trusts $Alice$ and is trusted by the second and so on until the last one trusts the second last and is
  trusted by $Bob$. This model of economy has some interesting implications, namely that conventional currency is simply
  viewed as government IOUs and checks as bank IOUs. Unfortunately, this proposal was made prior to the advent of bitcoin
  and thus had no concrete basis to be built upon, leaving room for malicious intermediary nodes to fake or disclose
  contradictory trust amounts to different parties. Furthermore, the distributed nature of the system and its resemblance to
  contemporary bank relations could sharply increase the time needed for a simple transaction, because active agreement of
  many intermediate parties would be required.

  \cite{davis} proposes a system closely related to ours, the mechanics of which are very similar to Trust Is Risk. The
  work is accomplished in a more economic vein, considering the dynamics that arise from charging a premium for the
  references (essentially direct trust) parties provide. Trust in itself is not strictly defined, it is just suggested that
  references be given only to trusted parties. Since this work came before the advent of blockchains, it relies on the
  honesty of parties to stay true to the references they have published, given that these references are a type of insurance.
  The assumption of honesty for most of the players in the long term is backed by the positive social result such a behavior
  would yield. Long term stabilization of the network and maximum gains for parties can be achieved if there are no
  fraudulent players. One further implication of the lack of a blockchain is that for every failed transaction a potentially
  great amount of time may be needed for the defrauded party to collect insurance money. Furthermore, a relatively small
  number of intermediate players refusing to return the insured money can result in the defrauded party losing money, thus
  very complex investment strategies are required to minimize risk while maintaining gains.

  \cite{dionyziz} describes and analyzes the OpenBazaar infrastructure. As Trust Is Risk can be a natural extension of
  OpenBazaar, the aforementioned work provides valuable insight on how closely related decentralized marketplace systems
  function. More precisely, its game theoretic analysis constitutes a basis for the future corresponding analysis of our
  work and the elaborate attacks described and mitigated solve a range of problems that could arise in Trust Is Risk.
  However, the concept of trust is not rigorously defined in \cite{dionyziz}, thus Trust Is Risk fills that vacuum in an
  elegant manner.

  The discussion in Synereo \cite{synereo} does not revolve around trust, but it aims to describe a decentralized social
  network. As a result, trust is not rigorously defined, but reputation is and its measure is called \texttt{Reo}. Risking
  oversimplification, we could say that the posts of users flow through the network like \textit{current} flows through
  cables. Some \textit{charge} dissipates for every node that receives the post until no charge is left. A user may pay with
  \texttt{AMP}s to help her post travel further. \texttt{AMP} is social capital created by previous popular posts the user
  has generated. Furthermore, content viewable by $Alice$ is personalized through her \textit{engagement} with other users.
  $Alice$'s engagement with $Bob$ is a measure of her amount of time/energy spent on $Bob$'s posts. For example, engagement
  increases when $Alice$ reads and likes a post by $Bob$. $Bob$'s \texttt{Reo} is another factor that determines how high
  $Bob$'s posts will be placed in $Alice$'s stream. $Bob$'s \texttt{Reo} as viewed by $Alice$ is calculated as the mean
  engagement of $Alice$'s community with $Bob$'s posts. Putting it all together, current is calculated as the product of
  charge, \texttt{AMP}s, engagement and \texttt{Reo}.

  While \cite{synereo} does not qualify as a pure trust model, it undoubtedly contains a host of interesting ideas and
  approaches on decentralized reputation and mathematical manipulation of arbitrarily many independent event and content
  generators. It also belongs in the same extended family of ideas as Trust Is Risk in that it attempts to definitively
  port to the decentralized setting a popular service that currently exists only through centralized solutions, namely the
  service of social networking.

  Freenet \cite{freenet} proposes a decentralized platform for secure, deniable file storage and retrieval. Files and file
  identifiers are hashed and files themselves are additionally encrypted and stored reduntantly in all nodes that receive
  them, whether said nodes initiated the file request or not. An efficient routing protocol similar to IP is used and a Least
  Recently Used protocol is used for file retention.

  While not directly related with the concept of trust, Freenet is an interesting example of a functional decentralized
  system that provides specific positive properties and can be trusted as a system from its users, requiring little to no
  trust between users themselves.

  A similar property is exhibited by Bitcoin \cite{bitcoin} which assures the existence of a currency with no central
  issuing authority where users do not have to trust anyone else, just the infrastructure itself. More precisely, the advent
  of blockchains put the parties that would otherwise be able to forge coins in a situation where they compete amongst
  themselves to generate a valid block, which will in turn be verified by everybody else. This circumvents elegantly the need
  for trust to any external party and confines trust to each one's local machine. This is a reasonable trust to demand, since
  everyone can read and understand \cite{bitcoin} and furthermore verify their clients execute the correct code, or even
  develop their own implementation of the protocol. What Bitcoin achieves is a trustless consensus.

  Consensus is a problem loosely related to trust and its best expression is in \cite{byzantine} as the problem of the
  Byzantine generals, who want to reach a common agreement (e.g. attack or retreat) in the presence of faulty, malicious and
  dishonest parties who can report inconsistent values to different generals or fail to reply whatsoever. A strong result
  achieved in \cite{byzantine} and put into revolutionary use in Bitcoin is that in the presence of unforgeable signatures,
  the honest generals can achieve a consensus no matter how many generals are corrupted.
