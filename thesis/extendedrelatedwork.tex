\documentclass[11pt]{llncs}
\usepackage{preamble}

\begin{document}
  Trust is a wide topic that exhibits very interesting properties and can be defined in several, often competing manners.
  Here we will present briefly several alternative approaches that have been followed in pursuit of a satisfactory model of
  trust and another tightly related and equally elusive concept, reputation.

  Mui and Halberstadt \cite{mui} have proposed an elaborate model based on the triptych "trust, reciprocity, reputation",
  where reciprocal actions of an agent $A$ generate a corresponding reputation, which in turn influences other agents' trust
  to $A$. Trusting $A$ inspires other agents to reciprocate, thus completing the cycle. In this model, actions are limited to
  $cooperate$ and $defect$, reciprocity and reputation between two agents are real numbers in $\left[0, 1\right]$, the latter
  also depending on the context of interest. Lastly trust is derived as a mean value based on the agent's reputation and the
  known history. The variables are connected using the Beta distribution from statistics.

  This model has little resemblance with Trust Is Risk not only in the formalities, but mainly in the approach taken. Trust
  Is Risk proposes a new financial game, whereas \cite{mui} attempts to model and predict all kinds of conceivable trust.
  Trust Is Risk does not use statistics nor scales trust to $\left[0, 1\right]$ and thus can provide strong results, such as
  the Risk Invariance theorem.

  FIRE \cite{fire} constitutes another attempt to tackle trust, this time in a practical setting. FIRE aims to create a
  decentralized rating system for services provided. It essentially calculates trust as "the sum of all the available ratings
  weighted by the rating relevance and normalized to the range of $\left[-1, 1\right]$." This setup needs two very disputable
  assumptions: Firstly that "[a]gents are willing to share their experiences with others" and secondly that "[a]gents are
  honest in exchanging information with one another." Trust Is Risk does not make these assumptions, but can function even
  when each player follows any strategy she desires.

  In \cite{jgs} it is stated that "willingness to take risks may be one of the few characteristics common to all trust
  situations" and \cite{mds} cites the same passage, adding "Trust is not taking risk \textit{per se}, but rather it is a
  \textit{willingness} to take risk." These observations corroborate our choice to define trust as risk.

  \cite{mds} proposes a concrete model for trust that incorporates several notions. For example, trust from agent $A$ to
  agent $B$ is a factor of $A$'s Propensity and $B$'s Ability, Benevolence and Integrity. Once more, the target of \cite{mds}
  is different from ours in that \cite{mds} attempts to explain the inner workings of trust, whereas we 

  \cite{jib} constitutes a thorough overview of trust and reputation systems up to the time of writing. Flow models are
  briefly discussed, however our case is not covered.

  Bartercast \cite{bartercast} uses the maximum flow algorithm in an innovative way to calculate trust towards unknown
  BitTorrent peers. MaxFlow usage there closely resembles to ours, however the abscence of a blockchain leaves room for
  sybil attacks. Nevertheless, simulation results show that freeriders that selfishly take advantage of the network obtain a
  progressively worse reputation, a fact that strengthens our reliance on MaxFlow as a suitable algorithm for trust
  calculation.

  Bazaar \cite{bazaar} proposes an encanhement to existing centralized marketplaces where subjective trust is calculated
  using the MaxFlow algorithm. The bootstrapping process of the network is extremely similar to how players join the Trust Is
  Risk network. However, their approach contains a weak point in the way new trust is calculated after each transaction.
  Furthermore, trust between parties is commutative because non-directed graphs are used. This may be viewed as a crucial
  restriction for Bazaar.  Nevertheless, inclusion of this system as an additional fraud detection system would probably
  decrease fraud cases and as a result insurance fees would diminish and customer satisfaction would increase.

  Beaver \cite{beaver} proposes an integrated decentralized marketplace solution that provides all the functionality of eBay
  or other centralized marketplaces. Up to one public review per transaction is permitted and user ratings are globally
  calculated and not subjective. On the downside, ad-hoc fees must be attached to several reputation generating actions to
  deter fraudulent merchants from arbitrarily improving their ratings through Sybil or other attacks. Our system promises to
  automate and integrate several comparable parts of Beaver in a more intuitive system with less hand-tuned parameters.

 % \import{thesis/}{riskinvalgsintro.tex}
\end{document}
