\section{Related Work}
  Trust is a wide topic that exhibits very interesting properties and can be defined in several, often competing manners.
  Here we will present briefly several alternative approaches that have been followed in pursuit of a satisfactory model of
  trust and another tightly related and equally elusive concept, reputation.

  The topic of trust has been repeatedly attacked with several approaches: Purely cryptographic infrastructure where trust is
  rather binary and transitivity is not possible is explored in PGP \cite{pgp}. A transitive web-of-trust for fighting spam
  is explored in Freenet \cite{freenet}. Other systems require central trusted third parties, such as PKI \cite{pki} and
  Bazaar \cite{bazaar}, or, in the case of BFT, authenticated membership \cite{byzantine}.

  Mui and Halberstadt \cite{mui} have proposed an elaborate model based on the triptych "trust, reciprocity, reputation",
  where reciprocal actions of an agent $A$ generate a corresponding reputation, which in turn influences other agents' trust
  to $A$. Trusting $A$ inspires other agents to reciprocate, thus completing the cycle. In this model, actions are limited to
  $cooperate$ and $defect$, reciprocity and reputation between two agents are real numbers in $\left[0, 1\right]$, the latter
  also depending on the context of interest. Lastly trust is derived as a mean value based on the agent's reputation and the
  known history. The variables are connected using the Beta distribution from statistics.

  This model has little resemblance with Trust Is Risk not only in the formalities, but mainly in the approach taken. Trust
  Is Risk proposes a new financial game, whereas \cite{mui} attempts to model and predict all kinds of conceivable trust.
  Trust Is Risk does not use statistics nor scales trust to $\left[0, 1\right]$ and thus can provide strong results, such as
  the Risk Invariance theorem.

  One relevant proposal is \cite{beta} which proposes a set of definitions and mathematical manipulations pertaining to
  trust, essentially providing the core for other integrated trust and reputation systems. Once more the Beta distribution is
  used to model the expected behavior of others, a fact that results in two major drawbacks. First of all, each agent's
  actions are confined to exactly two options, a constraint not applicable to real-world applications. Secondly, expecting
  people to act according to a certain distribution function is inviting them to trick and circumvent this assumption for
  personal gain. Furthermore, Sybil attacks are not addressed. Lastly, the system proposed has a centralized structure,
  however its core components could also function in a decentralized manner.

  FIRE \cite{fire} constitutes another attempt to tackle trust, this time in a practical setting. FIRE aims to create a
  decentralized rating system for services provided. It essentially calculates trust as "the sum of all the available ratings
  weighted by the rating relevance and normalized to the range of $\left[-1, 1\right]$." This setup needs two very disputable
  assumptions: Firstly that "[a]gents are willing to share their experiences with others" and secondly that "[a]gents are
  honest in exchanging information with one another." One side effect of the above assumptions is that FIRE is susceptible to
  Sybil attacks. Trust Is Risk does not make these assumptions, but can function even when each player follows any strategy
  she desires.

  \cite{sdt} design an ambitious framework that claims to cover all needs of decentralized trust models based on reputation.
  It consists of the 4C's: Content, Communication, Computation and Counteraction. Content refers to the agents' network
  structure (hierarchical, nested, etc.), the representation of the reputation (discrete, continuous, etc.), the context of
  the reputation (financial, medical, etc.) and the period of validity of the reputation. Communication refers to the
  protocol used to collect and transmit information (hierarchical or ad-hoc, etc.), to the allowed hop count of information
  and to the actual content of the messages exchanged, possibly of many different types (informational, revocation,
  confirmation, etc.) Computation engages with the mathematical and design details of trust derivation, such as whether a
  simple average of recommendations is used, how to combine external information with personal experience and how the period
  of validity influences the computation. Lastly counteraction expresses the particular model's method of feedback
  dissemination, with two methods being proposed, namely active and passive dissemination. An XML specification is proposed
  for all the aforementioned aspects of the desired trust and reputation system.

  It is currently unclear whether our model can be expressed in the terms of \cite{sdt}. More importantly, there is little
  insight into whether the task of expressing it in such terms will be a valuable asset for Trust Is Risk.

  \cite{kmrs} applies the MaxFlow algorithm into real-world situations of informal contract enforcement and money borrowing
  schemes. Their approach combines the algorithmic and sociological aspects of the issue in a productive way and their
  results constitute a strong confirmation of the validity of our assumption that trust is risk. More precisely, they show
  that money borrowing between residents in an area of Peru can be correctly predicted by deriving direct trust from the time
  residents spend together and calculating indirect trust with the MaxFlow algorithm. Their results show that our central
  design choice corresponds to real-world trust dynamics. The one important difference of their model with ours is that
  the graphs used in \cite{kmrs} are directionless because of the way direct trust is derived, thus making their case a
  special case of the Trust Is Risk graph, where all direct trust is obligatorily mutual and equal.

  In \cite{jgs} it is stated that "willingness to take risks may be one of the few characteristics common to all trust
  situations" and \cite{mds} cites the same passage, adding "Trust is not taking risk \textit{per se}, but rather it is a
  \textit{willingness} to take risk." These observations corroborate our choice to define trust as risk.

  \cite{mds} proposes a concrete model for trust that incorporates several notions. For example, trust from agent $A$ to
  agent $B$ is a factor of $A$'s Propensity and $B$'s Ability, Benevolence and Integrity. Once more, the target of \cite{mds}
  is different from ours in that \cite{mds} attempts to explain the inner workings of trust, whereas we define trust as risk
  and build a financial game atop of this assumption.

  \cite{jib} constitutes a thorough and highly informative overview of trust and reputation systems up to the time of
  writing. Flow models are briefly discussed, however our case is not covered.

  Bartercast \cite{bartercast} uses the maximum flow algorithm in an innovative way to calculate trust towards unknown
  BitTorrent peers. MaxFlow usage there closely resembles to ours, however the abscence of a blockchain leaves room for
  sybil attacks. Nevertheless, simulation results show that freeriders that selfishly take advantage of the network obtain a
  progressively worse reputation, a fact that strengthens our reliance on MaxFlow as a suitable algorithm for trust
  calculation.

  Bazaar \cite{bazaar} proposes an encanhement to existing centralized marketplaces where subjective trust is calculated
  using the MaxFlow algorithm. The bootstrapping process of the network is extremely similar to how players join the Trust Is
  Risk network. However, their approach contains a weak point in the way new trust is calculated after each transaction.
  Furthermore, trust between parties is commutative because non-directed graphs are used. This may be viewed as a crucial
  restriction for Bazaar.  Nevertheless, inclusion of this system as an additional fraud detection system would probably
  decrease fraud cases and as a result insurance fees would diminish and customer satisfaction would increase.

  Beaver \cite{beaver} proposes an integrated decentralized marketplace solution that provides all the functionality of eBay
  or other centralized marketplaces. Up to one public review per transaction is permitted and user ratings are globally
  calculated and not subjective. On the downside, ad-hoc fees must be attached to several reputation generating actions to
  deter fraudulent merchants from arbitrarily improving their ratings through Sybil or other attacks. Our system promises to
  automate and integrate several comparable parts of Beaver in a more intuitive system with less hand-tuned parameters and
  arbitrary fees that, while discouraging fraudulent action, they also reduce vendors' and customers' desire to participate.

  A very different direction is chosen by \cite{iou}, where an economy based on personal IOUs is proposed. According to this
  scheme, a payment from $Alice$ to $Bob$ can be completed by $Alice$ offering some of her IOUs. If $Bob$ trusts her, that is
  a valuable enough payment for him. Otherwise they can find a chain of trust, comprised by other intermediary agents, the
  first of which trusts $Alice$ and is trusted by the second and so on until the last one trusts the second last and is
  trusted by $Bob$. This model of economy has some interesting implications, namely that conventional currency is simply
  viewed as government IOUs and checks as bank IOUs. Unfortunately, this proposal was made prior to the advent of bitcoin
  and thus had no concrete basis to be built upon, leaving room for malicious intermediary nodes to fake or disclose
  contradictory trust amounts to different parties. Furthermore, the distributed nature of the system and its resemblance to
  contemporary bank relations could sharply increase the time needed for a simple transaction, because active agreement of
  many intermediate parties would be required.

  \cite{dionyziz} describes and analyzes the OpenBazaar infrastructure. As Trust Is Risk can be a natural extension of
  OpenBazaar, the aforementioned work provides valuable insight on how closely related decentralized marketplace systems
  function. More precisely, its game theoretic analysis constitutes a basis for the future corresponding analysis of our
  work and the elaborate attacks described and mitigated solve a range of problems that could arise in Trust Is Risk.
