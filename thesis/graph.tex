  \section{The Trust Graph}
     We now engage in the formal description of the proposed system, accompanied by helpful examples.
     \subimport{../common/definitions/}{gamegraph.tex}
     The nodes represent the players, the edges represent the existing direct trusts and the weights represent the amount of
     value attached to the corresponding direct trust. As we will see, the game evolves in turns. The subscript of the graph
     represents the corresponding turn.
     \subimport{../common/definitions/}{players.tex}
     Each node has a corresponding non-negative number that represents its capital. A node's capital is the total value that
     the node possesses exclusively and nobody else can spend.
     \subimport{../common/definitions/}{capital.tex}
     The capital of $A$ is subsequently modified only during her turns, according to her actions. We also note that a
     player's assets are the sum of her capital and her total outgoing trust.

     The formal definition of direct trust follows:
     \subimport{../common/definitions/}{directtrust.tex}
     This definition agrees with the title of this paper and coincides with the intuition and experimental results of
     \cite{kmrs} that the trust $Alice$ shows to $Bob$ in real-world social networks corresponds with the extent
     of danger in which $Alice$ is ready to expose herself to in order to help $Bob$. An example graph with its corresponding
     transactions in the UTXO can be seen below.

     \subimport{../common/figures/}{utxo.tikz}

     Any algorithm that has access to the graph $\mathcal{G}_j$ has implicitly access to all direct trusts of this graph.
     We use the notation $N^{+}(A)$ to refer to the nodes directly trusted by $A$ and $N^{-}(A)$ for the nodes that directly
     trust $A$. We also use the notation $in_{A, j}, out_{A, j}$ to refer to the total incoming and outgoing direct trust
     respectively. For a reference of common definitions, see the Appendix.
