\section{Evolution of Trust}
  \subimport{thesis/definitions/}{turns.tex}
  \subimport{common/}{turnexample.tex}
  We use $prev\left(j\right)$ and $next\left(j\right)$ to denote the previous and next turn respectively played by
  $Player(j)$. A formal definition can be found in the Appendix.
  \subimport{common/definitions/}{damage.tex}
  \subimport{common/definitions/}{history.tex}
  \noindent Knowledge of the initial graph $\mathcal{G}_0$, all players' initial capital and the history amount to full
  comprehension of the evolution of the game. Building on the example of figure \ref{fig:utxo}, we can see the resulting graph
  when $D$ plays
  \begin{equation}
  \label{turnexample}
    Turn_1 = \{Steal\left(1, A\right), Add\left(4, C\right)\} \enspace.
  \end{equation}
  \subimport{common/figures/}{turnexample.tikz}

  \noindent In the form presented here, Trust Is Risk is controlled by an algorithm that chooses a player, receives the turn
  that this player wishes to play and, if this turn is valid, executes it. These steps are repeated indefinitely. We assume
  players are chosen in a way that, after her turn, a player will eventually play again later.
  \subimport{thesis/algorithms/}{trustisriskgame.tex}

  \noindent \texttt{strategy[}$A$\texttt{]()} provides player $A$ with full knowledge of the game, except for the capitals of
  other players. This assumption may not be realistic because out of band knowledge and traffic analysis methods can be used
  to infer the capital of other players.

  \texttt{executeTurn()} checks the validity of \texttt{Turn} and substitutes it with an empty turn if invalid.
  Subsequently, it creates the new graph $\mathcal{G}_j$ and updates the history accordingly. For the routine code,
  see the Appendix.
