\begin{sepproof}{Proof of correctness for algorithm \texttt{abs}} \ \\
  First of all, we can note that, if $F_{prov} \leq F - V$ in line~\ref{abs:iffprovbig}, then the execution will enter the
  \texttt{else} clause of line~\ref{abs:elsefprovsmall}. Therefore, in line~\ref{abs:setlastfcur}, $F_{cur}$ will get the
  value $F - V$, as we can see by executing the lines~\ref{abs:saveprevreduction} -~\ref{abs:setlastfcur} by hand. This in
  turn means that the loop in line~\ref{abs:loop} will break right after the \texttt{else} clause is executed. Furthermore,
  the assignment in line~\ref{abs:setfcur} in combination with the truth of the statement $F_{prov} > F - V$ in
  line~\ref{abs:iffprovbig} shows that, if the execution enters the \texttt{if} clause of line~\ref{abs:iffprovbig}, then
  the loop of line~\ref{abs:loop} will be executed at least once more. These two observations amount to the fact that the
  \texttt{else} clause will be executed exactly one time and afterwards the \texttt{while} loop will break.

  We use the notation $F_{cur, 0}$, $reduction_0$ and $empty_0$ to refer to the initial values of the corresponding
  variables, as set in lines~\ref{abs:finit},~\ref{abs:reductioninit} and~\ref{abs:emptyinit} respectively. Furthermore, the
  notation $empty_j$, $reduction_j$, $F_{cur, j}$ and $i_j$ is used to refer to the values of the corresponding variables
  after the $j$-th iteration of the \texttt{while} loop. $i_j$ is chosen in line~\ref{abs:popmin}. From
  lines~\ref{abs:setfprov},~\ref{abs:setreduction},~\ref{abs:setfcur},~\ref{abs:iffprovbig} and~\ref{abs:saveprevreduction}
  to~\ref{abs:setlastfcur} we see that
  \begin{equation*}
    F_{cur, j} =
    \begin{cases}
       F_{prov, j},& F_{prov, j} > F - V \\
       F - V ,& F_{prov, j} \leq F - V
    \end{cases} \enspace, \mbox{ where}
  \end{equation*}
  \begin{equation*}
  \begin{gathered}
    F_{prov, j} = F_{cur, j-1} - \left(n - empty_{j-1}\right)\left(x_{i_j} - x_{i_{j-1}}\right), j \geq 1 \mbox{ and }
    x_{i_0} = 0 \enspace.
  \end{gathered}
  \end{equation*}

  It is worth noting that the maximum number of iterations is $n$, or else $j \leq n$. This holds because, if we suppose that
  $F_{cur, n+1}$ exists, it is
  \begin{equation}
  \label{abs:fnreq}
    F_{cur, n} > F - V \geq 0
  \end{equation}
  However, we can easily see that in this case
  \begin{equation*}
  \begin{gathered}
    F_{cur, n} = F_{cur, 0} - \sum\limits_{j=1}^n\left(n - \left(j-1\right)\right)\left(x_{i_j} - x_{i_{j-1}}\right) = \\
    = \sum\limits_{j=1}^nx_{i_j} - \sum\limits_{j=1}^n\left(n - \left(j - 1\right)\right)x_{i_j} +
    \sum\limits_{j=1}^{n-1}\left(n - j\right)x_{i_j} = \\
    = \sum\limits_{j=1}^nx_{i_j} - \sum\limits_{j=1}^{n-1}[\left(n - \left(j - 1\right)\right) - \left(n - j\right)]x_{i_j}
    - \left(n - \left(n - 1\right)\right)x_{i_n} = \\
    = \sum\limits_{j=1}^nx_{i_j} - \sum\limits_{j=1}^{n-1}x_{i_j} - x_{i_n} = 0\enspace,
  \end{gathered}
  \end{equation*}
  which is a contradiction to (\ref{abs:fnreq}), thus $F_{cur, n+1}$ does not exist and $j \leq n$. This means
  that \texttt{popMin()} will never fail.

  We will now show that $\forall j \in [n], empty_j < n$. At line~\ref{abs:emptyinit}, it is $empty_0 = 0 < n$. $empty$
  is again modified in line~\ref{abs:setempty}, where it is incremented by at most 1 at each iteration of the \texttt{while}
  loop (line~\ref{abs:loop}). As we saw above, the iterations cannot exceed $n$ and $empty$ is not incremented in the last
  iteration which consists of the \texttt{else} clause, thus $\forall j \in [n], empty_j < n$.

  Next, we will show that $\forall i \in [n], c'_i \leq x_i$, as per the requirement (\ref{naive:req1}). From
  line~\ref{abs:setcap}, we see that it suffices to prove that $reduction \geq 0$. In line~\ref{abs:reductioninit},
  $reduction$ is initialized to 0. In line~\ref{abs:setreduction}, $reduction$ is set to $x_i$, which is always
  a non-negative value. The last line where $reduction$ is modified is~\ref{abs:setlastreduction}. In this line, it is
  $F_{cur} > F - V$ or else the \texttt{while} loop would have broken before beginning this iteration and $n > empty$ as we
  previously saw. Thus the non-negative variable $reduction$ is increased and the resulting value is always positive.

  We will finally show that $\sum\limits_{i=1}^nc'_i = F - V$, which satisfies the requirement (\ref{naive:req2}).
  Let $k, 0 \leq k \leq n$ be such that at the end of the execution
  \begin{equation*}
    \forall j \leq k, c_{i_j} = 0 \wedge \forall j > k, c_{i_j} > 0 \enspace.
  \end{equation*}
  The following holds:
  \begin{equation*}
  \begin{gathered}
    \sum\limits_{j=1}^nc'_{i_j} = \sum\limits_{j=k+1}^nc'_{i_j} = \sum\limits_{j=k+1}^n\left(x_{i_j} -
    reduction_{k+1}\right) = \\
    \sum\limits_{j=k+1}^n\left(x_{i_j} - \left(x_{i_k} + \frac{F_{cur, k} - \left(F - V\right)}{n - k}\right)\right) \\
    \sum\limits_{j=k+1}^n\left(x_{i_j} - \left(x_{i_k} + \frac{F_{cur, 0} - \sum\limits_{l=1}^k\left(n
    - \left(l - 1\right)\right)\left(x_{i_l} - x_{i_{l-1}}\right) - F + V}{n - k}\right)\right) \\
    \sum\limits_{j=k+1}^n\left(x_{i_j} - \left(x_{i_k} + \frac{F - \sum\limits_{l=1}^k\left(n - l + 1\right)x_{i_l} +
    \sum\limits_{l=1}^{k-1}\left(n - l\right)x_{i_l} - F + V}{n - k}\right)\right) \\
    \sum\limits_{j=k+1}^nx_{i_j} - \left(n - k\right)x_{i_k} - \frac{n - k}{n - k}\left(-\sum\limits_{l=1}^{k-1}x_{i_l} -
    \left(n - k + 1\right)x_{i_k} + V\right) \\
    \sum\limits_{j=k+1}^nx_{i_j} - \left(n - k\right)x_{i_k} + \sum\limits_{j=1}^{k-1}x_{i_j} +
    \left(n - k + 1\right)x_{i_k} - V \\
    = \sum\limits_{j=1}^nx_{i_j} - V = F - V \enspace,
  \end{gathered}
  \end{equation*}
  thus the desired property holds.
\end{sepproof}
