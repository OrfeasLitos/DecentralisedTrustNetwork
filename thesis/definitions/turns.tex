\phantomsection
\addcontentsline{toc}{subsection}{Turns Definition}
\begin{definition}[Turns]
  The game we are describing is turn-based. In each turn $j$ exactly one player $A \in \mathcal{V}, A =
  Player\left(j\right)$, chooses one or more actions from the following two kinds:

  \noindent \textit{\textbf{Steal($y_B$,$\:B$)}}: Steal value $y_B$ from $B \in N^{-}\left(A\right)_{j-1}$, where
  $0 \leq y_B \leq DTr_{B \rightarrow A, j-1}$. Then:
  \begin{equation*}
    DTr_{B \rightarrow A, j} = DTr_{B \rightarrow A, j-1} - y_B
  \end{equation*}
  \noindent \textit{\textbf{Add($y_B$,$\:B$)}}:
  Add value $y_B$ to $B \in \mathcal{V}$, where $-DTr_{A \rightarrow B, j-1} \leq y_B$. Then:
  \begin{equation*}
    DTr_{A \rightarrow B, j} = DTr_{A \rightarrow B, j-1} + y_B
  \end{equation*}
  When $y_B < 0$, we say that $A$ reduces her trust to $B$ by $-y_B$. When $y_B > 0$, we say that $A$ increases her
  trust to $B$ by $y_B$. If $DTr_{A \rightarrow B, j-1} = 0$, then we say that $A$ starts directly trusting $B$.
  If player $A$ chooses no action in her turn, we say that she passes her turn. Also, let $Y_{st}, Y_{add}$ be the
  total value to be stolen and added respectively by $A$ in her turn, $j$. For a turn to be feasible, it must hold that
  \begin{equation}
    Y_{add} - Y_{st} \leq Cap_{A, j-1} \enspace.
  \end{equation}
  The capital is updated in every turn:
  \begin{equation*}
    Cap_{A, j} = Cap_{A, j-1} + Y_{st} - Y_{add} \enspace.
  \end{equation*}

  Moreover, player $A$ is not allowed to choose two actions of the same kind against the same player in one turn.
  The set of actions the player makes in turn $j$ is denoted by $Turn_j$. The new graph that emerges by applying
  the actions on $\mathcal{G}_{j-1}$ is $\mathcal{G}_j$.
\end{definition}
